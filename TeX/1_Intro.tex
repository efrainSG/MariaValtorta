\documentclass[12pt]{book} % use larger type; default would be 10pt

\usepackage[utf8]{inputenc} % set input encoding (not needed with XeLaTeX)
\usepackage{substr}
\usepackage[hidelinks]{hyperref}
%%% PAGE DIMENSIONS
\usepackage{geometry} % to change the page dimensions
\geometry{letterpaper} % or letterpaper (US) or a5paper or....
% \geometry{margin=2in} % for example, change the margins to 2 inches all round
% \geometry{landscape} % set up the page for landscape
%   read geometry.pdf for detailed page layout information
\usepackage[spanish]{babel}

\usepackage{graphicx} % support the \includegraphics command and options

% \usepackage[parfill]{parskip} % Activate to begin paragraphs with an empty line rather than an indent

%%% PACKAGES
\usepackage{booktabs} % for much better looking tables
\usepackage{array} % for better arrays (eg matrices) in maths
\usepackage{paralist} % very flexible & customisable lists (eg. enumerate/itemize, etc.)
\usepackage{verbatim} % adds environment for commenting out blocks of text & for better verbatim
\usepackage{subfig} % make it possible to include more than one captioned figure/table in a single float
\usepackage{titlesec}
% These packages are all incorporated in the memoir class to one degree or another...

%%% HEADERS & FOOTERS
\usepackage{fancyhdr} % This should be set AFTER setting up the page geometry
\pagestyle{fancy} % options: empty , plain , fancy
\renewcommand{\headrulewidth}{0pt} % customise the layout...
\lhead{}\chead{}\rhead{}
\lfoot{}\cfoot{\thepage}\rfoot{}

%%% SECTION TITLE APPEARANCE
\usepackage{sectsty}
\allsectionsfont{\sffamily\mdseries\upshape} % (See the fntguide.pdf for font help)
% (This matches ConTeXt defaults)

%%% ToC (table of contents) APPEARANCE
\usepackage[nottoc,notlof,notlot]{tocbibind} % Put the bibliography in the ToC
\usepackage[titles,subfigure]{tocloft} % Alter the style of the Table of Contents
\renewcommand{\cftsecfont}{\rmfamily\mdseries\upshape}
\renewcommand{\cftsecpagefont}{\rmfamily\mdseries\upshape} % No bold!

\titleformat{\chapter}{\normalfont\Large\scshape}{\thechapter}{1em}{}
%%% END Article customizations

%%% The "real" document content comes below...

\begin{document}

\begin{titlepage}
\centering
\vspace{3.5cm}
{\scshape\LARGE El Evangelio\\como me fue revelado \par}
\vfill
{\scshape\Large María Valtorta \par}
\vfill
{\itshape \url{http://www.reinadelcielo.org/el-evangelio-como-me-fue-revelado-poema-de-el-hombre-dios-mar%EE%A1%ADvaltorta/} \par}
\vfill
Reina del Cielo
\end{titlepage}

\frenchspacing
\chapter*{Dios quiso un seno sin mancha.}
\addcontentsline{toc}{chapter}{\normalfont\scshape{Dios quiso un seno sin mancha}}

Dice Jesús: 
\emph{"Hoy escribe esto sólo. La pureza tiene un valor tal, que un seno de criatura pudo contener al Incontenible, porque poseía la máxima pureza posible en una criatura de Dios. La Santísima Trinidad descendió con sus perfecciones, habitó con sus Tres Personas, cerró su Infinito en pequeño espacio- no por ello se hizo menor, porque el amor de la Virgen y la voluntad de Dios dilataron este espacio hasta hacer de él un Cielo – y se manifestó con sus características: El Padre, siendo Creador nuevamente de la Criatura como en el sexto día y teniendo una "hija" verdadera, digna, a su perfecta semejanza. La impronta de Dios estaba estampada en María tan nítidamente, que sólo en el Primogénito del Padre era superior. María puede ser llamada la "segundogénita" del Padre, porque, por perfección dada y sabida conservar, y por dignidad de Esposa y Madre de Dios y de Reina del Cielo, viene segunda después del Hijo del Padre y segunda en su eterno Pensamiento, que ab aeterno en Ella se complació. El Hijo, siendo también para Ella "el Hijo" enseñándole, por misterio de gracia, su verdad y sabiduría cuando aún era sólo un Embrión que crecía en su seno. El Espíritu Santo, apareciendo entre los hombres por un anticipado Pentecostés, por un prolongado Pentecostés, Amor en "Aquella que amó", Consuelo para los hombres por el Fruto de su seno, Santificación por la maternidad del Santo. Dios, para manifestarse a los hombres en la forma nueva y completa que abre la era de la Redención, no eligió como trono suyo un astro del cielo, ni el palacio de un grande. No quiso tampoco las alas de los ángeles como base para su pie. Quiso un seno sin mancha. Eva también había sido creada sin mancha., Mas, espontáneamente, quiso corromperse. María, que vivió en un mundo corrompido – Eva estaba, por el contrario, en un mundo puro – no quiso lesionar su candor ni siquiera con un pensamiento vuelto hacia el pecado. Conoció la existencia del pecado y vio de él sus distintas y horribles manifestaciones, las vio todas, incluso la más horrenda: el deicidio. Pero las conoció para expiarlas y para ser, eternamente, Aquella que tiene piedad de los pecadores y ruega por su redención. Este pensamiento será introducción a otras santas cosas que daré para consuelo tuyo y de muchos".}
 
\chapter*{Joaquín y Ana hacen voto al Señor}
\addcontentsline{toc}{chapter}{\normalfont\scshape{Joaquín y Ana hacen voto al Señor}}

Veo un interior de una casa. Sentada a un telar hay una mujer ya de cierta edad. Viéndola con su pelo ahora entrecano, antes ciertamente negro, y su rostro sin arrugas pero lleno de esa seriedad que viene con los años, yo diría que puede tener de cincuenta a cincuenta y cinco años, no más. 

 Al indicar estas edades femeninas tomo como base el rostro de mi madre, cuya efigie tengo, más que nunca, presente estos días que me recuerdan los últimos suyos cerca de mi cama... Pasado mañana hará un año que ya no la veo... Mi madre era de rostro muy fresco bajo unos cabellos precozmente encanecidos. A los cincuenta años era blanca y negra como al final de la vida. Pero, aparte de la madurez de la mirada, nada denunciaba sus años. Por eso, pudiera ser que me equivocase al dar un cierto número de años a las mujeres ya mayores. 

Ésta, a la que veo tejer, está en una habitación llena de claridad. La luz penetra por la puerta, abierta de par en par, que da a un espacioso huerto – jardín. Yo diría que es una pequeña finca rústica, porque se prolonga onduladamente sobre un suave columpiarse de verdes pendientes. Ella es hermosa, de rasgos sin duda hebreos. Ojos negros y profundos que, no sé por qué, me recuerdan al del Bautista. Sin embargo, estos ojos, además de tener gallardía de reina, son dulces, como si su centelleo de águila estuviera velado de azul Ojos dulces, con un trazo de tristeza, como de quien pensara nostálgicamente en cosas perdidas. El color del rostro es moreno, aunque no excesivamente. La boca, ligeramente ancha, está bien proporcionada, detenida en un gesto austero pero no duro. La nariz es larga y delgada, ligeramente combada hacia abajo: una nariz aguileña que va bien con esos ojos. Es fuerte, mas no obesa. Bien proporcionada. A juzgar por su estatura estando sentada, creo que es alta. 

Me parece que está tejiendo una cortina o una alfombra. Las canillas multicolores recorren, rápidas, la trama marrón oscura. Lo ya hecho muestra una vaga entretejedura de grecas y flores en que el verde, el amarillo, el rojo y el azul oscuro se intersecan y funden como en un mosaico. La mujer lleva un vestido sencillísimo y muy oscuro: un morado - rojo que parece copiado de ciertas trinitarias. 

Oye llamar a la puerta y se levanta. Es alta realmente. Abre. 

Una mujer le dice: Ana, ¿me dejas tu ánfora? Te la lleno. 

La mujer trae consigo a un rapacillo de cinco años, que se agarra inmediatamente al vestido de Ana. Ésta le acaricia mientras se dirige hacia otra habitación, de donde vuelve con una bonita ánfora de cobre. Se la da a la mujer diciendo: Tú siempre eres buena con la vieja Ana. Dios te lo pague, en éste y en los otros hijos que tienes y que tendrás. ¡Dichosa tú!

Ana suspira. 

La mujer la mira y no sabe qué decir ante ese suspiro. Para apartar la pena, que se ve que existe, dice: - Te dejo a Alfeo, si no te causa molestias; así podré ir más deprisa y llenarte muchos cántaros. 

Alfeo está muy contento de quedarse, y se ve el porqué una vez que se ha ido la madre: Ana le coge en brazos y lo lleva al huerto, lo aupa hasta una pérgola de uva de color oro como el topacio y dice: Come, come, que es buena" - y le besa en la carita embadurnada del zumo de las uvas que está desgranando ávidamente. 

Luego, cuando el niño, mirándola con dos ojazos de un gris azul oscuro todo abiertos, dice: 

- ¿Y ahora qué me das? - se echa a reír con ganas, y, al punto, parece más joven, borrados los años por la bonita dentadura y el gozo que viste su rostro. Y ríe y juega, metiendo su cabeza entre las rodillas y diciendo: 

- ¿Qué me das si te doy... si te doy?.. ¡Adivina! - Y el niño, dando palmadas con sus manecitas, todo sonriente, dice: 

- ¡Besos, te doy besos, Ana guapa, Ana buena, Ana mamá!... 

Ana, al sentirse llamar "Ana mamá", emite un grito de afecto jubiloso y abraza estrechamente al pequeñuelo, diciendo: - ¡0h, tesoro! ¡Amor! ¡Amor! ¡Amor! - Y por cada "amor" un beso va a posarse sobre las mejillitas rosadas. 

Luego van a un bazar y de un plato bajan tortitas de miel. 

Las he hecho para ti, hermosura de la pobre Ana, para ti que me quieres. Dime, ¿cuánto me quieres? Y el niño, pensando en la cosa que más le ha impresionado, dice: 

Como al Templo del Señor. 

Ana le da más besos: en los ojitos avispados, en la boquita roja. Y el niño se restriega contra ella como un gatito. 

La madre va y viene con un jarro colmado y ríe sin decir nada. Les deja con sus efusiones de afecto. 

Entra en el huerto un hombre anciano, un poco más bajo que Ana, de tupida cabellera completamente cana, rostro claro, barba cortada en cuadrado, dos ojos azules como turquesas, entre pestañas de un castaño claro casi rubio. Está vestido de un marrón oscuro. 

Ana no lo ve porque da la espalda a la puerta. El hombre se acerca a ella por detrás diciendo: ¿Y a mí nada? 

Ana se vuelve y dice:

- ¡Oh, Joaquín! ¿Has terminado tu trabajo? 

 Mientras tanto el pequeño Alfeo ha corrido a sus rodillas diciendo: 

También a ti, también a ti- y cuando el anciano se agacha y le besa, el niño se le ciñe estrechamente al cuello despeinándole la barba con las manecitas y los besos. 

También Joaquín trae su regalo: saca de detrás la mano izquierda y presenta una manzana tan hermosa que parece de cerámica, y, sonriendo, al niño que tiende ávidamente sus manecitas le dice: 

Espera, que te la parto en trozos. Así no puedes. Es más grande que tú - y con un pequeño cuchillo que tiene en el cinturón (un cuchillo de podador) parte la manzana en rodajas, que divide a su vez en otras más delgadas; y parece como si estuviera dando de comer en la boca a un pajarillo que no ha dejado todavía el nido, por el gran cuidado con que mete los trozos de manzana en esa boquita que muele incesantemente. 

- ¡Te has fijado qué ojos, Joaquín! ¿No parecen dos porcioncitas del Mar de Galilea cuando el viento de la tarde empuja un velo de nubes bajo el cielo? 

Ana ha hablado teniendo apoyada una mano en el hombro de su marido y apoyándose a su vez ligeramente en ella: gesto éste que revela un profundo amor de esposa, un amor intacto tras muchos años de vínculo conyugal. 

Joaquín la mira con amor, y asiente diciendo: 

- ¡Bellísimos! ¿Y esos ricitos? ¿No tienen el color de la mies secada por el sol? Mira, en su interior hay mezcla de oro y cobre. 

- ¡Ah, si hubiéramos tenido un hijo, lo habría querido así, con estos ojos y este pelo!... - Ana se ha curvado, es más, se ha arrodillado, y, con un fuerte suspiro, besa esos dos ojazos azul - grises. 

También suspira Joaquín, y, queriéndola consolar, le pone la mano sobre el pelo rizado y canoso, y le dice: 

Todavía hay que esperar. Dios todo lo puede. Mientras se vive, el milagro puede producirse, especialmente cuando se le ama y cuando nos amamos". Joaquín recalca mucho estas últimas palabras. 

Mas Ana guarda silencio, descorazonada, con la cabeza agachada, para que no se vean dos lágrimas que están deslizándose y que advierte sólo el pequeño Alfeo, el cual, asombrado y apenado de que su gran amiga llore como hace él alguna vez, levanta la manita y enjuga su llanto. 

- ¡No llores, Ana! Somos felices de todas formas. Yo por lo menos lo soy, porque te tengo a ti. 

Yo también por ti. Pero no te he dado un hijo... Pienso que he entristecido al Señor porque ha hecho infecundas mis entrañas... 

- ¡Oh, esposa mía! ¿En qué crees tú, santa, que has podido entristecerlo? Mira, vamos una vez más al Templo y por esto, no sólo por los Tabernáculos, hacemos una larga oración... Quizás te suceda como a Sara... o como a Ana de Elcana: esperaron mucho y se creían reprobadas por ser estériles, y, sin embargo, en el Cielo de Dios, estaba madurando para ellas un hijo santo. Sonríe, esposa mía. Tu llanto significa para mí más dolor que el no tener prole... Llevaremos a Alfeo con nosotros. Le diremos que rece. Él es inocente... Dios tomará juntas nuestra oración y la suya y se mostrará propicio. 

Sí. Hagamos un voto al Señor. Suyo será el hijo; si es que nos lo concede... ¡Oh, sentirme llamar "mamá"! Y Alfeo, espectador asombrado e inocente, dice: 

- ¡Yo te llamo "mamá"! 

Sí, tesoro amado... pero tú ya tienes mamá, y yo... yo no tengo niño.... La visión cesa aquí. 

\chapter*{En la fiesta de los Tabernáculos. \\ \normalfont\normalsize\textit{Joaquín y Ana poseían la Sabiduría.}}
\addcontentsline{toc}{chapter}{\normalfont\scshape{En la fiesta de los Tabernáculos.}}
 
Antes de proseguir hago una observación. 

La casa no me ha parecido la de Nazaret, bien conocida. Al menos la habitación es muy distinta. Con respecto al huerto - jardín, debo decir que es también más amplio; además, se ven los campos, no muchos, pero... los hay. Después, ya casada María, sólo está el huerto (amplio, eso sí, pero sólo huerto). Y esta habitación que he visto no la he observado nunca en las otras visiones. No sé si pensar que por motivos pecuniarios los padres de María se hubieran deshecho de parte de su patrimonio, o si María, dejado el Templo, pasó a otra casa, que quizás le había dado José. No recuerdo si en las pasadas visiones y lecciones recibí alguna vez alusión segura a que la casa de Nazaret fuera la casa natal. 

Mi cabeza está muy cansada. Además, sobre todo por lo que respecta a los dictados, olvido enseguida las palabras, aunque, eso sí, me quedan grabadas las prescripciones que contienen, y, en el alma, la luz. Pero los detalles se borran inmediatamente. Si al cabo de una hora tuviera que repetir lo que he oído, aparte de una o dos frases de especial importancia, no sabría nada más. Las visiones, por el contrario, me quedan vivas en la mente, porque las he tenido que observar por mi misma. Los dictados los recibo. Aquéllas, por el contrario, tengo que percibirlas; permanecen, por tanto, vivas en el pensamiento, que ha tenido que trabajar para advertir sus distintas fases. 

Esperaba un dictado sobre la visión de ayer, pero no lo ha habido. 

Empiezo a ver y escribo. 

Fuera de los muros de Jerusalén, en las colinas, entre los olivos, hay gran multitud de gente. Parece un enorme mercado, pero no hay ni casetas ni puestos de venta ni voces de charlatanes y vendedores ni juegos. Hay muchas tiendas hechas de lana basta, sin duda impermeables, extendidas sobre estacas hincadas en el suelo. Atados a las estacas hay ramos verdes, como decoración y como medio para dar frescor. Otras, sin embargo, están hechas sólo de ramos hincados en el suelo y atados así; éstas crean como pequeñas galerías verdes. Bajo todas ellas, gente de las más distintas edades y condiciones y un rumor de conversación tranquilo e íntimo en que sólo desentona algún chillido de niño. 

Cae la tarde y ya las luces de las lamparitas de aceite resplandecen acá y allá por el extraño campamento. En tomo a estas luces, algunas familias, sentadas en el suelo, están cenando; las madres tienen en su regazo a los más pequeños, muchos de los cuales, cansados, se han quedado dormidos teniendo todavía el trozo de pan en sus deditos rosados, cayendo su cabecita sobre el pecho materno, como los polluelos bajo las alas de la gallina. Las madres terminan de comer como pueden, con una sola mano libre, sujetando con la otra a su hijito contra su corazón. Otras familias, por el contrario, no están todavía cenando. Conversan en la semioscuridad del crepúsculo esperando a que la comida esté hecha. Se ven lumbres encendidas, desperdigadas; en torno a ellas trajinan las mujeres. Alguna nana muy lenta, yo diría casi quejumbrosa, mece a algún niño que halla dificultad para dormirse. 

Encima, un hermoso cielo sereno, azul cada vez más oscuro hasta semejar a un enorme toldo de terciopelo suave de un color negro - azul; un cielo en el que, muy lentamente, invisibles artífices y decoradores estuvieran fijando gemas y lamparitas, ya aisladas, ya formando caprichosas líneas geométricas, entre las que destacan la Osa Mayor y Menor, que tienen forma de carro con la lanza apoyada en el suelo una vez liberados del yugo los bueyes. La estrella Polar ríe con todos sus resplandores. 

Me doy cuenta de que es el mes de Octubre. 

Aparece en la escena Ana. Viene de una de las hogueras con algunas cosas en las manos y colocadas sobre el pan, que es ancho y plano, como una torta de las nuestras, y que hace de bandeja. Trae pegado a las faldas a Alfeo, que va parla que te parla con su vocecita aguda. Joaquín está a la entrada de su pequeña tienda (toda de ramajes). Habla con un hombre de unos treinta años, al que saluda Alfeo desde lejos con un gritito diciendo: "Papá". Cuando Joaquín ve venir a Ana se da prisa en encender la lámpara. 

Ana pasa con su majestuoso caminar regio entre las filas de tiendas; regio y humilde. No es altiva con ninguno. Levanta a un niñito, hijo de una pobre, muy pobre, mujer, el cual ha tropezado en su traviesa carrera y ha ido a caer justo a sus pies. Dado que el niñito se ha ensuciado de tierra la carita y está llorando, ella le limpia y le consuela y, habiendo acudido la madre disculpándose, se lo restituye diciendo: 

- ¡Oh, no es nada! Me alegro de que no se haya hecho daño. Es un niño muy majo. ¿Qué edad tiene?". 

Tres años. Es el penúltimo. Dentro de poco voy a tener otro. Tengo seis niños. Ahora querría una niña... Para una mamá es mucho una niña.... 

- ¡Grande ha sido el consuelo que has recibido del Altísimo, mujer! - Ana suspira. 

La otra mujer dice: Sí. Soy pobre, pero los hijos son nuestra alegría, y ya los más grandecitos ayudan a trabajar. Y tú, señora - todos los signos son de que Ana es de condición más elevada, y la mujer lo ha visto - ¿Cuántos niños tienes? - Ninguno. 

- ¿Ninguno! ¿No es tuyo éste? 

No. De una vecina muy buena. Es mi consuelo... 

 - ¡Oh! - La mujer pobre la mira con piedad. 

Ana la saluda con un gran suspiro y se dirige a su tienda. 

Te he hecho esperar, Joaquín. Me ha entretenido una mujer pobre, madre de seis hijos varones, ¡fíjate! Y dentro de poco va a tener otro hijo. 

Joaquín suspira. 

El padre de Alfeo llama a su hijo, pero éste responde: "Yo me quedo con Ana. Así la ayudo. 

Todos se echan a reír. 

Déjalo. No molesta. Todavía no le obliga la Ley. Aquí o allí... no es más que un pajarito que come- dice Ana, y se sienta con el niño en el regazo; le da un pedazo de torta y, creo, pescado asado. Veo que hace algo antes de dárselo. Quizás le ha quitado la espina. Antes ha servido a su marido. La última que come es ella. 

La noche está cada vez más poblada de estrellas y las luces son cada vez más numerosas en el campamento. Luego muchas luces se van poco a poco apagando: son los primeros que han cenado, que ahora se echan a dormir. Va disminuyendo también lentamente el rumor de la gente. No se oyen ya voces de niños. Sólo resuena la vocecita de algún lactante buscando la leche de su mamá. La noche exhala su brisa sobre las cosas y las personas, y borra penas y recuerdos, esperanzas y rencores. 

Bueno, quizás estos dos sobrevivan, aun cuando hayan quedado atenuados, durante el sueño, en los sueños. 

Ana está meciendo a Alfeo, que empieza a dormirse en sus brazos. Entonces cuenta a su marido el sueño que ha tenido: 

Esta noche he soñado que el próximo año voy a venir a la Ciudad Santa para dos fiestas en vez de para una sola. Una será el ofrecimiento de mi hijo al Templo... ¡Oh! ¡Joaquín!...

Espéralo, espéralo. Ana. ¿No has oído alguna palabra? ¿El Señor no te ha susurrado al corazón nada? 

Nada. Un sueño sólo... 

Mañana es el último día de oración. Ya se han efectuado todas las ofrendas. No obstante, las renovaremos solemnemente mañana. Persuadiremos a Dios con nuestro fiel amor. Yo sigo pensando que te sucederá como a Ana de Elcana. 

Dios lo quiera... ¡Si hubiera, ahora mismo, alguien que me dijera: "Vete en paz. El Dios de Israel te ha concedido la gracia que pides"!...

Si ha de venir la gracia, tu niño te lo dirá moviéndose por primera vez en tu seno. Será voz de inocente y, por tanto, voz de Dios. 

Ahora el campamento calla en la oscuridad de la noche. Ana lleva a Alfeo a la tienda contigua y lo pone sobre la yacija de heno junto a sus hermanitos, que ya están dormidos. Luego se echa al lado de Joaquín. Su lamparita también se apaga, una de las últimas estrellitas de la tierra. Quedan, más hermosas, las estrellas del firmamento, velando a todos los durmientes. 

Dice Jesús: 

Los justos son siempre sabios, porque, siendo como son amigos de Dios, viven en su compañía y reciben instrucción de Él, de Él que es Infinita Sabiduría. Mis abuelos eran justos; poseían, por tanto, la sabiduría. Podían decir con verdad cuanto dice la Escritura cantando las alabanzas de la Sabiduría en el libro que lleva su nombre: "Yo la he amado y buscado desde mi juventud y procuré tomarla por esposa". Ana de Aarón era la mujer fuerte de que habla el Antepasado nuestro. Y Joaquín, de la estirpe del rey David, no había buscado tanto belleza y riqueza cuanto virtud. Ana poseía una gran virtud. Toda las virtudes unidas como ramo fragante de flores para ser una única, bellísima cosa, que era la Virtud, una virtud real, digna de estar delante del trono de Dios. Joaquín, por tanto, había tomado por esposa dos veces a la sabiduría "amándola más que a cualquier otra mujer": la sabiduría de Dios contenida dentro del corazón de la mujer justa. Ana de Aarón no había tratado sino de unir su vida a la de un hombre recto, con la seguridad de que en la rectitud se halla la alegría de las familias. Y, para ser el emblema de la "mujer fuerte", no le faltaba sino la corona de los hijos, gloria de la mujer casada, justificación del vínculo matrimonial, de que habla Salomón; como también a su felicidad sólo le faltaban estos hijos, flores del árbol que se ha hecho uno con el árbol cercano obteniendo copiosidad de nuevos frutos en los que las dos bondades se funden en una, pues de su esposo nunca había recibido ningún motivo de infelicidad. Ella, ya tendente a la vejez, mujer de Joaquín desde hacía varios lustros, seguía siendo para éste "la esposa de su juventud, su alegría, la cierva amadísima, la gacela donosa", cuyas caricias tenían siempre el fresco encanto de la primera noche nupcial y cautivaban dulcemente su amor, manteniéndolo fresco como flor que el rocío refresca y ardiente como fuego que siempre una mano alimenta. Por tanto, dentro de su aflicción, propia de quien no tiene hijos, recíprocamente se decían "palabras de consuelo en las preocupaciones y fatigas". Y la Sabiduría eterna, llegada la hora, después de haberlos instruido en la vida, los iluminó con los sueños de la noche, lucero de la mañana del poema de gloria que había de llegar a ellos, María Santísima., la Madre mía. Si su humildad no pensó en esto, su corazón sí se estremeció esperanzado ante el primer tañido de la promesa de Dios. Ya de hecho hay certeza en las palabras de Joaquín: "Espéralo, espéralo... Persuadiremos a Dios con nuestro fiel amor". Soñaban un hijo, tuvieron a la Madre de Dios. Las palabras del libro de la Sabiduría parecen escritas para ellos: "Por ella adquiriré gloria ante el pueblo... por ella obtendré la inmortalidad y dejaré eterna memoria de mí a aquellos que vendrán después de mí". Pero, para obtener todo esto, tuvieron que hacerse reyes de una virtud veraz y duradera no lesionada por suceso alguno. Virtud de fe. Virtud de caridad. Virtud de esperanza. Virtud de castidad. ¡Oh, la castidad de los esposos! Ellos la vivieron, pues no hace falta ser vírgenes para ser castos. Los tálamos castos tienen por custodios a los ángeles, y de tales tálamos provienen hijos buenos que de la virtud de sus padres hacen norma para su vida. Mas ahora ¿dónde están? Ahora no se desean hijos, pero no se desea tampoco la castidad. Por lo cual Yo digo que se profana el amor y se profana el tálamo. 

\chapter*{Ana, con una canción, anuncia que es madre. \\ \normalfont\normalsize\textit{En su seno está el alma inmaculada de María.}}
\addcontentsline{toc}{chapter}{\normalfont\scshape{Ana, con una canción, anuncia que es madre.}}
 
Veo de nuevo la casa de Joaquín y Ana. Nada ha cambiado en su interior, si se exceptúan las muchas ramas florecidas, colocadas aquí y allá en jarrones (sin duda provienen de la podadura de los árboles del huerto, que están todos en flor: una nube que varía del blanco nieve al rojo típico de ciertos corales). 

También es distinto el trabajo que está realizando Ana. En un telar más pequeño, teje lindas telas de lino, y canta ritmando el movimiento del pie con la voz. Canta y sonríe... ¿A quién? A sí misma, a algo que ve en su interior. 

El canto, lento pero alegre, que he escrito aparte para seguirla, porque le repite una y otra vez, como gozándose en él, y cada vez con más fuerza y seguridad, como la persona que ha descubierto un ritmo en su corazón y primero lo susurra calladamente, y luego, segura, va más expedita y alta de tono, dice (y lo transcribo porque, dentro de su sencillez, es muy dulce):- ¡Gloria al Señor omnipotente que ha amado a los hijos de David! 

\begin{verse}
¡Gloria al Señor! \\
Su suprema gracia desde el Cielo me ha visitado. \\
El árbol viejo ha echado nueva rama y yo soy bienaventurada. \\
Por la Fiesta de las Luces echó semilla la esperanza; \\
ahora de Nisán la fragancia la ve germinar. \\
Como el almendro, se cubre de flores mi carne en primavera. \\
Su fruto, cercano ya el ocaso, ella siente llevar. \\
En la rama hay una rosa, hay uno de los más dulces pomos. \\
Una estrella reluciente, un párvulo inocente. \\
La alegría de la casa, del esposo y de la esposa. \\
Loor a Dios, a mi Señor, que piedad tuvo de mí. \\
Me lo dijo su luz: "Una estrella te llegará". \\
¡Gloria, gloria! Tuyo será este fruto del árbol, \\
primero y extremo, santo y puro como don del Señor. \\
Tuyo será. ¡Que por él venga alegría y paz a la tierra! \\
¡Vuela, lanzadera! Aprieta el hilo para la tela del recién nacido\\. 
¡Él nace! Laudatorio a Dios vaya el canto de mi corazón". 
\end{verse}
Entra Joaquín en el momento en que ella iba a repetir por cuarta vez su canto. 

- ¿Estás contenta, Ana? Pareces un ave en primavera. ¿Qué canción es ésta? A nadie se la he oído nunca. ¿De dónde nos viene? 

- De mi corazón, Joaquín. 

 Ana se ha levantado y ahora se dirige hacia su esposo, toda sonriente. Parece más joven y más guapa. 

- No sabía que fueras poetisa - dice su marido mirándola con visible admiración. No parecen dos esposos ya mayores. En su mirada hay una ternura de jóvenes cónyuges. 

- He venido desde la otra parte del huerto oyéndote cantar. Hacía años que no oía tu voz de tórtola enamorada. ¿Quieres repetirme esa canción? 

- Te la repetiría aunque no lo pidieras. Los hijos de Israel han encomendado siempre al canto los gritos más auténticos de sus esperanzas, alegrías y dolores. Yo he encomendado al canto la solicitud de anunciarme y de anunciarte una gran alegría. Sí, también a mí, porque es cosa tan grande que, a pesar de que yo ya esté segura de ella, me parece aún no verdadera... 

Y empieza a entonar de nuevo la canción. Pero cuando llega al punto: "En la rama hay una rosa, hay uno de los más dulces pomos, una estrella"..., su bien entonada voz de contralto primero se oye trémula y luego se rompe; se echa a llorar de alegría, mira a Joaquín y, levantando los brazos, grita: 

- ¡Soy madre, amado mío! - y se refugia en su corazón, entre los brazos que él ha tendido para volver a cerrarlos en torno a ella, su esposa dichosa. Es el más casto y feliz abrazo que he visto desde que estoy en este mundo. Casto y ardiente, dentro de su castidad. 

Y la delicada reprensión entre los cabellos blanco - negros de Ana: 

- ¿Y no me lo decías? 

- Porque quería estar segura. Siendo vieja como soy... verme madre... No podía creer que fuera verdad... y no quería darte la más amarga de las desilusiones. Desde finales de diciembre siento renovarse mis entrañas profundas y echar, como digo, una nueva rama. Mas ahora en esa rama el fruto es seguro... ¿Ves? Esa tela ya es para el que ha de venir. 

- ¿No es el lino que compraste en Jerusalén? 

Sí. Lo he hilado durante la espera... y con esperanza. Tenía esperanza por lo que sucedió el último día mientras oraba en el Templo, lo más que puede una mujer en la Casa de Dios, ya de noche. ¿Te acuerdas que decía: "Un poco más, todavía un poco más?" ¡No sabía separarme de allí sin haber recibido gracia! Pues bien, descendiendo ya las sombras, desde el interior del lugar sagrado al que yo miraba con arrobo para arrancarle al Dios presente su asentimiento, vi surgir una luz. Era una chispa de luz bellísima. Cándida como la luna pero que tenía en sí todas las luces de todas las perlas y gemas que hay en la tierra. Parecía como si una de las estrellas preciosas del Velo, las que están colocadas bajo los pies de los querubines, se separase y adquiriese esplendor de luz sobrenatural...Parecía como si desde el otro lado del Velo sagrado, desde la Gloria misma, hubiera salido un fuego y viniera veloz hacia mí, y que al cortar el aire cantara con voz celeste diciendo: "Recibe lo que has pedido". Por eso canto: "Una estrella te llegará". ¿Y qué hijo será éste, nuestro, que se manifiesta como luz de estrella en el Templo y que dice "existo" en la Fiesta de las Luces? ¿Será que has acertado al pensar en mí como una nueva Ana de Elcana? ¿Cómo la llamaremos a esta criatura nuestra que, dulce como canción de aguas, siento queme habla en el seno con su corazoncito, latiendo, latiendo, como el de una tortolita entre los huecos de las manos?". 

Si es varón, le llamaremos Samuel; si es niña, Estrella, la palabra que ha detenido tu canto para darme esta alegría de saber que soy padre, la forma que ha tomado para manifestarse entre las sagradas sombras del Templo. Estrella. Nuestra Estrella, porque... no lo sé, pero creo que es una niña. Pienso que unas caricias tan delicadas no pueden provenir sino de una dulcísima hija. Porque no la llevo yo, no me produce dolor; es ella la que me lleva por un sendero azul y florido, como si ángeles santos me sostuvieran y la tierra estuviera ya lejana... Siempre he oído decir a las mujeres que el concebir y el llevar al hijo en el seno supone dolor, pero yo no lo siento. Me siento fuerte, joven, fresca; más que cuando te entregué mi virginidad en la lejana juventud. Hija de Dios, porque es más de Dios que nuestra, siendo así que nacerá de un tronco aridecido, que no da dolor a su madre; sólo le trae paz y bendición: los frutos de Dios, su verdadero Padre. 

Entonces la llamaremos María. Estrella de nuestro mar, perla, felicidad, el nombre de la primera gran mujer de Israel. Pero no pecará nunca contra el Señor, que será el único al que dará su canto, porque ha sido ofrecida a Él como hostia antes de nacer. 

Está ofrecida a Él, sí. Sea niño o niña nuestra criatura, se la daremos al Señor, después de tres años de júbilo con ella. Nosotros seremos también hostias, con ella, para la gloria de Dios. 

No veo ni oigo nada más. 

Dice Jesús: 
\emph{La Sabiduría, tras haberlos iluminado con los sueños de la noche, descendió; Ella, que es "emanación de la potencia de Dios, genuino efluvio de la gloria del Omnipotente", y se hizo Palabra para la estéril. Quien ya veía cercano su tiempo de redimir, Yo, el Cristo, nieto de Ana, casi cincuenta años después, mediante la Palabra, obraría milagros en las estériles y en las enfermas, en las obsesas, en las desoladas; los obraría en todas las miserias de la tierra. Pero, entretanto, por la alegría de tener una Madre, he aquí que susurro una arcana palabra en las sombras del Templo que contenía las esperanzas de Israel, del Templo que ya estaba en la frontera de su vida. En efecto, un nuevo y verdadero Templo, no ya portador de esperanzas para un pueblo, sino certeza de Paraíso para el pueblo de toda la tierra, y por los siglos de los siglos hasta el fin del mundo, estaba para descender sobre la tierra. Esta Palabra obra el milagro de hacer fecundo lo que era infecundo, y de darme una Madre, la cual no tuvo sólo óptimo natural, como era de esperarse naciendo de dos santos, y no tuvo sólo un alma buena, como muchos también la tienen, y continuo crecimiento de esta bondad por su buena voluntad, ni sólo un cuerpo inmaculado... Tuvo, caso único entre las criaturas, inmaculado el espíritu. Tú has visto la generación continua de las almas por Dios. Piensa ahora cuál debió ser la belleza de esta alma que el Padre había soñado antes de que el tiempo fuera, de esta alma que constituía las delicias de la Trinidad, Trinidad que ardientemente deseaba adornarla con sus dones para donársela a sí misma. ¡Oh, Todo Santa que Dios creó para sí, y luego para salud de los hombres! Portadora del Salvador, tú fuiste la primera salvación; vivo Paraíso, con tu sonrisa comenzaste a santificar la tierra. ¡Oh, el alma creada para ser alma de la Madre de Dios!.. Cuando, de un más vivo latido del trino Amor, surgió esta chispa vital, se regocijaron los ángeles, pues luz más viva nunca había visto el Paraíso. Como pétalo de empírea rosa, pétalo inmaterial y preciado, gema y llama, aliento de Dios que descendía a animar a una carne de forma muy distinta que a las otras, con un fuego tan vivo que la Culpa no pudo contaminarla, traspasó los espacios y se cerró en un seno santo. La tierra tenía su Flor y aún no lo sabía. La verdadera, única Flor que florece eterna: azucena y rosa, violeta y jazmín, helianto y ciclamino sintetizados, y con ellas todas las flores de la tierra fusionadas en una Flor sola, María, en la cual toda virtud y gracia se unen. En Abril, la tierra de Palestina parecía un enorme jardín. Fragancias y colores deleitaban el corazón de los hombres. Sin embargo, aún ignorábase la más bella Rosa. Ya florecía para Dios en el secreto del claustro materno, porque mi Madre amó desde que fue concebida, mas sólo cuando la vid da su sangre para hacer vino, y el olor de los mostos, dulce y penetrante, llena las eras y el olfato, Ella sonreiría, primero a Dios y luego al mundo, diciendo con su superinocente sonrisa: "Mirad: la Vid que os va a dar el Racimo para ser prensado y ser Medicina eterna para vuestro mal está entre vosotros". He dicho que María amó desde que fue concebida. ¿Qué es lo que da al espíritu luz y conocimiento? La Gracia. ¿Qué es lo que quita la Gracia? El pecado original y el pecado mortal. María, la Sin Mancha, nunca se vio privada del recuerdo de Dios, de su cercanía, de su amor, de su luz, de su sabiduría. Ella pudo por ello comprender y amar cuando no era más que una carne que se condensaba en torno a un alma inmaculada que continuaba amando. Más adelante te daré a contemplar mentalmente la profundidad de las virginidades en María. Te producirá un vértigo celeste semejante a cuando te di a considerar nuestra eternidad. Entre tanto; piensa cómo el hecho de llevar en las entrañas a una criatura exenta de la Mancha que priva de Dios le da a la madre, que, no obstante, la concibió en modo natural, humano, una inteligencia superior, y la hace profeta, la profetisa de su hija, a la que llama "Hija de Dios". Y piensa lo que habría sido si de los Primeros Padres inocentes hubieran nacido hijos inocentes, como Dios quería. Éste, ¡oh, hombres que decís que vais hacia el "superhombre", y que de hecho con vuestros vicios estáis yendo únicamente hacia el super- demonio!, éste habría sido el medio que conduciría al "superhombre": saber estar libres de toda contaminación de Satanás, para dejarle a Dios la administración de la vida, del conocimiento, del bien; no deseando más de cuanto Dios os hubiera dado, que era poco menos que infinito, para poder engendrar, en una continua evolución hacia lo perfecto, hijos que fueran hombres en el cuerpo y, en el espíritu, hijos de la Inteligencia, es decir, triunfadores, es decir, fuertes, es decir, gigantes contra Satanás, que habría mordido el polvo muchos miles de siglos antes de la hora en que lo haga, y con él todo su mal. }
 
\chapter*{Nacimiento de la Virgen María. \\ \normalfont\normalsize\textit{Su virginidad en el eterno pensamiento del Padre.}}
\addcontentsline{toc}{chapter}{\normalfont\scshape{Nacimiento de la Virgen María.}}

Veo a Ana saliendo al huerto - jardín. Va apoyándose en el brazo de una pariente (se ve porque se parecen). Está muy gruesa y parece cansada, quizás también porque hace bochorno, un bochorno muy parecido al que a mí me hace sentirme abatida. 

A pesar de que el huerto sea umbroso, el ambiente es abrasador y agobiante. Bajo un despiadado cielo, de un azul ligeramente enturbiado por el polvo suspendido en el espacio, el aire es tan denso, que podría cortarse como una masa blanda y caliente. Debe persistir ya mucho la sequía, pues la tierra, en los lugares en que no está regada, ha quedado literalmente reducida a un polvo finísimo y casi blanco. Un blanco ligeramente tendente a un rosa sucio. Sin embargo, por estar humedecida, es marrón oscura al pie de los árboles, como también a lo largo de los cortos cuadros donde crecen hileras de hortalizas, .y en torno a los rosales, a los jazmines o a otras flores de mayor o menor tamaño (que están especialmente a lo largo de todo el frente de una hermosa pérgola que divide en dos al huerto hasta donde empiezan las tierras, ya despojadas de sus mieses). La hierba del prado, que señala el final de la propiedad, está requemada; se ve rala. Sólo permanece la hierba más verde y tupida en los márgenes del prado, donde hay un seto de espino blanco silvestre, ya todo adornado de los rubíes de los pequeños frutos; en ese lugar, en busca de pastos y de sombra, hay unas ovejas con su zagalillo. 

Joaquín, con otros dos hombres como ayuda, está dedicado a las hortalizas y a los olivos. A pesar de ser anciano, es rápido y trabaja con gusto. Están abriendo unas pequeñas protecciones de las lindes de una parcela para proporcionar agua a las sedientas plantas. Y el agua se abre camino borboteando entre la hierba y la tierra quemada, y se extiende en anillos que, en un primer momento, parecen como de cristal amarillento para luego ser anillos oscuros de tierra húmeda en torno a los sarmientos y a los olivos colmados de frutos. 

Lentamente, Ana, por la umbría pérgola, bajo la cual abejas de oro zumban ávidas del azúcar de los dorados granos de las uvas, se dirige hacia Joaquín, el cual, cuando la ve, se apresura a ir a su encuentro. 

- ¿Has llegado hasta aquí? 

La casa está caliente como un horno". 

Y te hace sufrir". 

Es mi único sufrimiento en este último período mío de embarazo. Es el sufrimiento de todos, de hombres y de animales. No te sofoques demasiado, Joaquín. 

El agua que hace tanto que esperamos, y que hace tres días que parece realmente cercana, no ha llegado todavía. Las tierras arden. Menos mal que nosotros tenemos el manantial cercano, y muy rico en agua. He abierto los canales. Poco alivio para estas plantas cuyas hojas ya languidecen cubiertas de polvo. No obstante, supone ese mínimo que las mantiene en vida. ¡Si lloviera!... 

Joaquín, con el ansia de todos los agricultores, escudriña el cielo, mientras Ana, cansada, se da aire con un abanico (parece hecho con una hoja seca de palma traspasada por hilos multicolores que la mantienen rígida). 

La pariente dice: Allí, al otro lado del Gran Hermón, están formándose nubes que avanzan velozmente. Viento del norte. Bajará la temperatura y dará agua. 

Hace tres días que se levanta y luego cesa cuando sale la Luna. Sucederá lo mismo esta vez - Joaquín está desalentado. 

Vamos a casa. Aquí tampoco se respira; además, creo que conviene volver - dice Ana, que ahora se le ha puesto de improviso pálida la cara. 

- ¿Sientes dolor? 

No. Siento la misma gran paz que experimenté en el Templo cuando se me otorgó la gracia, y que luego volví a sentir otra vez al saber que era madre. Es como un éxtasis. Es un dulce dormir del cuerpo, mientras el espíritu exulta y se aplaca con una paz sin parangón humano. Yo te he amado, Joaquín, y, cuando entré en tu casa y me dije: "Soy esposa de un justo", sentí paz, como todas las otras veces que tu próvido amor se prodigaba en mí. Pero esta paz es distinta. Creo que es una paz como la que debió invadir, como una deleitosa unción de aceite, el espíritu de Jacob, nuestro padre, después de su sueño de ángeles. O semejante, más bien, a la gozosa paz de los Tobías tras habérseles manifestado Rafael. Si me sumerjo en ella, al saborearla, crece cada vez más. Es como si yo ascendiera por los espacios azules del cielo... y, no sé por qué, pero, desde que tengo en mí esta alegría pacífica, hay un cántico en mi corazón: el del anciano Tobit. Me parece como si hubiera sido compuesto para esta hora... para esta alegría... para la tierra de Israel que es su destinataria... para Jerusalén, pecadora, mas ahora perdonada... bueno... no os riáis de los delirios de una madre... pero, cuando digo: "Da gracias al Señor por tus bienes y bendice al Dios de los siglos para que vuelva a edificar en ti su Tabernáculo", yo pienso que aquel que reedificará en Jerusalén el Tabernáculo del Dios verdadero, será este que está para nacer... y pienso también que, cuando el cántico dice: "Brillarás con una luz espléndida, todos los pueblos de la tierra se postrarán ante ti, las naciones irán a ti llevando dones, adorarán en ti al Señor y considerarán santa tu tierra, porque dentro de ti invocarán el Gran Nombre. Serás feliz en tus hijos porque todos serán bendecidos y se reunirán ante el Señor. ¡Bienaventurados aquellos que te aman y se alegran de tu paz!...", cuando dice esto, pienso que es profecía no ya de la Ciudad Santa, sino del destino de mi criatura, y la primera que se alegra de su paz soy yo, su madre feliz... 

El rostro de Ana, al decir estas palabras, palidece y se enciende, como una cosa que pasase de luz lunar a vivo fuego, y viceversa. Dulces lágrimas le descienden por las mejillas, y no se da cuenta, y sonríe a causa de su alegría. Y va yendo hacia casa entre su esposo y su pariente, que escuchan conmovidos en silencio. 

Se apresuran, porque las nubes, impulsadas por un viento alto, galopan y aumentan en el cielo mientras la llanura se oscurece y tirita por efectos de la tormenta que se está acercando. Llegando al fibra! de la puerta, un primer relámpago lívido surca el cielo. El ruido del primer trueno se asemeja al redoble de un enorme bombo ritmado con el arpegio de las primeras gotas sobre las abrasadas hojas. 

Entran todos. Ana se retira. Joaquín se queda en la puerta con unos peones que le han alcanzado, hablando de esta agua tan esperada, bendición para la sedienta tierra. Pero la alegría se transforma en temor, porque viene una tormenta violentísima con rayos y nubes cargadas de granizo. 

Si rompe la nube, la uva y las aceitunas quedarán trituradas como por rueda de molino. ¡Pobres de nosotros!". 

Joaquín tiene además otro motivo de angustia: su esposa, a la que le ha llegado la hora de dar a luz al hijo. La pariente le dice que Ana no sufre en absoluto. Él está, de todas formas, muy inquieto, y, cada vez que la pariente u otras mujeres (entre las cuales está la madre de Alfeo) salen de la habitación de Ana para luego volver con agua caliente, barreños y paños secados a la lumbre, que, jovial, brilla en el hogar central en una espaciosa cocina, él va y pregunta, y no le calman las explicaciones tranquilizadoras de las mujeres. También le preocupa la ausencia de gritos por parte de Ana. Dice: 

Yo soy hombre. Nunca he visto dar a luz. Pero recuerdo haber oído decir que la ausencia de dolores es fatal...

Declina el día antes de tiempo por la furia de la tormenta, que es violentísima. Agua torrencial, viento, rayos... de todo, menos el granizo, que ha ido a caer a otro lugar. 

Uno de los peones, sintiendo esta violencia, dice: 

Parece como si Satanás hubiera salido de la Gehena con sus demonios. ¡Mira qué nubes tan negras! ¡Mira qué exhalación de azufre hay en el ambiente, y silbidos y voces de lamento y maldición! Si es él, ¡está enfurecido esta noche! 

El otro peón se echa a reír y dice: 

Se le habrá escapado una importante presa, o quizás Miguel de nuevo le habrá lanzado el rayo de Dios, y tendrá cuernos y cola cortados y quemados. 

Pasa corriendo una mujer y grita: 

- ¡Joaquín! ¡Va a nacer de un momento a otro! ¡Todo ha ido rápido y bien! 

Y desaparece con una pequeña ánfora en las manos. 

Se produce un último rayo; tan violento, que lanza contra las paredes a los tres hombres. En la parte delantera de la casa, en el suelo del huerto, queda como recuerdo un agujero negro y humeante. Luego, de repente, cesa la tormenta. De detrás de la puerta de Ana viene un vagido (parece el lamento de una tortolita en su primer arrullo). Mientras, un enorme arco iris extiende su faja semicircular por toda la amplitud del cielo. Surge, o por lo menos lo parece, de la cima del Hermón (la cual, besada por un filo de sol, parece de alabastro de un blanco - rosa delicadísimo), se eleva hasta el más terso cielo septembrino y, salvando espacios limpios de toda impureza, deja debajo las colinas de Galilea y un terreno llano que aparece entre dos higueras, que está al Sur, y luego otro monte, y parece posar su punta extrema en el extremo horizonte, donde una abrupta cadena de montañas detiene la vista. 

- ¡Qué cosa más insólita! 

- ¡Mirad, mirad! 

- Parece como si reuniera en un círculo a toda la tierra de Israel, y... ya... ¡fijaos!, ya hay una estrella y el Sol no se ha puesto todavía. ¡Qué estrella! ¡Reluce como un enorme diamante!.. 

- ¡Y la Luna, allí, ya llena y aún faltaban tres días para que lo fuera! ¡Fijaos cómo resplandece! 

Las mujeres irrumpen, alborozadas, con un "ovillejo" rosado entre cándidos paños. 

¡Es María, la Mamá! Una María pequeñita, que podría dormir en el círculo de los brazos de un niño; una María que al máximo tiene la longitud de un brazo, una cabecita de marfil teñido de rosa tenue, y unos labiecillos de carmín que ya no lloran sino que instintivamente quieren mamar (tan pequeñitos, que no se ve cómo van a poder coger un pezón), y una naricita diminuta entre dos carrillitos redondetes. Si la estimulan abre los ojitos: dos pedacitos de cielo, dos puntitos inocentes y azules que miran, y no ven, entre sutiles pestañas de un rubio tan tenue que es casi rosa. También el vello de su cabeza redondita tiene una veladura entre rosada y rubia como ciertas mieles casi blancas. 

Tiene por orejas dos conchitas rosadas y transparentes, perfectas; y por manitas... ¿qué son esas dos cositas que gesticulan y buscan la boca? Cerradas, como están, son dos capullos de rosa de musgo que hubieran hendido el verde de los sépalos y asomaran su seda rosa tenue; abiertas, como están ahora, dos joyeles de marfil apenas rosa, de alabastro apenas rosa, con cinco pálidos granates por uñitas. ¿Cómo podrán ser capaces de secar tanto llanto esas manitas? ¿Y los piececitos? ¿Dónde están? Por ahora son sólo pataditas escondidas entre los lienzos. Pero, he aquí que la pariente se sienta y la destapa... ¡Oh, los piececitos! De la largura aproximada de cuatro centímetros, tienen por planta una concha coralina; por dorso, una concha de nieve veteada de azul; sus deditos son obras maestras de escultura liliputiense, coronados también por pequeñas esquirlas de granate pálido. Me pregunto cómo podrán encontrarse sandalias tan pequeñas que valgan para esos piececitos de muñeca cuando den sus primeros pasos, y cómo podrán esos piececitos recorrer tan áspero camino y soportar tanto dolor bajo una cruz. Pero esto ahora no se sabe. Se ríe o se sonríe de cómo menea los brazos y las piernas, de sus lindas piernecitas bien perfiladas, de los diminutos muslos, que, de tan gorditos como son, forman hoyuelos y aritos, de su barriguita (un cuenco invertido), de su pequeño tórax, perfecto, bajo cuya seda cándida se ve el movimiento de la respiración y se oye ciertamente, si, como hace el padre feliz ahora, en él se apoya la boca para dar un beso, latir un corazoncito... Un corazoncito que es el más bello que ha tenido, tiene y tendrá la tierra, el único corazón inmaculado de hombre. ¿Y la espalda? Ahora la giran y se ve el surco lumbar y luego los hombros, llenitos, y la nuca rosada, tan fuerte, que la cabecita se yergue sobre el arco de las vértebras diminutas, como la de un ave escrutadora en torno a sí del nuevo mundo que ve, y emite un gritito de protesta por ser mostrada en ese modo; Ella, la Pura y Casta, ante los ojos de tantos, Ella, que jamás volverá a ser vista desnuda por hombre alguno, la Toda Virgen, la Santa e Inmaculada. Tapad, tapad a este Capullo de azucena que nunca se abrirá en la tierra, y que dará, más hermosa aún que Ella, su Flor, sin dejar de ser capullo. Sólo en el Cielo la Azucena del Trino Señor abrirá todos sus pétalos. Porque allí arriba no existe vestigio de culpa que pudiera involuntariamente profanar ese candor. Porque allí arriba se trata de acoger, a la vista de todo el Empíreo, al Trino Dios - Padre, Hijo, Esposo - que ahora, dentro de pocos años, celado en un corazón sin mancha, vendrá a Ella. 

De nuevo está envuelta en los lienzos y en los brazos de su padre terreno, al que asemeja. No ahora, que es un bosquejo de ser humano. Digo que le asemeja una vez hecha mujer. De la madre no refleja nada; del padre, el color de la piel y de los ojos, y, sin duda, también del pelo, que, si ahora son blancos, de joven eran ciertamente rubios a juzgar por las cejas. Del padre son las facciones — más perfectas y delicadas en Ella por ser mujer, ¡y qué Mujer!; también del padre es la sonrisa y la mirada y el modo de moverse y la estatura. Pensando en Jesús como lo veo, considero que ha sido Ana la que ha dado su estatura a su Nieto, así como el color marfil más cargado de la piel; mientras que María no tiene esa presencia de Ana (que es como una palma alta y flexible), sino la finura del padre. También las mujeres, mientras entran con Joaquín donde se encuentra la madre feliz para devolverle a su hijita, hablan de la tormenta y del prodigio de la Luna, de la estrella, del enorme arco iris. 

Ana sonríe ante un pensamiento propio: 

Es la estrella – dice Su signo está en el cielo. ¡María, arco de paz! ¡María, estrella mía! ¡María, Luna pura! ¡María, perla nuestra! 

¿María la llamas? 

Sí. María, estrella y perla y luz y paz... 

Pero también quiere decir amargura... ¿No temes acarrearle alguna desventura? 

Dios está con Ella. Es suya desde antes de que existiera. El la conducirá por sus vías y toda amargura se transformará en paradisíaca miel. Ahora sé de tu mamá... todavía un poco, antes de ser toda de Dios.... 

Y la visión termina en el primer sueño de Ana madre y de María recién nacida. 

Dice Jesús: 
\emph{Levántate y apresúrate, pequeña amiga. Siento ardiente deseo de llevarte conmigo al azul paradisíaco de la contemplación de la Virginidad de María. Saldrás de él con el alma fresca como si tú también hubieras sido recientemente creada por el Padre, una pequeña Eva antes de conocer carne; saldrás con el espíritu lleno de luz, pues te habrás abismado en la contemplación de la obra maestra de Dios; con todo tu ser repleto de amor, pues habrás comprendido cómo sabe amar Dios. Hablar de la concepción de María, la Sin Mancha, significa sumergirse en lo azul, en la luz, en el amor. Ven y lee sus glorias en el Libro del Antepasado: "Dios me poseyó al inicio de sus obras, desde el principio, antes de la creación. Ab aeterno fui erigida, al principio, antes de que la tierra fuera hecha; aún no existían los abismos, y yo ya había sido concebida. Aún no manaba agua de los manantiales, aún no se elevaban con su pesada mole los montes, aún las colinas no eran para el Sol collares... y yo ya había nacido. Dios no había hecho todavía la tierra ni los ríos ni las columnas del mundo, y yo ya existía. Cuando preparaba los cielos, yo estaba presente, cuando con ley inmutable clausuró el abismo bajo la bóveda, cuando fijó arriba la bóveda celeste y colgó de ella las fuentes de las aguas, cuando al mar le establecía sus confines y daba leyes a las aguas, cuando daba leyes a las aguas de no sobrepasar su límite, cuando echaba los fundamentos de la tierra, yo estaba con Él ordenando todas las cosas. Siempre alegre jugueteaba ante Él continuamente, jugueteaba en el universo...". Las habéis aplicado a la Sabiduría, pero hablan de Ella: la hermosa Madre, la santa Madre, la Virgen Madre de la Sabiduría, que soy Yo, el que te habla. He querido que escribieras, como encabezamiento del libro que habla de Ella, el primer verso de este himno, para que fuera confesado y conocido el consuelo y la alegría de Dios; la razón de la constante, perfecta, íntima alegría de este Dios Uno y Trino que os sostiene y ama y que del hombre recibió tantos motivos de tristeza; la razón de que perpetuara la raza aun cuando ésta, con la primera prueba, había merecido la destrucción; la razón del perdón que habéis recibido. Que María le amara... ¡Oh, bien merecía la pena crear al hombre y dejarlo vivir y decretar perdonarlo, para tener a la Virgen bella, a la Virgen santa, a la Virgen inmaculada, a la Virgen enamorada, a la Hija dilecta, a la Madre purísima, a la Esposa amorosa! Mucho os ha dado, y más aún os habría dado, Dios, con tal de poseer a la Criatura de sus delicias, al Sol de su sol y Flor de su jardín. Y mucho os sigue dando por Ella, a petición de Ella, para alegría de Ella, porque su alegría se vierte en la alegría de Dios y la aumenta con destellos que llenan de resplandores la luz, la gran luz del Paraíso, y cada resplandor es una gracia para el universo, para la raza del hombre, para los mismos bienaventurados, que responden con un esplendoroso grito de aleluya a cada milagro que sale de Dios, creado por el deseo del Dios Trino de ver la esplendorosa sonrisa de alegría de la Virgen. Dios quiso poner un rey en ese universo que había creado de la nada. Un rey que, por naturaleza material, fuera el primero entre todas las criaturas creadas con materia y dotadas de materia. Un rey que, por naturaleza espiritual, fuera poco menos que divino, fundido con la Gracia, como en su inocente primer día. Pero la Mente suprema, que conoce la totalidad de los hechos más lejanos en el tiempo, la Mente cuya vista ve incesantemente todo cuanto era, es y será, y que, mientras contempla el pasado y observa el presente, hunde su mirada en el extremo futuro, no ignorando cómo será el morir del último hombre, sin confusión ni discontinuidad, esa Mente no ignoró nunca que ese rey, creado para ser semidivino a su lado en el Cielo, heredero del Padre, cuando llegara como adulto a su Reino después de haber vivido en la casa de su madre — la tierra con la que fue hecho —, durante su niñez de párvulo del Eterno en su jornada sobre la tierra, cometería hacia sí mismo el delito de matarse en la Gracia y el latrocinio de despojarse del cielo. ¿Por qué lo creó entonces? Sin duda muchos se hacen esta pregunta. ¿Habríais preferido no existir? ¿No merece ser vivida esta jornada incluso por sí misma, a pesar de ser tan pobre y desnuda, y tan severa a causa de vuestra maldad, para conocer y admirar la Belleza infinita que la mano de Dios ha sembrado en el universo? ¿Para quién, si no, habría hecho estos astros y planetas que pasan como saetas, como flechas, rayando la bóveda del firmamento, o van — y parecen lentos —, van majestuosos con su paso veloz de bólidos, regalándoos luces y estaciones, y dándoos, eternos, inmutables aunque siempre mutables, a leer en el cielo una nueva página, cada noche, cada mes, cada año, como queriendo deciros: "Olvidaos de la cárcel, abandonad esa imagen vuestra llena de cosas oscuras, podridas, sucias, venenosas, mentirosas, blasfemas, corruptoras, y elevaos, al menos con la mirada, a la ilimitada libertad de los firmamentos; haceos un alma azul mirando tanta limpidez de cielo, haceos con una reserva de luz que podáis llevar a vuestra oscura cárcel; leed la palabra que escribimos cantando en coro nuestra melodía sideral, más armoniosa que si proviniera de un órgano de catedral, la palabra que escribimos resplandeciendo, la palabra que escribimos amando, porque siempre tenemos presente a Aquel que nos dio la alegría de existir, y le amamos por habernos dado este existir, este resplandecer, este movemos, este ser libres y bellos en medio de este cielo delicado allende el cual vemos un cielo aún más sublime, el Paraíso; a Aquel cuyo precepto de amor en su segunda parte cumplimos al amaros a vosotros, prójimo universal nuestro, al amaros proporcionándoos guía y luz, calor y belleza. Leed la palabra que decimos, la palabra a la que ajustamos nuestro canto, nuestro resplandecer, nuestro reír: Dios"? ¿Para quién habría hecho ese líquido azul: para el cielo, espejo; para la tierra, camino; sonrisa de aguas; voz de olas; palabra, también, que, con frufrú de roce de seda, con risitas de muchachas serenas, con suspiros de ancianos que recuerdan y lloran, con bofetadas de violentos, y con envites y bramidos y estruendos, siempre habla y dice: "Dios"? El mar es para vosotros, como lo son el cielo y los astros. Y con el mar los lagos y los ríos, los estanques y los arroyos, y los manantiales puros, que sirven, todos, para transportaros, para nutriros, para apagar vuestra sed y limpiaros, y que os sirven, sirviendo al Creador, sin salir a sumergiros, como merecéis. ¿Para quién habría hecho las innumerables familias de los animales, que son flores que vuelan cantando, que son siervos que trabajan, que corren, que os alimentan, que os recrean a vosotros, los reyes? ¿Para quién habría hecho las innumerables familias de las plantas y de las flores, que parecen mariposas, que parecen gemas e inmóviles avecillas; de los frutos, que parecen collares de oro y piedras preciosas o cofres de gemas? Son alfombra para vuestros pies, protección para vuestras cabezas, recreo, beneficio, alegría para la mente, para los miembros del cuerpo, para la vista y el olfato. ¿Para quién, si no, habría hecho los minerales en las entrañas de la Tierra y las sales disueltas en manantiales de álgidas aguas o de agua hirviendo: los azufres, los yodos, los bromos?.. Ciertamente, para que los gozara uno que no fuera Dios, sino hijo de Dios. Uno: el hombre. Nada le faltaba a la alegría de Dios, nada necesitaba Dios. El se basta a sí mismo. No tiene sino que contemplarse para deleitarse, nutrirse, vivir y descansar. Toda la creación no ha aumentado ni en un átomo su infinidad de alegría, de belleza, de vida, de potencia. He aquí que todo lo ha hecho para la criatura a la que ha querido poner como rey de la obra de sus manos: para el hombre. Aunque sólo fuera por ver una obra divina de tal magnitud y por manifestarle reconocimiento a Dios, que os la otorga, merecería la pena vivir. Y debéis sentir gratitud por el hecho de vivir. Gratitud que deberíais haber tenido aunque no hubierais sido redimidos sino al final de los siglos, porque, a pesar de que hayáis sido, en los Primeros, y ahora aun individualmente, prevaricadores, soberbios, lujuriosos, homicidas, Dios os concede todavía gozar de lo bello del universo, de lo bueno del universo, y os trata como si fuerais personas buenas, hijos buenos a los cuales todo se enseña y todo se concede para hacerles más suave y sana la vida. Cuanto sabéis, lo sabéis por luz de Dios. Cuanto descubrís, lo descubrís porque Dios os lo señala. Esto, en el Bien. Los otros conocimientos y descubrimientos que llevan el signo del mal vienen del Mal supremo: Satanás. La Mente suprema, que nada ignora, antes de que el hombre fuese, sabía que sería ladrón y homicida de sí mismo. Y, dado que la Bondad eterna no conoce límites en su ser buena, antes de que la Culpa fuera, pensó el medio para anular la Culpa. El medio, Yo; el instrumento para hacer del medio un instrumento operante, María. Y la Virgen fue creada en el pensamiento sublime de Dios. Todas las cosas han sido creadas para mí, Hijo dilecto del Padre. Yo- Rey habría debido tener bajo mi pie de Rey divino alfombras y joyas como palacio alguno jamás tuviera, y cantos y voces, y tantos siervos y ministros en torno a Mí como soberano alguno jamás tuviera, y flores y gemas, y todo lo sublime, lo grandioso, lo fino, lo delicado que es posible extraer del pensamiento de todo un Dios. Mas Yo debía ser Carne además de Espíritu. Carne para salvar a la carne. Carne para sublimar la carne, llevándola al Cielo muchos siglos antes de la hora. Porque la carne habitada por el espíritu es la obra maestra de Dios, y para ella había sido hecho el Cielo. Para ser Carne tenía necesidad de una Madre. Para ser Dios tenía necesidad de que el Padre fuese Dios. He aquí que entonces Dios se crea a su Esposa y le dice: "Ven conmigo. Junto a mí ve cuanto Yo hago para el Hijo nuestro. Mira y regocíjate, eterna Virgen, Doncella eterna, y tu risa llene este empíreo y dé a los ángeles la nota inicial y al Paraíso le enseñe la armonía celeste. Yo te miro, y te veo como serás, ¡oh, Mujer inmaculada que ahora eres sólo espíritu: el espíritu en que Yo me deleito! Yo te miro y doy al mar y al firmamento el azul de tu mirada; el color de tus cabellos, al trigo santo; el candor, a la azucena; el color rosa como tu epidermis de seda, a la rosa; de tus dientes delicados copio las perlas; hago las dulces fresas mirando tu boca; a los ruiseñores les pongo en la garganta tus notas y a las tórtolas tu llanto. Leyendo tus futuros pensamientos, oyendo los latidos de tu corazón, tengo el motivo guía para crear. Ven, Alegría mía, séante los mundos juguete hasta que me seas luz danzarina en el pensamiento, sean los mundos para reír tuyo. Tente las guirnaldas de estrellas y los collares de astros, ponte la luna bajo tus nobles pies, adórnate con el chal estelar de Galatea. Son para ti las estrellas y los planetas. Ven y goza viendo las flores que le servirán a tu Niño como juego y de almohada al Hijo de tu vientre. Ven y ve crear las ovejas y los corderos, las águilas y las palomas. Estate a mi lado mientras hago las cuencas de los mares y de los ríos, y alzo las montañas y las pinto de nieve y de bosques; mientras siembro los cereales y los árboles y las vides, y hago el olivo para ti, Pacífica mía, y la vid para ti, Sarmiento mío que llevarás el Racimo eucarístico. Camina, vuela, regocíjate, ¡oh, Hermosa mía!, y que el mundo universo, que en diversas fases voy creando, aprenda de ti a amarme, Amorosa, y que tu risa le haga más bello, Madre de mi Hijo, Reina de mi Paraíso, Amor de tu Dios". Y, viendo a quien es el Error y mirando a la Sin Error, dice: "Ven a mí, tú que cancelas la amargura de la desobediencia humana, de la fornicación humana con Satanás y de la humana ingratitud. Contigo me tomaré la revancha contra Satanás". Dios, Padre Creador, había creado al hombre y a la mujer con una ley de amor tan perfecta, que vosotros no podéis ni siquiera comprender sus perfecciones; vuestra mente se pierde pensando en cómo habría venido la especie si el hombre no la hubiera obtenido con la enseñanza de Satanás. Observad las plantas de fruto y de grano. ¿Obtienen la semilla o el fruto mediante fornicación, mediante una fecundación por cada cien uniones? No. De la flor masculina sale el polen y, guiado por un complejo de leyes meteóricas y magnéticas, va hacia el ovario de la flor femenina. Éste se abre y lo recibe y produce; no como hacéis vosotros, para experimentar al día siguiente la misma sensación, se mancha y luego lo rechaza. Produce, y hasta la nueva estación no florece, y cuando florece es para reproducirse. Observad a los animales. Todos. ¿Habéis visto alguna vez a un macho y a una hembra ir el uno hacia el otro para estéril abrazo y lascivo comercio? No. Desde cerca o desde lejos, volando, arrastrándose, saltando o corriendo, van, llegada la hora, al rito fecundativo, y no se substraen a él deteniéndose en el goce, sino que van más allá de éste, van a las consecuencias serias y santas de la prole, única finalidad que en el hombre, semidiós por el origen de gracia, de esa Gracia que Yo he devuelto completa, debería hacer aceptar la animalidad del acto, necesario desde que descendisteis un grado hacia los brutos. Vosotros no hacéis como las plantas y los animales. Vosotros habéis tenido como maestro a Satanás, lo habéis querido y lo queréis como maestro. Y las obras que realizáis son dignas del maestro que habéis querido. Mas si hubieseis sido fieles a Dios, habríais recibido la alegría de los hijos santamente, sin dolor, sin extenuaros en cópulas obscenas, indignas, ignoradas incluso por las bestias, las bestias sin alma racional y espiritual. Dios quiso oponer, frente al hombre y a la mujer pervertidos por Satanás, al Hombre nacido de una Mujer suprasublimada por Dios hasta el punto de generar sin haber conocido varón: Flor que genera Flor sin necesidad de semilla; sólo por el beso del Sol en el cáliz inviolado de la Azucena- María. ¡La revancha de Dios!Echa resoplidos de odio, Satanás, mientras Ella nace. ¡Esta Párvula te ha vencido! Antes de que fueras el Rebelde, el Tortuoso, el Corruptor, eras ya el Vencido, y Ella es tu Vencedora. Mil ejércitos en formación nada pueden contra tu potencia, ceden las armas de los hombres contra tus escamas, ¡oh, Perenne!, y no hay viento capaz de llevarse el hedor de tu hálito. Y sin embargo este calcañar de recién nacida, tan rosa que parece el interior de una camelia rosada, tan liso y suave que comparada con él la seda es áspera, tan pequeño que podría caber en el cáliz de un tulipán y hacerse un zapatito de ese raso vegetal, he aquí que te comprime sin miedo, te confina en tu caverna. Y su vagido te pone en fuga, a ti que no tienes miedo de los ejércitos; y su aliento libera al mundo de tu hedor. Estás derrotado. Su nombre, su mirada, su pureza son lanza, rayo, losa que te traspasan, que te abaten, que te encierran en tu madriguera de Infierno, ¡oh, Maldito, que le has arrebatado a Dios la alegría de ser Padre de todos los hombres creados! Se demuestra inútil ahora el haber corrompido a quienes habían sido creados inocentes, conduciéndolos a conocer y a concebir por caminos sinuosos de lujuria, privándole a Dios, en su criatura dilecta, de ser Él quien distribuyera magnánimamente los hijos según reglas que, si hubieran sido respetadas, habrían mantenido en la tierra un equilibrio entre los sexos y las razas que hubiera podido evitar guerras entre los hombres y desgracias en las familias. Obedeciendo, habrían conocido también el amor. Es más, sólo obedeciendo lo habrían conocido y lo habrían poseído. Una posesión llena y tranquila de esta emanación de Dios, que de lo sobrenatural desciende hacia lo inferior, para que la carne también se goce santamente en ella, la carne que está unida al espíritu y que ha sido creada por el Mismo que le creó el espíritu. ¿Ahora, ¡oh, hombres!, vuestro amor, vuestros amores, qué son? O libídine vestida de amor o miedo incurable de perder el amor del cónyuge por libídine suya y de otros. Desde que la libídine está en el mundo, ya nunca os sentís seguros de la posesión del corazón del esposo o de la esposa; y tembláis y lloráis y enloquecéis de celos, asesináis a veces para vengar una traición, os desesperáis otras veces u os volvéis abúlicos o dementes. Eso es lo que has hecho, Satanás, a los hijos de Dios. Estos que tú has corrompido habrían conocido la dicha de tener hijos sin padecer dolor, la dicha de nacer y no tener miedo a morir. Mas ahora has sido derrotado en una Mujer y por la Mujer. De ahora en adelante quien la ame volverá a ser de Dios, venciendo a tus tentaciones para poder mirar a su inmaculada pureza. De ahora en adelante, no pudiendo concebir sin dolor, las madres la tendrán a Ella como consuelo. De ahora en adelante será guía para las esposas y madre para los moribundos, por lo que dulce será el morir sobre ese seno que es escudo contra ti, Maldito, y contra el juicio de Dios. María, (se dirige aquí a María Valtorta) pequeña voz, has visto el nacimiento del Hijo de la Virgen y el nacimiento de la Virgen al Cielo. Has visto, por tanto, que los sin culpa desconocen la pena del dar a luz y la pena del morir. Y, si a la superinocente Madre de Dios le fue reservada la perfección de los dones celestes, igualmente, si todos hubieran conservado la inocencia y hubieran permanecido como hijos de Dios en los Primeros, habrían recibido el generar sin dolores (como era justo por haber sabido unirse y concebir sin lujuria) y el morir sin aflicción. La sublime revancha de Dios contra la venganza de Satanás ha consistido en llevar la perfección de la dilecta criatura a una superperfección que anulara, al menos en una, cualquier vestigio de humanidad susceptible de recibir el veneno de Satanás, por lo cual el Hijo vendría no de casto abrazo de hombre sino de un abrazo divino que, en el éxtasis del Fuego, arrebola el espíritu. ¡La Virginidad de la Virgen!... Ven. Medita en esta virginidad profunda que produce al contemplarla vértigos de abismo! ¿Qué es, comparada con ella, la pobre virginidad forzada de la mujer con la que ningún hombre se ha desposado? Menos que nada. ¿Y la virginidad de la mujer que quiso ser virgen para ser de Dios, pero sabe serlo sólo en el cuerpo y no en el espíritu, en el cual deja entrar muchos pensamientos de otro tipo, y acaricia y acepta caricias de pensamientos humanos? Empieza a ser una sombra de virginidad. Pero bien poco aún. ¿Qué es la virginidad de una religiosa de clausura que vive sólo de Dios? Mucho. Pero nunca es perfecta virginidad comparada con la de mi Madre. Hasta en el más santo ha habido al menos un contubernio: el de origen, entre el espíritu y la Culpa, esa unión que sólo el Bautismo disuelve. La disuelve, sí, pero, como en el caso de una mujer separada de su marido por la muerte, no devuelve la virginidad total como era la de los Primeros antes del pecado. Una cicatriz queda, y duele, recordando así su presencia, cicatriz que puede siempre en cualquier momento traducirse de nuevo en una llaga, como ciertas enfermedades agudizadas periódicamente por sus virus. En la Virgen no existe esta señal de un disuelto ligamen con la Culpa. Su alma aparece bella e intacta como cuando el Padre la pensó reuniendo en Ella todas las gracias. Es la Virgen. Es la Única. Es la Perfecta. Es la Completa. Pensada así. Engendrada así. Que ha permanecido así. Coronada así. Eternamente así. Es la Virgen. Es el abismo de la intangibilidad, de la pureza, de la gracia que se pierde en el Abismo de que procede, es decir, en Dios, Intangibilidad, Pureza, Gracia perfectísimas. Así se ha desquitado el Dios Trino y Uno: Él ha alzado contra la profanación de las criaturas esta Estrella de perfección; contra la curiosidad malsana, esta Mujer Reservada que sólo se siente satisfecha amando a Dios; contra la ciencia del mal, esta Sublime Ignorante. Ignorante no sólo en lo que toca al amor degradado, o al amor que Dios había dado a los cónyuges, sino más todavía: en Ella se trata de ignorancia del fomes, herencia del Pecado. En Ella sólo se da la gélida e incandescente sabiduría del Amor divino. Fuego que encoraza de hielo la carne para que sea espejo transparente en el altar en que un Dios se desposa con una Virgen, y no por ello se rebaja, porque su perfección envuelve a Aquella que, como conviene a una esposa, es sólo inferior en un grado al Esposo, sujeta a Él por ser Mujer, pero, como Él, sin mancha". }

\chapter*{Purificación de Ana y ofrecimiento de María, \\ \normalfont\normalsize\textit{que es la Niña perfecta para el reino de los Cielos.}}
\addcontentsline{toc}{chapter}{\normalfont\scshape{Purificación de Ana y ofrecimiento de María,}}
 
Veo a Joaquín y a Ana, junto a Zacarías y a Isabel, saliendo de una casa de Jerusalén de amigos o familiares. Se dirigen hacia el Templo para la ceremonia de la Purificación. 

Ana lleva en brazos a la Niña, envuelta toda en fajos, toda envuelta en un amplio tejido de lana ligera, pero que debe ser suave y caliente. ¡Con cuánto cuidado y amor lleva a su criaturita! De vez en cuando levanta el borde del fino y caliente tejido para ver si María respira a gusto, y luego vuelve a taparla para protegerla del aire helador de un día sereno pero frío, de pleno invierno. 

Isabel lleva unos paquetes en las manos. Joaquín lleva de una cuerda a dos corderos blanquísimos bien cebados, ya más carneros que corderos. Zacarías no lleva nada. ¡Qué apuesto con ese vestido de lino que un grueso manto de lana, también blanca, deja entrever! Es un Zacarías mucho más joven que el que se veía en el nacimiento del Bautista, entonces ya en plena edad adulta. Isabel es una mujer madura, pero todavía de apariencia fresca; cada vez que Ana mira a la Niña, se curva extasiada hacia esa carita dormida. También Isabel está guapísima con su vestido de un azul tendente al morado oscuro y con el velo que le cubre la cabeza y cae sobre los hombros y sobre el manto, que es más oscuro que el vestido. 

¿Y Joaquín y Ana? ¡Ah..., solemnes con sus vestidos de fiesta! Contrariamente a lo normal, él no lleva la túnica marrón oscura, sino un largo vestido de un rojo oscurísimo (hoy diríamos: rojo S. José). Las orlas de su manto son bonitas y muy nuevas. En la cabeza lleva también una especie de velo rectangular, ceñido con una cinta de cuero. Todo nuevo y fino. 

Ana... ¡oh!, hoy no viste de oscuro. Lleva un vestido de un amarillo muy tenue, casi color marfil viejo, ceñido en la cintura, cuello y muñecas, con una gruesa cinta que parece de plata y oro. Su cabeza está cubierta por un velo ligerísimo y como adamascado, sujeto a la frente con un aro sutil, valioso. En el cuello lleva un collar de filigrana; en las muñecas, pulseras. Parece una reina, incluso por la dignidad con que lleva el vestido, y especialmente el manto, amarillo tenue, orlado con una greca en bordadura muy bonita, también amarilla. 

Me pareces como en el día de tu boda. Entonces yo era poco más que una niña. Todavía me acuerdo de lo guapa y dichosa que se te veía - dice Isabel. 

Pues más feliz me siento ahora... Y he querido ponerme el mismo vestido para este rito. Lo había conservado siempre para esto... aunque ya, para esto, no tenía esperanzas de ponérmelo. 

El Señor te ha amado mucho... - dice suspirando Isabel. 

Por eso precisamente le doy lo que más quiero. Esta flor mía. 

- ¿Y vas a tener fuerzas para arrancártela de tu seno cuando llegue el momento? 

Sí, porque recordaré que no la tenía y que Dios me la dio. En todo caso me sentiré más feliz que entonces. Y, sabiendo que está en el Templo, me diré: "Está orando ante el Tabernáculo, está rezando al Dios de Israel, y también por su madre". Ello me dará paz. Y más paz todavía al decir: "Ella es toda suya. Cuando estos dos felices ancianos, que la recibieron del Cielo, ya no estén en este mundo, Él, el Eterno, seguirá siendo su Padre". Créeme, tengo la firme convicción de que esta pequeñuela no es nuestra. Yo ya no podía hacer nada... Él la puso en mi seno como don divino para enjugar mi llanto y confortar nuestras esperanzas y oraciones. Por tanto, es suya. Nosotros somos los encargados, felices encargados, de cuidarla... ¡y que por ello sea bendito! 

Llegan a los muros del Templo. 
 
Mientras vais a la Puerta de Nicanor, yo voy a advertir al sacerdote. Luego os alcanzo - dice Zacarías; y desaparece tras un arco que introduce a un amplio patio circundado de pórticos. 

La comitiva continúa adentrándose por las sucesivas terrazas (porque — no sé si lo he dicho alguna vez — el recinto del Templo no es una superficie plana, sino que sube escalonadamente en niveles cada vez más altos; a cada uno de ellos se accede mediante escalinatas, y en todos hay patios y pórticos y portones labradísimos, de mármol, bronce y oro). 

Antes de llegar al lugar establecido, se paran para desenvolver las cosas que traen, o sea, tortas — me parece — muy untadas, anchas y finas, harina blanca, dos palomas en una jaulita de mimbre y unas monedas grandes de plata, tan pesadas que era una suerte que en aquella época no hubiera bolsillos, porque los habrían roto. 

Ahí está la bonita Puerta de Nicanor; es por entero un bordado en pesado bronce laminado de plata. Ya está allí Zacarías, al lado de un sacerdote que está todo pomposo con su vestido de lino. 

Asperjan a Ana con agua lustral — supongo — y luego le indican que se dirija hacia el ara del sacrificio. Ya no lleva a la Niña en brazos. La ha tomado en brazos Isabel, que se ha quedado a este lado de la Puerta. 

Joaquín, sin embargo, entra siguiendo a su mujer, y llevando tras sí un desgraciado cordero que va balando. Y yo... hago como para la purificación de María: cierro los ojos para no ver ningún tipo de degüello. 

Ana ya está purificada. 

Zacarías dice en voz baja unas palabras a su compañero de ministerio, el cual, sonriendo, da señales de asentimiento y luego se acerca al grupo, rehecho de nuevo, y, congratulándose con la madre y el padre por su gozo y por su fidelidad a las promesas, recibe el segundo cordero, la harina y las tortas. 

Entonces ¿esta hija está consagrada al Señor? Que su bendición os acompañe a Ella y a vosotros. Mirad, ahí viene Ana. Va a ser una de sus maestras. Ana de Fanuel, de la tribu de Aser. Ven, mujer. Esta pequeñuela ha sido ofrecida al Templo como hostia de alabanza. Tú serás para ella maestra. A tu amparo crecerá santa. 

Ana de Fanuel, ya completamente encanecida, hace mimos a la Niña, que ya se ha despertado y que observa toda esa blancura con esos inocentes y atónitos ojos suyos, y todo ese oro que el sol enciende. 

La ceremonia debe haber terminado. No he visto ningún rito especial para el ofrecimiento de María. Quizás era suficiente con decírselo al sacerdote, y sobre todo a Dios, en el lugar santo. 

Querría dar mi ofrenda al Templo e ir al lugar en que el año pasado vi la luz - dice Ana. 

Ana de Fanuel va con ellos. No entran en el Templo propiamente dicho. Es natural que, siendo mujeres y tratándose de una niña, no vayan ni siquiera a donde fue María para ofrecer a su Hijo. Pero, eso sí, desde muy cerquita de la puerta, que está abierta de par en par, miran hacia el semioscuro interior del que vienen dulces cantos de niñas y en el que brillan ricas lámparas, que expanden luz de oro sobre dos cuadros de flores de cabecitas veladas de blanco, dos verdaderos cuadros de azucenas. 

Dentro de tres años estarás ahí, Azucena mía - le promete Ana a María, que mira como embelesada hacia el interior y sonríe al oír el lento canto. 

Parece como si entendiera - dice Ana de Fanuel - ¡Es una niña muy bonita! La querré como si fuera fruto de mis entrañas. Te lo prometo, madre. Si la edad me lo concede. 

Te lo concederá, mujer - dice Zacarías - La recibirás entre las niñas consagradas. Yo también estaré presente. Quiero estar ese día para decirle que pida por nosotros desde el primer momento... - y mira a su mujer, la cual, habiendo comprendido, suspira. 

La ceremonia ha concluido. Ana de Fanuel se retira, mientras los otros, hablando entre sí, salen del Templo. 

Oigo a Joaquín que dice: 

- ¡No sólo dos, y los mejores, sino que habría dado todos mis corderos por este gozo y para alabar a Dios! 

No veo nada más. 

Dice Jesús: 
\emph{Salomón pone en boca de la Sabiduría estas palabras: "Quien sea niño venga a mí". Y verdaderamente, desde la roca, desde los muros de su ciudad, la eterna Sabiduría le decía a la eterna Niña: "Ven a mí". Se consumía por tenerla. Pasado un tiempo, el Hijo de la Doncella purísima dirá: "Dejad que los niños vengan a mí, porque el Reino de los Cielos es de ellos, y quien no se haga como ellos no tendrá parte en mi Reino". Las voces se buscan recíprocamente y, mientras la voz proveniente del Cielo grita a la pequeñuela María: "Ven a mí", la voz del Hombre dice: "Venid a mí si sabéis ser niños", y al decirlo piensa en su Madre. Os doy el modelo en mi Madre. Ella es la perfecta Niña con corazón de paloma sencillo y puro, Aquélla a quien ni los años ni el contacto con el mundo enrudecen bárbaramente, corrompiendo su espíritu o haciéndole tortuoso o mentiroso. Porque Ella no lo quiere. Venid a mí mirando a María. Tú, que la ves, dime: ¿su mirada de infante es muy distinta de la que viste al pie de la Cruz; o en el júbilo de Pentecostés; o en la hora en que los párpados cubrieron su ojo de gacela para el último sueño? No. Aquí se trata de la mirada incierta y atónita del infante; luego se tratará de esa mirada atónita y ruborosa de la Virgen de la Anunciación, o beata como la de la Madre de Belén, o adoradora, como la de mi primera, sublime Discípula; luego será la mirada lastimera de la Torturada del Gólgota, o radiante, como en la Resurrección y en Pentecostés; luego será esa mirada velada: la del extático sueño de la última visión. Pero, ya se abra para ver por primera vez, ya se cierre, cansado, con la última luz, habiendo visto tanto gozo y tanto horror, este ojo es ese apacible, puro, sosegado trocito de cielo que resplandece siempre igual bajo la frente de María. Ira, mentira, soberbia, lujuria, odio, curiosidad, no lo ensucian jamás con sus fumosas nubes. Es la mirada que mira a Dios con amor, ya llore, ya ría, y que por amor a Dios acaricia y perdona, y todo lo soporta; el amor a su Dios le ha hecho inmune a los asaltos del Mal, que muchas veces se sirve del ojo para penetrar en el corazón; es el ojo puro, tranquilizante, bendecidor que tienen los puros, los santos, los enamorados de Dios. Ya lo dije: "El ojo es luz de tu cuerpo. Si el ojo es puro, todo tu cuerpo estará iluminado; mas si el ojo es túrbido, toda tu persona estará en las tinieblas". Los santos han tenido estos ojos, que son luz para el espíritu y salvación para la carne, porque, como María, durante toda su vida sólo han mirado a Dios; o, más aún, han tenido recuerdo de Dios. Ya te explicaré, pequeña voz, el sentido de estas palabras mías. }
 
\chapter*{María niña con Ana y Joaquín. \\ \normalfont\normalsize\textit{En sus labios ya está la Sabiduría del Hijo.}}
\addcontentsline{toc}{chapter}{\normalfont\scshape{María niña con Ana y Joaquín.}}
 
Sigo viendo todavía a Ana. Desde ayer por la tarde la veo así: sentada donde empieza la pérgola umbrosa; dedicada a un trabajo de costura. Está vestida de un solo color gris arena; es un vestido muy sencillo y suelto, quizás por el mucho calor que parece que hace. 

En el otro extremo de la pérgola se ve a los dalladores segando el heno; heno que no debe ser de mayo. Efectivamente, la uva ya está detrás coloreándose de oro, y un grueso manzano muestra entre sus oscuras hojas sus frutos, que están tomando un color de lúcida cera amarilla y roja; y además el campo de trigo es ya sólo un rastrojal en que ondean ligeras las llamitas de las amapolas y los lirios se elevan, rígidos y serenos, radiados como una estrella, azules como el cielo de oriente. 

De la pérgola umbrosa sale caminando una María pequeñita, que, no obstante, es ya ágil e independiente. Su breve paso es seguro y sus sandalitas blancas no tropiezan en los cantos. Tiene ya esbozado su dulce paso ligeramente ondulante de paloma, y está toda blanca, como una palomita, con su vestidito de lino que le llega a los tobillos, amplio, fruncido en torno al cuello con un cordoncito de color celeste, y con unas manguitas cortas que dejan ver los antebrazos regordetes. Con su pelito sérico y rubio- miel, no muy rizado pero sí todo él formando suaves ondas que en el extremo terminan en un leve ensortijado, con sus ojos de cielo y su dulce carita tenuemente sonrosada y sonriente, parece un pequeño ángel. El vientecillo que le entra por las anchas mangas y le hincha por detrás el vestidito de lino contribuye también a darle aspecto de un pequeño ángel cuando despliega las alas para el vuelo. 

Lleva en sus manitas amapolas y lirios y otras florecillas que crecen entre los trigos y cuyo nombre desconozco. Se dirige hacia su madre. Cuando está ya cerca, inicia una breve carrera, emitiendo una vocecita festiva, y va, como una tortolita, a detener su vuelo contra las rodillas maternas, abiertas un poco para recibirla. Ana ha depositado al lado el trabajo que estaba haciendo para que Ella no se pinche, y ha extendido los brazos para ceñirla. 

Hasta este punto, ayer por la tarde; hoy por la mañana se ha vuelto a presentar y continúa así: 

- ¡Mamá! ¡Mamá! 

La tortolita blanca está toda en el nido de las rodillas maternas, apoyando sus piececitos sobre la hierba corta, y la carita en el regazo materno. Sólo se ve el oro pálido de su pelito sobre la sutil nuca que Ana se inclina a besar con amor. 

Luego la tortolita levanta su pequeña cabeza y entrega sus florecillas: todas para su mamá. Y de cada flor cuenta una historia creada por Ella. 

Ésta, tan azul y tan grande, es una estrella que ha caído del cielo para traerle a su mamá el beso del Señor... ¡Que bese en el corazón, en el corazón, a esta florecilla celeste, y percibirá que tiene sabor a Dios!.. 

Y esta otra, de color azul más pálido, como los ojos de su papá, lleva escrito en las hojas que el Señor quiere mucho a su papá porque es bueno. 

Y esta tan pequeñita, la única encontrada de ese tipo (una miosota), es la que el Señor ha hecho para decirle a María que la quiere. 

Y estas rojas, ¿sabe su mamá qué son? Son trozos de la vestidura del rey David, empapados de sangre de los enemigos de Israel, y esparcidos por los campos de batalla y de victoria. Proceden de esos limbos de regia vestidura hecha jirones en la lucha por el Señor. 

En cambio ésta, blanca y delicada, que parece hecha con siete copas de seda que miran al cielo, llenas de perfumes, y que ha nacido allí, junto al fontanar, se la ha cogido su papá de entre las espinas, está hecha con la vestidura que llevaba el rey Salomón cuando, el mismo mes en que nació esta Niña descendiente suya, muchos años, ¡oh, cuántos, cuántos antes; muchos años antes, él, con la pompa cándida de sus vestiduras, caminó entre la multitud de Israel ante el Arca y ante el Tabernáculo, y se regocijó por la nube que volvía a circundar su gloria, y cantó el cántico y la oración de su gozo. 

Yo quiero ser siempre como esta flor, y, como el rey sabio, quiero cantar toda la vida cánticos y oraciones ante el Tabernáculo" termina así la boquita de María. 

- ¡Tesoro mío! ¿Cómo sabes estas cosas santas? ¿Quién te las dice? ¿Tu padre? 

No. No sé quién es. Es como si las hubiera sabido siempre. Pero quizás me las dice alguien, alguien a quien no veo. Quizás uno de los ángeles que Dios envía a hablarles a los hombres buenos. Mamá, ¿me sigues contando alguna otra historia?... 

- ¡Oh, hija mía! ¿Cuál quieres saber? 

María se queda pensando; seria y recogida como está, habría que pintarla para eternizar su expresión. En su carita infantil se reflejan las sombras de sus pensamientos. Sonrisas y suspiros, rayos de sol y sombras de nubes pensando en la historia de Israel. Luego elige: 

Otra vez la de Gabriel y Daniel, en que está la promesa del Cristo. 

Y escucha con los ojos cerrados, repitiendo en voz baja las palabras que su madre le dice, como para recordarlas mejor. Cuando Ana termina, pregunta: 

- ¿Cuánto falta todavía para tener con nosotros al Emmanuel? 

Treinta años aproximadamente, querida mía. 

- ¡Cuánto todavía! Y yo estaré en el Templo... Dime, si rezase mucho, mucho, mucho, día y noche, noche y día, y deseara 

ser sólo de Dios, toda la vida, con esta finalidad, ¿el Eterno me concedería la gracia de dar antes el Mesías a su pueblo? 

No lo sé, querida mía. El Profeta dice: "Setenta semanas". Yo creo que la profecía no se equivoca. Pero el Señor es tan bueno — se apresura a añadir Ana, al ver que las pestañas de oro de su niña se perlan de llanto — que creo que si rezas mucho, mucho, mucho, se te mostrará propicio. 

La sonrisa aparece de nuevo en esa carita ligeramente alzada hacia la madre, y un ojalito de sol que pasa entre dos pámpanas hace brillar las lágrimas del ya cesado llanto, cual gotitas de rocío colgando de los tallitos sutilísimos del musgo alpino. - Entonces rezaré y me consagraré virgen para esto. - Pero, ¿sabes lo que quiere decir eso? 

Quiere decir no conocer amor de hombre, sino sólo de Dios. Quiere decir no tener ningún pensamiento que no sea para el Señor. Quiere decir ser siempre niña en la carne y ángel en el corazón. Quiere decir no tener ojos sino para mirar a Dios, oídos para oírle, boca para alabarle, manos para ofrecerse como hostias, pies para seguirle velozmente, corazón y vida para dárselos a El. 

- ¡Bendita tú! Pero entonces no tendrás nunca niños, ¿sabes? ; y a ti te gustan mucho los niños y los corderitos y las tortolitas. Un niño para una mujer es como un corderito blanco y crespo, como una palomita de plumas de seda y boca de coral: se le puede amar, besar; se puede oír que nos llama "mamá". 

No importa. Seré de Dios. En el Templo rezaré. Y quizás un día vea al Emmanuel. La Virgen que debe ser Madre suya, como dice el gran Profeta, ya debe haber nacido y estar en el Templo... Yo seré compañera suya... y sierva suya. ¡Oh, sí! Si pudiera conocer, por luz de Dios, a esa mujer bienaventurada, querría servirla. Luego Ella me traería a su Hijo, me conduciría hacia su Hijo y así le serviría también a Él. ¡Fíjate, mamá!.. ¡¡Servir al Mesías!!.. - María se siente sobrepujada por este pensamiento que la sublima y la deja anonadada al mismo tiempo. Con las manitas cruzadas sobre su pecho y la cabecita un poco inclinada hacia adelante, y encendida de emoción, parece una infantil reproducción de la Virgen de la Anunciación que yo vi. Y sigue diciendo: 

- ¿Pero, el Rey de Israel, el Ungido de Dios, me permitirá servirle? 

No lo dudes. ¿No dice el rey Salomón: "Sesenta son las reinas y ochenta las otras esposas y sin número las doncellas" En ello puedes ver que en el palacio del Rey serán sin número las doncellas vírgenes que servirán a su Señor. 

- ¡Oh! ¿Lo ves como debo ser virgen? Debo serlo. Si Él por madre quiere una virgen, es señal de que estima la virginidad por encima de todas las cosas. Yo quiero que me ame a mí, su sierva, por esa virginidad que me hará un poco similar a su dilecta Madre... Esto es lo que quiero... Querría también ser pecadora, muy pecadora, si no temiera ofender al Señor... Dime, mamá, ¿puede una ser pecadora por amor a Dios? 

Pero, ¿qué dices, tesoro? No entiendo. 

Quiero decir: pecar para poder ser amada por Dios hecho Salvador. Se salva a quien está perdido, ¿no es verdad? Yo querría ser salvada por el Salvador para recibir su mirada de amor. Para esto querría pecar, pero no cometer un pecado que le disgustase. ¿Cómo puede salvarme si no me pierdo? 

Ana está atónita. No sabe ya qué decir. 

Viene en su ayuda Joaquín, el cual, caminando sobre la hierba, se ha ido acercando, sin hacer ruido, por detrás del seto de sarmientos bajos. 

Te ha salvado antes porque sabe que le amas y quieres amarle sólo a Él. Por ello tú ya estás redimida y puedes ser virgen como quieres - dice Joaquín. 

- ¿Sí, padre mío?- María se abraza a sus rodillas y le mira con las claras estrellas de sus ojos, muy semejantes a los paternos, y muy dichosos por esta esperanza que su padre le da. 

Verdaderamente, pequeño amor. Mira, yo te traía este pequeño gorrión que en su primer vuelo había ido a posarse junto a la fuente. Habría podido dejarlo, pero sus débiles alas no tenían fuerza para elevarlo en nuevo vuelo, ni sus patitas de seda para fijarlo a las musgosas piedras, que resbalaban. Se habría caído en la fuente. No he esperado a que esto sucediera. Lo he cogido y ahora te lo regalo. Haz lo que quieras con él. El hecho es que ha sido salvado antes de caer en el peligro. Lo mismo ha hecho Dios contigo. Ahora, dime, María: ¿he amado más al gorrión salvándolo antes, o lo habría amado más salvándolo después? 

Ahora lo has amado, porque no has permitido que se hiciera daño con el agua helada. 

Y Dios te ha amado más, porque te ha salvado antes de que tú pecaras. 

Pues entonces yo le amaré completamente, completamente. Gorrioncito bonito, yo soy como tú. El Señor nos ha amado de la misma manera, salvándonos... Ahora voy a criarte y luego te dejaré suelto. Tú cantarás en el bosque y yo en el Templo las alabanzas del Señor, y diremos: "Envía a tu Prometido, envíaselo a quien espera". ¡Oh, papá mío! ¿Cuándo me vas a llevar al Templo? 

Pronto, perla mía. Pero, ¿no te duele dejar a tu padre? 

- ¡Mucho! Pero tú vendrás... y, además, si no doliese, ¿qué sacrificio sería? 

 - ¿Y te vas a acordar de nosotros? - Siempre. Después de la oración por el Emmanuel rezaré por vosotros. Para que Dios os haga dichosos y os dé una larga vida... hasta el día en que Él sea Salvador. Luego diré que os tome para llevaros a la Jerusalén del Cielo. 

La visión me cesa con María estrechada en el lazo de los brazos de su padre... 

Dice Jesús: 
\emph{Llegan ya a mis oídos los comentarios de los doctores de los tiquismiquis: "¿Cómo puede hablar así una niña que no ha cumplido aún tres años? Es una exageración". Pero no piensan que ellos, alterando mi infancia con actos propios de adultos, dan de mí una imagen monstruosa. La inteligencia no llega a todos de la misma manera y al mismo tiempo. La Iglesia ha establecido los seis años como la edad de responsabilidad de las acciones, porque esa es la edad en que incluso un niño retrasado puede distinguir, al menos rudimentariamente, el bien y el mal. Pero hay niños que mucho antes son capaces de discernir, entender y querer, con una razón ya suficientemente desarrollada. Que las pequeñas Imelda Lambertini, Rosa de Viterbo, Nellie Organ, Nennolina os proporcionen una base para creer, ¡oh, doctores difíciles!, que mi Madre podía pensar y hablar así. Sólo he considerado cuatro nombres al azar entre los millares de niños santos que, después de haber razonado como adultos en la tierra durante más o menos años, han venido a poblar mí Paraíso. ¿Qué es la razón? Un don de Dios. Él, por tanto, puede darla con la medida que quiera, a quien quiera y cuando quiera. Es, además, una de las cosas que más os asemejan a Dios, Espíritu inteligente y que razona. La razón y la inteligencia fueron gracias otorgadas por Dios al Hombre en el Paraíso Terrenal. ¡Y qué vivas estaban cuando la Gracia moraba, aún intacta y operante, en el espíritu de los dos Primeros! En el libro de Jesús Bar Sirac está escrito: "Toda sabiduría viene del Señor Dios y con Él ha estado siempre, incluso antes de los siglos". ¿Qué sabiduría, pues, habrían tenido los hombres si hubieran conservado su filiación para con Dios? Vuestras lagunas de inteligencia son el fruto natural de haber venido a menos en la Gracia y en la honestidad. Perdiendo la Gracia, habéis alejado de vosotros, durante siglos, la Sabiduría. Cual estrella fugaz que se oculta tras nebulosidades de kilómetros, la Sabiduría no ha seguido llegándoos con sus netos destellos, sino sólo a través de neblinas cada vez más oprimentes a causa de vuestras prevaricaciones. Luego ha venido el Cristo y os ha vuelto a dar la Gracia, don supremo del amor de Dios. Pero ¿sabéis custodiar limpia y pura esta gema? No. Cuando no la rompéis con la voluntad individual de pecar, la ensuciáis con continuas culpas menores, con debilidades, o gravitando hacia el vicio (y ello, a pesar de no significar una verdadera unión con el septiforme vicio, debilita la luz de la Gracia y su actividad). Luego, además, siglos y siglos de corrupciones, que, deletéreas, repercuten en lo físico y en la mente, han ido debilitando la magnífica luz de la inteligencia que Dios había dado a los Primeros. Pero María era no sólo la Pura, la nueva Eva recreada para alegría de Dios, era la super- Eva, era la Obra Maestra del Altísimo, era la Llena de Gracia, era la Madre del Verbo en la mente de Dios. "Fuente de la Sabiduría" dice Jesús Bar Sirac "es el Verbo". ¿Y el Hijo no va a haber puesto su sabiduría en los labios de su Madre? Si a un Profeta que debía decir las palabras que el Verbo, la Sabiduría, le confiaba para transmitírselas a los hombres, le fue purificada la boca con carbones encendidos, ¿no va a haber depurado y elevado el Amor el habla de esa su Esposa niña que debía llevar en sí la Palabra, a fin de que no hablase primero como niña y luego como mujer, sino sólo y siempre como criatura celeste fundida con la gran luz y sabiduría de Dios? El milagro no está en el hecho de que María, como luego Yo, mostrara en edad infantil una inteligencia superior. El milagro está en el hecho de contener a la Inteligencia infinita, que en Ella moraba, en los diques convenientes para no pasmar a las multitudes y para no despertar la atención satánica. En otra ocasión seguiré hablando de esto, que está en relación con ese "recordarse" que los santos tienen de Dios. }
 
\chapter*{María recibida en el Templo. \\ \normalfont\normalsize\textit{En su humildad, no sabía que era la Llena de Sabiduría.}}
\addcontentsline{toc}{chapter}{\normalfont\scshape{María recibida en el Templo.}}
 
Veo a María caminando entre su padre y su madre por las calles de Jerusalén. 

Los que pasan se paran a mirar a la bonita Niña vestida toda de blanco nieve y arrollada en un ligerísimo tejido que, por sus dibujos, de ramas y flores, más opacos que el tenue fondo, creo que es el mismo que tenía Ana el día de su Purificación. Lo único es que, mientras que a Ana no le sobrepasaba la cintura, a María, siendo pequeñita, le baja casi hasta el suelo, envolviéndola en una nubecita ligera y lúcida de singular gracia. 

El oro de la melena suelta sobre los hombros, mejor: sobre la delicada nuca, se transparenta a través del sutilísimo fondo, en las partes del velo no adamascadas. Éste está sujeto a la frente con una cinta de un azul palidísimo que tiene, obviamente hecho por su mamá, unas pequeñas azucenas bordadas en plata. 

El vestido, como he dicho, blanquísimo, le llega hasta abajo, y los piececitos, con sus pequeñas sandalias blancas, apenas se muestran al caminar. Las manitas parecen dos pétalos de magnolia saliendo de la larga manga. Aparte del círculo azul de la cinta, no hay ningún otro punto de color. Todo es blanco. María parece vestida de nieve. 

Joaquín lleva el mismo vestido de la Purificación. Ana, en cambio, un oscurísimo morado; el manto, que le tapa incluso la cabeza, es también morado oscuro; lo lleva muy bajo, a la altura de los ojos, dos pobres ojos de madre rojos de llanto, que no quisieran llorar, y que no quisieran, sobre todo, ser vistos llorar, pero que no pueden no llorar al amparo del manto. Éste protege, por una parte, de los que pasan; también, de Joaquín, cuyos ojos, siempre serenos, hoy están también enrojecidos y opacos por las lágrimas (las que ya han caído y las que aún siguen cayendo). Camina muy curvado, bajo su velo a guisa casi de turbante que le cubre los lados del rostro. 

Joaquín está muy envejecido. Los que le ven deben pensar que es abuelo o quizás bisabuelo de la pequeñuela que lleva de la mano. El pobre padre, a causa de la pena de perderla, va arrastrando los pies al caminar; todo su porte es cansino y le hace unos veinte años más viejo de lo que en realidad es; su rostro parece el de una persona enferma además de vieja, por el mucho cansancio y la mucha tristeza; la boca le tiembla ligeramente entre las dos arrugas — tan marcadas hoy — de los lados de la nariz. 

Los dos tratan de celar el llanto. Pero, si pueden hacerlo para muchos, no pueden para María, la cual, por su corta estatura, los ve de abajo arriba y, levantando su cabecita, mira alternativamente a su padre y a su madre. Ellos se esfuerzan en sonreírle con su temblorosa boca, y aprietan más con su mano la diminuta manita cada vez que su hijita los mira y les sonríe. Deben pensar: "Sí. Otra vez menos que veremos esta sonrisa". 

Van despacio, muy despacio. Da la impresión de que quieren prolongar lo más posible su camino. Todo es ocasión para detenerse... Pero, ¡siempre debe tener un fin un camino!.. Y éste está ya para acabarse. En efecto, allí, en la parte alta de este último tramo en subida, están los muros que circundan el Templo. Ana gime, y estrecha más fuertemente la manita de María. - ¡Ana, querida mía, aquí estoy contigo! - dice una voz desde la sombra de un bajo arco echado sobre un cruce de calles. Isabel estaba esperando. Ahora se acerca a Ana y la estrecha contra su corazón, y, al ver que Ana llora, le dice: - Ven, ven un poco a esta casa amiga; también está Zacarías. 

Entran todos en una habitación baja y oscura cuya luz es un vasto fuego. La dueña, que sin duda es amiga de Isabel, si bien no conoce a Ana, amablemente se retira, dejando a los llegados libertad de hablar. 

No creas que estoy arrepentida, o que entregue con mala voluntad mi tesoro al Señor — explica Ana entre lágrimas — ... Lo que pasa es que el corazón... ¡oh, cómo me duele el corazón, este anciano corazón mío que vuelve a su soledad, a esa soledad de quien no tiene hijos!.. Si lo sintieras... 

Lo comprendo, Ana mía... Pero tú eres buena y Dios te confortará en tu soledad. María va a rezar por la paz de su mamá, ¿verdad? 

María acaricia las manos maternas y las besa, se las pone en la cara para ser acariciada a su vez, y Ana cierra entre sus manos esa carita y la besa, la besa... no se sacia de besarla. 

Entra Zacarías y saluda diciendo: 

A los justos la paz del Señor. 

Sí — dice Joaquín —, pide paz para nosotros porque nuestras entrañas tiemblan, ante la ofrenda, como las de nuestro padre Abraham mientras subía el monte; y nosotros no encontraremos otra ofrenda que pueda recobrar ésta; ni querríamos hacerlo, porque somos fieles a Dios. Pero sufrimos, Zacarías. Compréndenos, sacerdote de Dios, y no te seamos motivo de escándalo. 

Jamás. Es más, vuestro dolor, que sabe no traspasar lo lícito, que os llevaría a la infidelidad, es para mí escuela de amor al Altísimo. ¡Ánimo! La profetisa Ana cuidará con esmero esta flor de David y Aarón. En este momento es la única azucena que David tiene de su estirpe santa en el Templo, y cual perla regia será cuidada. A pesar de que los tiempos hayan entrado ya en la recta final y de que deberían preocuparse las madres de esta estirpe de consagrar sus hijas al Templo — puesto que de una virgen de David vendrá el Mesías — no obstante, a causa de la relajación de la fe, los lugares de las vírgenes están vacíos. Demasiado pocas en el Templo; y de esta estirpe regia ninguna, después de que, hace ya tres años, Sara de Elíseo salió desposada. Es cierto que aún faltan seis lustros para el final, pero bueno, pues esperemos que María sea la primera de muchas vírgenes de David ante el Sagrado Velo. Y... ¿quién sabe?... — Zacarías se detiene en estas palabras y... mira pensativo a María. Luego prosigue diciendo: - También yo velaré por Ella. Soy sacerdote y ahí dentro tengo mi influencia. Haré uso de ella para este ángel. Además, Isabel vendrá a menudo a verla... 

- ¡Oh, claro! Tengo mucha necesidad de Dios y vendré a decírselo a esta Niña para que a su vez se lo diga al Eterno. 

Ana ya está más animada. Isabel, buscando confortarla aún más, pregunta: 

- ¿No es éste tu velo de cuando te casaste?, ¿o has hilado más muselina? 

Es aquél. Lo consagro con Ella al Señor. Ya no tengo ojos para hilar... Además, por impuestos y adversidades, las posibilidades económicas son mucho menores... No me era lícito hacer gastos onerosos. Sólo me he preocupado de que tuviera un ajuar considerable para el tiempo que transcurra en la Casa de Dios y para después... porque creo que no seré yo quien la vista para la boda... Pero quiero que sea la mano de su madre, aunque esté ya fría e inmóvil, la que la haya ornado para la boda y le haya hilado la ropa y el vestido de novia. 

- ¡Oh, por qué tienes que pensar así? 

Soy vieja, prima. Jamás me he sentido tan vieja como ahora bajo el peso de este dolor. Las últimas fuerzas de mi vida se las he dado a esta flor, para llevarla y nutrirla, y ahora... y ahora... el dolor de perderla sopla sobre las postreras y las dispersa. - No digas eso. Queda Joaquín. 

Tienes razón. Trataré de vivir para mi marido. 

Joaquín ha hecho como que no ha oído, atento como está a lo que le dice Zacarías; pero sí que ha oído, y suspira fuertemente, y sus ojos brillan de llanto. 

Estamos entre tercia y sexta. Creo que sería conveniente ponernos en marcha" dice Zacarías. 

Todos se levantan para ponerse los mantos y comenzar a salir. 

Pero María se adelanta y se arrodilla en el umbral de la puerta con los brazos extendidos, un pequeño querubín suplicante: 

- ¡Padre, Madre, vuestra bendición! 

No llora la fuerte pequeña; pero los labiecitos sí tiemblan, y la voz, rota por un interno sollozo, presenta más que nunca el tembloroso gemido de una tortolita. La carita está más pálida y el ojo tiene esa mirada de resignada angustia que — más fuerte, hasta el punto de llegar a no poderse mirar sin que produzca un profundo sufrimiento — veré en el Calvario y ante el Sepulcro. 

Sus padres la bendicen y la besan. Una, dos, diez veces. No se sacian de besarla... Isabel llora en silencio. Zacarías, aunque quiera no dar muestras de ello, está también conmovido. 

Salen. María entre su padre y su madre, como antes; delante, Zacarías y su mujer... 

Ahora están dentro del recinto del Templo. 

- Voy a ver al Sumo Sacerdote. Vosotros subid hasta la Gran Terraza. 

Atraviesan tres atrios y tres patios superpuestos... Ya están al pie del vasto cubo de mármol coronado de oro. Cada una de las cúpulas, convexas como una media naranja enorme, resplandece bajo el sol, que cae a plomo, ahora que es aproximadamente mediodía, en el amplio patio que rodea a la solemne edificación, y llena el vasto espacio abierto y la amplia escalinata que conduce al Templo. Sólo el pórtico que hay frente a la escalinata, a lo largo de la fachada, está en sombra, y la puerta, altísima, de bronce y oro, con tanta luz, aparece aún más oscura y solemne. 

Por el intenso sol, María parece aún más de nieve. Ahí está, al pie de la escalinata, entre sus padres. ¡Cómo debe latirles el corazón a los tres! Isabel está al lado de Ana, pero un poco retrasada, como medio paso. 

Un sonido de trompetas argentinas y la puerta gira sobre los goznes, los cuales, al moverse sobre las esferas de bronce, parecen producir sonido de cítara. Se ve el interior, con sus lámparas en el fondo. Un cortejo viene desde allí hacia el exterior. Es un pomposo cortejo acompañado de sonidos de trompetas argénteas, nubes de incienso y luces. 

Ya ha llegado al umbral; delante, el que debe ser el Sumo Sacerdote: un anciano solemne, vestido de lino finísimo, cubierto con una túnica más corta, también de lino, y sobre ésta una especie de casulla, recuerda en parte a la casulla y en parte al paramento de los Diáconos, multicolor: púrpura y oro, violáceo y blanco se alternan en ella y brillan como gemas al sol; y dos piedras preciosas resplandecen encima de los hombros más vivamente aún (quizás son hebillas con un engaste precioso); al pecho lleva una ancha placa resplandeciente de gemas sujeta con una cadena de oro; y colgantes y adornos lucen en la parte de abajo de la túnica corta, y oro en la frente sobre la prenda que cubre su cabeza (una prenda que me recuerda a la de los sacerdotes ortodoxos, con su mitra en forma de cúpula en vez de en punta como la mitra católica). 

El solemne personaje avanza, solo, hasta el comienzo de la escalinata, bajo el oro del sol, que le hace todavía más espléndido. Los otros esperan, abiertos en forma de corona, fuera de la puerta, bajo el pórtico umbroso. A la izquierda hay un cándido grupo de niñas, con Ana, la profetisa, y otras maestras ancianas. 

El Sumo Sacerdote mira a la Pequeña y sonríe. ¡Debe parecerle bien pequeñita al pie de esa escalinata digna de un templo egipcio! Levanta los brazos al cielo para pronunciar una oración. Todos bajan la cabeza como anonadados ante la majestad sacerdotal en comunión con la Majestad eterna. 

Luego... una señal a María, y Ella se separa de su madre y de su padre y sube, sube como hechizada. Y sonríe, sonríe a la zona del Templo que está en penumbra, al lugar en que pende el preciado Velo... Ha llegado a lo alto de la escalinata, a los pies del Sumo Sacerdote, que le impone las manos sobre la cabeza. La víctima ha sido aceptada. ¿Alguna vez había tenido el Templo una hostia más pura? 

Luego se vuelve y, pasando la mano por el hombro de la Corderita sin mancha, como para conducirla al altar, la lleva a la puerta del Templo y, antes de hacerla pasar pregunta: 

María de David, ¿conoces tu voto? 

Ante el "sí" argentino que le responde, él grita: 

Entra, entonces. Camina en mi presencia y sé perfecta. 

Y María entra y desaparece en la sombra, y el cortejo de las vírgenes y de las maestras, y luego de los levitas, la ocultan cada vez más, la separan... Ya no se la ve... 

La puerta se vuelve, girando sobre sus armoniosos goznes. Una abertura, cada vez más estrecha, permite todavía ver al cortejo, que se va adentrando hacia el Santo. Ahora es sólo una rendija. Ahora ya nada. Cerrada. 

Al último acorde de los sonoros goznes responde un sollozo de los dos ancianos y un grito único: " ¡María! ¡Hija!". Luego dos gemidos invocándose mutuamente: " ¡Ana, Joaquín!". Luego, como final: "Glorifiquemos al Señor, que la recibe en su Casa y la conduce por sus caminos". 

Y todo termina así. 

Dice Jesús: 
\emph{El Sumo Sacerdote había dicho: "Camina en mi presencia y sé perfecta". El Sumo Sacerdote no sabía que estaba hablándole a la Mujer que, en perfección, es sólo inferior a Dios. Mas hablaba en nombre de Dios y, por tanto, su imperativo era sagrado. Siempre sagrado, pero especialmente a la Repleta de Sabiduría. María había merecido que la "Sabiduría viniera a su encuentro tomando la iniciativa de manifestarse a Ella", porque "desde el principio de su día Ella había velado a su puerta y, deseando instruirse, por amor, quiso ser pura para conseguir el amor perfecto y merecer tenerla como maestra". En su humildad, no sabía que la poseía antes de nacer y que la unión con la Sabiduría no era sino un continuar los divinos latidos del Paraíso. No podía imaginar esto. Y cuando, en el silencio del corazón, Dios le decía palabras sublimes, Ella, humildemente, pensaba que fueran pensamientos de orgullo, y elevando a Dios un corazón inocente suplicaba: "¡Piedad de tu sierva, Señor!". En verdad, la verdadera Sabia, la eterna Virgen, tuvo un solo pensamiento desde el alba de su día: "Dirigir a Dios su corazón desde los albores de la vida y velar para el Señor, orando ante el Altísimo", pidiendo perdón por la debilidad de su corazón, como su humildad le sugería creer, sin saber que estaba anticipando la solicitud de perdón para los pecadores que haría al pie de la Cruz junto con su Hijo moribundo. "Luego, cuando el gran Señor lo quiera, Ella será colmada del Espíritu de inteligencia", y entonces comprenderá su sublime misión. Por ahora no es más que una párvula que, en la paz sagrada del Templo, anuda, "reanuda", cada vez de forma más estrecha, sus coloquios, sus afectos, sus recuerdos, con Dios. Esto es para todos. Pero, para ti, pequeña María (se dirige aquí a María Valtorta), ¿no tiene ninguna cosa particular que decir tu Maestro? "Camina en mi presencia, sé por tanto perfecta". Modifico ligeramente la sagrada frase y te la doy por orden. Perfecta en el amor, perfecta en la generosidad, perfecta en el sufrir. Mira una vez más a la Madre. Y medita en eso que tantos ignoran, o quieren ignorar, porque el dolor es materia demasiado ingrata para su paladar y para su espíritu. El dolor. María lo tuvo desde las primeras horas de la vida. Ser perfecta como Ella era poseer también una perfecta sensibilidad. Por eso, el sacrificio debía serle más agudo; mas, por eso mismo, más meritorio. Quien posee pureza posee amor, quien posee amor posee sabiduría, quien posee sabiduría posee generosidad y heroísmo, porque sabe el porqué de por qué se sacrifica. ¡Arriba tu espíritu, aunque la cruz te doble, te rompa, te mate! Dios está contigo". }
 
\chapter*{La muerte de Joaquín y Ana fue dulce \\ \normalfont\normalsize\textit{después de una vida de sabia fidelidad a Dios en las pruebas.}}
\addcontentsline{toc}{chapter}{\normalfont\scshape{La muerte de Joaquín y Ana fue dulce}}
 
Dice Jesús: 
\emph{Como un rápido crepúsculo de invierno en que un viento de nieve acumule nubes en el cielo, la vida de mis abuelos conoció rápida la noche, una vez que su Sol se había quedado fijo resplandeciendo ante la sagrada Cortina del Templo. Pero, ¿acaso no fue dicho: "La Sabiduría inspira vida a sus hijos, toma bajo su protección a los que la buscan... Quien la ama ama la vida, y quien está en vela por ella gozará de su paz. Quien la posee heredará la vida... Quien la sirve rendirá obediencia al Santo, y a quien la ama Dios lo ama mucho... Si cree en ella la tendrá como herencia y le será como tal confirmada a su posteridad porque lo acompaña en la prueba. En primer lugar le elige, luego enviará sobre él temores, miedos y pruebas, le atormentará con el flagelo de su disciplina, hasta haberle probado en sus pensamientos y poder fiarse de él. Mas luego le dará estabilidad, volverá a él por recto camino y le alegrará. Le descubrirá sus arcanos, pondrá en él tesoros de ciencia y de inteligencia en la justicia"? Sí, todo esto fue dicho. Los libros sapienciales son aplicables a todos los hombres, que en ellos tienen un espejo de sus comportamientos y una guía. Mas dichosos aquellos que puedan ser reconocidos como amantes espirituales de la Sabiduría. Yo me circundé de una parentela mortal de sabios. Ana, Joaquín, José, Zacarías y, más aún, Isabel y luego el Bautista, ¿no son, acaso, verdaderos sabios? Y eso sin hablar de mi Madre, en la cual la Sabiduría había hecho morada. Desde la juventud hasta la tumba, la Sabiduría había inspirado a mis abuelos la manera de vivir de forma grata a Dios; y, como un toldo que protege de la violencia de los elementos, los había protegido del peligro de pecar. El santo temor de Dios es base del árbol de la sabiduría, que, a partir de aquél, se desarrolla impetuoso con todas sus ramas para alcanzar con su copa el amor tranquilo en su paz, el amor pacífico en su seguridad, el amor seguro en su fidelidad, el amor fiel en su intensidad, el amor total, generoso, activo de los santos. "Quien la ama ama la vida y recibirá en herencia la Vida" dice el Eclesiástico. Pues bien, esto se funde con mi: 'Aquel que pierda la vida por amor mío, la salvará". Porque no se habla de la pobre vida de esta tierra, sino de la eterna; no de las alegrías de una hora, sino de las inmortales. Joaquín y Ana la amaron en ese sentido. Y ella estuvo con ellos en las pruebas.¡Cuántas, vosotros, que, pensando que no sois completamente malvados, querríais no tener que llorar ni sufrir nunca! ¡Cuántas pruebas sufrieron estos dos justos que merecieron tener por hija a María! La persecución política que los arrojó de la tierra de David, empobreciéndolos excesivamente. La tristeza de ver caer en la nada los años sin que una flor les dijese: "Yo os continuaré". Y luego la congoja por haberla tenido a una edad en que ciertamente no la iban a ver hacerse mujer. Y, más tarde, el tener que arrancarse de su corazón esta flor para depositarla sobre el altar de Dios. Y el vivir en un silencio más oprimente aún que el primero, ahora que se habían acostumbrado al gorjeo de su tortolita, al rumor de sus pasitos, a las sonrisas, a los besos de su criatura; y esperar en el recuerdo la hora de Dios. Y más, y más todavía: enfermedades, calamidades por la intemperie, abusos de los poderosos... muchos golpes de ariete contra el débil castillo de su modesta prosperidad. Y no acaba aquí todo: el dolor de esa criatura lejana, que se quedaba sola y pobre, y que, a pesar de todas las atenciones y todos los sacrificios, no tendría sino un resto del bien paterno. ¿Y cómo podía encontrarlo, si durante años todavía quedaría yermo, cerrado, esperándola? Temores, miedos, pruebas y tentaciones. Y fidelidad, fidelidad, fidelidad, siempre, a Dios. La tentación más fuerte: no negarse el consuelo de su hija en torno a su vida ya declinante. Pero, los hijos son de Dios antes que de los padres. Todos los hijos pueden decir lo que Yo le dije a mi Madre: "¿No sabes que debo ocuparme de los intereses del Padre de los Cielos?". Y todas las madres y todos los padres deben aprender la actitud a guardar en estos casos, mirando a María y a José en el Templo, a Ana y a Joaquín en la casa de Nazaret, cada vez más vacía y triste, aunque, no obstante, en ella una cosa no disminuyese nunca, sino que, al contrario, crecía cada vez más: la santidad de dos corazones, la santidad de una unión matrimonial. ¿Qué luz le queda a Joaquín, enfermo; qué luz le queda a su dolorida esposa en las largas y silenciosas tardes propias de ancianos que se sienten morir? Los vestiditos, las primeras sandalitas, los pobres juguetitos de su criatura lejana, y los recuerdos, los recuerdos, los recuerdos. Y, con éstos, una paz que proviene del poder decir: "Sufro, pero he cumplido mi deber de amor hacia Dios". Pues bien, he aquí que se produce una alegría sobrehumana de celestial brillo, no conocida por los hijos de este mundo, y que no se opaca por el hecho de que un grave párpado descienda sobre dos ojos que mueren, sino que en la postrera hora resplandece más, e ilumina verdades que habían estado dentro durante toda la vida, cerradas como mariposas en su capullo, que daban señales de estar dentro de ellos sólo por unos suaves movimientos de ligeros destellos, mientras que ahora abren sus alas de sol mostrando las palabras que las decoran. Y la vida se apaga en el conocimiento de un futuro beato para ellos y para su estirpe, bendiciendo a su Dios. Así fue la muerte de mis abuelos, como era justo que fuera por su vida santa. Por la santidad merecieron ser los primeros depositarios de la Amada de Dios, y, sólo cuando un Sol mayor se mostró en su vital ocaso, ellos intuyeron la gracia que Dios les había concedido. Por la santidad que tuvieron, Ana no padeció la tortura propia de la puérpera, sino que experimentó el éxtasis de quien llevó a la Sin Culpa. No sufrieron la angustia de la agonía, sino que fueron languidez que se apaga, como dulcemente se apaga una estrella cuando el Sol sale con la aurora. Y, si bien no experimentaron el consuelo de tenerme como Encamada Sabiduría, como me tuvo José, Yo, no obstante, estaba allí, invisible Presencia que decía sublimes palabras, inclinado hacia su almohada para adormecerlos en la paz en espera del triunfo. Hay quien dice: "¿Por qué no debieron sufrir al generar y al morir, puesto que eran hijos de Adán?". A éste le respondo: "Si el Bautista, hijo de Adán y concebido con la culpa de origen, fue presantificado en el seno de su madre porque Yo le visité, ¿ninguna gracia va a haber recibido la madre santa de la Santa sin Mancha, de la Preservada por Dios que llevó consigo a Dios en su espíritu casi divino y en el corazón embrional, y que no se separó nunca de Él desde que fue pensada por el Padre, desde que fue concebida en un seno, hasta que retornó a poseer a Dios plenamente en el Cielo para una eternidad gloriosa?". A éste le respondo: "La recta conciencia proporciona una muerte serena y las oraciones de los santos os obtienen tal muerte". Joaquín y Ana tenían toda una vida de recta conciencia a sus espaldas, y ésta se alzaba como sosegado panorama y los guió hasta el Cielo; y tenían a la Santa en oración por ellos, sus padres lejanos, ante el Tabernáculo de Dios. Dios, Bien supremo, era antes que ellos, pero Ella amaba a sus padres, como querían la ley y el sentimiento, con un amor sobrenaturalmente perfecto. }

\chapter*{Cántico de María. \\ \normalfont\normalsize\textit{Ella recordaba cuanto su espíritu había visto en Dios.}}
\addcontentsline{toc}{chapter}{\normalfont\scshape{Cántico de María.}}
 
Hasta ayer por la tarde, viernes, no se me ha iluminado la mente para ver. Y he visto solamente esto. He visto a una María muy joven, una María de como mucho doce años, cuyo rostro no presenta ya esas redondeces propias de la infancia, sino que devela los futuros contornos de la mujer en el perfil oval que ya se va alargando. Por lo que respecta al pelo, ya no es aquel que caía suelto sobre el cuello con sus ligeros rizos, sino que está recogido en dos gruesas trenzas de un oro palidísimo, de lo claro que es el pelo, parece como si estuviera mezclado con plata, que siguiendo los hombros bajan hasta las caderas. El rostro aparece más pensativo, más maduro, aunque siga siendo el rostro de una niña, de una hermosa y pura niña que, toda vestida de blanco, cose en una habitacioncita muy pequeña y también toda blanca, por cuya ventana abierta de par en par se ve el edificio imponente y central del Templo, y toda la bajada de las escalinatas de los patios, de los pórticos, y, al otro lado de la muralla, la ciudad con sus calles y casas y jardines, y, al fondo, la cima protuberante y verde del Monte de los Olivos. 

Cose y canta en voz baja. No sé si se trata de un canto sacro. Dice:

\begin{verse}
Como una estrella dentro de un agua clara\\
me resplandece una luz en el fondo del corazón.\\
Desde la infancia, de mí no se separa\\
y dulcemente me guía con amor.

En lo más hondo del corazón hay un canto. \\
¿De dónde venir podrá? \\
¡Oh, hombre, tú lo ignoras! \\
De donde descansa el Santo.

Yo miro mi estrella clara\\
y no quiero cosa que no sea,\\
aunque fuera la más dulce y estimada,\\
esta dulce luz que es toda mía. 

Me trajiste de los altos Cielos, \\
Estrella, al interior de un seno de madre.\\
Ahora vives en mí; mas allende los velos\\
te veo, rostro glorioso del Padre.

¿Cuándo a tu sierva darás el honor\\
de ser humilde esclava del Salvador? \\
Manda, del Cielo mándanos al Mesías.\\ 
Acepta, Padre Santo, la ofrenda de María.
\end{verse} 

María calla, sonríe y suspira, y luego se pone de rodillas en oración. Su carita es toda una luz. Alta, elevada hacia el azul terso de un bonito cielo estival, parece como si aspirase toda su luminosidad y la irradiara. O, más exactamente, parece como si de su interior un escondido Sol irradiase sus luces y encendiera la nieve apenas rosada de la carne de María y se vertiera, llegando a las cosas y al Sol que resplandece sobre la tierra, bendiciendo y prometiendo abundancia de bienes. 

Estando María a punto de ponerse en pie después de su amorosa oración, permaneciendo en su rostro una luminosidad de éxtasis, entra la anciana Ana de Fanuel y se detiene atónita, o, por lo menos, admirada del acto y del aspecto de María. 

La llama: "María", y la Niña se vuelve con una sonrisa, distinta pero como siempre muy bonita, y saluda diciendo: "Ana, 

paz a ti". 

- ¿Estabas orando? ¿No te es suficiente nunca la oración? 

La oración me sería suficiente. Pero yo hablo con Dios. Ana, tú no puedes saber qué cercano a mí lo siento; más que cercano, en el corazón. Dios me perdone tal soberbia. Es que yo no me siento sola. ¿Ves? Allí, en aquella casa de oro y de nieve, detrás de la doble Cortina, está el Santo de los Santos, y jamás ojo alguno, aparte del del Sumo Sacerdote, puede detenerse en el Propiciatorio, sobre el que descansa la gloria del Señor. Mas yo no tengo necesidad de mirar con toda el alma veneradora a ese doble Velo bordado, que palpita con las ondas de los cantos virginales y de los levitas y que huele a preciosos inciensos, como para perforar su cohesión y ver así la luz irradiada por el Testimonio. ¡Pero sí que miro! No temas que no mire con ojo venerador como todo hijo de Israel. No temas que el orgullo me ciegue haciéndome pensar esto que ahora te digo. Yo miro, y no hay ningún humilde siervo en el pueblo de Dios que mire más humildemente la Casa de su Señor que como yo la miro, convencida como estoy de ser la más pequeña de todos. Pero, ¿qué es lo que veo? Un velo. ¿Qué pienso al otro lado del Velo? Un Tabernáculo. ¿Y en él? Mas si miro a mi corazón, he aquí que veo a Dios resplandecer en su gloria de amor y decirme: "Te amo", y yo le digo: "Te amo", y me deshago y me rehago con cada uno de los latidos del corazón en este beso recíproco... Estoy entre vosotras, mis queridas maestras y compañeras, pero un círculo de fuego me aísla de vosotras. Dentro de ese círculo, Dios y yo. Y os veo a través del Fuego de Dios y así os amo... mas no puedo amaros según la carne, como jamás podré amar a nadie según la carne, sino sólo a Este que me ama, y según el espíritu. Conozco mi destino. La Ley secular de Israel quiere de toda niña una esposa y de toda esposa una madre. Pero yo, no sin obedecer a la Ley, obedezco a la Voz que me dice: "Yo te quiero para mí", y permaneceré siempre virgen. ¿Cómo podré hacerlo? Esta dulce, invisible Presencia que está conmigo me ayudará, porque ella desea eso. Yo no temo. Ya no tengo ni padre ni madre... y sólo el Eterno sabe cómo en ese dolor se quemó cuanto yo tenía de humano. Ardió con dolor atroz. Ahora sólo tengo a Dios. A Él, por tanto, le presto obediencia ciegamente... Lo habría hecho incluso contra el padre y la madre, porque la Voz me enseña que quien quiere seguirla debe pasar por encima del padre y de la madre, amorosas patrullas de ronda en torno a los muros del corazón filial, al que quieren conducir a la alegría según sus caminos... y no saben que hay otros caminos de infinita alegría. Yo les habría dejado los vestidos y el manto, con tal de seguir la Voz que me dice: "¡Ven, dilecta mía, esposa mía!". Les habría dejado todo; y las perlas de las lágrimas — porque habría llorado por tener que desobedecer —, y los rubíes de mi sangre — que hasta a la muerte habría desafiado por seguir la Voz que llama — les habrían dicho que hay algo más grande que el amor de un padre y una madre, y más dulce: la Voz de Dios. Pero ahora su voluntad me ha dejado libre incluso de este lazo de piedad filial. Ya de por sí no habría habido lazo. Eran dos justos, y Dios, ciertamente, hablaba en ellos como me habla a mí. Habrían seguido la justicia y la verdad. Cuando pienso en ellos, pienso que están en la quietud de la espera entre los Patriarcas, y acelero con mi sacrificio la venida del Mesías para abrirles las puertas del Cielo. En la tierra yo me rijo, o sea, es Dios quien rige a su pobre sierva diciéndole sus preceptos, y yo los cumplo, porque cumplirlos es mi alegría. Cuando llegue la hora, le diré a mi esposo mi secreto... y él lo acogerá en su interior. 

Pero, María... ¿con qué palabras lo vas a persuadir? Tendrás en contra el amor de un hombre, la Ley y la vida. 

Tendré conmigo a Dios... Dios abrirá a la luz el corazón de mi esposo... la vida perderá sus aguijones de sentido para ser pura flor con perfume de caridad. La Ley... Ana, no me llames blasfema. Yo creo que la Ley pronto va a sufrir un cambio. Pensarás: "¿quién puede cambiarla, si es divina?". Sólo quien la puede mutar: Dios. El tiempo está más próximo de lo que pensáis, yo os lo digo. Leyendo a Daniel, una gran luz que venía del centro del corazón se me ha iluminado, y la mente ha comprendido el sentido de las arcanas palabras. Serán abreviadas las setenta semanas por las oraciones de los justos. ¿Será cambiado el número de los años? No. La profecía no miente; mas, la medida del tiempo profético no es el curso del Sol, sino el de la Luna, y por ello os digo: "Cercana está la hora que oirá el vagido del Nacido de una Virgen". ¡Oh, si esta Luz que me ama quisiera decirme — pues muchas cosas me dice — dónde está la mujer feliz que dará a luz el Hijo a Dios y el Mesías a su pueblo! Caminando descalza recorrería la tierra; ni frío y hielo, ni polvo y canícula, ni fieras y hambre me serían obstáculo para llegar a Ella y decirle: "Concédele a tu sierva y a la sierva de los siervos del Cristo vivir bajo tu techo. Haré girar la rueda del molino y la prensa; como esclava ponme en el molino; como pastora, a tu rebaño; o para lavar los pañalitos a tu Nacido; ponme en tus cocinas, en tus hornos... donde tú quieras, pero recíbeme. ¡Que yo lo pueda ver, que pueda oír su voz, recibir su mirada!". Y, si no me admitiese, yo viviría, mendiga, a su puerta, de limosnas y escarnios, al raso o bajo el sol intenso, con tal de oír la voz del Mesías niño y el eco de su risa, y luego verle pasar... y, quizás, un día recibiría de Él el óbolo de un pan... ¡Oh, aunque el hambre me desgarrara las entrañas y desfalleciera después de tanto ayuno, yo no me comería ese pan! Lo tendría como un saquito de perlas contra mi corazón y lo besaría para sentir el perfume de la mano del Cristo, y ya no tendría ni hambre ni frío, porque su contacto me proporcionaría éxtasis y calor, éxtasis y alimento... 

- ¡Tú deberías ser la Madre del Cristo, tú que le amas de esa forma! ¿Por eso es por lo que quieres permanecer virgen? 

- ¡Oh, no! Yo soy miseria y polvo. No oso levantar la mirada hacia la Gloria. Por eso es por lo que prefiero mirar dentro de mi corazón más que mirar al doble Velo, tras el cual sé que está la invisible Presencia de Yeohveh. Allí está el Dios terrible del Sinaí. Aquí, en mí, veo al Padre nuestro, veo un amoroso Rostro que me sonríe y bendice, porque soy pequeña como un pajarillo que el viento sujeta sin sentir su peso, y débil como tallito de muguete silvestre que sólo sabe florecer y perfumar, y no opone más resistencia al viento que la de su perfumada y pura dulzura. ¡Dios, mi viento de amor! No, no es por eso, sino porque al Nacido de Dios y de una Virgen, al Santo del Santísimo no le puede gustar sino lo que en el Cielo ha elegido como Madre y lo que en la tierra le habla del Padre celestial: la Pureza. Si la Ley meditara en esto, si los rabíes, que la han multiplicado con todas las sutilezas de su enseñanza, volviendo la mente a horizontes más altos, se sumergieran en lo sobrenatural, dejando de lado lo humano y la ganancia que pretenden olvidando el Fin supremo, deberían, sobre todo, volver su enseñanza a la Pureza, para que el Rey de Israel, cuando venga, la encuentre. Con el olivo del Pacífico, con las palmas del Triunfador, esparcid azucenas y azucenas y azucenas... ¡Cuánta Sangre tendrá que derramar para redimirnos el Salvador! ¡Cuánta! De los miles de heridas que Isaías vio en el Hombre de dolores, cae, cual rocío de un recipiente poroso, una lluvia de Sangre. ¡Que no caiga en el lugar de la profanación y la blasfemia esta Sangre divina, sino en copas de fragante pureza que la acojan y recojan, para luego esparcirla sobre los enfermos del espíritu, sobre los leprosos del alma, sobre los muertos a Dios! ¡Dad azucenas, azucenas dad para enjugar, con la cándida vestidura de los pétalos puros, los sudores y las lágrimas del Cristo! ¡Dad azucenas, azucenas dad para el ardor de su fiebre de Mártir! ¡Oh, ¿dónde estará esa Azucena que te lleva dentro; dónde, la que aplacará la quemazón que padeces; dónde, la que se pondrá roja con tu Sangre y morirá por el dolor de verte morir; dónde, la que llorará ante tu Cuerpo desangrado?! ¡Oh, Cristo, Cristo, suspiro mío!... 

María queda en silencio, llorando y abatida. 

Ana está un rato en silencio. Luego, con su voz blanca de anciana conmovida, dice: 

- ¿Tienes algo más que enseñarme, María? 

María se estremece. Debe haber creído, en su humildad, que su maestra la haya reprendido y dice: - ¡Perdón! Tú eres maestra, yo soy una pobre nada. Es que esta Voz me sube del corazón. Yo la tengo bien vigilada, para no hablar; pero, cual río que por el ímpetu de la ola rompe las presas, ahora me ha prendido y se ha desbordado. No tengas en cuenta mis palabras y mortifica mi presunción. Las arcanas palabras deberían estar en el arca secreta del corazón al que Dios, en su bondad, favorece. Lo sé. Pero, tan dulce es esta invisible Presencia, que me embriaga... ¡Ana, perdona a tu pequeña sierva! 

Ana la estrecha contra sí, y todo el viejo rostro rugoso tiembla y brilla de llanto. Las lágrimas se insinúan entre las arrugas como agua por terreno accidentado que se transforma en un trémulo regatillo. No obstante, la anciana maestra no suscita risa, sino que, al contrario, su llanto promueve la más alta veneración. 

María está entre sus brazos, su carita contra el pecho de la anciana maestra, y todo termina así. 

Dice Jesús: 
\emph{María tenía el recuerdo de Dios. Soñaba con Dios. Creía soñar. No hacía sino ver de nuevo cuanto su espíritu había visto en el fulgor del Cielo de Dios, en el instante en que había sido creada para ser unida a la carne concebida en la tierra. Condividía con Dios, si bien de forma mucho menor, por exigencia de justicia, una de las propiedades de Dios: la de recordar, ver y prever, por el atributo de una inteligencia no lesionada por la Culpa, y, por tanto, poderosa y perfecta. El hombre ha sido creado a imagen y semejanza de Dios. Una de las semejanzas está en la posibilidad, para el espíritu, de recordar, ver y prever. Esto explica la facultad de leer el futuro, facultad que viene, muchas veces y directamente, por voluntad divina, otras por el recuerdo, que se alza, como Sol en una mañana, iluminando un cierto punto del horizonte de los siglos precedentemente visto desde el seno de Dios. Son misterios demasiado altos como para que podáis comprenderlos plenamente. Eso sí, reflexionad. ¿Esa Inteligencia suprema, ese Pensamiento que lo sabe todo, esa Vista que lo ve todo, que os crea con un movimiento de su voluntad y con el hálito de su amor infinito, haciéndoos hijos suyos por origen e hijos suyos por destino, podrá daros algo que sea distinto de Él? Os lo da en proporción infinitesimal, porque la criatura no podría contener al Creador, mas esa parte es, en su infinitesimal, perfecta y completa. ¡Cuán grande el tesoro de inteligencia que dio Dios al hombre, a Adán! La culpa lo ha menoscabado, mas mi Sacrificio lo reintegra y os abre los fulgores de la Inteligencia, sus ríos, su ciencia. ¡Oh, sublimidad de la mente humana unida por la Gracia a Dios, copartícipe de la capacidad de Dios de conocer!.. De la mente humana unida por la Gracia a Dios. No hay otro modo; que lo tengan presente los que anhelan conocer secretos ultrahumanos. Toda cognición que no venga de alma en gracia — y no está en gracia aquel que se manifiesta contrario a la Ley divina, cuyos preceptos son muy claros — sólo puede venir de Satanás, y difícilmente corresponde a verdad por lo que se refiere a cuestiones humanas, y nunca responde a verdad por lo que respecta a lo sobrehumano, porque el Demonio es padre de la mentira y a quien arrastra consigo lo lleva por el sendero de la mentira. No existe ningún otro método para conocer la verdad, sino el que viene de Dios. Y Dios habla y dice o hace recordar, del mismo modo como un padre a un hijo le hace recordar la casa paterna y dice: "¿Te acuerdas cuando conmigo hacías esto, veías aquello, oías aquello otro? ¿Te acuerdas cuando yo te despedía con un beso? ¿Te acuerdas cuando me viste por primera vez, cuando viste el fulgurante sol de mi rostro en tu alma virgen, instantes antes creada y aún exenta — puesto que acababa de salir de mí — de la debilidad que después te consumiera? ¿Te acuerdas de cuando comprendiste en un latido de amor lo que es el Amor y cuál es el misterio de nuestro Ser y Proceder?". Y cuando la capacidad limitada del hombre en gracia no llega a comprender, entonces el Espíritu de ciencia habla y enseña. Pero para poseer al Espíritu es necesaria la Gracia. Y para poseer la Verdad y la Ciencia es necesaria la Gracia. Y para tener consigo al Padre es necesaria la Gracia, Tienda en que las tres Personas hacen morada, Propiciatorio en que reside el Eterno y habla, no desde dentro de la nube, sino mostrando su Rostro al hijo fiel. Los santos tienen el recuerdo de Dios, de las palabras oídas en la Mente creadora y resucitadas por la Bondad en su corazón para elevarlos como águilas en la contemplación de la Verdad, en el conocimiento del Tiempo. María era la Llena de Gracia. Toda la Gracia Una y Trina estaba en Ella. Toda la Gracia Una y Trina la preparaba como esposa para la boda, como tálamo para la prole, como divina para su maternidad y para su misión. Ella es la que cierra el ciclo de las profetisas del Antiguo Testamento y abre el de los "portavoces de Dios" en el Nuevo Testamento. Verdadera Arca de la Palabra de Dios, mirando en su interior eternamente inviolado, descubría, trazadas por el dedo de Dios sobre su corazón inmaculado, las palabras de ciencia eterna, y recordaba, como todos los santos, haberlas oído ya al ser generada con su espíritu inmortal por Dios Padre, creador de todo lo que tiene vida. Y, si no recordaba todo de su futura misión, era porque en toda perfección humana Dios deja algunas lagunas, por ley de una divina prudencia que es bondad y mérito para y hacia la criatura. María, segunda Eva, tuvo que conquistarse su parte de mérito de ser la Madre del Cristo; con una fiel, buena voluntad. Esto quiso también Dios en su Cristo para hacerle Redentor. El espíritu de María estaba en el Cielo. Su parte moral y su carne estaban en la tierra, y tenían que pisotear tierra y carne para llegar hasta el espíritu y unirlo al Espíritu en un abrazo fecundo.}

Nota mía. Todo el día de ayer había estado pensando que vería la noticia de la muerte de los padres, y, además — por qué, no lo sé —, dado por Zacarías. Igualmente pensaba, a mi manera, cómo trataría Jesús el punto del "recuerdo de Dios por parte de los santos". Esta mañana, cuando empezó la visión, he dicho: "Eso es, ahora le dirán que es huérfana". Y ya sentía encogido mi corazón porque... se trataba de oír y ver la misma tristeza mía de estos días. Sin embargo, no hay nada de cuanto había pensado ver y oír; pero es que ni una palabra por equivocación. Esto me consuela porque me dice que verdaderamente no hay nada mío, ni siquiera una honesta sugestión respecto a un determinado punto. Todo viene realmente de otra fuente. Mi continuo miedo cesa... hasta la próxima vez, porque este miedo de ser engañada y de engañar me acompañará siempre. 
 
\chapter*{María confía su voto al Sumo Sacerdote. \\ \normalfont\normalsize\textit{3 de septiembre de 1944.}}
\addcontentsline{toc}{chapter}{\normalfont\scshape{María confía su voto al Sumo Sacerdote.}}
 
¡Qué noche de infierno! Verdaderamente parecía como si los demonios hubieran salido a la Tierra a pasear. Cañonazos, truenos, relámpagos, peligro, miedo, sufrimiento por estar en una cama que no es mía... (estaban en la Segunda Guerra Mundial y la guerra se desarrollaba cerca de su pueblo) Y, en medio, como una flor toda blanca y suave entre fogonazos y angustias, la presencia de María, un poco más adulta que en la visión de ayer, pero todavía jovencita, con sus trenzas rubias sobre los hombros, su vestido blanco y su mansa, recogida sonrisa, una sonrisa interior, vuelta al misterio glorioso que lleva dentro de su corazón. Paso la noche comparando su aspecto dulce con la crueldad que hay en el mundo, y evocando sus palabras de ayer por la mañana, canto de caridad viva, en contraste con el odio que hace que los hombres se despedacen... 

Pues bien, esta mañana, de nuevo en el silencio de mi habitación, presencio esta escena. 

María sigue estando en el Templo, y ahora sale del Templo propiamente dicho entre otras vírgenes. 

Debe haberse llevado a cabo alguna ceremonia, pues un olor a inciensos se esparce por la atmósfera toda roja de un hermoso ocaso, que yo diría que es de otoño avanzado, porque un cielo ya dulcemente cansado, como lo está en un octubre sereno, se arquea sobre los jardines de Jerusalén, en los que el amarillo ocre de las hojas que pronto caerán dispone manchas dorado- rojizas entre el verde- plata de los olivos. 

La comitiva — mejor sería llamarla enjambre — cándida de las vírgenes cruza el patio posterior, sube la escalinata, atraviesa un pórtico, entra en otro patio menos suntuoso, cuadrado, que como aperturas no tiene sino la que sirve para acceder a él. Debe ser el patio dedicado a acoger las pequeñas moradas de las vírgenes reservadas para el Templo, porque cada una de las jovencitas se dirige a su celda como una palomita a su nido, y asemejan verdaderamente a una bandada de palomas separándose tras haberlas tenido agrupadas. Muchas — podría decir todas — hablan entre sí antes de dejarse, en voz baja pero al mismo tiempo festiva. María guarda silencio. Sólo las saluda con afecto antes de separarse; luego se dirige a su habitacioncita, que está en una de las esquinas a la derecha. 

Se llega hasta Ella una maestra anciana, aunque no tanto como Ana de Fanuel. 

María, el Sumo Sacerdote te espera. 

María la mira con cierto asombro, pero no hace preguntas. Se limita a responder: 

Voy inmediatamente. 

No sé si la espaciosa sala en que entra es de la casa del Sacerdote o forma parte de los aposentos de las mujeres que están dedicadas al Templo. Sé que es vasta y luminosa, puesta con gusto, y que en ella, además del Sumo Sacerdote (que con las vestiduras que lleva aparece muy elegante), están Zacarías y Ana de Fanuel. 

María se inclina profundamente en el umbral de la puerta y no entra hasta que el Sumo Sacerdote no le dice: "Pasa, María. No temas". Ella se yergue y alza la cara, y entra lentamente, no por desgana, sino por un algo de involuntaria solemnidad que la hace parecer más mujer. 

Ana le sonríe para animarla y Zacarías la saluda con un: "Paz a ti, prima". 

El Pontífice la observa atentamente. Luego le dice a Zacarías: 

Es patente en Ella la estirpe de David y Aarón. 

Hija, conozco tu gracia y tu bondad. Sé que cada día has ido creciendo en ciencia y gracia ante los ojos de Dios y de los hombres. Sé que la voz de Dios susurra a tu corazón las más dulces palabras. Sé que eres la Flor del Templo de Dios y que un tercer querubín está ante el Testimonio desde que tú llegaste; y quisiera que tu perfume siguiera subiendo con los inciensos cada nuevo día. Pero, la Ley se expresa en modo distinto. Tú ya no eres una niña, sino una mujer. Y en Israel todas las mujeres deben casarse para ofrecer a su hijo varón al Señor. Tú seguirás el precepto de la Ley. No temas, no te ruborices. No me olvido de tu ofrecimiento. De hecho ya te la tutela la Ley al ordenar que todo hombre reciba de su estirpe la mujer; pero, aunque no fuera así, yo lo haría, para no corromper tu magnífica sangre. ¿No conoces, María, a alguno de tu estirpe que pudiera ser tu marido? 

María levanta su cara, todo roja de pudor, y, con un primer titileo de llanto, que resplandece orlando los párpados, y con voz temblorosa, responde: 

Ninguno. 

No puede conocer a ninguno, puesto que entró aquí siendo niña, y la estirpe de David está demasiado castigada y demasiado dispersa como para que las distintas ramas puedan reunirse y formar con sus frondas la copa de la palma regia - dice Zacarías. 

Entonces le dejaremos a Dios que elija. 

Las lágrimas, contenidas hasta ese momento, brotan y descienden hasta la trémula boca. María dirige una mirada suplicante a su maestra. 

Ana la socorre diciendo: 

María se ha prometido al Señor para gloria de Dios y para la salvación de Israel. Era sólo una niña que apenas sabía pronunciar y ya se había ligado con un voto. 

Se debe a esto entonces tu llanto. No es por resistencia a la Ley. 

Es por esto... no por otro motivo. Yo te obedezco, Sacerdote de Dios. 

Esto confirma cuanto de ti me ha sido referido siempre. ¿Desde hace cuántos años eres virgen consagrada? 

Yo creo que desde siempre. Antes de venir a este Templo ya me había ofrecido al Señor. 

Pero, ¿no eres tú la Niña que vino hace doce inviernos a pedirme entrar? 

Sí. 

Y ¿cómo, entonces, puedes decir que ya eras de Dios? 

Si miro hacia atrás yo me veo ya consagrada... No tengo memoria de la hora en que nací, ni de cómo empecé a amar a mi madre y a decirle a mi padre: "¡Oh, padre, yo soy tu hija!"... Pero sí recuerdo, aunque no a partir de cuándo, haber dado mi corazón a Dios. Quizás fue con el primer beso que supe dar, con la primera palabra que supe pronunciar, con el primer paso que supe dar... Sí, eso es, creo que mi primer recuerdo de amor lo encuentro junto a mi primer paso seguro... Mi casa... Mi casa tenía un jardín lleno de flores... un huerto de árboles frutales y campos cultivados... y había un manantial allí, en el fondo, al pie del monte, que manaba de una roca ahuecada en forma de gruta... estaba llena de hierbas largas y finas que pendían de todas partes asemejando cascaditas verdes, y parecía como si llorasen porque las livianas hojitas, que en su espesura parecían un bordado, tenían, todas, una gotita de agua que al caer sonaba como un cascabelito diminuto. Y también cantaba el manantial. Y había aves en los olivos y en los manzanos de la pendiente que estaba hacia arriba del manantial, y palomas blancas venían a lavarse en la balsa límpida de la fuente... Ya no me acordaba de todo esto porque había puesto todo mi corazón en Dios y, aparte de mi padre y de mi madre, a quienes amé en vida y después de muertos, todas las demás cosas de la tierra habían desaparecido de mi corazón... Pero tú me haces pensar en ello, Sacerdote... Debo buscar el momento en que me di a Dios... y vuelven a la mente las cosas de los primeros años... Me gustaba esa gruta porque en ella oía una Voz, más dulce que el canto del agua y de los pájaros, que me decía: "Ven, dilecta mía". Me gustaban esas hierbas diamantinas con sus gotas sonoras porque en ellas veía el signo de mi Señor y me perdía diciéndome: "¿Ves qué grande es tu Dios, alma mía! El mismo que ha hecho los cedros del Líbano para el aquilón ha hecho estas hojitas que ceden bajo el peso de un mosquito para alegría de tu ojo y para que protejan tu piececito". Me gustaba aquel silencio de cosas puras: el viento leve, el agua de plata, la pulcritud de las palomas... me gustaba esa paz que amparaba la gruta, descendiendo de los manzanos y de los olivos, ya enteramente en flor, ya repletos de frutos... Y, no sé... me parecía que la Voz me dijese a mí, justamente a mí: "Ven, tú, aceituna especiosa; ven, tú, dulce pomo; ven, tú, fuente sigilada; ven, tú, paloma mía"... Dulce era el amor de mi padre y de mi madre... dulce su voz cuando me llamaba... ¡Ah, pero ésta, ésta...! ¡Oh!, yo creo que así la oiría en el Paraíso Terrenal aquella que fue culpable, y no sé cómo pudo preferir un silbido a esta Voz de amor, cómo pudo apetecer otro conocimiento que no fuera Dios... Aún con el sabor a leche materna en los labios, pero con el corazón ebrio de miel celestial, yo dije entonces: "Sí, voy. Tuya. Y mi carne no tendrá otro señor aparte de Ti, Señor, de la misma forma que mi espíritu no tiene otro amor"... Y al decir esto me parecía estar repitiendo cosas ya dichas precedentemente y estar cumpliendo un rito que ya había sido cumplido, y no me resultaba extraño el Esposo elegido, puesto que yo ya conocía su ardor y mi vista se había formado bajo su luz y mi capacidad de amar había hallado cumplimiento entre sus brazos. ¿Cuándo?.. No lo sé. Yo diría que más allá de la vida, porque tengo la impresión de que siempre ha sido mío, y de que yo siempre he sido suya, y de que yo existo porque Él me ha querido para sí, para alegría de su Espíritu y del mío... 'Ahora obedezco, Sacerdote; pero, dime tú cómo debo actuar... No tengo ni padre ni madre. Sé tú mi guía. - Dios te dará el esposo, y será santo, dado que en Dios te abandonas. Lo que harás será manifestarle tu voto. 

- ¿Y aceptará? 

- Espero que sí. Ora, hija, para que él pueda comprender tu corazón. Ahora puedes marcharte. Que Dios te acompañe siempre. 

María se retira con Ana y Zacarías se queda con el Pontífice. La visión cesa aquí. 
 
\chapter*{José designado para esposo de la Virgen.}
\addcontentsline{toc}{chapter}{\normalfont\scshape{José designado para esposo de la Virgen.}}
 
Veo una rica sala, con un suelo bonito, cortinas, alfombras y muebles taraceados. Debe formar parte del Templo todavía. Se deduce que hay sacerdotes (entre los cuales Zacarías) y muchos hombres de las más diversas edades, o sea, de los veinte a los cincuenta años aproximadamente. 

Están hablando unos con otros, bajo pero animadamente. Se los ve inquietos por algo que desconozco. Todos están vestidos de fiesta, con vestidos nuevos o, al menos, recién lavados, como si estuvieran ataviados para una celebración. Muchos se han quitado el paño con que se cubren la cabeza, otros todavía lo tienen puesto, especialmente los ancianos, mientras que los jóvenes muestran sus cabezas descubiertas: unas rubio- oscuras, otras moreno- oscuras, algunas negrísimas, una — sólo ella — rojo- cobre. Las cabelleras son generalmente cortas, pero algunas de ellas llegan hasta los hombros. No deben conocerse todos entre sí porque se están observando con curiosidad. Pero parecen relacionados pues se ve que los apremia un pensamiento común. 

En una de las esquinas veo a José. Está hablando con un anciano de aspecto robusto y vigoroso. José tendrá unos treinta años. Es un hombre apuesto; pelo corto, más bien rizado, de un castaño oscuro como el de la barba y el bigote, que velan un mentón bien conformado y suben hacia las mejillas moreno- rojizas, no aceitunadas como en el caso de otras personas morenas; tiene ojos oscuros, buenos y profundos, muy serios, incluso yo diría que un poco tristes. Sin embargo, cuando sonríe — como está haciendo en este momento —aparecen alegres y juveniles. Está vestido todo de marrón claro, de forma muy simple pero muy ordenada. 

Entra un grupo de jóvenes levitas. Se disponen entre la puerta y una mesa larga y estrecha que está cerca de la pared en cuyo centro se encuentra la puerta, la cual queda abierta de par en par; sólo una cortina tensa, que pende hasta unos veinte centímetros del suelo, sigue cubriendo el vano. 

La curiosidad se acentúa. Y más aún cuando una mano separa la cortina para dejar paso a un levita que lleva en los brazos un haz de ramas secas sobre el cual ha sido depositada delicadamente una ramilla florecida, una ligera espuma de pétalos blancos que apenas muestran un rosáceo esfumado que desde el centro se irradia, atenuándose cada vez más, hasta el extremo de los livianos pétalos. El levita deposita el haz de ramas encima de la mesa con exquisito cuidado para no lesionar el milagro de esa rama en flor en medio de tanta hojarasca. 

Un murmullo recorre la sala. Los cuellos se alargan, las miradas se hacen más penetrantes, como para poder ver. Zacarías, con los sacerdotes, también trata de ver, estando como está más cerca de la mesa, pero no ve nada. 

José, desde su esquina, apenas dirige los ojos hacia el haz de ramas, y, cuando su interlocutor le dice algo, él hace un gesto denegatorio como de quien dice: "¡Imposible!", y sonríe. 

Un toque de trompeta desde el otro lado de la cortina. Todos guardan silencio y se disponen en perfecto orden mirando hacia la puerta, ahora enteramente abierta, dado que a la cortina la hacen deslizarse sobre sus anillos. Rodeado de otros ancianos, entra el Sumo Pontífice. Todos se postran. El Pontífice se acerca a la mesa y, en pie, comienza a hablar: 

Hombres de la estirpe de David, que habéis convenido en este lugar por convocatoria mía, escuchad. El Señor ha hablado, ¡gloria a Él! De su Gloria un rayo ha descendido y, como sol de primavera, ha dado vida a una rama seca, y ésta ha florecido milagrosamente cuando ninguna rama de la tierra hoy está en flor, hoy, último día de las Luminarias, cuando aún no se ha derretido la nieve caída sobre las alturas de Judá y es lo único cándido que hay entre Sión y Betania. Dios ha hablado haciéndose padre y tutor de la Virgen de David, que no tiene tutor alguno aparte de Dios. Santa doncella, gloria del Templo y de la estirpe, ha merecido la palabra de Dios para conocer el nombre del esposo grato al Eterno. ¡Muy justo debe ser para haber sido elegido por el Señor para tutelar a su amada Virgen! Por ello nuestro dolor de perderla se aplaca, y cesa toda preocupación acerca de su destino como esposa. Y a aquel que ha sido señalado por Dios le confiamos, plenamente seguros, la Virgen que posee la bendición de Dios y la nuestra. El nombre del prometido es José de Jacob, betlemita, de la tribu de David, carpintero en Nazaret de Galilea. José, acércate; el Sumo Sacerdote te lo ordena. 

Gran murmullo. Cabezas que se vuelven, ojos y manos que señalan, expresiones de desilusión y expresiones de alivio. Alguno, especialmente entre los viejos, debe haberse sentido contento de no haber sido destinado para ello. 

José, muy colorado y visiblemente turbado, se abre paso. Ya está ante la mesa, frente al Pontífice, al cual ha saludado con reverencia. 

Venid todos y mirad el nombre grabado en la rama. Coja cada uno su ramilla, para asegurarse de que no hay trampa. 

Los hombres obedecen. Miran la ramilla que delicadamente tiene el Sumo Sacerdote; cada uno coge la suya: unos la rompen, otros la guardan. Todos miran a José: hay quien mira y calla, otros lo felicitan. El anciano con el que antes estaba hablando dice: 

- ¿No te lo había dicho, José? ¡Quien menos se siente seguro es el que vence la partida! Ya han pasado todos. 

El Sumo Sacerdote da a José la ramilla florecida, y, poniéndole la mano en el hombro, le dice: 

No es rica, y tú lo sabes, la esposa que Dios te dona, pero posee todas las virtudes. Hazte cada día más digno de Ella. En Israel no hay flor alguna tan linda y pura como Ella. Salid todos ahora. Que se quede José; y tú, Zacarías, pariente, trae a la prometida. 

Salen todos, excepto el Sumo Sacerdote y José. Vuelven a correr la cortina, cubriendo así la puerta. 

José está todo humilde junto al majestuoso Sacerdote. Una pausa silenciosa y éste le dice: María debe manifestarte un voto que ha hecho. Ayúdala en su timidez. Sé bueno con la mujer buena. 

Pondré mi virilidad a su servicio y ningún sacrificio por Ella me pesará. Estáte seguro de ello. 

 Entra María con Zacarías y Ana de Fanuel. 

Ven, María - dice el Pontífice - Éste es el esposo que Dios te ha destinado. Es José de Nazaret. Regresarás, por tanto, a tu ciudad. Ahora os voy a dejar. Que Dios os dé su bendición. Que el Señor os mire y os bendiga, os muestre su rostro y tenga siempre piedad de vosotros. Que vuelva a vosotros su rostro y os dé la paz. 

Zacarías sale escoltando al Pontífice. Ana felicita al prometido y luego también sale. 

Los dos prometidos están el uno enfrente del otro. María, toda colorada, tiene la cabeza agachada. José, también ruborizado, la observa buscando las primeras palabras que decir. 

Al fin las encuentra y una sonrisa ilumina su rostro. Dice: 

Te saludo, María. Te vi cuando eras una niña de pocos días... Yo era amigo de tu padre y tengo un sobrino de mi hermano Alfeo que era muy amigo de tu madre, su pequeño amigo, pues ahora no tiene más que dieciocho años, y, cuando tú todavía no habías nacido, siendo sólo un niñito, ya alegraba las tristezas de tu madre, que lo quería mucho. No nos conoces porque viniste aquí siendo muy pequeñita. Pero en Nazaret todos te quieren y piensan en ti, y hablan de la pequeña María de Joaquín, cuyo nacimiento fue un milagro del Señor, que hizo verdecer a la estéril... Yo me acuerdo de la tarde en que naciste... Todos la recordamos por el prodigio de una gran lluvia que salvó los campos, y de una violenta tormenta durante la cual los rayos no quebraron ni siquiera un tallito de brezo silvestre, tormenta que terminó con un arco iris de dimensiones y belleza no vistas nunca más. Y... ¿quién no recuerda la alegría de Joaquín? Te mecía enseñándote a los vecinos... Considerándote una flor venida del Cielo, te admiraba, y quería que todos te admirasen. ¡Oh, dichoso y anciano padre que murió hablando de su María, tan bonita y buena y que decía palabras llenas de gracia y de saber!.. ¡Tenía razón al admirarte y al decir que no existe ninguna más hermosa que tú! ¿Y tu madre? Llenaba con su canto el ángulo en que estaba tu casa. Parecía una alondra en primavera durante la gestación, y luego, cuando te amamantaba. Yo hice tu cuna, una cunita toda de entalladuras de rosas, porque así la quiso tu madre. Quizás esté todavía en la casa, ahora cerrada... Yo soy viejo, María. Cuando naciste, yo ya hacía mis primeros trabajos. Ya trabajaba... ¡Quién me iba a decir que te hubiera tenido por esposa! Quizás hubieran muerto más felices los tuyos, porque éramos amigos. Yo enterré a tu padre, llorándole con corazón sincero porque fue para mí maestro bueno durante la vida. 

María levanta muy despacio el rostro, sintiéndose cada vez más segura al oír cómo le habla José, y cuando alude a la cuna sonríe levemente, y cuando José habla de su padre le tiende una mano y dice: Gracias, José - Un "gracias" tímido y delicado. 

José toma entre sus cortas y fuertes manos de carpintero esa manita de jazmín, y la acaricia con un afecto que pretende inspirar cada vez más tranquilidad. Quizás espera otras palabras, pero María vuelve a guardar silencio. Entonces continúa hablando él: 

La casa, como sabes, está intacta, menos la parte que fue derribada por orden consular para transformar en calle el sendero para los convoyes de Roma. Pero las parcelas de cultivo, las que te han quedado — porque ya sabes... la enfermedad de tu padre consumió mucho tus haberes — están un poco abandonadas. Hace ya más de tres primaveras que los árboles y las cepas no conocen podadera de hortelano, y la tierra está sin cultivar y, por tanto, dura. Pero los árboles que te vieron cuando eras pequeñita están todavía allí, y, si me lo permites, yo me ocuparé inmediatamente de ellos. 

Gracias, José. Pero, ya trabajas... 

Trabajaré en tu huerto durante las primeras y las últimas horas del día. Ahora el tiempo de luz se va alargando cada vez más. Para la primavera quiero que todo esté en orden, para alegría tuya. Mira, ésta es una ramilla del almendro que está frente a la casa. Quise coger ésta... — se puede entrar por cualquier parte por el seto destruido, pero ahora le haré de nuevo sólido y fuerte —, quise coger ésta pensando que si yo hubiera sido el elegido — no lo esperaba porque soy consagrado nazareno, y he obedecido porque se trataba de una orden del Sacerdote, no por deseos de casamiento —, pensando, te decía, que el tener una flor de tu jardín te habría alegrado. Aquí la tienes, María. Con ella te doy mi corazón, que, como ella, hasta ahora, ha florecido sólo para el Señor, y que ahora florece para ti, esposa mía. 

María coge la ramita. Se la ve emocionada, y mira a José con una cara cada vez más segura y radiante. Se siente segura de él. Cuando él dice: "Soy consagrado nazareno", su rostro se muestra todo luminoso y encuentra fuerzas para decir: Yo también soy toda de Dios, José. No sé si el Sumo Sacerdote te lo ha dicho... 

Me ha dicho sólo que tú eres buena y pura y que debes manifestarme un voto tuyo, y que fuera bueno contigo. Habla, María. Tu José desea hacerte feliz en todos tus deseos. No te amo con la carne. ¡Te amo con mi espíritu, santa doncella que Dios me otorga! Debes ver en mí un padre y un hermano, además de un esposo. Ábrete a mí como con un padre, abandónate en mí como con un hermano. 

Ya desde la infancia me consagré al Señor. Sé que esto no se hace en Israel, pero yo sentía una Voz que me pedía mi virginidad en sacrificio de amor por la venida del Mesías. ¡Hace mucho tiempo que Israel lo espera!.. ¡No es demasiado el renunciar por esto a la alegría de ser madre! 

José la mira fijamente, como queriendo leer en su corazón, y luego coge las dos manitas que tienen todavía entre los 

dedos la ramita florecida, y dice: Pues yo también uniré mi sacrificio al tuyo, y amaremos tanto con nuestra castidad al Eterno, que Él dará antes a la Tierra al Salvador, permitiéndonos ver su Luz resplandecer en el mundo. Ven, María. Vamos ante su Casa y juremos amarnos como lo hacen los ángeles entre sí. 'Luego iré a Nazaret a prepararlo todo para ti, en tu casa si quieres ir a ella, en otra parte si así lo deseas. 

En mi casa... En el fondo había una gruta... ¿Todavía está? 

Está, pero ya no es tuya... Yo, de todas formas, te haré otra gruta donde estarás fresca y tranquila en las horas más calurosas. La haré lo más parecida posible. Y... dime, ¿quién quieres que esté contigo? 

Nadie. No tengo miedo. La madre de Alfeo, que siempre viene a verme, me hará compañía un poco durante el día, y por la noche prefiero estar sola. Ningún mal me puede suceder. 

Bueno, y ahora estoy yo... ¿Cuándo debo venir a recogerte? - Cuando tú quieras, José. 

Pues entonces vendré cuando la casa esté en orden. No pienso tocar nada. Quiero que encuentres todo como lo dejó tu madre, pero quiero también que esté llena de luz y bien limpia para acogerte sin tristeza. Ven, María. Vamos a decirle al Altísimo que le bendecimos. 

Y no veo nada más. Me queda, eso sí, en el corazón el sentido de seguridad que experimenta María... 
 
\chapter*{Esponsales de la Virgen y José \\ \normalfont\normalsize\textit{que fue instruido por la Sabiduría para ser custodio del Misterio.}}
\addcontentsline{toc}{chapter}{\normalfont\scshape{Esponsales de la Virgen y José}}
 
¡Qué guapa está María, rodeada de sus amigas y sus maestras jubilosas, vestida para los esponsales! Entre aquéllas está también Isabel. 

Va toda vestida de blanquísimo lino, tan seríceo y fino que parece de preciosa seda. Ciñe su grácil cintura un cinturón burilado de oro y plata, hecho todo de medallones unidos por delgadas cadenas — cada uno de los medallones es una filigrana engastada en la pesada plata bruñida por el tiempo — y, quizás porque es demasiado largo para Ella, que todavía es una delicada jovencita, le pende por delante con los tres últimos medallones, cayendo entre los pliegues del vestido amplísimo, que a su vez termina en una pequeña cola debido a su largura. Calzan sus piececitos unas sandalias de piel blanquísima con hebillas de plata. 

El vestido está sujeto al cuello por una cadenita de rosetas de oro y de filigrana de plata, que presentan en pequeño el mismo motivo del cinturón. La cadenita pasa a través de los anchos ojales del amplio cuello del vestido, acortándolo, por tanto, en frunces que forman como una pequeña puntilla. El cuello de María sobresale entre ese candor fruncido, con la gracia de un tierno tallo fajado con una gasa preciada, y así parece aún más grácil y blanco: un tallito de azucena culminado por su rostro de lirio, el cual, por la emoción, se ve aún más pálido y más puro: un rostro de hostia purísima. 

El pelo ya no le pende sobre los hombros. Está graciosamente dispuesto en nudo de trenzas. Unas valiosas horquillas de plata bruñida, con un trabajo de filigrana que cubre enteramente la parte superior del arco, sujetan las trenzas. El velo materno se apoya sobre ellas y desciende, formando lindos pliegues, por debajo del estrecho aro que lleva ajustado a la frente blanquísima; desciende hasta las caderas, porque María no tiene la altura de su madre y el velo le llega más abajo de ellas, mientras que a Ana le llegaba sólo a la cintura. 

No lleva anillos en las manos; en las muñecas, unas pulseras. Pero estas muñecas son tan delgadas, que las pesadas pulseras maternas se apoyan sobre el dorso de las manos y quizás, si sacudiera las manos, se caerían al suelo. 

Las compañeras la miran absortas desde todos los puntos, y con maravilla. Con sus preguntas y con sus frases de admiración crean un festivo trinar de gorrioncillos. 

- ¿Son de tu madre? 

Antiguas, ¿verdad? 

- ¡Qué bonito, Sara, ese cinturón! 

- ¿Y este velo, Susana? ¡Mira que finura! ¡Fíjate estas azucenas tejidas en el velo! 

¡Déjame ver las pulseras, María! ¿Eran de tu madre? 

Las llevó ella, pero son de la madre de Joaquín, mi padre. 

- ¡Oh, mira! Tienen el sigilo de Salomón entrelazado con sutiles ramitas de palma y olivo, y entre ellas hay azucenas y rosas. ¡Oh! ¿Quién habrá realizado un trabajo tan perfecto y minucioso? 

Son de la casa de David - explica María - Hace ya siglos que las llevan las mujeres de esta estirpe cuando se van a casar, y van pasando a las herederas. 

- ¡Ah, ya! Tú eres hija heredera... 

- ¿Te han traído todo de Nazaret? 

No. Cuando murió mi madre, mi prima se llevó a su casa el ajuar para conservarlo sin que se dañase. Ahora me lo ha traído. 

- ¿Dónde está? ¿Dónde está? Enséñanoslo a las amigas. 

María no sabe qué hacer... Quisiera ser amable, pero no querría remover todas las cosas, que están ordenadas en tres pesados baúles. 

Vienen en su ayuda las maestras: 

El novio está para llegar. No es el momento de crear confusión. Dejadla. Que la cansáis. Id a prepararos". 

El gárrulo enjambre se aleja un poco enfadado. María puede así gozar en paz de la compañía de sus maestras, las cuales le dirigen palabras de alabanza y bendición. 

Isabel también se ha acercado, y, dado que María, emocionada, llora porque Ana de Fanuel la llama hija y la besa con un afecto verdaderamente maternal, le dice: 

María, tu madre no está presente, pero sí está presente. Su espíritu se regocija junto al tuyo, y, mira, las cosas que llevas te traen de nuevo su caricia. En ellas sientes aún el sabor de sus besos. Un día ya lejano, el día en que viniste al Templo, me dijo: "Le he preparado los vestidos y el ajuar para cuando se case, porque quiero ser yo la que le haya hilado las telas y le haya hecho los vestidos, para no estar ausente en el día de su alegría". Mira, al final, cuando yo la asistía, ella quería todas las noches acariciar tus primeros vestidos y este que llevas ahora, y decía: "Aquí siento el olor de jazmín de mi pequeñuela, aquí quiero que Ella sienta el beso de su mamá". ¡Cuántos besos dio a este velo que cubre tu frente! ¡Más besos que hilos tiene!.. Y, cuando uses estas telas hiladas por ella, piensa que más que la estambre los ha hecho el amor de tu madre. Y estas joyas... Tu padre las salvó para ti incluso en los momentos difíciles, para que te embellecieran, como corresponde a una princesa de David, en este momento. Alégrate, María. No estás huérfana; los tuyos están contigo, y quien va a ser tu marido es tan perfecto, que es para ti padre y madre... 

- ¡Oh, sí! ¡Eso es verdad! No puedo quejarme de él, ciertamente. En menos de dos meses ha venido dos veces, y hoy viene por tercera vez, desafiando a las lluvias y al tiempo ventoso, declarándose sujeto a mí... Fíjate: ¡sujeto a mí! ¡Yo, que soy una pobre mujer, y mucho más joven que él! Y no me ha negado nada. Es más, ni siquiera espera a que yo pida. Parece como si un ángel le dijera lo que deseo, y me lo dice él antes de que yo hable. La última vez me dijo: "María, creo que preferirás estar en tu casa paterna. Dado que eres hija heredera, lo puedes hacer, si lo ves oportuno. Yo iré a tu casa. Solamente para observar el rito, tú vas durante una semana a casa de Alfeo, mi hermano. María te quiere ya mucho. De allí partirá la tarde de la boda el cortejo que te llevará a casa". ¿No es amable por su parte? No le ha importado ni siquiera el dar pie a la gente para decir que él no tiene una casa que me guste... A mí me hubiera gustado en todo caso, por estar él, que es tan bueno, en ella. Pero sin duda prefiero la mía... por los recuerdos... ¡Oh, José es bueno! 

- ¿Qué dijo del voto? Todavía no me has comentado nada. 

- No puso ninguna objeción. Es más, conocidas las razones del mismo, dijo: "Uniré mi sacrificio al tuyo". 

- ¡Es un joven santo!- dice Ana de Fanuel. 

El "joven santo" entra en este momento, acompañado de Zacarías. 

Su figura es, literalmente hablando, espléndida. Todo de amarillo oro, parece un soberano oriental. Bolsa y puñal penden de un espléndido cinturón: aquélla, de tafilete bordado en oro; el puñal, en una vaina con guarniciones bordadas en oro, también de tafilete. Cubre su cabeza un turbante, la típica faja de tela como la llevan todavía ciertos pueblos de África, los beduinos por ejemplo; lo sujeta en torno un valioso arito de oro, delgado, que ciñe unos ramitos de mirto. Viste majestuosamente un manto completamente nuevo con muchas franjas. Está radiante de alegría. En las manos lleva unos ramitos de mirto en flor. 

Saluda diciendo: 

- ¡A ti la paz, mi prometida! Paz a todos. 

Recibido el saludo de respuesta, dice: 

Vi tu alegría el día en que te di la ramita de tu huerto. He pensado traerte este mirto que procede de la gruta que tanto estimas. Quería haberte traído las rosas que están enfrente de tu casa, las primeras que están floreciendo ahora; pero las rosas no duran varios días de viaje... Habría llegado trayendo sólo espinas, y yo a ti, dilecta mía, te quiero ofrecer sólo rosas, y quiero sembrar tu camino de flores blandas y perfumadas, para que apoyes tu pie sobre ellas y no encuentres ni inmundicias ni asperezas. 

- ¡Oh, gracias, hombre de corazón bueno! ¿Cómo has logrado que llegara fresco? 

He atado a la silla un recipiente y he metido dentro estas ramitas con las flores todavía en capullo. Durante el viaje han florecido. Tómalas, María. Que tu frente se enguirnalde de pureza, símbolo de la mujer prometida; aunque siempre será mucho menor que la pureza que hay en tu corazón. 

Isabel y las maestras engalanan a María con la florida guirnaldita que se forma al fijar en el precioso aro los ramitos cándidos del mirto, e intercalan unas pequeñas, cándidas rosas, que había en un jarrón encima de un arca. 

María hace ademán de coger su amplio manto cándido para colocárselo prendido a los hombros. Pero su prometido le precede en el gesto y le ayuda a fijar con dos hebillas de plata, en los hombros, este amplio manto suyo. Las maestras disponen los pliegues con amor y gracia. 

Todo está preparado. Mientras esperan a no sé qué, José dice (lo dice apartándose un poco con María): 

He pensado este tiempo en tu voto. Ya te dije que lo comparto. Pero, cuanto más pienso en ello, más me doy cuenta de que no es suficiente el nazireato temporal, aunque se vaya renovando. Yo te he comprendido, María. No merezco todavía la palabra de la Luz, pero sí me llega un murmullo de su voz, y ello me pone en condiciones de leer tu secreto, al menos en sus líneas maestras. Soy un pobre ignorante, María. Soy un pobre obrero. Ni sé de letras ni tengo tesoros, mas a tus pies pongo mi tesoro, para siempre. Mi castidad absoluta, para ser digno de estar a tu lado, Virgen de Dios, "hermana mía, novia, cerrado huerto, fuente sellada", como dice el Antepasado nuestro, que quizás escribió el Cantar viéndote a ti... Yo seré el guardián de este huerto de perfumes en que se dan las más preciadas frutas, donde mana una vena de agua viva con ímpetu suave: ¡tu dulzura, prometida mía, que con tu candor — ¡oh, llena de hermosura! — me has conquistado el espíritu! ¡Oh, tú, más hermosa que una aurora; Sol, que resplandeces porque te resplandece el corazón; oh, toda amor para con tu Dios y para con el mundo al que quieres dar el Salvador con tu sacrificio de mujer! ¡Ven, mi amada! 

Y coge delicadamente su mano para guiarla hacia la puerta. 

Los siguen todos los demás. Afuera se añaden las joviales compañeras, enteramente de blanco todas ellas y con velos. 

Van por patios y pórticos, entre la muchedumbre observadora, hasta llegar a un punto que ya no pertenece al Templo; parece, más bien, una sala dada para el culto, como se deduce de la existencia en ella de lámparas y rollos de pergaminos como en las sinagogas. Los novios caminan hasta llegar frente a un alto atril (casi una cátedra), y esperan. Los demás, perfectamente en orden, se ponen detrás de ellos. Otros sacerdotes y gente simplemente curiosa se agolpan en el fondo de la sala. Entra, solemne, el Sumo Sacerdote. Rumor de los curiosos: - ¿Es él el que los casa? 

Sí, porque es de casta real y sacerdotal. La novia es flor de David y Aarón, y virgen del Templo; el novio, de la tribu de David. 

El Pontífice pone la mano derecha de la novia en la del novio y los bendice solemnemente: 

El Dios de Abraham, Isaac y Jacob esté con vosotros. Que El os una y se cumpla en vosotros su bendición, dándoos su paz y una numerosa descendencia con larga vida y muerte beata en el seno de Abraham. 

Luego se retira, solemne como había entrado. 

Se lleva a cabo la promesa recíproca. María es la prometida- esposa de José. 

Todos salen y, en perfecto orden, van a una sala, en la cual se redacta el contrato de matrimonio, donde se dice que María, hija heredera de Joaquín de David y Ana de Aarón, da como dote a su prometido- esposo su casa y bienes anejos y su ajuar personal así como cualquier otro bien heredado de su padre. Todo queda cumplido. 

Los esposos salen al patio, lo atraviesan, van hacia la salida, que está cerca de la sección de las mujeres dedicadas al Templo. Los está esperando un carro cómodo y voluminoso. Va provisto de una cortina protectora. En él ya están colocados los pesados baúles de María. 

Despedidas, besos y lágrimas, bendiciones, consejos, recomendaciones... María sube con Isabel y se pone en el interior del carro; en la parte de delante se ponen José y Zacarías. Se han quitado los mantos de fiesta y se han arrollado en unas capas oscuras. 

El carro se pone en marcha, al trote pesado de un caballazo oscuro. Los muros del Templo se alejan, y luego los de la ciudad. Ya se ve el campo, nuevo, fresco, florido bajo los primeros soles de la primavera, con los trigos ya alzados un buen palmo del suelo, que parecen esmeraldas transformadas en hojitas ondulantes bajo una brisa ligera con sabor a flores de melocotonero y manzano, con sabor a tréboles en flor y a hierbabuenas silvestres. 

María llora en voz baja, al amparo de su velo, y, de vez en cuando, corre un poco la cortina y mira una vez más al Templo lejano, a la ciudad dejada... 

La visión cesa así. 

Dice Jesús: 
\emph{¿Qué dice el libro de la Sabiduría al cantar sus alabanzas?: "En la sabiduría está presente, efectivamente, el espíritu de inteligencia, santo, único, múltiple, sutil". Y continúa enumerando sus dotes, para terminar el período con estas palabras: "... que todo lo puede, todo lo prevé; que comprende a todos los espíritus, inteligente, puro, sutil. La sabiduría penetra con su pureza, es vapor de la virtud de Dios... por ello en ella no hay nada impuro... imagen de la bondad de Dios. Es única y, no obstante, lo puede todo; es inmutable y da vida nueva a todas las cosas; se comunica a las almas santas; forma a los amigos de Dios y a los profetas". Ya has visto cómo José, no por cultura humana, sino por instrucción sobrenatural, sabe leer en el libro sellado de la Virgen sin mancha; y cómo se acerca extremamente a las verdades proféticas con ese su "ver" un misterio sobrehumano donde los demás veían únicamente una gran virtud. Impregnado de esta sabiduría, que es vapor de la virtud de Dios y emanación cierta del Omnipotente, se conduce con espíritu seguro por el mar de este misterio de gracia que es María, se armoniza con Ella con espirituales contactos — en que se hablan, más que los labios, los dos espíritus en el sagrado silencio de las almas — donde sólo Dios oye voces que perciben también los que le son gratos por servirle con fidelidad y por estar llenos de Él. La sabiduría del Justo, que aumenta por la unión con la Toda Gracia y por la cercanía a Ella, le prepara a penetrar en los secretos más altos de Dios y a poderlos tutelar y defender de insidias humanas y demoníacas. Y contemporáneamente lo va renovando. Del justo hace un santo; del santo, el custodio de la Esposa y del Hijo de Dios. Sin quitar el sello de Dios, él, el casto, que ahora lleva su castidad a heroísmo angélico, puede leer la palabra de fuego escrita sobre el diamante virginal por el dedo de Dios, y en él lee aquello que su prudencia no dice, y que es mucho más grande que lo que leyó Moisés en las tablas de piedra. Y a fin de que ningún ojo profano alcance este Misterio, él se pone, como sello sobre el sello, como arcángel de fuego, a la entrada del Paraíso, dentro del cual el Eterno encuentra sus delicias "paseando al fresco del atardecer" y hablando con Aquella que es su amor, bosque de azucena en flor, aura perfumada de aromas, viento suave de frescura matutina, hermosa estrella, delicia de Dios. La nueva Eva está allí, en su presencia. No es hueso de sus huesos ni carne de su carne; sí, compañera de su vida, Arca viva de Dios. Él la recibe para tutelarla, y a Dios debe restituírsela, pura como la ha recibido. "Desposada con Dios" estaba escrito en ese libro místico de inmaculadas páginas... Y cuando la duda, sibilante, en la hora de la prueba, le sugirió su tormento, él, como hombre y como siervo de Dios, sufrió, como ninguno, por causa del temido sacrilegio. Pero ésta fue la prueba futura. Ahora, en este tiempo de gracia, él ve y se pone a sí mismo al servicio más auténtico de Dios. Luego vendrá la tempestad de la prueba, como para todos los santos, para ser probados y venir así a ser ayudantes de Dios. ¿Qué se lee en el Levítico? "Di a Aarón, tu hermano, que no entre en cualquier tiempo en el santuario que está detrás del Velo, ante el Propiciatorio que cubre al Arca, para no morir — pues Yo apareceré en la nube sobre el oráculo —, si no hace antes estas cosas: ofrecerá un novillo por el pecado y un carnero como holocausto; llevará la túnica de lino y con calzones de lino cubrirá su desnudez". Y verdaderamente José entra, cuando Dios quiere y cuanto Dios quiere, en el santuario de Dios; y traspasa el velo que cela el Arca sobre la cual está suspendido el Espíritu de Dios; y se ofrece a sí mismo y ofrecerá al Cordero, holocausto por el pecado del mundo, expiación de tal pecado? Y esto lo hace, vestido de lino, mortificados los miembros viriles para abolir su sensualidad, la cual, una vez, al inicio de los tiempos, triunfó, lesionando el derecho de Dios sobre el hombre; mas ahora será conculcada en el Hijo, en la Madre y en el padre adoptivo, para restituir a los hombres a la Gracia y devolverle a Dios su derecho sobre el hombre. Esto lo hace con su castidad perpetua. ¿No estaba José en el Gólgota? ¿Os parece que no está en el número de los corredentores? En verdad os digo que fue el primero de ellos, y que grande es, por tanto, ante los ojos de Dios. Grande por el sacrificio, la paciencia, la constancia y la fe. ¿Qué fe será mayor que ésta, que creyó sin haber visto los milagros del Mesías? Sea alabado mi padre adoptivo, ejemplo para vosotros de aquello que en vosotros más falta: pureza, fidelidad y perfecto amor. Gloria al magnífico lector del Libro sellado, que fue instruido por la Sabiduría para saber comprender los misterios de la Gracia y que fue elegido para tutelar la Salvación del mundo contra las insidias de todos los enemigos. }
 
\chapter*{Los Esposos llegan a Nazaret.}
\addcontentsline{toc}{chapter}{\normalfont\scshape{Los Esposos llegan a Nazaret.}}
 
El más azul de los cielos de un apacible febrero se extiende sobre las colinas de Galilea. Las suaves colinas que no he visto nunca en este ciclo de la Virgen niña, y que me son ya tan familiares al ojo como si hubiera nacido entre ellas. 

La calzada principal, refrescada por lluvia reciente, caída quizás la noche anterior, no tiene polvo, mas tampoco barro. Presenta aspecto compacto y limpio, como si fuera una calle de ciudad, y avanza, sinuosa, entre dos hileras de espino albar en flor: una nevada con sabor amargoso y a bosque, interrumpida una y otra vez por las monstruosas aglomeraciones de los cactus, con sus hojas carnosas en forma de paleta, erizadas de pinchos y decoradas con los enormes granates de sus originales frutos, crecidos sin tallo sobre las hojas, las cuales, por su color y forma, evocan siempre en mí profundidades marinas y bosques de corales y medusas, u otros animales de los mares profundos. 

Las hileras de espino sirven como cercas de las propiedades privadas, por lo cual se extienden en todas las direcciones formando un caprichoso trazado geométrico de curvas y de ángulos, de rombos, cuadrados, semicírculos, triángulos con las más inverosímiles formas agudas u obtusas; es un trazado enteramente asperjado de blanco: como una cinta llena de fantasía que hubieran extendido así, por diversión, a lo largo de los campos; sobre ella vuelan, pían, cantan, a centenares, pajaritos de toda especie, sintiendo la alegría del amor y dedicados a rehacer sus nidos. Al otro lado de las hileras de espino están los campos, con los trigos todavía verdes, pero aquí ya más altos que en los campos de Judea, y prados llenos de flores, y en ellos — como contrapunto de las ligeras nubecillas del cielo, que el ocaso tiñe de rosa o de un lila tenue o violeta o de un opalino colorado de azul o de un naranja- coral —, a centenares, las nubes vegetales de los árboles frutales, blancas, rosadas, rojas, en todas las tonalidades del blanco, rosa y rojo. 

Con el suave viento de la tarde, caen revoloteando de los árboles florecidos los primeros pétalos: parecen bandadas de mariposas buscando polen en las flores del campo. Entre árbol y árbol, festones de vid aún desnuda: sólo en la parte alta de los festones, en la parte donde más da el sol, las primeras hojitas se abren, inocentes, extrañadas, palpitantes. 

El Sol se pone, sereno, en el cielo — ¡qué apacible con ese azul suyo que la luz hace aún más claro! — y a lo lejos titilan, reflejándolo, las nieves del Hermón y de otras cumbres lejanas. 

Un carro avanza por la calzada, el carro que lleva a José y a María y a los primos de Ella; el viaje está tocando a su fin. 

María mira con el ojo ansioso de quien quiere conocer, o mejor, reconocer, aquello que ya un día vio, pero no lo recuerda, y sonríe cuando una sombra de recuerdo vuelve y se posa, como una luz, en esta o aquella cosa, en este o aquel punto. Isabel le ayuda a recordar, y también Zacarías y José, señalando esta o aquella cumbre, esta o aquella casa. 

Casas, sí. Porque Nazaret ya aparece extendida sobre la ondulación de su colina. Recibiendo por la izquierda el Sol ya ocultándose, muestra, con pinceladas de rosa, el color blanco de sus casitas, anchas y bajas, culminadas por una terraza. Algunas de ellas, al darles el sol de lleno, parecen, de lo rojas que se han puesto las fachadas, estar al lado de un fuego. Y el sol enciende también el agua de los bajos pozos, que no tienen casi brocal, de donde suben, chirriando, los cubos para la casa o los odres para la huerta. 

Niños y mujeres se acercan al borde de la calzada, queriendo ver el interior del carro, y saludan a José, que es muy conocido en el lugar. Pero luego se muestran titubeantes y tímidos ante las otras tres personas. 

Sin embargo, dentro ya de la pequeña ciudad, no hay titubeos ni temor. Mucha, mucha gente de todas las edades está a la entrada del pueblo bajo un rústico arco hecho con flores y ramas, y nada más que el carro aparece por detrás del recodo de la última casa de campo, que está colocada oblicuamente, se produce un verdadero gorjeo de voces agudas y un agitarse de ramas y flores. Son las mujeres, las chiquillas y los niños de Nazaret que saludan a la novia. Los hombres, más contenidos, están detrás de este seto agitado y gorjeante, y saludan con gravedad. 

María, ahora que la cortina ha sido quitada, dejando al descubierto el carro — lo habían hecho ya antes de llegar al pueblo, porque el sol ya no molestaba, y para permitirle a María el ver bien su tierra natal — aparece en su belleza de flor. Blanca y rubia como un ángel, sonríe con bondad a los niños, que le echan flores y besos, a las jóvenes de su edad, que la llaman por el nombre, a las mujeres casadas, a las madres, a las ancianas, que la bendicen con sus voces cantadoras. Inclina su cabeza ante los hombres, y especialmente ante uno de ellos, que quizás es el rabino o la personalidad principal del pueblo. 

El carro prosigue por la calle principal a paso lento, seguido de la muchedumbre por un buen trecho, muchedumbre para la que esta llegada es un acontecimiento. 

Esa es tu casa, María- dice José señalando con el látigo una casita que está justo en la base de una ondulación de la colina, y que tiene en la parte de atrás un hermoso y amplio huerto, exuberante, que termina en un pequeño olivar. Más allá, la consabida cerca de espino albar y cácteas señala el límite de la propiedad. Las tierras, que fueron de Joaquín, están al otro lado... 

Te ha quedado poco, ¿ves?- dice Zacarías - La enfermedad de tu padre fue larga y económicamente cara. Y caros fueron también los gastos para reparar el daño que hizo Roma. ¿Lo ves? La calle le ha cortado a la casa sus tres principales habitaciones. Se ha quedado más pequeña. Para ampliarla sin gastos excesivos, se cogió una parte del monte que forma una gruta; Joaquín tenía en ese lugar las provisiones y Ana sus telares. Haz con esto lo que creas más oportuno. 

- ¡Que sea poco no importa! Siempre me será suficiente. Me pondré a trabajar... 

No, María — es José quien habla — Yo seré quien trabaje. Tú sólo tejerás y coserás las cosas de la casa. Soy joven y fuerte, y soy tu esposo. No me atormentes viéndote trabajar. 

Haré como tú quieras. 

Sí, en esto yo quiero. Para todas las demás cosas tu deseo es ley, pero en esto no. 

Ya han llegado. El carro se detiene. 

Dos mujeres y dos hombres, respectivamente de unos cuarenta y cincuenta años, están a la puerta, y muchos niños y jovencitos están con ellos. 

Dios te dé paz, María - dice el hombre más anciano. Una de las mujeres se acerca a María, la abraza y la besa. 

Es mi hermano Alfeo, y María, su mujer, y éstos son sus hijos. Han venido expresamente para recibirte y felicitarte y decirte que su casa es tuya, si así lo deseas - dice José. 

Sí, ven, María, si te resulta penoso vivir sola. El campo es bonito en primavera y nuestra casa está en medio de campos floridos. Tú serás su más hermosa flor - dice María de Alfeo. 

Gracias, María. Yo iría con mucho gusto, y alguna vez iré; iré, sin duda, para la boda... Pero, deseo vivamente ver, reconocer mi casa. La dejé siendo muy pequeña y se me ha desdibujado su imagen... Ahora esta imagen la encuentro de nuevo... y me parece como si encontrara de nuevo a mi madre perdida, a mi padre amado, el eco de las palabras de ellos... y el aroma de su último respiro. Siento como si ya no fuera huérfana, porque me abrazan de nuevo estas paredes... Compréndeme, María - Aparece un poco el llanto en la voz de María, y también en sus pestañas. María de Alfeo responde: 

Querida mía, como tú quieras. Quiero que me sientas hermana y amiga y un poco madre incluso, porque soy mucho más mayor que tú. 

La otra mujer, que se ha acercado entretanto, dice: 

María, quiero saludarte. Soy Lía, la amiga de tu madre. Te vi nacer. Este es Alfeo, sobrino de Alfeo y muy amigo de tu madre. Lo que hice por tu madre, si quieres, lo haré por ti. Mira, mi casa es la que está más cerca de la tuya y tus parcelas de terreno son ahora nuestras. Pero, si quieres venir hazlo cuando te apetezca, en cualquier momento. Abrimos un paso en el cercado y así estaremos juntas, sin dejar de estar cada una en su casa. Este es mi marido. 

Os doy las gracias a todos y por todo; por todo el amor que habéis tenido a los míos, y por todo el amor que me tenéis a mí. Que Dios todopoderoso os bendiga por ello. 

Descargan los pesados baúles y los meten en la casa. Entran. Reconozco ahora que es la casita de Nazaret, como será luego, durante la vida de Jesús. 

José toma de la mano — un gesto habitual en él — a María, y entra así. Pero en el umbral de la puerta le dice: 

Ahora, aquí, en el umbral de esta puerta, quiero de ti una promesa: que cualquier cosa que te suceda, o cualquier cosa que necesites, tu único amigo, la única persona en quien pienses para solicitar ayuda, sea yo, y que, bajo ningún motivo, debas sufrir sola ninguna pena. Yo estoy a tu entera disposición, y para mí será una satisfacción el hacerte feliz el camino, y, dado que la felicidad no siempre está en nuestra mano, al menos, hacértelo tranquilo y seguro. 

Te lo prometo, José. 

La siguiente cosa es abrir puertas y ventanas... El último sol entra curioso. 

María se ha quitado el manto y el velo. Menos las flores de mirto, todavía va vestida como en los esponsales. Sale al huerto, que presenta un aspecto exuberante. Mira, sonríe, y, todavía de la mano de José, da un paseo. Se la ve como quien volviera a tomar posesión de un lugar perdido. 

José le muestra el resultado de sus trabajos: 

Mira, aquí he cavado para recoger el agua de la lluvia, porque estas cepas están siempre sedientas. A este olivo le he vuelto a cortar las ramas más viejas para darle vigor; y he plantado estos manzanos, porque dos estaban muertos; y luego, allí he plantado unas higueras. Cuando crezcan resguardarán a la casa del sol excesivo y de las miradas curiosas. La pérgola es la misma que había; lo único que he hecho ha sido cambiar los palos que estaban deteriorados, y también una labor de poda. Espero que dé muchas uvas. Y aquí, mira - y la lleva, orgulloso, hacia el terreno en pendiente que resguarda la casa por detrás y que es límite del huerto por el lado de tramontana - y. aquí he excavado una pequeña gruta, y la he reforzado, y, cuando agarren estas plantas, será casi igual que la que tenías. Falta el manantial... pero, espero hacer llegar aquí desde el manantial un regatillo. 

Pienso trabajar durante las largas tardes de verano cuando venga a verte... 

- ¿Cómo es eso? - dice Alfeo. " ¿No vais a celebrar la boda este verano? 

No. María quiere tejer los paños de lana, que es lo único que le falta a su ajuar. Y a mí eso me satisface. María es tan joven, que el esperar un año o más no es nada. Entretanto se ambienta a la casa... 

- ¡Bueno! Tú siempre has sido un poco distinto de los demás, y lo sigues siendo. No sé quién pudiera no tener prisa en tener por esposa a una flor como María, ¡y tú metes meses por medio!.. 

Alegría muy esperada, alegría más intensamente gustada - responde José con una sonrisa sutil. 

El hermano se encoge de hombros y dice: 

- ¿Y entonces? Según tus planes, ¿cuándo vas a pensar en la boda? 

Cuando María cumpla dieciséis años. Después de la fiesta de los Tabernáculos. ¡Dulces serán las tardes de invierno para los recién casados!.. - Y sigue sonriendo mirando a María: una sonrisa que conlleva un pacto secreto y delicado; de una castidad fraterna consoladora. 

Luego continúa caminando y explicando: 

Ésta es la habitación grande que había en el monte. Si te parece bien, cuando venga, instalaré en ella mi taller. Está unida, pero no forma parte de la casa. Así no molestaré con los ruidos, o creando otros trastornos. No obstante, si no quieres que sea así... 

No, José; así está muy bien. 

Vuelven a entrar en la casa. Encienden las lámparas. 

María está cansada - dice José - Dejémosla tranquila con sus primos. 

Saludos de todos los que se marchan... José se queda todavía unos minutos y habla con Zacarías en voz baja. 

Tu primo te deja a Isabel durante un poco. ¿Contenta? Yo sí, porque te ayudará a... ser una perfecta ama de casa; con ella podrás colocar como quieras tus cosas y tu ajuar, y yo vendré todas las tardes a ayudarte; con ella podrás conseguir lana y todo lo que necesites, y yo me encargaré de los gastos. Acuérdate de que has prometido que recurrirías a mí para todo. Adiós, María. Duerme el primer sueño de señora en esta casa tuya, y que el ángel de Dios te lo haga sereno. Que el Señor sea siempre contigo. 

Adiós, José. Queda tú también bajo las alas del ángel de Dios. 

Gracias, José, por todo. En la medida en que pueda, te pagaré por tu amor, con el mío. José saluda a los primos y sale. 

Y con él cesa la visión. 
 
\chapter*{Como conclusión del Pre- Evangelio. \\ \normalfont\normalsize\textit{6 de Septiembre de 1944.}}
\addcontentsline{toc}{chapter}{\normalfont\scshape{Como conclusión del Pre- Evangelio.}}

Dice Jesús: 
\emph{El ciclo ha terminado. Y con él, tan dulce y delicado como ha sido, tu Jesús te ha mantenido (habla a María Valtorta), sin movimientos bruscos, al margen de la agitación de estos días. Como a niño envuelto en blandos paños de lana y depositado sobre mullidos almohadones, a ti te han envuelto estas beatas visiones, para que no sintieras, con el consiguiente terror, la crueldad de los hombres que se odian en vez de amarse. No serías capaz ya de soportar ciertas cosas, y no quiero que mueras por causa de ello: Yo cuido a mi "portavoz". Está para desaparecer del mundo ya la causa de todas las desesperaciones que han torturado a las víctimas. Por tanto, María, también cesa para ti el tiempo de este tremendo sufrimiento por demasiadas causas tan en contraposición con tu modo de sentir. No terminará tu sufrir: eres víctima; pero, parte de él, ésta, cesa. Después llegará el día en que Yo te diga, como a María de Magdala moribunda: "Descansa. Ahora es tiempo de descanso para ti. Dame tus espinas. Ahora es tiempo de rosas. Descansa y espera. Te bendigo, mujer bendita". Esto es lo que te decía — y era una promesa y tú no la entendiste — cuando llegaba, el tiempo en que habías de ser sumergida, revolcada, en espinas, encadenada, colmada de espinas hasta en los más hondos recovecos de tu ser... Esto es lo que ahora te repito, con una alegría como sólo el Amor puede experimentar — y Yo soy el Amor — cuando puede hacer cesar un dolor en su dilecto amado. Esto es lo que te digo ahora, ahora que ese tiempo de sacrificio cesa. Y Yo, que sé, por el mundo que no sabe, por Italia, por Viareggio, por esta pequeña población, a donde tú me has portado — medita el sentido de estas palabras - Yo te expreso mi agradecimiento, como corresponde a las víctimas por su sacrificio. Cuando te mostré a Cecilia (Santa Cecilia), virgen- esposa, te dije que ella se había echado mis perfumes, y con ellos atrajo a su marido, a su cuñado, a sus domésticos, a sus familiares, a sus amigos. Tú has hecho — no lo sabes, pero te lo digo Yo, Yo que conozco las cosas — el papel de Cecilia en medio de este mundo enloquecido. Te has saturado de Mí, de mi palabra, has llevado mis deseos a las personas, y las mejores han comprendido, y siguiéndote a ti, que eres víctima, muchísimas otras víctimas han surgido. Si tu patria, y los lugares que tú más quieres, no han sido completamente destruidos, ha sido porque muchas hostias han sido sacrificadas a raíz de tu ejemplo y de tu ministerio. Gracias, mujer bendita. Continúa así. Tengo gran necesidad de salvar a la Tierra, de volver a comprar la Tierra; las monedas sois vosotras, las víctimas. La Sabiduría que ha instruido a los santos, y que te instruye a ti con un magisterio directo, te eleve cada vez más en la comprensión de la Ciencia de vida y en practicarla. Levanta tú también tu pequeña tienda ante la casa del Señor. Te digo más aún: hinca las estacas que sostienen tu misma morada en la morada de la Sabiduría y mora en ella sin jamás dejarla. Descansarás así, protegida por el Señor, que te ama, como ave entre ramas florecidas, y Él será tu amparo ante cualquier tipo de intemperie espiritual y estarás en la luz de la gloria de Dios de donde descenderán para ti palabras de paz y verdad. Puedes ir en paz. Te bendigo, mujer bendita". }

Inmediatamente después dice María (la Virgen): 
\emph{A María el regalo de Mamá por su fiesta. Una cadena de regalos. Y si hay alguna espina entre ellos, no te quejes al Señor, que te ha amado como a bien pocos ama. Te dije al principio: "Escribe acerca de mí. De todo lo que sufras recibirás consuelo". ¿Ves como ha sido verdad? Te estaba reservado este regalo para este tiempo de agitación, porque no sólo cuidamos el espíritu, sino que sabemos también cuidar la materia, que no es reina, sino sierva útil al espíritu en el cumplimiento de su misión. Sé agradecida al Altísimo, que, incluso en el sentido afectuosamente humano, es verdaderamente Padre tuyo, que te acuna con éxtasis suaves para ocultarte lo que te asustaría. Ámame cada vez más. Te he conducido conmigo al secreto de mis primeros años. Ahora ya sabes todo acerca de Mamá. Ámame como hija y como hermana en el destino victimal. Y ama a Dios Padre, a Dios Hijo, a Dios Espíritu Santo, con perfección de amor. La bendición del Padre, del Hijo y del Espíritu pasa por mis manos; recibe el perfume de mi materno amor hacia ti, a ti desciende y en ti se deposita. Sé sobrenaturalmente devota. }

\chapter*{La Anunciación.}
\addcontentsline{toc}{chapter}{\normalfont\scshape{La Anunciación.}}
 
Lo que veo. María, muchacha jovencísima (al máximo quince años a juzgar por su aspecto), está en una pequeña habitación rectangular; verdaderamente, una habitación de jovencita. Contra una de las dos paredes más largas, está el lecho: una cama baja, sin armadura, cubierta por gruesas esteras o tapetes — diríase que éstos están extendidos sobre una tabla o sobre un entramado de cañas porque están muy rígidos y sin pliegues como los de nuestras camas —. Contra la otra pared, un estante con una lámpara de aceite, unos rollos de pergamino y una labor de costura — parece un bordado — cuidadosamente doblada. 

A uno de los lados del estante, hacia la puerta, que da al huerto, abierta ahora, aunque tapada por una cortina que se mueve movida por un ligero vientecillo, en un taburete bajo está sentada la Virgen. Está hilando un lino candidísimo y suave como la seda. Sus manitas, sólo un poco más oscuras que el lino, hacen girar rápidamente el huso. Su carita juvenil, preciosa, está ligeramente inclinada y ligeramente sonriente, como si estuviera acariciando o siguiendo algún dulce pensamiento. 

Hay un gran silencio en la casita y en el huerto. Y mucha paz, tanto en la cara de María como en el espacio que la rodea. Paz y orden. Todo está limpio y ordenado. La habitación, de humildísimo aspecto y mobiliario, casi desnuda como una celda, tiene un aire austero y regio, debido a su gran limpieza y a la cuidadosa colocación de la cobertura del lecho, de los rollos, de la lámpara y del jarroncito de cobre que está cerca de ésta con un haz de ramitas floridas dentro, ramitas de melocotonero o de peral, no lo sé; lo que sí está claro es que son de árboles frutales, de un blanco ligeramente rosado. 

María comienza a cantar en voz baja. Luego alza ligeramente la voz. No llega al pleno canto, pero su voz ya vibra en la habitación, sintiéndose en aquélla una vibración del alma. No entiendo la letra, que sin duda es en hebreo, pero, dado que, de vez en cuando repite "Yeohveh", intuyo que se trata de algún canto sagrado, acaso un salmo. Quizás María recuerda los cantos del Templo. Debe tratarse de un dulce recuerdo. Efectivamente, deja sobre su regazo sus manos, y con ellas el hilo y el huso, y levanta la cabeza para apoyarla en la pared, hacia atrás. Su rostro está encendido de un lindo rubor; los ojos, perdidos tras algún dulce pensamiento, brillantes por un golpe de llanto, que no los rebosa pero sí los agranda. Y, a pesar de todo, loa ojos ríen, sonríen ante ese pensamiento que ven y que los abstrae de lo sensible. Resaltando de su vestido blanco sencillísimo, circundado por las trenzas, que lleva recogidas como corona en torno a la cabeza, el rostro rosado de María parece una linda flor. 

El canto pasa a ser oración: 

- Señor Dios Altísimo, no te demores más en mandar a tu Siervo para traer la paz a la tierra. Suscita el tiempo propicio y la virgen pura y fecunda para la venida de tu Cristo. Padre, Padre santo, concédele a tu sierva ofrecer su vida para esto. Concédeme morir tras haber visto tu Luz y tu Justicia en la Tierra, sabiendo que la Redención se ha cumplido. ¡Oh, Padre Santo, manda a la Tierra el Suspiro de los Profetas! Envía el Redentor a tu sierva. Que cuando cese mi día se me abra tu Casa por haber sido abiertas sus puertas por tu Cristo para todos aquellos que en ti hayan esperado. Ven, ven, Espíritu del Señor. Ven a los fieles tuyos que te esperan. ¡Ven, Príncipe de la Paz!.. 

María se queda así ensimismada... 

La cortina late más fuerte, como si alguien la estuviera aventando con algo o quisiera descorrerla. Y una luz blanca de perla fundida con plata pura hace más claras las paredes tenuemente amarillentas, hace más vivos los colores de las telas, más espiritual el rostro alzado de María. En la luz se prosterna el Arcángel. La cortina no ha sido descorrida ante el misterio que se está verificando; es más, ya no late: pende, rígida, pegada a las jambas, separando, como una pared, el interior del exterior. 

El Arcángel necesariamente debe adquirir un aspecto humano; pero es un aspecto ultra- humano. ¿De qué carne está compuesta esta figura bellísima y fulgurante? ¿Con qué sustancia la ha materializado Dios para hacerla sensible a los sentidos de la Virgen? Sólo Dios puede poseer estas substancias y usarlas de esa manera perfecta. Es un rostro, es un cuerpo, son ojos, boca, cabellos y manos como las nuestras; pero no se trata de nuestra opaca materia: es una luz que ha tomado color de carne, de ojos, de cabellera, de labios, una luz que se mueve y sonríe y mira y habla. 

- ¡Salve, María, llena de Gracia, salve! - La voz es un dulce arpegio como de perlas chocadas contra un metal precioso. 

María se estremece y baja la mirada. Su estremecimiento aumenta cuando ve a la fúlgida criatura arrodillada aproximadamente a un metro de distancia de Ella, con las manos cruzadas sobre el pecho, mirándola con una veneración infinita. 

María, repentinamente, se pone en pie y aprieta su cuerpo contra la pared. Palidece y se ruboriza alternativamente. Su rostro expresa estupor y turbación. Inconscientemente aprieta sus manos contra el pecho escondiéndolas en las anchas mangas. Se recoge sobre sí misma como queriendo esconder lo más posible su cuerpo: un acto de delicado pudor. 

- No. No temas. ¡El Señor está contigo! ¡Bendita tú entre todas las mujeres! 

A pesar de estas palabras, María sigue temiendo. ¿De dónde viene ese ser extraordinario? ¿Es un enviado de Dios, o del Engañador? 

- ¡No temas, María! - insiste el Arcángel - Yo soy Gabriel, el Ángel de Dios. Mi Señor me ha enviado a ti. No temas, porque has hallado gracia ante Dios. Tú concebirás en tu seno y darás a luz un Hijo, y le pondrás por nombre "Jesús". Será grande, será llamado Hijo del Altísimo, y verdaderamente lo será. El Señor Dios le dará el trono de David, su padre, y reinará para siempre en la casa de Jacob, y su Reino no tendrá fin nunca. Comprende, santa Virgen amada del Señor, Hija bendita suya, llamada a ser Madre de su Hijo, comprende qué Hijo vas a engendrar. 

- ¿Cómo puede suceder esto si yo no conozco hombre? ¿Acaso el Señor ya no acoge el ofrecimiento de su sierva? ¿Es que ya no quiere que sea virgen por amor a Él? 

No vas a ser madre por obra de varón, María. Tú eres la eterna Virgen, la Santa de Dios. El Espíritu Santo descenderá sobre ti y la potencia del Altísimo te cubrirá con su sombra. Por eso, Santo se llamará el que nacerá de ti, e Hijo de Dios. Todo lo puede el Señor, Dios nuestro. Isabel, la estéril, en su vejez ha concebido un hijo que será el Profeta de tu Hijo, el que preparará sus caminos. El Señor la ha liberado de su oprobio y su memoria quedará en las gentes unida a tu nombre, como el nombre de su hijo al de tu Hijo Santo, y hasta el final de los siglos las gentes os llamarán bienaventuradas por la gracia del Señor que habéis recibido, y a ti especialmente, María, porque habrán recibido la Gracia por medio de ti. Isabel está ya en su sexto mes, y su peso, paradójicamente, la regocija, y más aún la regocijará cuando conozca el motivo de tu gozo. Para Dios nada es imposible, María, llena de Gracia. ¿Qué debo responderle a mi Señor? No te turbe ningún tipo de pensamiento. Él tutelará tus intereses si te pones en sus manos. ¡El mundo, el Cielo, Dios eterno esperan tu respuesta! 

María, cruzando a su vez sus manos sobre el pecho e inclinándose con gesto reverente dice: 

He aquí la esclava de Dios. Hágase de mí según su palabra. 

El Ángel resplandece de alegría y se pone en actitud adorante, puesto que, sin duda, ve al Espíritu de Dios descender sobre la Virgen, inclinada en gesto de adhesión; luego desaparece sin mover la cortina, dejándola cerrada cubriendo el Misterio santo. 

\chapter*{La desobediencia de Eva y la obediencia de María.}
\addcontentsline{toc}{chapter}{\normalfont\scshape{La desobediencia de Eva y la obediencia de María.}}
 
Dice Jesús: 
\emph{¿No se lee en el Génesis que Dios hizo al hombre dominador de todo lo que había sobre la tierra, es decir, de todo excepto de Dios y de sus ángeles ministros? ¿No se lee que hizo a la mujer como compañera del hombre en la alegría y en el dominio sobre todos los seres vivos? ¿No se lee que de todo podían comer excepto del árbol de la ciencia del Bien y del Mal? ¿Por qué? ¿Cuál es el sentido que subyace en las palabras "para que domine"; cuál, en el árbol de la ciencia del Bien y del Mal? ¿Os habéis preguntado alguna vez esto, vosotros, que os hacéis tantas preguntas inútiles y que no sabéis preguntarle nunca a vuestra alma acerca de las celestes verdades? Vuestra alma, si estuviera viva, os las manifestaría. Esa alma que, cuando está en gracia, es como una flor entre las manos de vuestro ángel; esa alma que, cuando está en gracia, es como una flor besada por el sol y asperjada por el rocío, besada y asperjada por el Espíritu Santo, que le da calor y la ilumina, que la riega y la adorna de celestes luces. ¡Cuántas verdades os manifestaría vuestra alma, si supierais conversar con ella, si la amarais como a quien os proporciona la semejanza con Dios, que es Espíritu, como espíritu es vuestra alma! ¡Qué gran amiga tendríais, si amarais a vuestra alma en vez de odiarla hasta matarla; qué grande, sublime amiga con quien hablar de cosas celestes; vosotros que tenéis tanta avidez de hablar y os destruís los unos a los otros con amistades que, aun no siendo indignas (alguna vez lo son), sí son casi siempre inútiles, y se os transforman en un bullicio vano o nocivo de palabras y sólo palabras, todas terrenas! ¿No dije Yo: "Quien me ama observará mi palabra y el Padre mío le amará e iremos a él y haremos morada en él"? El alma que está en gracia posee el amor y, poseyéndolo, posee a Dios, o sea, al Padre que la conserva, al Hijo que la instruye, al Espíritu que la ilumina. Posee, por tanto, el Conocimiento, la Ciencia, la Sabiduría. Posee la Luz. Imaginaos, pues, qué conversaciones más sublimes podría establecer con vosotros vuestra alma, que son las conversaciones que han llenado los silencios de las cárceles, los silencios de las celdas, los silencios del yermo, los silencios de las habitaciones de los enfermos santos; las que han confortado a los presos que en la cárcel esperaban el martirio, a los cenobitas, que habían elegido el claustro en pos de la Verdad, a los eremitas, que anhelaban conocer anticipadamente a Dios, a los enfermos, para que soportaran o, mejor dicho, amaran su cruz. Si supierais preguntar a vuestra alma, ella os diría que el significado verdadero, exacto, vasto cuanto la creación, de la palabra "domine" es éste: "Para que el hombre domine todo: sus tres estratos (el inferior, animal; el estrato intermedio, moral; el estrato superior, espiritual), y oriente los tres hacia un único fin: poseer a Dios". Poseerlo mereciéndolo con este férreo dominio que tiene sujetas todas las fuerzas del yo haciéndolas esclavas de esta única finalidad: merecer poseer a Dios. Vuestra alma os diría que Dios había prohibido el conocimiento del Bien y del Mal, porque el Bien lo había donado con generosidad y gratuitamente a sus criaturas, y el Mal no quería que lo conocierais, porque es un fruto dulce al paladar, pero que, una vez que baja con su jugo a la sangre, ocasiona una fiebre que mata y produce ardiente sequedad en la garganta, por lo cual, cuanto más se bebe de su jugo traidor, más sed de él se tiene. Vuestra objeción será: "¿Y por qué lo ha puesto?". ¿Por qué! El Mal es una fuerza que ha nacido sola, como ciertos males monstruosos en el más sano de los cuerpos. Lucifer era un ángel, el más hermoso de los ángeles. Espíritu perfecto. Sólo Dios era superior a él. Pues bien, con todo, en su ser luminoso nació un vapor de soberbia, y Lucifer no lo dispersó, sino que, por el contrario, lo condensó dándole vida en su interior. De esta incubación nació el Mal. Este ya existía antes del hombre. Dios había arrojado fuera del Paraíso al Incubador maldito del Mal, al que ensuciaba el Paraíso. Pero ha seguido siendo y es el eterno Incubador del Mal, y, no pudiendo seguir ensuciando el Paraíso, ha ensuciado la Tierra. Ese metafórico árbol pone en evidencia esta verdad. Dios había dicho al hombre y a la mujer: "Conoced todas las leyes y los misterios de la creación. Pero no pretendáis usurparme el derecho de ser el Creador del hombre. Para propagar la estirpe humana bastará el amor mío que circulará por vosotros, y, sin libídine sensual, sólo por latido de caridad, dará vida a los nuevos hombres como Adán de la estirpe. Todo os lo doy; sólo me reservo este misterio de la formación del hombre". Satanás quiso quitarle al hombre esta virginidad intelectual y, con su lengua serpentina, hechizó y halagó miembros y ojos de Eva, suscitando en ellos reflejos y sutilezas que antes no tenían porque no estaban intoxicados de Malicia. Ella "vio", y, viendo, quiso probar. Había sido despertada la carne. ¡Ah, si hubiera llamado a Dios¡ Si hubiera corrido a decirle: "¡Padre, estoy enferma; la serpiente me ha halagado y me siento turbada!". El Padre la habría purificado, la habría curado con su aliento, pues lo mismo que le había infundido la vida podía infundirle de nuevo la inocencia, quitándole el recuerdo del tóxico serpentino, es más, introduciendo en ella una repugnancia hacia la Serpiente (como les sucede a los que han sufrido una enfermedad, que, una vez curados, sienten hacia ella una instintiva repugnancia). Pero no, Eva no va al Padre, Eva vuelve donde la Serpiente. Esa sensación le es dulce. "Viendo que el fruto del árbol se podía comer y que era bonito y de aspecto agradable, lo cogió y comió de él". Y "comprendió". Ya la malicia había penetrado y le mordía las entrañas. Vio con ojos nuevos y oyó con oídos nuevos los usos y la voz de las bestias; y los deseó febrilmente. Inició sola el pecado. Lo consumó con su compañero. Por eso sobre la mujer pesa una condena mayor. Por ella el hombre se hizo rebelde a Dios, y por ella conoció la lujuria y la muerte. Por ella perdió el dominio sobre sus tres reinos: el del espíritu, porque permitió que el espíritu desobedeciera a Dios; el de lo moral, porque permitió que las pasiones le sometieran a su señorío; el de la carne, porque la rebajó a las leyes instintivas de las bestias. "La Serpiente me ha seducido" dice Eva. "La mujer me ha ofrecido el fruto y yo he comido de él" dice Adán. Y el triple, desenfrenado apetito, desde entonces, tiene entre sus garras los tres reinos del hombre. Sólo la Gracia logra aflojar la presa de este monstruo despiadado; y, si vive, si está vivísima, si la voluntad del hijo fiel la mantiene cada vez más viva, llega incluso a estrangular al monstruo. Ya no habrá nada que temer: ni a los tiranos internos (o sea, la carne y las pasiones), ni a los tiranos externos (el mundo y los que en el mundo tienen poder), ni a las persecuciones, ni a la muerte. Es como dice el apóstol Pablo: "Nada de esto yo temo, y no considero ya mía mi vida, con tal de cumplir mi misión y llevar a cabo el ministerio recibido del Señor Jesús para dar testimonio del Evangelio de la Gracia de Dios"". }

Dice María (la Virgen): 
\emph{Gozoso — pues, efectivamente, cuando comprendí la misión a que Dios me llamaba, mi corazón se llenó de gozo — mi corazón se abrió como una azucena en capullo y vertió la sangre que habría de ser terreno para la Semilla del Señor. Alegría de ser madre. Me había consagrado a Dios desde mi más tierna edad, porque la luz del Altísimo me había iluminado acerca de la causa del mal del mundo; yo deseé, por lo que de mí dependía, borrar de mí la huella de Satanás. No sabía que no tenía mancha. No podía pensarlo. El solo hecho de pensarlo habría sido presunción y soberbia porque, habiendo nacido de padre y madre humanos, no me era lícito pensar que justamente yo era la Elegida para ser la Sin Mancha. El Espíritu de Dios me había instruido acerca del dolor del Padre ante la corrupción de Eva, que había aceptado degradarse — siendo una criatura de gracia — a un nivel de criatura inferior. Yo tenía la intención de suavizar ese dolor, poniendo de nuevo mi carne en la situación de pureza angélica, conservándome intacta de pensamientos, deseos y contactos humanos. Sólo para Él sería mi latido de amor; sólo para El, mi ser. No había en mí sed camal, pero sí sentía el sacrificio de no ser madre. La maternidad, exenta de lo que ahora la humilla, le había sido concedida por el Padre creador también a Eva. ¡Dulce y pura maternidad sin el peso del sentido! ¡Yo la experimenté! ¡Cuán grande la pérdida de Eva, renunciando a esta riqueza! Mayor que la pérdida de la inmortalidad. No, no creáis que es una exageración. Mi Jesús, y con Él yo, su Madre, conocimos la languidez de la muerte. Yo, el dulce languidecer de quien, cansado, se duerme; Él, ese languidecer atroz de quien muere por haber sido condenado. A nosotros, pues, también nos vino la muerte. Sin embargo, la maternidad exenta de cualquier tipo de violación me vino solamente a mí, la nueva Eva, para que yo pudiera manifestarle al mundo cuan dulce era el destino de la mujer, llamada a ser madre sin el dolor de la carne. El deseo de esta pura maternidad, siendo, como es, la gloria de la mujer, podía estar, y estaba, en la Virgen toda de Dios. Añadid a vuestra consideración el honor en que era tenida la mujer madre en el pueblo israelita, y comprenderéis mejor la naturaleza del sacrificio cumplido al consagrarme a esta privación. Ahora a su sierva el eterno Bueno le ofrecía este don, sin privarme del candor de que yo me había vestido para ser flor en su trono. Por ello exultaba, con el doble gozo de ser madre de un hombre y de ser Madre de Dios. Alegría porque a través de mí se restablecía la paz entre el Cielo y la Tierra. ¡Oh... haber deseado esta paz por amor a Dios y por amor al prójimo, y saber que por medio de mí, pobre esclava del Poderoso, aquélla venía al mundo! ¡Decir: "Hombres, no lloréis más. Yo traigo conmigo el secreto que os hará felices. No os lo puedo manifestar, porque está sellado en mí, en mi corazón, de la misma forma que el Hijo dentro del intacto seno. Ya os lo traigo, ya cada hora que pasa está más cercano el momento en que le veréis y sabréis su Nombre santo"! Alegría de haber hecho feliz a Dios: alegría del creyente que ve feliz a su Dios. ¡Oh... haber quitado del corazón de Dios la amargura de la desobediencia de Eva, de la soberbia de Eva, de su incredulidad! Mi Jesús ha explicado con qué culpa se manchó la Pareja primera. Yo he anulado esa culpa recorriendo en sentido inverso, para ascender, las etapas de su descenso. El principio de la culpa estuvo en la desobediencia: "No comáis y no toquéis de ese árbol", había dicho Dios. Pero el hombre y la mujer, los reyes de la creación, que podían tocar todo y comer todo excepto aquello — porque Dios quería hacerlos sólo inferiores a los ángeles — no tomaron en consideración ese veto. El árbol: el medio para probar la obediencia de los hijos. ¿Qué es la obediencia al mandato divino? Es un bien porque Dios no ordena sino el bien. ¿Qué es la desobediencia? Es un mal porque pone al corazón en las disposiciones de rebelión sobre las cuales Satanás puede obrar. Eva va al árbol, a ese árbol del que vendría: alejándose, su bien; acercándose, su mal. La arrastra a él la curiosidad ingenua de ver qué es lo que podía tener en sí de especial; la arrastra la imprudencia, que hace que le parezca inútil el mandato divino, dado que ella es fuerte y pura, reina del Edén, donde todo le presta obediencia, donde nada podrá causarle mal alguno. Su presunción la pierde. La presunción es ya levadura de soberbia. En el árbol encuentra al Seductor, el cual, a su inexperiencia, a su tan hermosa y virgen inexperiencia, a esa inexperiencia que no supo tutelar, le canta la canción de la mentira: "¿Tú crees que aquí hay mal? No. Dios te lo ha dicho porque quiere teneros bajo la esclavitud de su poder. ¿Creéis que sois reyes? No tenéis ni siquiera la libertad de las fieras. Ellas tienen concedido el amarse con amor verdadero, vosotros no. A las fieras se les ha concedido el ser creadoras como Dios. Ellas engendrarán hijos y verán a su gusto crecer la familia, vosotros no. A vosotros os ha sido negado este contento. ¿En razón de qué, pues, que seáis hombre y mujer, para tener que vivir de ese modo? Sed dioses. ¿No sabéis qué alegría supone el ser dos en una sola carne creadora de una tercera, de muchas otras terceras! No creáis en las promesas de Dios acerca del gozo de una descendencia viendo a vuestros hijos crearse nuevas familias, dejando por ellas padre y madre. Os ha dado un simulacro de vida. La verdadera vida está en conocer las leyes de la vida. Entonces seréis como dioses y podréis decirle a Dios: 'Somos tus iguales'". Y la seducción continuó, porque no hubo voluntad de interrumpirla, sino, más bien, de continuarla, y de conocer aquello que no le pertenecía al hombre. He aquí pues que el árbol prohibido vino a ser, para la raza, realmente mortal, porque de sus ramos pendía el fruto del amargo saber que venía de Satanás; y la mujer vino a ser hembra, y, con la levadura del conocimiento satánico en el corazón, fue a Adán a corromperlo. Humillada así la carne, corrompida la parte moral, degradado el espíritu, conocieron el dolor y la muerte: del espíritu privado de la Gracia; de la carne privada de la inmortalidad. Y la herida de Eva engendró el sufrimiento, que no se calmará hasta la extinción de la última pareja de la tierra. Yo recorrí en sentido inverso el camino de los dos pecadores. Obedecí. Obedecí en todos los modos. Dios me había pedido ser virgen. Obedecí. Habiendo amado la virginidad, que me hacía pura como la primera de las mujeres antes de conocer a Satanás, Dios me pidió ser esposa. Obedecí, llevando al matrimonio a la pureza que tuvo, a ese grado de pureza que Dios tenía en su pensamiento cuando creó a los dos Primeros. Convencida de mi destino de soledad en el matrimonio y de desprecio del prójimo por mi esterilidad santa, ahora Dios me pedía ser Madre. Obedecí. Creí que ello era posible y que esa palabra venía de Dios, porque la paz iba entrando en mí al oírla. No pensé: "Lo he merecido". No me dije a mí misma: "Ahora el mundo me admirará, porque soy semejante a Dios dando ser a la carne de Dios". No. Me anonadé en la humildad. La alegría brotó de mi corazón como un tallo de rosa florecida. Pero enseguida se adornó de punzantes espinas y quedó abrazada por la maraña del dolor, como esas ramas envueltas en campanillas de enredadera. El dolor del dolor de mi esposo: ésta era la angustia dentro de mi gozo. El dolor del dolor de mi Hijo: éstas eran las espinas de mi gozo. Eva quiso el disfrute, el triunfo, la libertad: yo acepté el dolor, el anonadamiento, la esclavitud. Renuncié a mi vida tranquila, a la estima de mi esposo, a la propia libertad. No me quedé con nada. Me hice la Esclava de Dios en la carne, en la parte moral, en el espíritu, confíándome a Él, no sólo respecto a la concepción virginal, sino también a la defensa de mi honor, a la consolación de mi esposo, al medio con que conducirlo a él también a la sublimación del matrimonio, de manera que los dos fuéramos quienes devolvieran al hombre y a la mujer la dignidad perdida. Abracé la voluntad del Señor por mí, por mi esposo, por mi Hijo. Dije "sí" por los tres, segura como estaba de que Dios no faltaría a su promesa de socorrerme en mi dolor de esposa que se ve juzgada culpable, en mi dolor de madre que ve que engendra para entregar a su Hijo al dolor. "Sí" dije. Sí, y basta. Ese "sí" ha anulado el "no" que Eva opuso al mandato divino. "Sí, Señor, como Tú quieras. Conoceré lo que Tú quieras. Viviré como Tú quieras. Estaré gozosa si Tú lo quieres. Sufriré por lo que Tú quieras. Sí, siempre sí, mi Señor, desde el momento en que tu rayo me hizo Madre hasta el momento en que me llamaste a ti. Sí, siempre sí. Todas las voces de la carne, todas las pasiones de lo moral, bajo el peso de este sí mío perpetuo. Y encima, como encima de un pedestal de diamante, mi espíritu, al cual le faltan las alas para volar a ti, pero es señor de todo el yo, domado y siervo tuyo, siervo en la alegría, siervo en el dolor. ¡Sonríe, oh Dios! ¡Alégrate! La culpa ha sido vencida, cancelada, destruida; yace bajo mi talón, ha sido lavada en mi llanto, destruida por mi obediencia. De mi seno nacerá el Árbol nuevo que dará el Fruto que conocerá todo el Mal por haberlo padecido en sí y dará todo el Bien. A éste sí podrán acercarse los hombres, y yo me sentiré feliz de que cojan de él, aunque no piensen que de mí nace. Con tal de que el hombre se salve y Dios sea amado, hágase de su esclava lo mismo que de la base de terreno en que un árbol crece: escalón para subir". María, hay que saber ser siempre escalón para que los demás suban a Dios. Si nos pisan, no importa, con tal de que logren ir a la Cruz. Es el nuevo árbol que posee el fruto del conocimiento del Bien y del Mal, porque le dice al hombre lo que está mal y lo que está bien, para que sepa elegir y vivir; y sabe, al mismo tiempo, hacer de sí elixir para curar a los que se han intoxicado con el mal que quisieron gustar. Nuestro corazón bajo los pies de los hombres, con tal de que el número de los redimidos crezca y que la Sangre de mi Jesús no sea derramada sin fruto. Este es el destino de las esclavas de Dios. Mas luego mereceremos recibir en nuestro seno la Hostia santa, y, a los pies de la Cruz, embebida en su Sangre y en nuestro llanto, decir: "He aquí, oh Padre, la Hostia inmaculada que te ofrecemos para salud del mundo. Míranos, oh Padre, fundidas con Ella, y por sus méritos infinitos danos tu bendición". Y yo te doy una caricia. Descansa, hija (María Valtorta). El Señor está contigo. }

Dice Jesús: 
\emph{Las palabras de mi Madre deberían disolver cualquier vacilación de pensamiento, incluso en los más atrapados por las fórmulas. Había dicho: "metafórico árbol"; ahora diré: "simbólico árbol". Quizás así entenderéis mejor. Su símbolo es claro: de cómo los dos hijos de Dios actuasen respecto a él, se comprendería la medida de su tendencia al Bien y al Mal. Cual agua regia que prueba el oro, cual balanza del orfebre que pesa los quilates del oro, ese árbol; que vino a ser una "misión" a causa del mandato divino respecto a él, dio la medida de la pureza del metal de Adán y de Eva. Llega ya a mis oídos vuestra objeción: "¿No fue excesiva la condena y pueril el medio que condujo a ella?". No lo fue. Una desobediencia actualmente en vosotros, que sois sus herederos, es menos grave de lo que lo fue en ellos. Vosotros estáis redimidos por Mí, pero el veneno de Satanás, como ciertos morbos que no desaparecen nunca totalmente de la sangre, está siempre pronto para reactivarse. Ellos, los dos progenitores, eran posesores de la Gracia sin haber tenido nunca el más mínimo contacto con la Desgracia. Por tanto, eran más fuertes, estaban más respaldados por esa Gracia que generaba inocencia y amor. Infinito era el don que Dios les había dado; mucho más grave, por tanto, su caída poseyendo ese don. También el fruto ofrecido, y comido, era simbólico. Era el fruto de una experiencia voluntariamente llevada a cabo por instigación satánica contra el imperativo de Dios. Yo no les había prohibido a los hombres el amor. Quería únicamente que se amaran sin malicia; de la misma forma que Yo los amaba con mi santidad, ellos habrían de amarse en santidad de afectos, de afectos limpios de toda libídine. No se debe olvidar que la Gracia es foco de luz, y, que quien la posee conoce aquello que es útil y bueno conocer. La Llena de Gracia conoció todo, porque la Sabiduría la instruía (la Sabiduría, que es Gracia), y supo guiarse a sí misma santamente. Eva conocía, por tanto, aquello que le era bueno conocer; no más de eso. Porque es inútil conocer lo que no es bueno. No tuvo fe en las palabras de Dios y no fue fiel a su promesa de obediencia. Prestó fe a Satanás, infringió la promesa, quiso conocer lo no bueno, lo amó sin remordimiento, transformó en cosa corrompida, envilecida, ese amor que Yo había otorgado tan santo. Ángel caído, se revolcó en barro y paja, mientras que podía haber corrido dichosa entre las flores del Paraíso Terrenal y ver florecer a su alrededor la prole, de la misma forma que un árbol se cubre de flores sin combar su copa y meterla en el pantano. No seáis como esos niños estúpidos de que hablo en el Evangelio, los cuales oían cantar y se tapaban los oídos, oían tocar y no bailaban, oían llorar y querían reír. No seáis mezquinos ni negadores. Aceptad la Luz, aceptadla sin malicia, sin testarudez, sin ironía o incredulidad. Y ya basta sobre esto. Para que entendáis cuánto debéis sentiros agradecidos a Aquel qué murió para levantaros y orientaros de nuevo al Cielo y para vencer la concupiscencia de Satanás, he querido hablaros, en este tiempo de preparación a la Pascua, de este primer eslabón de la cadena con que el Verbo del Padre, el Cordero Divino, fue llevado a la muerte, al matadero. Os he querido hablar de ello porque al presente el noventa por ciento de vosotros está, como Eva, intoxicado por el hálito y por la palabra de Lucifer, y no vivís para amaros sino para saciaros de sensualidad, no vivís para el Cielo sino para el barro; ya no sois criaturas dotadas de alma y razón, sino perros sin alma y sin razón. Habéis matado el alma, habéis depravado la razón. En verdad os digo que las bestias, en sus amores, son más honestas que vosotros. }

\chapter*{María anuncia a José la maternidad de Isabel \\ \normalfont\normalsize\textit{y confía a Dios la justificación de la suya.}}
\addcontentsline{toc}{chapter}{\normalfont\scshape{María anuncia a José la maternidad de Isabel}}

Ante mi vista la casita de Nazaret, y María dentro, jovencita, como cuando el Ángel de Dios se le apareció. El solo hecho de ver, ya me llena el alma del perfume virginal de esa morada; del perfume angélico aún presente en esa estancia en que el Ángel agitó sus alas de oro; del perfume divino, que se ha concentrado enteramente en María para hacer de Ella una Madre y que ahora de Ella revierte. 

Las sombras empiezan a invadir la estancia a la que antes había descendido tanta luz de Cielo. Está anocheciendo. 

María, de rodillas al lado de su lecho, ora con las manos cruzadas sobre el pecho y con el rostro muy inclinado hacia el 

suelo. Lleva el mismo vestido del momento del Anuncio. Todo está como entonces. La ramita florecida en su jarrón, los muebles en el mismo orden. La única variación es que la rueca y el huso están apoyados en un rincón: con su penacho de estambre, aquélla; con su brillante hilo envuelto en torno, éste. 

María deja de rezar y se pone en pie, con el rostro encendido como por una llama. La boca sonríe, pero el llanto hace brillar sus ojos azules. Coge la lámpara de aceite y con una piedra de chispa la enciende. Mira si todo está ordenado en la habitación. Endereza la cobija de la cama, que se había torcido. Añade agua al jarrón de la ramita florecida y le saca de la habitación, al fresco de la noche. Luego entra otra vez. Coge el bordado que estaba doblado encima del mueble de anaqueles, y la lámpara encendida, y, cerrando la puerta, sale. 

Da unos pasos por el huertecillo bordeando la casa, luego entra en la habitación donde vi que Jesús se despidió de María. La reconozco, a pesar de que falten ahora algunos objetos del mobiliario que entonces había. María se marcha a otra pequeña habitación cercana a ésta, llevando la lámpara consigo, y yo me quedo, me quedo con la sola compañía de su labor depositada en la esquina de la mesa. Oigo ir y venir el paso leve de María; le oigo agitar agua, como quien estuviera lavando algo. Luego, romper unas ramitas. Comprendo que se trata de leña rota por el sonido que hace. Oigo que enciende el fuego. 

Vuelve. Sale al jardincito. Vuelve a entrar; trae unas manzanas y verdura. Deja las manzanas en la mesa, en una bandeja de metal grabado (creo que se trata de cobre burilado). Vuelve a la cocina (está claro que allí está la cocina). Ahora la llama de la lumbre se proyecta alegre desde la puerta abierta hasta aquí dentro, representando una danza de sombras en las paredes. 

Pasa un rato y María regresa con un pan pequeño y oscuro y un cuenco de leche caliente. Se sienta. Moja unas rodajas de pan en la leche. Come tranquila y despacio. Luego, dejando la mitad del tazón de leche, entra de nuevo en la cocina y vuelve con las verduras, les echa un poco de aceite y se las come con el pan. Para la sed, bebe la leche. Luego coge una manzana y se la come. Una cena de niña. 

María piensa mientras come, y sonríe ante un íntimo pensamiento. Levanta la mirada, recorre con ella las paredes; parece como si les comunicase un secreto suyo. De vez en cuando, sin embargo, se pone seria, casi triste; pero luego le torna la sonrisa. 

Se oye llamar a la puerta. María se levanta y abre. Entra José. Se saludan. José se sienta en un taburete, de la otra parte de la mesa, frente a María. 

José es un hombre apuesto, en la plenitud de la vida. Tendrá unos treinta y cinco años como mucho. Su pelo castaño oscuro y su barba del mismo color le enmarcan un rostro proporcionado con dos dulces ojos castaños casi negros. Su frente es amplia y lisa; su nariz, delgada, ligeramente arqueada; carrillos más bien llenos, de un moreno no aceitunado, incluso rosado en los pómulos. No es muy alto, sí de complexión fuerte y bien proporcionado. 

Antes de sentarse se ha quitado el manto, que — es el primero que veo hecho de esa manera — es circular y se lleva sujeto al cuello con un ganchito o algo parecido, y tiene capucha. Es de color marrón claro y parece hecho de una tela impermeable de lana basta. Parece un manto de montañés, bueno para resguardar de las inclemencias del tiempo. 

También antes de sentarse, le ha ofrecido a María dos huevos y un racimo de uvas, un poco arrugadas pero bien conservadas. Y sonríe diciendo: 

Me las han traído de Cana. Los huevos me los ha dado el Centurión por un trabajo que le hice a un carro suyo — se había roto una rueda y el que trabaja para ellos estaba enfermo... —. Son frescos. Los ha cogido de su gallinero. Bébetelos. Te vendrán bien.- 

Mañana, José. Acabo de comer. 

Las uvas sí te las puedes comer. Son buenas. Dulces como la miel. Las he traído despacio para no estropearlas. Cómetelas. Tengo más. Te las traigo mañana en una cesta. Esta noche no podía porque vengo directamente de casa del Centurión. 

Entonces, no has cenado todavía». 

No. Pero no importa. 

María se levanta inmediatamente y va a la cocina. Vuelve con leche, aceitunas y queso. 

No tengo otra cosa- dice - Cómete un huevo. 

José no quiere. Los huevos son para María. Come con gusto su pan con queso y se bebe la leche, que está todavía tibia. Luego acepta una manzana. La cena ha terminado. 

María coge su bordado — primero ha despejado la mesa de las cosas de la cena con la ayuda de José, que se ha quedado en la cocina incluso cuando Ella vuelve aquí. Le oigo mover las cosas poniendo todo en su sitio. Atiza el fuego de nuevo porque la noche está fresca. Cuando vuelve, María le da las gracias. 

Se ponen a hablar. José cuenta cómo ha pasado el día. Habla de sus sobrinitos. Se interesa por el trabajo de María y por sus flores. Le promete que le traerá unas flores muy bonitas que el Centurión le ha ofrecido. - Nosotros no tenemos esas flores. Las han traído de Roma. Me ha prometido que, apenas hayan germinado, me dará las plantas. Ahora, cuando la Luna sea propicia, te las planto. Tienen colores bonitos y un perfume muy bueno. Las he visto el verano pasado, porque florecen en verano. Perfumarán toda tu casa. Los árboles los podaré más tarde, con la Luna favorable. Es ése el momento. 

María sonríe y de nuevo le da las gracias. Silencio. José fija su mirada en la rubia cabeza de María inclinada hacia su trabajo de bordado. Es una mirada de amor angelical. Sin duda alguna, si un ángel amara a una mujer con amor de esposo, la miraría así. 

María, como quien hubiese tomado una decisión, pone en su regazo el bordado y dice: José, yo también tengo algo que decirte. Nunca recibo nada, pues tú sabes qué retirada vivo. Pero, hoy he recibido una noticia. He tenido noticia de que nuestra parienta Isabel, mujer de Zacarías, va a tener pronto un hijo... José abre enormemente los ojos y dice: 

- ¿A su edad? 

A su edad - responde sonriendo María - El Señor todo lo puede, y ahora ha querido darle esta alegría a nuestra parienta. 

- ¿Cómo lo has sabido? ¿Es segura esta noticia? 

Ha venido un mensajero; y es uno que no puede mentir. Yo quisiera ir donde Isabel, para servirla y decirle que exulto con ella. Si tú lo permites... 

María, tú eres mi señora y yo tu siervo. Todo lo que haces está bien hecho. ¿Cuándo quisieras partir? 

Lo antes posible. Pero estaré fuera algunos meses. 

Y yo contaré los días esperándote. Ve tranquila. Me ocuparé de la casa y de tu huertecito. Cuando vuelvas encontrarás tus flores tan bonitas como si tú misma las hubieras estado cuidando. Sólo una cosa... Espera. Antes de la Pascua tengo que ir a Jerusalén, para comprar unas cosas para mi trabajo. Si esperas unos días, te acompaño hasta allí; no más lejos, porque debo volver rápidamente; pero hasta allí podemos ir juntos. Estoy más tranquilo si no pienso que vas sola por los caminos. Para la vuelta, házmelo saber, y así saldré a tu encuentro. 

Eres muy bueno, José. Que el Señor te recompense con sus bendiciones y mantenga lejos de ti el dolor. Le pido siempre por esto. 

Los dos castos esposos se sonríen angelicalmente. Silencio de nuevo durante un tiempo. 

Luego José se pone en pie. Se pone el manto, se pone la capucha, se despide de María, que también se ha levantado, y sale. 

María le sigue con la mirada y con un suspiro como de pena. Luego levanta los ojos al cielo. Está, sin duda, orando. Cierra la puerta con cuidado. Dobla el bordado. Va a la cocina. Apaga, o cubre, la lumbre. Mira a ver si todo está como debe. Coge la lámpara y sale, cerrando la puerta. Con su mano protege la llamita, temblorosa en el viento fresquito de la noche. Entra en su habitación y sigue orando. 

La visión cesa así. 

Dice María: 
\emph{Hija mía querida, cuando, terminado el éxtasis que me había henchido de inefable alegría, regresé a los sentidos de la Tierra, el primer pensamiento que, punzante como espina de rosas, hirió mi corazón envuelto en las rosas del Divino Amor, desposado conmigo unos instantes antes, fue José. Yo ya amaba entonces a este santo y providente custodio mío. Desde el momento en que la voluntad de Dios, a través de la palabra de su Sacerdote, quiso que fuera esposa de José, pude ir conociendo y apreciando la santidad de este Justo. Unida a él, sentí cesar mi estado de desorientación por mi orfandad, y dejé de añorar el perdido amparo del Templo. Él era tan dulce como el padre que había perdido. Junto a él me sentía tan segura como junto al Sacerdote. Toda vacilación había cesado; es más, había quedado olvidada — efectivamente, mucho se habían alejado de mi corazón de virgen las vacilaciones, porque había comprendido que no tenía motivo alguno de vacilar, que no tenía nada que temer respecto a José —. Mi virginidad, confiada a José, estaba más segura que un niño en brazos de su madre. ¿Cómo decirle ahora que era Madre? Trataba de encontrar las palabras con que anunciárselo. Difícil búsqueda. No quería yo, en efecto, alabarme por el don divino recibido, y no podía justificar mi maternidad en ningún modo sin decir: "El Señor me ha amado entre todas las mujeres y de mí, su sierva, ha hecho su Esposa". Tampoco quería engañarle, ocultándole mi estado. Pero, mientras oraba, el Espíritu que me llenaba me había dicho: "Guarda silencio. Déjame a mí la tarea de justificarte ante tu esposo". ¿Cuándo? ¿Cómo? No lo había preguntado. Siempre me había abandonado en Dios, como una flor se abandona a la ola que la lleva. Jamás el Eterno me había dejado sin su ayuda. Su mano me había sujetado, protegido, guiado hasta aquí; esta vez, pues, también lo haría. Hija mía, ¡qué hermosa y confortante es la fe en nuestro eterno y buen Dios! Nos pone entre sus brazos como si fueran una cuna; nos lleva, como una barca, al radiante puerto del Bien; da calor a nuestro corazón, nos consuela, nos nutre, nos proporciona descanso y júbilo, nos ilumina y nos guía. La confianza en Dios lo es todo, y Dios da todo a quien tiene confianza en Él: se da El mismo. Aquella tarde llevé hasta la perfección mi confianza de criatura. Ahora podía hacerlo, porque Dios estaba en mí. Antes, mi confianza era la de una pobre criatura como era; siempre una nada, aunque fuera la Tan Amada que era la Sin Mancha. Pero ahora poseía la confianza divina porque Dios era mío: ¡mi Esposo, mi Hijo! ¡Oh, gran gozo! Ser Una con Dios. No para gloria mía, sino para amarle en una unión total y poderle decir: "Tú, Tú solo, que estás en mí, actúa con tu divina perfección en todas las cosas que yo haga". Si Él no me hubiera dicho: "¡Calla!", quizás habría osado, con el rostro en tierra, decirle a José: "El Espíritu ha penetrado en mí y llevo la Semilla de Dios". Él me habría creído, porque me estimaba y además porque, como todos los que nunca mienten, no podía creer que otro mintiera. Sí, con tal de no causarle un dolor subsiguiente, yo habría vencido la reticencia a proporcionarme a mí misma esa alabanza. Mas, presté obediencia al mandato divino. A partir de ese momento, y durante meses, sentí esa primera herida que me ensangrentaba el corazón. Ese fue el primer dolor de mi destino de Corredentora. Lo ofrecí y lo sufrí para expiar, y para daros una norma de vida en momentos análogos a éste, de sufrimiento por deber guardar silencio o por un hecho que da una mala imagen de vosotros a quien os ama. Confiadle a Dios la tutela de vuestro buen nombre y de vuestros intereses afectivos. Mereced, con una vida santa, la tutela de Dios, y... caminad seguros. Podrá el mundo entero ponerse en contra de vosotros; Él os defenderá ante quien os ama, y hará brillar la verdad.}

\chapter*{María y José camino de Jerusalén.}
\addcontentsline{toc}{chapter}{\normalfont\scshape{María y José camino de Jerusalén.}}

Asisto al momento de la partida para ir donde Sta. Isabel. José ha venido a recoger a María con dos borriquillos grises: uno para él, el otro para María. Los dos animalitos llevan la acostumbrada albardilla; una de ellas agrandada, por un arnés, que sólo luego comprendo que ha sido hecho para llevar la carga (es una especie de portaequipajes), sobre el cual José asegura una pequeña arca de madera — un pequeño baúl, diríamos ahora — que le ha traído a María para que pueda colocar en ella sus indumentos sin peligro de que el agua los moje. 

Le oigo a María agradecer mucho a José este regalo providente, donde ordena todo lo que llevaba en un talego que había preparado antes. 

Cierran la puerta de casa y se ponen en camino. Está naciendo el día; efectivamente, veo que la aurora tenuemente empieza a rosear a Oriente. Nazaret duerme todavía. Los dos viajeros madrugadores encuentran en su camino únicamente a un pastor, el cual va arreando a las ovejas para que avancen; y las ovejas van trotando, chocándose unas contra otras balando. Los corderitos son los que más balan, con sonido agudo y ligero; quisieran buscar, incluso mientras caminan, la mama materna. Pero las madres van deprisa al pasto y los invitan con su balido, más fuerte, a que también troten. 

María mira y sonríe. Se ha detenido para dejar pasar al rebaño, y se inclina desde su albardilla y acaricia a estos mansos animalitos que pasan rozando al borriquillo. Cuando llega el pastor, con un corderillo recién nacido en sus brazos, y se para saludar, María ríe acariciando en el morrito rosado al corderito, que bala como un desesperado, y dice: 

- Está buscando a su mamá. Ésta es la mamá, aquí está. No te abandona, no, pequeñuelo. Efectivamente, la oveja madre se restriega contra el pastor y se pone de manos para lamer en el morrito a su hijo. 

Pasa el rebaño con rumor de agua entre frondas, dejando tras sí el polvo que han levantado las veloces pezuñitas, y todo un bordado de pisadas sobre la tierra del camino. 

José y María reanudan la marcha. José lleva su capa; María va arropada con una especie de toquilla de rayas porque la mañana está muy fresca. 

Ya están en el campo y van el uno al lado del otro. Hablan raras veces. José piensa en sus asuntos y María sigue sus propios pensamientos, y, recogida en sí, sonríe ante éstos y ante las cosas cuando, saliendo de su concentración, dirige la mirada hacia lo que la rodea. De vez en cuando mira a José, y un velo de seriedad triste le nubla la cara; luego le torna la sonrisa, incluso al mirar a este esposo suyo providente, que habla poco pero que si lo hace es para preguntarle si va cómoda y si no necesita nada. 

Ahora ya han afluido otras personas a los caminos, especialmente en las cercanías de algún pueblo o dentro de él. Pero ninguno de los dos hace mucho caso de las personas que se cruzan con ellos. Van en sus burritos trotadores en medio de un gran rumor de cascabeles. Se detienen sólo una vez, a la sombra de un bosquecillo, para comer un poco de pan y aceitunas y beber en una fuente que baja de una cuevecilla, y, otra vez, para protegerse de un chaparrón violento que rompe al improviso de un nubarrón oscurísimo. 

Están al amparo del monte, contra un saliente de una roca que los protege de lo más intenso del agua. Pero José quiere a toda costa que María se ponga su capa de lana impermeable, por la que el agua resbala sin mojar. María se ve obligada a ceder ante la premurosa insistencia de su esposo, el cual para tranquilizarla en lo que toca a su propia inmunidad, se pone sobre la cabeza y sobre los hombros una mantita parda que cubría la albardilla. La manta del burro probablemente. Ahora María, enmarcada su cara con la capucha y cubierta por entero con la capa marrón que lleva sujeta al cuello, parece un frailecito. 

El chaparrón amaina, aunque se transforma en una lluvia fastidiosa y fina. Los dos reanudan la marcha por el camino todo lleno de barro. De todas formas, es primavera, y, pasado un poco de tiempo, torna el sol a hacer más cómoda la marcha. 

Los dos burritos trotan de mejor gana por el camino. 

No veo nada más porque la visión cesa aquí. 

\chapter*{Salida de Jerusalén. \\ \normalfont\normalsize\textit{El aspecto beatífico de María. Importancia de la oración para María y José.}}
\addcontentsline{toc}{chapter}{\normalfont\scshape{Salida de Jerusalén.}}

Estamos en Jerusalén. La conozco bien ya con sus calles y sus puertas. 

Los dos esposos lo primero que hacen es dirigirse hacia el Templo. Reconozco la cuadra donde José dejó el burro el día de la Presentación en el Templo. También ahora deja allí — primero les ha dado de comer — a los dos burros, y con María va a adorar al Señor. 

Salen. Van a una casa de personas conocidas según parece; allí comen y beben algo. María se pone a descansar hasta que vuelve José con un viejecillo. 

- Este hombre va por el mismo camino que tú. Deberás recorrer bien poco camino sola para llegar donde tu parienta. Fíate de él, que le conozco. 

Vuelven a subirse a los burros. José acompaña a María hasta la Puerta (no la puerta por la que entraron; otra) y allí se despiden... 

María va sola con el viejecillo, que habla por todo lo que no hablaba José, y que se interesa de mil cosas. María contesta pacientemente. 

Ahora, en la parte de delante de la albardilla lleva el baulillo (hasta entonces lo había llevado siempre José en su burrito), y ya no tiene la capa; tampoco lleva su toquilla, la cual está ahora doblada encima del baúl. Está guapísima con su vestido azul oscuro y con su velo blanco que la protege del sol. ¡Qué guapa está! 

El viejecillo debe ser un poco sordo, porque, para que la oyera, María ha tenido que hablar bien fuerte; Ella, que habla siempre bajo. Ahora está ya cansado; ha agotado todo su repertorio de preguntas y de noticias y se ha quedado transpuesto sobre el burro, dejándose guiar por él, que conoce bien el camino. 

María aprovecha esta tregua para recogerse en sus pensamientos y para orar. Debe ser una oración la que Ella va cantando en voz baja, mirando al cielo azul y con los brazos sobre el pecho y con rostro iluminado y beato por la emoción interior. 

No veo más cosas. 

Y también ahora, cuando la visión se me detiene, como ayer, queda presente conmigo la Madre, tan nítidamente visible a mi interna vista, que le puedo describir el color rosado tenue del carrillo que bien poco tiene de grueso y sí de dulcemente blando; le puedo describir el rojo vivo de su pequeña boca y el brillar dulce de sus ojos azulinos entre el rubio oscuro de las pestañas. 

Le puedo decir cómo sus cabellos, divididos por el medio de la cabeza, caen esponjosos con tres ondulaciones por cada parte hasta tapar la mitad de sus pequeñas orejas rosadas, y desaparecen con su oro pálido y brillante bajo el velo que le cubre la cabeza (en efecto, la veo cubierta con su manto, vestida con su vestido de seda paradisíaca, y con su manto fino como un velo, aunque opaco, de la misma tela que el vestido). 

Le puedo decir que su vestido está como ceñido al cuello por una vaina atravesada por un cordón cuyos extremos se anudan por delante en la base del cuello; y que el vestido está recogido en torno a la cintura por un cordón más grueso, también de seda blanca, del que penden lateralmente dos borlas. 

Le puedo incluso decir que el vestido, estando ceñido al cuello y a la cintura, forma sobre el pecho siete pliegues ondulados y esponjosos, único ornato del castísimo indumento. 

Le puedo expresar la castidad que emana de todo el aspecto de María, de esas formas suyas tan delicadas y armoniosas que la hacen tan angélicamente mujer. 

Y, cuanto más la miro, más sufro pensando en cuánto la hicieron sufrir, y me pregunto cómo pudieron no tener piedad de Ella, tan mansa y gentil, tan delicada incluso en su aspecto físico. Mirándola, llegan de nuevo a mis oídos todos los gritos del Calvario — que también iban contra Ella —, todos los escarnios y burlas, todas las maldiciones por ser la Madre del Condenado. La veo bella y tranquila, ahora; pero, su aspecto actual no me borra el recuerdo de su trágico rostro de aquellas horas de agonía, ni el de su rostro desolado en la casa de Jerusalén, después de la muerte de Jesús. Y quisiera poderla acariciar y besarle esa mejilla tan delicadamente rosada y suave, para hacer desaparecer con mi beso ese recuerdo de llanto que, igual que en mí, ciertamente está en Ella. 

No puede imaginarse qué paz me da el tenerla cerca. Creo que morir viéndola tiene que ser tan dulce como la más dulce hora de vida; más dulce aún. Durante este tiempo en que no la veía así — toda para mí — he sufrido su ausencia como se sufre por la ausencia de una madre. Experimento de nuevo la inefable alegría que me acompañó en el mes de diciembre y al principio de enero. Y me siento feliz. Feliz, a pesar de que el haber visto el suplicio de la Pasión extienda un velo de dolor sobre toda dicha mía. 

Es difícil decir y hacer comprender lo que siento y lo que se ha producido desde el 11 de febrero, desde la tarde en que vi sufrir a Jesús en su Pasión. Ha sido una visión que me ha cambiado radicalmente. Ya muriese ahora, ya dentro de cien años, esa visión permanecería siempre igual en su intensidad y en sus efectos. Antes pensaba en los dolores de Cristo; ahora los vivo, porque me basta una palabra, una mirada a una imagen, para volver a sufrir cuanto sufrí aquella tarde y para horrorizarme ante aquellos suplicios y angustiarme por aquel padecimiento suyo desolado; y, aunque nada lo recuerde, el recuerdo y su suplicio están vivos en mí. 

María empieza a hablar y yo me callo. 

Dice María: 
\emph{ Voy a hablar poco porque estás muy cansada, pobre hija mía. Sólo quiero que pongas — como también quien lee — tu atención en la costumbre constante de José y mía de reservar siempre el primer puesto a la oración. Ni el cansancio ni la prisa ni los pesares ni las ocupaciones impedían la oración; antes al contrario, la favorecían. Era siempre la reina de nuestras ocupaciones. Nuestro refrigerio, nuestra luz, nuestra esperanza. Si en las horas tristes era consuelo, en las felices canto; pero siempre, la amiga constante de nuestra alma: era la que nos desligaba de la tierra, del destierro, y nos mantenía en suspensión hacía el Cielo, la Patria. No sólo yo — que ya tenía dentro de mí a Dios y me bastaba con mirarme dentro para adorar al Santo de los santos — me sentía unida a Dios cuando oraba, sino que también lo sentía José, porque nuestra oración era adoración verdadera de todo el ser, que se fundía con Dios adorándole y recibiendo a su vez su abrazo. Fijáos que ni siquiera yo, que ya tenía en mí al Eterno, me sentí exenta de prestar veneración al Templo. La más alta santidad no exime de sentirse una nada respecto a Dios y de humillar esta nada, puesto que Él nos lo permite, en un continuo grito de júbilo a su gloria. ¿Sois débiles, pobres, imperfectos? Invocad la santidad del Señor: "¡Santo, Santo, Santo!". Invocad al Santo bendito para que socorra vuestra miseria. Vendrá, transfundiéndoos su santidad. ¿Sois santos, ricos de méritos ante sus ojos? Invocad igualmente la santidad del Señor, la cual, siendo infinita, aumentará cada vez más la vuestra. Los ángeles, seres que están por encima de las debilidades de la humanidad, no cesan un instante de cantar su "Sanctus", y su belleza sobrenatural crece con cada acto de invocación de la santidad de nuestro Dios. Imitad, pues, a los ángeles. No os despojéis nunca del amparo de la oración. Contra ella se despuntan las armas de Satanás, las malicias del mundo, los apetitos de la carne, las soberbias de la mente. No bajéis jamás esta arma, por la cual los Cielos se abren, lloviendo así gracias y bendiciones. La tierra tiene necesidad de un baño de oraciones para purificarse de las culpas que atraen los castigos de Dios. Y, dado que pocos oran, esos pocos deben orar como si fueran muchos, multiplicar sus oraciones vivas para obtener con ellas esa suma necesaria para conseguir gracia; y las oraciones viven cuando están sazonadas con verdadero amor y sacrificio. Que tú, hija, sufras, además de por tu sufrimiento, por el mío y el de mi Jesús, es bueno, es meritorio y grato a Dios. Tengo en gran estima tu amor compasivo. ¿Querías besarme? Besa las llagas de mi Hijo. Úngelas con el bálsamo de tu amor. Yo sentí espiritualmente el agudo dolor de los azotes y de las espinas y la tortura de los clavos y de la cruz. Mas, de la misma forma, siento espiritualmente todas las caricias hechas a mi Jesús, y son otros tantos besos que yo recibo. Bueno, ven de todas formas; verdad es que soy la Reina del Cielo, pero sigo siendo la Madre... Y yo me siento bendecida. }

\chapter*{La llegada de María a Hebrón \\ \normalfont\normalsize\textit{y su encuentro con Isabel.}}
\addcontentsline{toc}{chapter}{\normalfont\scshape{La llegada de María a Hebrón}}

Me encuentro en un lugar montañoso. No son grandes montañas, pero tampoco puede decirse que sean simples colinas. Tienen cimas y sinuosidades ya propias de las verdaderas montañas, como las que se ven en nuestros Apeninos toscoumbrianos. La vegetación es tupida y bonita. Abunda el agua fresca que mantiene verdes los pastos y fértiles los huertos, casi todos plantados de manzanos, higueras y vid; esta última, en torno a las casas. Debe ser primavera, como se deduce de que las uvas sean ya de un cierto volumen, como semillas de veza; y de que las flores de los manzanos asemejen a numerosas bolitas de color verde intenso; así como del hecho de que en lo alto de las ramas de las higueras hayan aparecido ya los primeros frutos, todavía en estado embrional, pero ya bien definidos. Y los prados son una verdadera alfombra esponjosa y de mil colores en que pacen, o descansan, las ovejas: manchas blancas sobre el fondo de esmeralda de la hierba. 

María sube en su burrito por una vía que está en bastante buen estado, y que debe ser de primer orden. Sube, porque, efectivamente, el pueblo, de aspecto bastante ordenado, está más arriba. Mi interno consejero me dice: 

- Este lugar es Hebrón». Usted me hablaba de Montana. Yo no sé qué hacer. A mí se me indica con este nombre. No sé si será «Hebrón» toda la zona o sólo el pueblo. Yo oigo esto, y esto es lo que digo. 

María está entrando en el pueblo. Atardece. Algunas mujeres, en las puertas de las casas, observan la llegada de la forastera y chismean entre sí. La siguen con la mirada y no se quedan tranquilas hasta que la ven detenerse delante de una de las casas más lindas, situada en el centro del pueblo y que tiene delante un huerto- jardín, y detrás y alrededor un huerto de árboles frutales bien cuidado, que se extiende luego dando lugar a un vasto prado que sube y baja por las sinuosidades del monte, para terminar en un bosque de altos árboles, tras el cual no sé qué más hay. Todo ello cercado por un seto de morales o rosales silvestres. No lo distingo bien porque — no sé si usted lo tiene presente — tanto la flor como el ramaje de estas matas espinosas son muy semejantes, y mientras no aparece el fruto en las ramas es fácil confundirse. En la parte delantera de la casa, es decir, por el lado paralelo al pueblo, la propiedad está cercada por un pequeño muro blanco, a lo largo de cuya parte alta hay ramas de verdaderos rosales, todavía sin flores, aunque ya llenas de capullos. En el centro, una cancilla de hierro, cerrada. Se comprende que se trata de la casa de una de las personalidades del pueblo, y de gente que vive desahogadamente, pues, efectivamente, todo en ella da signos, si no de riqueza y de pompa, sí, sin duda, de bienestar. Y mucho orden. 

María se baja del burrito y se acerca a la puerta de hierro. Mira por entre las barras. No ve a nadie. Entonces trata de que la oigan. Una mujercita (la más curiosa de todas, que la ha seguido) le hace señales para que se fije en un extraño objeto que sirve para llamar: dos piezas de metal dispuestas en equilibrio en una especie de yugo, las cuales, moviendo el yugo con una gruesa cuerda, chocan entre sí haciendo el sonido de una campana o de un gong. 

María tira de la cuerda, pero lo hace de forma tan delicada que el sonido es sólo un ligero tintineo que nadie oye. Entonces la mujercita, una viejecilla toda ella nariz y barbilla puntiaguda, y con una lengua que vale por diez juntas, se agarra a la cuerda y se pone a tirar, a tirar, a tirar. Una llamada que despertaría a un muerto. 

Se hace así, mujer. Si no, ¿cómo va a querer que la oigan? Sepa que Isabel es anciana, y también Zacarías. Y ahora, además de sordo, está mudo. Los dos sirvientes son también viejos, ¿sabe? ¿Ha venido alguna otra vez? ¿Conoce a Zacarías? ¿Es usted...? 

Aparece un viejecillo renco que salva a María de este diluvio de informaciones y preguntas. Debe ser jardinero o labrador. Lleva en la mano un pequeño rastrillo y una hoz atada a la cintura. Abre. María entra mientras le da las gracias a la mujer, pero... ¡ay!, la deja sin respuesta. ¡Qué desilusión para la curiosa! 

Nada más entrar, dice: 

Soy María de Joaquín y Ana, de Nazaret. Prima de vuestros señores. E1 viejecillo inclina la cabeza y saluda, luego da una voz: - ¡Sara! ¡Sara! 

Y abre otra vez la verja para coger el borriquillo, que se había quedado afuera porque María, para librarse de la pegajosa mujercita, se había colado dentro muy rápida, y el jardinero, tan rápidamente como Ella, había cerrado la verja delante de las narices de la chismosa. Pasa al burro y, mientras lo hace, dice: 

- ¡Ah... gran dicha y gran desgracia para esta casa! El Cielo ha concedido un hijo a la estéril. ¡Bendito sea por ello el Altísimo! Pero Zacarías volvió de Jerusalén mudo hace ya siete meses. Se hace entender con gestos, o escribiendo. ¿Ha tenido noticia de ello? Mi señora, en medio de esta alegría y este dolor, la ha echado mucho de menos. Siempre hablaba de usted con Sara. Decía: "¡Si estuviese aquí conmigo mi pequeña 

María... ! Si hubiera seguido hasta ahora en el Templo, habría enviado a Zacarías a traerla. Pero el Señor ha querido que fuese la esposa de José de Nazaret. Sólo Ella podría consolarme en este dolor y ayudarme a rezar a Dios, porque todo en Ella es bondad. En el Templo todos la echan de menos y están tristes. La pasada fiesta, cuando fui con Zacarías la última vez a Jerusalén a dar gracias a Dios por haberme dado un hijo, oí de sus maestras estas palabras: "Al Templo parecen faltarle los querubines de la Gloria desde que la voz de María no suena ya entre estas paredes". ¡Sara! ¡Sara! Mi mujer es un poco sorda. Ven, ven, que te llevo yo». 

En vez de Sara, aparece, en la parte alta de una escalera adosada a un lado de la casa, una mujer ya muy anciana, ya llena de arrugas, con el pelo muy canoso — pero que ha debido ser negrísimo, a juzgar por lo negras que tiene las pestañas y las cejas y por el color moreno de su cara —. Contrasta en modo extraño, con su visible vejez, su estado, ya muy patente, a pesar de la ropa amplia y suelta que lleva. Mira protegiéndose los ojos de la luz con la mano. Reconoce a María. Levanta los brazos hacia el cielo con una exclamación de asombro y de alegría, y se apresura, en la medida en que puede, hacia abajo al encuentro de la recién llegada. Y María — cuyos movimientos son siempre moderados — esta vez se echa a correr rápida como un cervatillo y llega al pie de la escalera al mismo tiempo que Isabel. Y recibe en su pecho con viva efusión de afecto a su prima, que, al verla, llora de alegría. 

Permanecen abrazadas un momento. Luego Isabel se separa con una exclamación de dolor y alegría al mismo tiempo, y se lleva las manos al abultado vientre. Agacha la cabeza, palideciendo y sonrojándose alternativamente. María y el sirviente extienden los brazos para sujetarla, pues ella vacila como si se sintiera mal. 

Pero Isabel, después de un minuto de estar como recogida dentro de sí, alza su rostro, tan radiante que parece rejuvenecido, mira a María sonriendo con veneración como si estuviera viendo un ángel y se inclina en un intenso saludo diciendo: 

- ¡Bendita tú entre todas las mujeres! ¡Bendito el Fruto de tu vientre! (lo dice así, dos frases bien separadas) ¿Cómo he merecido que venga a mí, sierva tuya, la Madre de mi Señor? Sí, ante el sonido de tu voz, el niño ha saltado en mi vientre como jubiloso, y cuando te he abrazado el Espíritu del Señor me ha dicho una altísima verdad en el corazón. ¡Dichosa tú, porque has creído que a Dios le fuera posible lo que posible no aparece a la humana mente! ¡Bendita tú, que por tu fe harás realidad lo que te ha sido predicho por el Señor y fue predicho a los Profetas para este tiempo! ¡Bendita tú, por la Salud que engendras para la estirpe de Jacob! ¡Bendita tú, por haber traído la Santidad a este hijo mío que siento saltar de júbilo en mi vientre como cabritillo alborozado porque se siente liberado del peso de la culpa, llamado a ser el precursor, santificado antes de la Redención por el Santo que se está desarrollando en ti! 

María, con dos lágrimas como perlas, que le bajan desde los risueños ojos hasta la boca sonriente, el rostro alzado hacia el cielo, levantados también los brazos, en la posición que luego tantas veces tendrá su Jesús, exclama: 

El alma mía magnifica a su Señor – y continúa el cántico como nos ha sido transmitido. Al final, en el versículo: «Ha socorrido a Israel, su siervo etc», recoge las manos sobre el pecho y se arrodilla muy curvada hacia el suelo adorando a Dios. 

El sirviente, cuando había visto que Isabel no se sentía mal y que quería manifestar su pensamiento a María, se había retirado prudentemente; ahora vuelve del huerto acompañado de un anciano de aspecto majestuoso, de barba y pelo enteramente blancos, el cual, con vistosos gestos y sonidos guturales, saluda desde lejos a María. 

Zacarías está llegando - dice Isabel tocando en el hombro a la Virgen, que está orando absorta - Mi Zacarías está mudo. Está bajo sanción divina por no haber creído. Ya te contaré luego. Ahora espero en el perdón de Dios porque has venido tú; tú, llena de Gracia. 

María se levanta. Va hacia Zacarías. Se inclina hasta el suelo ante él. Le besa la orla de la vestidura blanca que le cubre hasta los pies. Esta vestidura es muy amplia y está sujeta a la cintura por una ancha franja bordada. 

Zacarías, con gestos, da la bienvenida a María, y juntos van donde Isabel. Entran todos en una vasta habitación, muy bien puesta, de la planta baja. Ofrecen asiento a María y mandan que le sirvan una taza de leche recién ordeñada — todavía tiene la espuma — y unas pequeñas tortas. 

Isabel da órdenes a la sirvienta, quien, embadurnadas de harina todavía las manos y el pelo más blanco de cuanto en realidad lo es, por la harina que tiene, por fin ha hecho acto de presencia. Quizás estaba haciendo el pan. Da órdenes también al sirviente — al que oigo llamar Samuel — para que lleve el baulillo de María a la habitación que le indica. Todos los deberes de una señora de casa para con su huésped. 

Entretanto, María responde a las preguntas que Zacarías le hace escribiendo con un estilo en una tablilla encerada. Por las respuestas, comprendo que le está preguntando por José y por cómo se encuentra siendo su prometida. Y comprendo también que a Zacarías le es negada toda luz sobrenatural acerca de la gravidez de María y su condición de Madre del Mesías. Es Isabel quien, acercándose a su marido y poniéndole con amor una mano en el hombro, como para hacerle una casta caricia, le dice: 

- María también es madre. Regocíjate por su felicidad - Y no dice nada más. Mira a María; y María la mira, pero no la invita a decir nada más, por lo cual guarda silencio. 

¡Dulce, dulcísima visión que me cancela el horror que me quedó al ver el suicidio de Judas! 

Ayer por la tarde, antes del sopor, vi el llanto de María, inclinada hacia la piedra de la unción, sobre el cuerpo sin vida del Redentor. Estaba a su lado derecho, dando la espalda a la boca de la gruta sepulcral. La luz de las antorchas iluminaba su cara y me hacía ver su pobre rostro devastado por el dolor, lavado por el llanto. Cogía la mano de Jesús, la acariciaba, se la calentaba en sus mejillas, la besaba, extendía los dedos... besaba uno a uno estos dedos ya inmóviles. Luego acariciaba el rostro de Jesús, se inclinaba a besar la boca abierta, los ojos semicerrados, la frente herida. La luz rojiza de las antorchas daba un aspecto más vivo aún a las llagas de todo ese cuerpo torturado y hacía más verídica la crudeza del suplicio padecido y la realidad de su estar muerto. 

Y así me quedé contemplando mientras permaneció lúcida mi inteligencia. Luego, despertada del sopor, he orado y me tranquilicé para dormir verdaderamente. Entonces me comenzó la visión que he descrito. Pero la Madre me dijo: «No te muevas. Únicamente mira. Mañana escribirás». Durante el sueño he vuelto a soñar todo. Me he despertado a las 6'30 y he vuelto a ver cuanto ya había visto despierta y en sueño. He escrito mientras veía. Luego ha venido usted (el sacerdote con quien ella consultaba y a quien daba los escritos) y le he podido preguntar si tenía que meter lo que sigue. Son pequeños cuadros separados que tratan del tiempo de permanencia de María en casa de Zacarías. 

\chapter*{Las jornadas en Hebrón. \\ \normalfont\normalsize\textit{Los frutos de la caridad de María hacia Isabel.}}
\addcontentsline{toc}{chapter}{\normalfont\scshape{Las jornadas en Hebrón.}}
 
Veo a María cosiendo sentada en la sala de la planta baja. Parece que es por la mañana. Isabel va y viene, ocupándose de la casa. Cada vez que entra, se acerca a depositar una caricia en la rubia cabeza de María, más rubia aún ahora por el contraste con las paredes; más bien oscuras, y bajo el rayo del luminoso sol que entra por la puerta abierta que da al jardín. Isabel se inclina a mirar el trabajo de María — es el bordado que tenía en Nazaret — y alaba su belleza. - Tengo también lino para hilar - dice María. 

- ¿Para tu Niño? 

No. Lo tenía ya cuando todavía no pensaba que... - María no acaba la frase, pero yo entiendo: «... cuando todavía no pensaba que iba a ser Madre de Dios. 

Pero ahora tendrás que usarlo para Él. ¿Es bonito? ¿Es fino? Ya sabes que los niños necesitan una tela suavísima. 

Sí, lo sé. 

Yo había empezado... Tarde, porque quería estar segura de que no era un engaño del Maligno; a pesar de que... sentía en mí una alegría, tal, que, no, no podía provenir de Satanás. Luego... he sufrido mucho. Soy vieja, María, para encontrarme en este estado. "He sufrido mucho. Tú no sufres... 

Yo no. Nunca me he sentido tan bien. 

- ¡Ya! ¡Claro! En ti no hay mancha, si Dios te ha elegido para ser Madre suya. Por tanto, no estás sujeta a los sufrimientos de Eva. El Fruto concebido en ti es santo. 

Es como si tuviera un ala en el corazón y no un peso; es como llevar dentro todas las flores y todas las avecillas que cantan en primavera, y toda la miel y todo el sol... ¡Oh, me siento dichosa! 

- ¡Bendita eres! Yo también, desde que te he visto, he dejado de sentir peso, cansancio y dolor. Me siento nueva, joven, liberada de las miserias de mi carne de mujer. Mi hijo saltó primero dichoso ante el sonido de tu voz, luego se tranquilizó gozoso. Y me parece como si lo llevase dentro en una cuna viva, y como si le viera dormir completamente satisfecho y dichoso, y respirar como un pajarito feliz bajo el ala de su madre... Ahora me voy a poner manos a la obra. No sentiré ya el peso. Veo poco, pero... 

- ¡Deja, Isabel! Me encargo yo de hilar y tejer para ti y para tu niño. Yo soy rápida y veo bien. 

Pero tendrás que ocuparte del tuyo.... 

- ¡Bueno, hay tiempo de sobra!.. Primero me ocuparé de ti, que ya vas a tener pronto al pequeñuelo; luego de mi Jesús. 

Decirle lo dulce de la expresión y voz de María, decirle cómo se adornaran sus ojos de un suave, dichoso llanto, cómo Ella sonríe al pronunciar este Nombre, mirando al cielo luminoso y azul, es superior a las posibilidades humanas. Parece como si el éxtasis la arrobara por el solo hecho de pronunciar «Jesús». 

Isabel dice: 

- ¡Qué nombre más hermoso! ¡El Nombre del Hijo de Dios, Salvador nuestro! 

- ¡Oh..., Isabel! - María revela una expresión tristísima y ha aferrado las manos que su parienta tenía cruzadas sobre el vientre abultado - Dime, tú que, cuando yo llegué, fuiste investida del Espíritu del Señor y que profetizaste lo que el mundo ignora. Dime, ¿qué tendrá que hacer para salvar al mundo mi Criatura? Los Profetas... ¡Oh!.. ¡Los Profetas que hablan del Salvador!.. Isaías... ¿recuerdas Isaías! "Él es el Varón de los dolores. Por sus moretones recibimos la salud. Él ha sido traspasado y está llagado por nuestras iniquidades... Plugo al Señor quebrantarlo con dolores... Tras la condena fue levantado..." ¿De qué elevación habla? Le llaman Cordero, y yo pienso... yo pienso en el cordero pascual, el cordero mosaico, y concateno esto con la serpiente que Moisés levantó en una cruz. ¡Isabel!.. ¡Isabel! ¿Qué le harán a mi Criatura? ¿Qué tendrá que sufrir para salvar al mundo? - María se echa a llorar. 

Isabel la quiere consolar diciendo: 

María, no llores. Es tu Hijo, pero también es Hijo de Dios. Dios se preocupará de su Hijo y de ti, que eres su Madre. Si bien es cierto que muchos lo tratarán cruelmente, también lo es que otros muchos lo amarán. ¡Muchos!.. Por los siglos de los siglos. El mundo dirigirá su mirada al que de ti nacerá y, junto con El, te bendecirá a ti, que eres Manantial de redención. ¡La suerte de tu Hijo! Proclamado Rey de toda la creación. Piensa en esto, María. Rey, por haber rescatado toda la creación; como tal, será su Rey universal. Y también en la tierra, en el tiempo, será amado. El que nacerá de mí precederá al tuyo y lo amará. Se lo dijo el ángel a Zacarías. Él me lo escribió... ¡Qué dolor ver mudo a mi Zacarías! De todas formas, espero que cuando nazca el niño el padre sea liberado de este castigo. Pide tú por ello, tú que eres la Sede de la Potencia de Dios y la Causa de la alegría del mundo. Yo, para obtener esto, como puedo hago ofrenda de mi criatura al Señor, porque es suya, pues Él se la ha prestado a su sierva para proporcionarle la alegría de ser llamada "madre". Es el testimonio de cuanto Dios me ha hecho. Quiero que se llame Juan. ¿No es él, mi niño, acaso, una gracia? Y ¿no es Dios quien me la ha dado? 

Y Dios — yo también estoy convencida de ello — te concederá esa gracia. Yo oraré... contigo. 

- ¡Siento tanto dolor viéndolo mudo!.. - Isabel llora - Cuando escribe, pues ya no puede hablarme, es como si montes y mares estuvieran entre mí y mi Zacarías. Después de tantos años de dulces palabras, ahora sólo silencio de su boca... sobre todo ahora, que sería verdaderamente hermoso hablar del que ha de venir. Incluso yo misma evito hablar para no verlo cómo se fatiga respondiéndome con gestos. ¡He llorado tanto... ! ¡Cuánto te he echado de menos! El pueblo mira, chismorrea y critica. El mundo es así. Cuando se padece una pena o se tiene una alegría, tenemos necesidad de alguien capaz de comprender, no de criticar. Ahora es como si toda la vida fuera mejor. Estoy alegre desde que llegaste; siento que mi prueba pronto quedará superada y que pronto mi dicha será completa. Será así, ¿no es verdad? Yo me resigno a todo, pero... ¡si Dios perdonara a mi marido! ¡Oh, poder oírle orar de nuevo!.. 

María la acaricia y la anima, y le propone, para distraerla, salir un poco al soleado jardín. 

Caminan bajo una pérgola bien cuidada, hasta una torrecilla rural, en cuyos agujeros hacen sus nidos las palomas. 

María les echa comida sonriendo, pues se le han echado encima arrullando intensamente. Su revoloteo dibuja en torno a Ella círculos iridiscentes. Se le posan sobre la cabeza, sobre los hombros, en los brazos y en las manos, alargando los picos rosados para arrebatarle los granitos de la concavidad de las manos, picoteando con gracia los róseos labios de la Virgen, y los dientes, que le brillan con el sol. María saca de un saquito el blondo trigo, y ríe en medio de ese carrusel de avidez impetuosa. 

- ¡Cuánto te quieren! - dice Isabel - Pocos días llevas con nosotros y ya te quieren más que a mí, que las he cuidado siempre. 

El paseo continúa hasta llegar a un recinto cerrado en el fondo del huerto. Hay unas veinte cabritas con sus cabritillos. 

- ¿Has vuelto del pasto? - pregunta María a un pastorcillo acariciándolo. 

- Sí, porque mi padre me ha dicho: "Vete a casa, que dentro de poco va a llover y hay ovejas que pronto van a parir. Preocúpate de que tengan hierba seca y cama de paja preparada". Viene por allí - Y señala hacia más allá del bosque, de donde llega un trémulo balitar. 

María acaricia a un cabritillo que se restriega en ella, rubio como un niño. Y ella e Isabel beben la leche recién ordeñada que el pastorcillo les ofrece. 

Llegan las ovejas con un pastor hirsuto como un oso. Debe ser, no obstante, un buen hombre porque lleva sobre sus hombros una oveja quejumbrosa. La deja en el suelo despacio; explica que está para dar a luz un cordero, que no podía caminar sino con dificultad, que se la ha puesto sobre los hombros y que se ha dado una buena carrera para llegar a tiempo. Y el niño conduce al redil a la oveja, que va cojeando a causa de los dolores. 

María se ha sentado en una piedra y juega con los cabritillos y los corderos, ofreciendo a sus rosados morritos flores de trébol. Un cabritillo blanco y negro le pone las patitas sobre un hombro y le olisquea los cabellos. «No es pan» dice María riendo. «Mañana te traigo una corteza. Ahora tranquilo». 

También Isabel, ya sosegada, ríe. 

"Veo a María hilando premurosamente bajo la pérgola en que la uva aumenta de volumen. Debe haber pasado ya un poco de tiempo, pues las manzanas comienzan a tomar color rojo en los árboles, y las abejas zumban cerca de las flores de la higuera ya formadas. 

Isabel está verdaderamente gruesa y camina pesadamente. María la mira con atención y amor. También a María, que se ha levantado para recoger el huso, que se le ha caído lejos, se la ve más llena a la altura de los costados, y su expresión ha cambiado. Ahora es más madura. Antes era niña, ahora es mujer. 

Está anocheciendo y las mujeres entran en casa; en la habitación se encienden las lámparas. En espera de la cena, María teje. 

- ¿No te cansa nunca? - pregunta Isabel señalando el telar. - No, tenlo por seguro. 

A mí este calor me deja sin fuerzas. No he vuelto a tener dolores, pero ahora el peso es grande para mis riñones, que ya son viejos». 

- ¡Ánimo! Pronto serás liberada de ese peso. ¡Qué feliz te sentirás entonces! Yo ardo en deseos de ser madre. ¡Mi Niño, mi Jesús! ¿Cómo será? 

Tan guapo como tú, María. 

- ¡Oh, no! ¡Más guapo! Él es Dios, yo soy su sierva. Me refería a si será rubio o moreno, si tendrá los ojos como el cielo sereno o como los de los ciervos de las montañas. Yo me le imagino más hermoso que un querubín, de cabellos rizados y color oro; los ojos del color de nuestro mar de Galilea cuando las estrellas empiezan a asomarse al confín del cielo; una boquita pequeñina y roja como el corte de una granada apenas abierta por el sol que la madura; sus mejillas, un rosáceo como éste de esta pálida rosa; dos manitas que, de lo pequeñitas y lindas que serán, podrán estar dentro de la corola de una azucena; dos piececitos que podrían caberme en el hueco de la mano, más delicados y lisos que un pétalo de flor. Mira, yo pongo en la idea que me he hecho de El todo lo que de hermoso me sugiere la tierra. Ya oigo su voz. Cuando llore — un poco llorará por hambre o por sueño mi Niño, y ello causará siempre un gran dolor a su Mamá, que no podrá, no, no podrá oírle llorar sin sentirse traspasar el corazón cuando llore, su voz será como ese balido que ahora oímos, de corderito de pocas horas que está buscando la mama y el calor de la lana materna para dormir. En la risa, en esa risa que llenará de cielo mi corazón, enamorado de mi Criatura — puedo estar enamorada de Él porque es mi Dios, y amarle con amor de enamorada no es contravenir a mi consagrada virginidad —, en la risa, su voz será como el zurear jubiloso de este pichoncito, contento porque ha comido, satisfecho en el nido calentito. Pienso en Él dando sus primeros pasos... un pajarillo saltando en un prado florido. El prado será el corazón de su Mamá, que estará bajo sus piececitos de rosa con todo su amor para que no encuentre nada que le produzca dolor. ¡Cuánto le voy a querer a mi Niño, a mi Hijo! ¡Y también José lo amará! 

Sí, pero tendrás que decírselo también a José. 

Se le nubla el rostro a María, que suspira. 

Tendré que decírselo... Yo habría querido que se lo dijera el Cielo, porque es muy difícil de decir. 

- ¿Quieres que se lo diga yo? Lo llamamos para la circuncisión de Juan... 

No. Mira, he dejado en manos de Dios la tarea de instruirle, y lo hará, acerca del feliz destino de nutricio del Hijo de Dios. El Espíritu me dijo aquella tarde: "Guarda silencio. Déjame a mí la tarea de justificarte". Y lo hará. Dios no miente nunca. Es una gran prueba, pero con la ayuda del Eterno será superada. De mi boca, ninguno, aparte de ti, a quien el Espíritu se lo ha revelado, debe saber lo que la benevolencia del Señor ha hecho a su sierva. 

He guardado silencio siempre, incluso con Zacarías, que hubiera exultado de gozo si lo hubiera sabido. Él cree que eres madre según la naturaleza. 

Sí, lo sé. Así lo he querido por prudencia. Los secretos de Dios son santos. El ángel del Señor no le ha revelado a Zacarías mi maternidad divina. Habría podido hacerlo, si Dios hubiese querido, porque Dios sabía que ya era inminente el momento de la Encarnación de su Verbo en mí. Pero Dios le ha tenido escondida esta luz de gozo a Zacarías, que no aceptaba, por considerarlo imposible, vuestra paternidad y maternidad tardías. Me he puesto en sintonía con la voluntad de Dios, y, ya ves, tú has sentido el secreto que vive en mí, y él no ha advertido nada. Hasta que no se desprenda el diafragma de su incredulidad ante la potencia de Dios, se verá separado de las luces sobrenaturales. 

Isabel suspira y guarda silencio. 

Entra Zacarías. Ofrece unos rollos a María. Es la hora de la oración de la cena. María reza en voz alta en vez de Zacarías. Luego se sientan a la mesa. 

Cuando te marches, ¡cómo echaremos de menos el no tener quien ore en lugar de nosotros! - dice Isabel mirando a su mudo. 

Tú rezarás para ese entonces, Zacarías - dice María. 

Él menea la cabeza y escribe: «No podré volver a orar en representación de otros. Me he hecho indigno de ello desde que dudé de Dios». 

Zacarías, tú rezarás. Dios perdona. 

El anciano se enjuga una lágrima y suspira. 

Terminada la cena, María vuelve al telar. 

- ¡Vale ya! - dice Isabel - Es demasiado cansancio. 

Está próxima la hora, Isabel. Quiero hacerle a tu niño un equipo digno del predecesor del Rey de la estirpe de David. Zacarías escribe: « ¿De quién nacerá Él, y dónde?». María responde: 

Donde han dicho los Profetas, y de quien elija el Eterno. Todo lo que nuestro Señor altísimo hace está bien hecho. 

Zacarías escribe: « ¡Entonces, en Belén! En Judea. Mujer, iremos a venerarlo. Tú también vendrás con José a Belén». Y María, inclinando hacia su telar la cabeza, dice: 

Iré. 

La visión cesa así. 

Dice María: 
\emph{El primer acto de caridad para con el prójimo ha de ejercitarse con el prójimo. No veas en esto un juego de palabras. La caridad se tiene hacia Dios y hacia el prójimo. En la caridad hacia el prójimo está comprendida también la que tiene por objeto nosotros mismos. Pero, si nos amamos más que a los demás, ya no somos caritativos, somos egoístas. Incluso en las cosas lícitas debemos ser tan santos, que demos siempre prioridad a las necesidades de nuestro prójimo. Estad seguros, hijos, de que Dios completa la deficiencia de los generosos con medios de su potencia y bondad. Esta certeza me impulsó a ir a Hebrón para ayudar en su estado a mi parienta. Pues bien, a este detalle mío de ayuda humana, Dios, dando sin medida como El hace, añadió un inesperado don de ayuda sobrenatural. Yo había ido para aportar ayuda material; Dios santificó mi recta intención haciendo, de la misma, santificación del fruto del vientre de Isabel y anulando, a través de esta santificación, por la cual el Bautista fue presantificado, el sufrimiento físico de esta madura hija de Eva que había concebido a una edad inusitada. Isabel, mujer de fe intrépida y de confiado abandono a la voluntad de Dios, mereció comprender el misterio encerrado en mí. El Espíritu le habló a través de ese vuelco de su vientre. El Bautista pronunció su primer discurso de Anunciador del Verbo a través de los velos y los diafragmas de venas y de carne que lo separaban de su santa madre, y que a la vez la unían a ella. No oculté mi condición de Madre del Señor a esta mujer que merecía saberlo, a quien además la Luz se había manifestado. Ocultarla habría sido negarle a Dios la alabanza que era justo darle, el sentimiento de alabanza que yo llevaba en mí y que, no pudiéndolo manifestar a nadie, lo manifestaba a la hierba, a las flores, a las estrellas, al sol, a los canoros pájaros, a las pacientes ovejas, a las aguas cantarinas y a la luz de oro que me besaba descendiendo del cielo. Pero, orar dos juntos es más dulce que decir uno solo su oración. Yo hubiera querido que el mundo entero hubiera conocido mi destino; no por mí, sino porque todos se hubiesen unido a mí para alabar a mi Señor. La prudencia me prohibió revelarle a Zacarías la verdad. Habría significado ir más allá de la obra de Dios, y, si bien era cierto que yo era su Esposa y Madre, seguía siendo su Sierva y no debía — porque Él me había amado sin medida — permitirme colocarme en su lugar y sobrepasar un decreto suyo. Isabel, en su santidad, comprendió y guardó silencio, porque el que es santo es siempre sumiso y humilde. El don de Dios debe hacernos cada vez mejores. Cuanto más recibimos de Él, más debemos dar, porque cuanto más recibimos, más es signo de que Él está en nosotros y con nosotros, y cuanto más está en nosotros y con nosotros, más debemos esforzarnos en alcanzar su perfección. Ello explica por qué yo, posponiendo mi labor, trabajé para Isabel. No me dejé llevar del miedo de la falta de tiempo. Dios es dueño del tiempo, y provee a las necesidades de quien en El espera, incluso en las cosas ordinarias. El egoísmo no acelera, retarda; la caridad no retarda, acelera: tenedlo siempre en cuenta. ¡Cuánta paz en la casa de Isabel! Si no hubiera tenido la preocupación de José y esa, esa, esa preocupación de que mi Niño era el Redentor del mundo, me habría sentido feliz. Pero ya la Cruz extendía su sombra sobre mi vida, ya me era sonido fúnebre la voz de los Profetas... Yo me llamaba María. La amargura siempre se mezclaba con las dulzuras que Dios vertía en mi corazón, amargura que fue cada vez más en aumento, hasta la muerte de mi Hijo. Y, no obstante, cuando Dios nos destina a ser víctimas por su honor, ¡oh, qué dulce es ser trituradas en el molino, como el trigo, para hacer de nuestro dolor el pan que consolide a los débiles y los haga capaces de obtener el Cielo! }

\chapter*{Nacimiento de Juan el Bautista. \\ \normalfont\normalsize\textit{Todo sufrimiento se aplaca sobre el seno de María.}}
\addcontentsline{toc}{chapter}{\normalfont\scshape{Nacimiento de Juan el Bautista.}}
 
En medio de las cosas repugnantes que nos ofrece el mundo de ahora, baja del Cielo, y no sé cómo puede hacerlo, dado que yo soy como una ramita seca a merced del viento en estos continuos choques contra la maldad humana, tan discordante con lo que vive en mí, baja del Cielo, digo, esta visión de paz. 

Continúa la casa de Isabel. Es una hermosa tarde de verano, aún clara con un último sol, y de todas formas ya adornada en el cielo por un arco falcado de luna, que parece una coma de plata en una vasta tela azul intenso de fina seda. 

Los rosales huelen fuertemente, y las abejas, gotas de oro zumbadoras, dan sus últimos vuelos en el aire quieto y caliente de la tarde. De los prados viene un gran olor de heno secado al sol, un olor casi de pan, de pan caliente, recién hecho. 

Quizás viene también de los muchos lienzos que están tendidos por todas partes para secarse y que ahora Sara está plegando. 

María pasea dándole el brazo a su prima. Muy despacito van y vienen, bajo el emparrado semioscuro. 

María está pendiente de todo y, a pesar de estar dedicada a Isabel, se da cuenta de que Sara está atareada en doblar un largo lienzo que ha quitado de un seto. 

Espérame aquí, sentada - le dice a su parienta; y va a ayudar a la anciana sirvienta, estirando la tela para alisarla, y doblándola con cuidado. 

Se siente todavía el sol, están calientes - dice sonriendo; y, para que se sienta contenta la mujer, añade: 

Esta tela después de tu blanqueo ha quedado más bonita que nunca. Nadie tiene tanta maña como tú - Sara se marcha toda contenta con su carga de fragantes telas. 

María vuelve con Isabel y dice: 

Otros poquitos pasos. Te vendrán bien - Y, dado que Isabel está cansada y no le apetece moverse, le dice: 

Vamos sólo a ver si todas tus palomas están en sus nidos y si el agua de su pilón está limpia. Luego nos volvemos a casa. 

Las palomas deben ser las predilectas de Isabel. Llegadas ante la rústica torrecilla donde ya se han recogido todas las palomas (las hembras están en los nidos; los machos, delante de éstos y no se mueven, pero en viendo a las dos mujeres las saludan con su arrullo), Isabel se emociona. La debilidad de su estado la vence y le produce temores que le hacen llorar. Se los manifiesta a su prima: 

Si yo muriese... ¡pobres palomitas mías! Tú no permanecerás aquí. Si te quedaras en mi casa, no me importaría morirme. He gozado de la máxima alegría que una mujer puede recibir, una alegría que ya me había resignado a no conocer nunca. Ni de la misma muerte puedo presentarle quejas al Señor, porque Él, ¡bendito sea!, me ha colmado de su benevolencia. Pero, está Zacarías... y estará el niño: uno, viejo, que se encontraría como perdido en un desierto sin su mujer; el otro, tan pequeñito, que sería como una flor destinada a morir helada, por no tener a su mamá. ¡Pobre niño, sin las caricias de su madre!.. 

Pero, ¿por qué estás tan triste? Dios te ha dado la alegría de ser madre, y no te la va a quitar cuando llega a su plenitud. El pequeño Juan tendrá todos los besos de su mamá y Zacarías gozará de todos los cuidados de su fiel esposa hasta la más avanzada ancianidad. Sois dos ramas de un mismo árbol. No morirá uno dejando al otro solo. 

Tú eres buena y quieres consolarme, pero yo soy muy anciana para tener un hijo, y ahora que estoy para darlo a luz tengo miedo! 

- ¡Oh, no! ¡Está aquí Jesús! Donde está Jesús no se debe tener miedo. Mi Niño te quitó el dolor cuando era como un capullo recién formado; tú lo dijiste. Ahora, que cada vez va desarrollándose más y que vive ya como criatura mía; ahora, que siento palpitar su corazón en mi garganta y es como si tuviera posado en ella un pajarito de nido con un corazoncito de suave palpitar, alejará de ti todo peligro. Debes tener fe. 

La tengo. Pero, si yo muriese... no dejes a Zacarías inmediatamente. Sé que piensas en tu casa, pero, quédate un poco, para ayudarle a mi marido en el momento del primer dolor. 

Me quedaré, para complacerme en la alegría de ambos, y sólo te dejaré cuando estés fuerte y te sientas aliviada. Estate tranquila, Isabel; todo irá bien. En tu casa no faltará nada mientras dure tu dolor. Zacarías será servido por la más amorosa de las siervas, y tus flores y tus palomas estarán cuidadas y a unas y a otras las encontrarás avivadas y bonitas para recibir cálidamente a la dueña cuando vuelva. Regresemos a casa ahora, te estás poniendo pálida... 

Sí, me parece que tengo otra vez dolores. Quizás haya llegado la hora. María, ora por mí. 

Te sostendré con la oración hasta que tus dolores se transformen en gozo. 

Y las dos mujeres entran despacio en la casa. Isabel se retira a sus habitaciones. María, hábil y previsora, da órdenes y prepara todo lo que puede necesitarse, y trata de confortar a Zacarías, que está preocupado. 

En la casa que vela esta noche, con voces nuevas, de mujeres llamadas para ayudar, María está en pie, vigilante como un faro en una noche de tormenta. Toda la casa gravita sobre Ella, que, dulce y sonriente, provee a todo; y ora. Cuando no se le llama para esto o aquello, se recoge en oración. Está en la habitación en que se reunían siempre para las comidas y el trabajo. 

Con Ella está Zacarías, paseando turbado. Ya han orado juntos. María luego ha seguido orando; incluso ahora, que el anciano, cansado, se ha sentado en su sillón junto a la mesa y se ha quedado en silencio, soñoliento. Cuando ve que está dormido del todo — la cabeza sobre los brazos cruzados apoyados en la mesa —, Ella se desata las sandalias para hacer menos ruido, y camina descalza; luego, con menos rumor del que puede hacer una mariposa volando por una habitación, coge el manto de Zacarías y se lo extiende encima al anciano con una suavidad tal, que éste continúa durmiendo bajo el calorcito de la lana protectora del fresco nocturno, que entra a ondas por la puerta, frecuentemente abierta. Luego sigue orando; cada vez con más intensidad; de rodillas, con los brazos levantados, cuando el quejido de Isabel, que sufre, se agudiza. Sara entra y la llama con señas. María sale con sus pies descalzos al jardín. - La señora la llama - dice. 

Voy. 

María va por el lado externo de la casa, sube la escalera... Parece un ángel blanco moviéndose en la noche quieta llena de astros. Entra en la habitación de Isabel. 

- ¡Oh! ¡María! ¡María! ¡Cuánto dolor! ¡No puedo más, María! ¡Cuánto dolor hay que padecer para ser madre! 

María la acaricia con amor y la besa. 

- ¡María! ¡María! ¡Deja que ponga mis manos sobre tu vientre! 

María coge esas dos manos rugosas e hinchadas, las pone sobre su abdomen ya algo abultado y las mantiene apretadas con sus manitas lisas y gráciles. Y ahora, que están las dos solas, habla en tono suave y dice: 

Jesús está aquí, oyéndote y viéndote. Ten confianza, Isabel. Su corazón santo late con más fuerza, porque está actuando para bien tuyo. Lo siento latir como si lo tuviera entre una mano y otra. Yo entiendo las palabras de mi Niño hechas de latidos. Ahora me está diciendo: "Dile a la mujer que no tema. Todavía un poco de dolor. Luego, con el primer sol, entre las tantas rosas que esperan ese rayo matutino para abrir sus pétalos sobre su tallo, su casa tendrá la rosa más bonita, Juan, mi Precursor". 

Isabel apoya también la cara en el vientre de María y llora silenciosamente. María está un tiempo así, pues parece que el dolor va pasando a una fase de relajación reparadora. Luego indica a todos que estén tranquilos. Ella permanece en pie, blanca y hermosa bajo el tenue claror de una lámpara de aceite, como un ángel al lado de quien sufre. 

Ora. La veo mover los labios. De todas formas, aun cuando no se los viese mover, comprendería que está orando por la expresión arrobada del rostro. 

El tiempo pasa. Le vuelve el dolor a Isabel. María la besa de nuevo y se retira. Baja rápida a la luz de la luna y corre a ver si el anciano duerme todavía. Duerme, gimiendo en el sueño. María hace un gesto de piedad. Se pone de nuevo a orar. 

Pasa el tiempo. El anciano sale bruscamente de su sueño y levanta su rostro, confuso, como de quien no recordase bien por qué estaba ahí. Luego recuerda, hace un gesto y profiere una exclamación gutural, y escribe: «¿No ha nacido todavía?». 

María indica que no, y Zacarías: «¡Cuánto dolor! ¡Pobre esposa mía! ¿Lo logrará sin morir a cambio?». 

María coge la mano del anciano tratando de infundirle ánimo: 

Para el alba, dentro de poco, el niño ya habrá nacido. Todo irá bien. Isabel es fuerte. ¡Qué bonito va a ser este día — pues está cercana la aurora — en que tu niño va a ver la luz! ¡El más bello de tu vida! Grandes gracias te tiene reservadas el Señor, y tu hijo es su anunciador. 

Zacarías menea tristemente la cabeza y señala a su boca muda. Quisiera decir muchas cosas, pero no puede. 

María se da cuenta de ello y responde: 

El Señor hará completa tu alegría. Cree en Él completamente, espera infinitamente, ama totalmente. El Altísimo te escuchará más de lo que pudieras esperar. Él quiere esta fe tuya total como purificación de tu pasada desconfianza. Di en tu corazón conmigo: "Creo". Dilo a cada uno de los latidos de tu corazón. Los tesoros de Dios se abren para quien cree en Él y en su poderosa bondad. 

La puerta está entornada y la luz comienza a penetrar por ella. María la abre. El alba ha puesto toda blanca la tierra aljofarada de rocío. Se percibe un fuerte olor de tierra húmeda y hierba, y los primeros silbos de pájaros se llaman de rama a rama. 

El anciano y María salen a la puerta. Están pálidos por la noche pasada en vela; la luz del alba los pone aún más pálidos. María calza de nuevo sus sandalias y va al pie de la escalera, atenta a ver si se oye algo. Una mujer se asoma, María hace unos gestos y vuelve. Todavía nada. 

Luego va a una habitación y regresa con leche caliente. Se la da a beber al anciano. Después va donde las palomas, y desaparece de nuevo en esa habitación; quizás es la cocina. Se mueve aquí y allá, está atenta a todo. Se la ve tan ágil y tan serena, que parece como si hubiera dormido el mejor de los sueños. 

Zacarías pasea arriba y abajo nerviosamente por el jardín. María lo mira con piedad. Luego entra otra vez en la misma habitación y, arrodillada junto a su telar, ora intensamente, pues la queja de la sufriente se hace más aguda. Se curva hasta el suelo para suplicarle al Eterno. Zacarías vuelve, entra y la ve postrada en ese modo; el pobre anciano llora. María se alza y le coge de la mano. Es mucho más joven que él, pero parece Ella la madre de esa vejez desolada sobre la que extiende sus consuelos. 

Permanecen así, el uno al lado del otro, bajo este sol que pone rosáceo el aire de la mañana. Estando así, llega a sus oídos el jubiloso anuncio: 

- ¡Ha nacido! ¡Ha nacido! ¡Un niño! ¡Oh, padre dichoso! ¡Un niño lozano como una rosa, bonito como el Sol, fuerte y bueno como la madre! ¡Alégrate, padre bendecido por el Señor, que te ha dado un hijo para que lo ofrezcas a su Templo! ¡Gloria a Dios, que ha concedido posteridad a esta casa! ¡Benditos seáis tú y el hijo que te ha nacido! ¡Que su linaje perpetúe tu nombre por los siglos de los siglos, generación tras generación, y permanezca siempre en alianza con el Señor eterno! 

María, llorando de alegría, bendice al Señor. Luego, los dos acogen al pequeñuelo, que le ha sido traído al padre para que lo bendiga. Zacarías no va con Isabel; coge al niño, que grita como un desesperado. Pero no va donde su esposa. 

María sí que va, llevando amorosa al pequeñuelo, el cual se ha quedado callado nada más que María lo ha cogido en brazos. La comadre, que va tras Ella, se percata de este hecho. 

Mujer — dice a Isabel — tu hijo se ha callado enseguida, cuando ella lo ha cogido en sus brazos. ¡Mira qué tranquilo duerme; y bien sabe el Cielo lo inquieto y fuerte que es! ¡Mira, ahora parece un pichoncito! 

María deposita a la criatura junto a la madre y acaricia a Isabel, poniendo en orden su pelo gris. 

La rosa ha nacido — le dice con voz suave — y tú vives. Zacarías está dichoso. - ¿Habla? 

Todavía no. Pero, espera en el Señor. Ahora descansa. Yo estoy contigo. 

Dice María: 

\emph{Mi presencia había santificado al Bautista, pero no había cancelado a Isabel la condena proveniente de Eva. "Darás a luz con dolor" había dicho el Eterno. Sólo yo, sin mancha y sin haber tenido unión matrimonial humana, quedé exenta de engendrar con dolor. La tristeza y el dolor son los frutos de la culpa. Yo, que era la Inculpable, tuve que conocer también el dolor y la tristeza, porque era la Corredentora. Pero no conocí el tormento del generar; no, este tormento no lo conocí. Y, no obstante, créeme, hija, no hubo, ni habrá jamás tormento puerperal semejante al mío de Mártir de una Maternidad espiritual cumplida en el más duro lecho, el de mi cruz, al pie del patíbulo del Hijo que se me moría. ¿Qué madre se verá obligada a generar de esa manera? ¿Qué madre se verá obligada a amalgamar el suplicio del desgarro de sus entrañas por los estertores de su Hijo moribundo, con el suplicio de sentírsele retorcer las entrañas al tener que superar el horror de deber decir: "Os amo; venid a mí, que soy Madre vuestra" a los que estaban matando a ese Hijo nacido del más sublime amor que jamás haya visto el Cielo, del amor de un Dios con una virgen, del beso de Fuego, del abrazo de Luz, que se hicieron Carne, y que del vientre de una mujer hicieron el Tabernáculo de Dios? - ¡Cuánto dolor para ser madre! - dice Isabel. - ¡Mucho! Sí, pero insignificante, comparado con el mío. - Déjame poner las manos en tu vientre". ¡Ah, si cuando sufrís me pidierais siempre esto! Yo soy la eterna Portadora de Jesús. Él está dentro de mi pecho, como tú lo viste el año pasado, cual Hostia en el ostensorio. Quien a mí viene, a Él lo encuentra; quien en mí se apoya, a Él lo toca; quien a mi se dirige, con Él habla. Yo soy su vestidura. Él es el alma mía. Mi Hijo está ahora más unido a mí que durante los nueve meses de gestación. A quien a mí viene y apoya su cabeza en mi regazo, todo dolor se le adormece, toda esperanza le florece, toda gracia le fluye. Yo oro por vosotros. Recordadlo. La beatitud de estar en el Cielo, viviendo en el esplendor de Dios, no me distrae de mis hijos que padecen en la tierra. Yo oro. Todo el Cielo ora porque el Cielo ama. El Cielo es caridad que vive, y la Caridad tiene piedad de vosotros. Pero, aunque sólo estuviera yo, habría suficiente oración para cubrir las necesidades de quien espera en Dios. Porque no ceso de orar por todos vosotros, santos y malvados, para dar: a los santos, la alegría; a los malvados, el salvífico arrepentimiento. Venid, venid, hijos de mi dolor. Os espero al pie de la Cruz para distribuir gracias. }
 
\chapter*{La circuncisión de Juan el Bautista. \\ \normalfont\normalsize\textit{María es Fuente de Gracia para quien acoge la Luz.}}
\addcontentsline{toc}{chapter}{\normalfont\scshape{La circuncisión de Juan el Bautista.}}
 
Veo ambiente de fiesta en la casa. Es el día de la circuncisión. 

María se ha preocupado de que todo esté lindo y en orden. Las habitaciones resplandecen de luz. Lucen por todas partes los más bellos paños, los más bellos atavíos. Hay mucha gente. María se mueve ágil entre los grupos, toda hermosa con su más bonito vestido blanco. 

Isabel, reverenciada como una matrona, goza feliz su fiesta. El niño está en su regazo, saciado ya de leche. 

Llega la hora de la circuncisión. 

Zacarías le llamaremos. Tú eres anciano. Justo sería ponerle tu nombre al niño - dicen unos hombres. 

- ¡De ninguna manera! - exclama la madre - Su nombre es Juan. Su nombre debe dar testimonio de la potencia de Dios. 

- ¿Pero se puede saber cuándo ha habido un Juan en nuestra parentela? 

No importa. Tiene que llamarse Juan. 

- ¿Tú qué dices, Zacarías? ¿Quieres tu nombre, no es verdad? 

Zacarías dice que no, con gestos. Coge una tablilla y escribe: «Su nombre es Juan», y, nada más terminar de escribir, añade, ya su liberada lengua: «porque Dios nos ha hecho objeto de una gran gracia, a mí, su padre, y a su madre, como también a este nuevo siervo suyo, el cual consumirá su vida en aras de la gloria del Señor y será llamado grande por los siglos y ante los ojos de Dios, porque pasará convirtiendo a los corazones al Señor altísimo. Lo dijo el ángel y yo no lo creí. Mas ahora creo y entra la Luz en mí. La Luz está entre nosotros y vosotros no la veis. Su destino es el de no ser vista, pues el espíritu de los hombres está lleno de estorbos, y además es perezoso. Pero mi hijo sí que la verá y hablará de Ella y hará que a Ella se vuelvan los corazones de los justos de Israel. ¡Bienaventurados los que crean en Ella y crean siempre en la Palabra del Señor! Y bendito seas Tú, Señor eterno, Dios de Israel, porque has visitado y redimido a tu pueblo, suscitando en él un poderoso Salvador en la casa de su siervo David. Como prometiste por boca de los santos Profetas, ya desde los tiempos antiguos: librarnos de nuestros enemigos y de las manos de los que nos odian, para ejercitar tu misericordia hacia nuestros padres y mostrar que te acuerdas de tu santa alianza. Este es el juramento que hiciste a Abraham, nuestro padre: concedernos que, sin temor, de las manos de nuestros enemigos libres, te sirviéramos con santidad y justicia en presencia tuya toda la vida» 

Los presentes se quedan estupefactos, tanto del nombre como del milagro, como de las palabras de Zacarías. 

Isabel, que al oír la primera palabra de Zacarías ha gritado de alegría, ahora está llorando abrazada a María, que la acaricia contenta. 

No veo la circuncisión. Veo sólo que traen a Juan y que chilla desesperado. No le calma ni siquiera la leche de su mamá. Tira patadas como un potrillo. Pero María le toma en sus brazos y le acuna, y él se calla y se queda tranquilo. 

- ¡Fijáos!- dice Sara - ¡sólo se calla cuando le coge en brazo ella! 

La gente se va marchando lentamente. En la habitación se quedan únicamente María, con el pequeñín en sus brazos, e Isabel, dichosa. 

Entra Zacarías y cierra la puerta. Mira a María con lágrimas en los ojos. Hace ademán de hablar. Guarda silencio. Continúa adelante. Se arrodilla ante María y le dice: 

Bendice al mísero siervo del Señor. Bendícelo. Tú puedes hacerlo, tú que lo llevas en tu seno. La palabra de Dios me ha hablado cuando he reconocido mi error, cuando he creído en todo cuanto me había sido dicho. Yo te veo a ti y veo tu destino feliz. Adoro en ti al Dios de Jacob. Tú, mi primer Templo, donde el sacerdote, regresado, puede de nuevo orar al Eterno. Bendita tú, que has obtenido gracia para el mundo y le traes el Salvador. Perdona a tu siervo si no ha visto antes tu majestad. Con tu venida nos has traído todas las gracias. En efecto, doquiera que vas, ¡oh Llena de Gracia!, Dios obra sus prodigios; santas son las paredes en que tú entras, santos se hacen los oídos que oyen tu voz y la carne que tú tocas, santos los corazones, porque tú confieres Gracia, Madre del Altísimo, Virgen profetizada y esperada para darle al pueblo de Dios el Salvador. 

María sonríe, encendida de humildad, y habla: 

Gloria al Señor, a Él sólo. De Él y no de mí viene toda gracia, y Él te la dona para que lo ames y sirvas con perfección en 

los años que te quedan, para merecer su Reino, que será abierto por mi Hijo a los Patriarcas, a los Profetas, a los justos del Señor. Y tú, ahora que puedes orar ante el Santo, ora por la sierva del Altísimo; que, si ser Madre del Hijo de Dios es destino dichoso, ser Madre del Redentor debe ser destino de atroz sufrimiento. Ora por mí, que hora a hora siento crecer mi peso de dolor, y durante toda una vida tendré que llevarlo; no lo veo en sus detalles particulares, pero sí siento que será un peso mayor que si sobre estos hombros míos de mujer se posase el mundo y tuviera que ofrecérsele al Cielo. ¡Yo, yo sola, una pobre mujer! ¡Mi Niño! ¡El Hijo mío! El tuyo no llora si yo le acuno; pero, ¿voy a poder acunar yo al mío para calmarle el dolor?.. Ora por mí, sacerdote de Dios. Mi corazón tiembla como una flor en medio de un temporal. Miro a los hombres y los amo, pero detrás de sus rostros veo aparecer al Enemigo, y veo cómo los hace enemigos de Dios, de Jesús, de mi Hijo... 

Y la visión cesa con la palidez de María y esas lágrimas suyas que hacen luciente su mirada. 

Dice María: 

\emph{- A quien reconoce su error arrepintiéndose y acusándose con humildad y corazón sincero, Dios lo perdona; no sólo lo perdona, sino que lo recompensa. ¡Oh, qué bueno es mi Señor con los humildes y sinceros, con los que creen en Él y en Él se abandonan! Arrojad de vuestro espíritu todo lo que lo traba y lo hace perezoso. Disponedlo para que acoja la Luz, que es, cual faro en las tinieblas, guía y santo conforto. ¡Amistad con Dios, beatitud de sus fieles, riqueza no igualada por nada, quien te posee nunca está solo ni siente la amargura de la desesperación! No anulas el dolor, santa amistad, porque el dolor fue destino de un Dios encarnado y puede ser destino del hombre; eso sí, lo haces dulce en su amargura, y añades una luz y una caricia que, cuales celestes toques, alivian la cruz. Y, cuando la Bondad divina os dé una gracia, usad el bien recibido para dar gloria a Dios. No seáis como esos insensatos que de un objeto bueno se hacen un arma dañosa, o como los derrochadores que de la abundancia acaban haciendo miseria. Me causáis demasiado dolor, hijos tras cuyos rostros veo aparecer al Enemigo, a aquel que arremete contra mi Jesús. ¡Demasiado dolor! Yo quisiera ser para todos el Manantial de la Gracia, pero hay demasiados entre vosotros que no quieren la Gracia. Pedís "gracias", pero con el alma privada de Gracia. ¿Cómo podrá la Gracia socorreros si sois enemigos suyos? El gran misterio del Viernes Santo se aproxima. Todo en los templos lo recuerda y lo celebra. Pero es necesario que lo celebréis y lo recordéis en vuestros corazones, y que os deis golpes de pecho, como los que bajaban del Gólgota, y que digáis: "Este es realmente el Hijo de Dios, el Salvador", y que digáis: 'Jesús, por tu Nombre, sálvanos", y que digáis: "Padre, perdónanos", y, en fin, es necesario decir: "Señor, yo no soy digno; pero, si Tú me perdonas y vienes a mí, mi alma quedará curada. Yo no quiero, no, no quiero pecar ya más, para no volver a enfermarme y para no ser de nuevo detestado por ti". Orad, hijos, con las palabras de mi Hijo. Decidle al Padre por vuestros enemigos: "Padre, perdónalos". Invocad al Padre, que se ha apartado indignado por vuestros errores: "Padre, Padre, ¿por qué me has abandonado? Yo soy pecador, pero, si me abandonas, moriré. Vuelve, Padre santo, que yo me salve". Poned vuestro eterno bien, vuestro espíritu, en manos del Único que lo puede conservar ileso del demonio: "Padre, en tus manos dejo mi espíritu". Si humilde y amorosamente cedéis vuestro espíritu a Dios, El ciertamente le guiará como hace un padre con su pequeñuelo; no permitirá que nada dañe vuestro espíritu. Jesús, en sus agonías, oró para enseñaros a orar. Os lo recuerdo en estos días de Pasión. Y tú, María, (se dirige la Virgen a María Valtorta) tú que ves mi gozo de Madre y te extasías con ello, piensa y recuerda que he poseído a Dios a través de un dolor progresivamente más intenso, que bajó a mí con la Semilla de Dios y, cual árbol gigante, fue creciendo hasta tocar el Cielo con su copa y el Infierno con sus raíces, cuando recibí en mi regazo el despojo exánime de la Carne de mi carne, y vi y conté sus laceraciones, y toqué su Corazón desgarrado, para apurar aquél hasta su última gota. }

\chapter*{Presentación de Juan el Bautista en el Templo \\ \normalfont\normalsize\textit{Y partida de María. La Pasión de José.}}
\addcontentsline{toc}{chapter}{\normalfont\scshape{Presentación de Juan el Bautista en el Templo}}

Zacarías, Isabel, María (ésta con el pequeño Juan en brazos) y Samuel (con un cordero y una cesta con la paloma) están bajando de un cómodo carro, al que viene atado el burrito de María. Se apean delante de la caballeriza de costumbre — que debe ser la etapa de todos los peregrinos que vienen al Templo — para dejar sus cabalgaduras. 

María llama a un hombre de baja estatura, el dueño de la caballeriza, y le pregunta si durante el día precedente o en las primeras horas de la mañana ha llegado algún nazareno. 

Ninguno, mujer - contesta el viejecillo. 

María se queda extrañada, pero no dice nada más. 

Le encarga a Samuel que le busque un puesto al burro. Luego alcanza a los dos ancianos padres y refiere el retardo de José: 

Algo le habrá entretenido, pero seguro que viene hoy. 

Vuelve a coger al niño — se lo había dejado a Isabel — y se encaminan hacia el Templo. 

Los hombres que están de guardia le reciben a Zacarías con honor, y los otros sacerdotes lo saludan y felicitan. Zacarías, hoy, con sus vestiduras sacerdotales y la alegría del padre que se siente feliz, está guapísimo. Parece un patriarca. Creo que Abraham debía asemejarse a él cuando jubilaba por ofrecer a Isaac al Señor. 

Veo la ceremonia de la presentación del nuevo israelita y la purificación de la madre. Es todavía más pomposa que la de María, porque por el hijo de un sacerdote los sacerdotes hacen mucha fiesta. Acuden en masa y se ponen manos a la obra diligentes en torno al grupito de las mujeres y del recién nacido. 

También otras personas se han acercado curiosas. Oigo los comentarios. Dado que María lleva en brazos al pequeñuelo mientras se dirigen al lugar establecido, la gente cree que es la madre. 

Pero una mujer dice: 

No puede ser. ¿No veis que está encinta? El niño no tiene más de unos pocos días y Ella está ya abultada. 

Ya... pero- dice otro - sólo puede ser Ella la madre. La otra es vieja. Será una parienta. No puede ser madre a esa edad. - Vamos detrás de ellos y así vemos quién tiene razón. 

Bien grande viene a ser el asombro cuando se ve que la que cumple el rito de la purificación es Isabel, que ofrece su corderillo balante para el holocausto y su paloma por el pecado. - La madre es aquélla. ¿Has visto? 

- ¡No! 

Sí. 

La gente, incrédula, sigue cuchicheando. Cuchichean tanto, que el grupo sacerdotal que está presente en el rito se ve obligado a emitir un « ¡Chsss!» imperativo. La gente se calla un momento, pero musita aún más fuerte cuando Isabel, radiante de santo orgullo, toma al niño y se adentra en el Templo para presentarlo al Señor. 

Es ella realmente. 

Es siempre la madre quien lo ofrece. 

Y entonces, ¿qué milagro es éste?». 

- ¿Qué será ese niño concedido en edad tan tardía a esa mujer? - ¿Qué signo es éste? 

- ¿No sabéis — dice uno que en ese momento llega jadeante — que es hijo del sacerdote Zacarías, de la estirpe de Aarón, aquel que quedó mudo estando ofreciendo el incienso en el Santuario? 

- ¡Misterio! ¡Misterio! ¡Y ahora ya puede hablar otra vez! El nacimiento del hijo le ha soltado la lengua. 

- ¿Qué espíritu será el que le habló y le incapacitó la lengua para acostumbrarlo al silencio sobre los secretos de Dios? - ¡Misterio! ¿Qué verdad será la que conoce Zacarías? 

- ¿No será que su hijo es el Mesías esperado por Israel? 

Ha nacido en Judea, no en Belén, ni de una virgen. No puede ser Mesías. 

- ¿Y entonces quién? 

Mas la respuesta queda en los silencios de Dios y la gente se queda con su curiosidad. 

Cumplido el ceremonial, los sacerdotes ahora también agasajan a la madre y al pequeñuelo; la única que pasa poco observada es María; es más, incluso la evitan casi con repulsión cuando se dan cuenta del estado suyo. 

Terminadas todas las felicitaciones, la mayor parte vuelve a la calle. María quiere pasar de nuevo por la caballeriza para ver si ya ha llegado José... No ha llegado. Y se queda desilusionada y pensativa. Isabel se preocupa por Ella. 

Hasta la hora sexta podemos estar aquí, pero luego tenemos que irnos para llegar a casa antes de la primera vigilia... es todavía demasiado pequeño para estar más tiempo de noche. 

Y María, tranquila y triste, dice: 

Me quedaré en un patio del Templo, iré donde mis maestras... No sé. Algo haré. 

Zacarías interviene con una propuesta que enseguida aceptan como una buena resolución. 

Vamos a casa de los familiares de Zebedeo. José, sin duda, te buscará allí, y, si él no fuera allí, te será fácil encontrar a alguien que te acompañe hacia Galilea, porque en esa casa hay un continuo ir y venir de pescadores de Genesaret. 

Toman el borriquillo y van a donde estos parientes de Zebedeo, los cuales son los mismos de la casa en que se detuvieron José y María cuatro meses antes. 

Las horas pasan deprisa y José no aparece. María domina su contrariedad acunando al niño; pero se la ve pensativa. Como para esconder su estado, no se ha quitado nunca el manto, a pesar de que el intenso calor les hace sudar a todos. 

Por fin se oye llamar fuerte a la puerta. Es el anuncio de la llegada de José. El rostro de María resplandece sosegado. José la saluda, porque Ella se ha presentado antes y le ha saludado con reverencia: 

- ¡La bendición de Dios sea contigo, María! 

Y contigo, José. ¡Alabado sea el Señor porque has venido! Zacarías e Isabel iban a marcharse ya para estar en casa antes de que fuera de noche. 

Tu mensajero llegó a Nazaret estando yo en Cana para unos trabajos. Lo supe anteayer por la tarde. Me puse en marcha enseguida. Pero, por mucho que haya venido sin detenerme, he llegado tarde, porque había perdido una herradura el burro. ¡Perdona! 

- ¡Perdona tú, por haber estado tanto tiempo lejos de Nazaret! La verdad es que se sentían tan felices de tenerme con ellos, que pensé darles hasta ahora esta satisfacción. 

Has hecho bien, Mujer. ¿Dónde está el niño? 

Entran en la habitación donde Isabel está dando de mamar a Juan, antes de marcharse. José felicita a los padres por la fortaleza del niño, que ha sido separado del pecho para mostrárselo a José, y que chilla y patalea como si le estuvieran despellejando. Ante esta protesta, todos se echan a reír. También ríen los parientes de Zebedeo, y se unen a la conversación. Habían venido trayendo fruta fresca, leche y pan para todos, y una gran bandeja de pescado. 

María habla muy poco. Está tranquila y silenciosa, sentada en su rinconcito, con las manos bajo su manto sobre el regazo. Habla poco y se mueve poco, incluso cuando bebe una taza de leche, y al comer un racimo de uvas doradas con un poco de pan. Mira a José apenada y escrutadora al mismo tiempo. 

También él la mira. Pasado un rato, inclinándose hacia su hombro, le pregunta: 

- ¿Estás cansada? ¿Te duele algo? Estás pálida y triste. 

Me duele separarme de Juanín. Lo quiero. Le he tenido sobre mi corazón desde pocos momentos después de nacer... José no pregunta nada más. 

Ha llegado la hora de la partida de Zacarías. El carro se para delante de la puerta. Todos se acercan. Las dos primas se abrazan con amor. María besa una y otra vez al pequeñuelo antes de depositarlo sobre el regazo de su madre, que ya está sentada en el carro. Luego saluda a Zacarías y le pide su bendición. Al arrodillarse delante del sacerdote, el manto se le desliza de los hombros y las formas le aparecen en la luz intensa de la tarde estival. No sé si José las percibe en este momento en que está ocupado en saludar a Isabel. El carro se pone en movimiento. 

José con María entran de nuevo en casa. Ella vuelve a su sitio del rincón semioscuro. 

Si no te importa viajar de noche, yo propondría salir con la puesta del Sol. El calor, durante el día, es fuerte; la noche, en cambio, estará fresca y serena. Lo digo por ti, para que no cojas demasiado sol. Para mí no es nada el estar bajo el sol intenso, pero tú... 

Como quieras, José. Yo también veo conveniente caminar de noche. 

La casa — dice José — está toda en orden, como también el huertecillo. ¡Vas a ver qué flores más bonitas! Vas a llegar a tiempo de verlas florecer todas. El manzano, la higuera y la vid están repletos de frutos como nunca lo han estado; y he tenido que apuntalar el granado, pues sus ramas están cargadísimas de frutos, maduros ya como jamás se vio en esta época. Y el olivo... Dispondrás de aceite en abundancia. Ha tenido una florescencia milagrosa y no se ha perdido ni una flor. Todas son ya pequeñas aceitunas. Cuando estén maduras, el árbol parecerá lleno de oscuras perlas. Tan bonito como tu huerto no hay ningún otro en Nazaret. La familia está asombrada. Alfeo dice que se trata de un prodigio. 

Obra de tus cuidados». 

- ¡Oh, no! ¡Yo soy sólo un pobre hombre! ¿Qué he hecho yo realmente? Cuidar un poco los árboles, echar un poco de agua a las flores... Mira, te he hecho una fuente donde acaba el huerto, al lado de la gruta, y he dispuesto allí un pilón. Así no tendrás que salir para coger agua. La he traído de ese manantial que está encima del olivar de Matías. Es pura y abundante. Te he hecho llegar un pequeño regato. He construido un canalillo bien tapado, y ahora llega y canta como un arpa. Me dolía el que tuvieras que ir a la fuente del pueblo y volver cargada con las ánforas llenas de agua. 

Gracias José. ¡Tú eres bueno! 

Los dos esposos guardan silencio ahora, como cansados. José incluso se queda transpuesto. María ora. Cae la tarde. 

Los huéspedes insisten en que antes de ponerse en camino coman otra vez. José come pan y pescado; María, sólo fruta y leche. 

Luego se inicia la marcha. Montan sus burritos. José ha atado sobre su asno, como cuando venían, el baulillo de María, y, antes de que Ella monte en el borriquillo, comprueba que la albardilla esté bien segura. Veo que José observa a María cuando se monta, pero no dice nada. 

Bajo las primeras estrellas que empiezan a latir en el cielo, comienza el viaje. Se apresuran, quizás para llegar a las puertas ante de que las cierren. Al salir de Jerusalén y coger la vía de Galilea, ya el cielo sereno está repleto de estrellas y hay un gran silencio en el campo. Sólo se oye el canto de algún ruiseñor y el choque de las pezuñas de los dos borriquillos contra el terreno duro de la vía abrasada por el verano. 

Dice María: 

\emph{Es la víspera de Jueves Santo. A algunos les parecerá que la visión está fuera de lugar. Y, sin embargo, tu dolor de amante de mi Jesús Crucificado está en tu corazón, y permanece aunque se presente una dulce visión. Ésta es como el calorcillo producido por una llama: por una parte, fuego todavía; por otra, ya no. El fuego es la llama, no su calor, que no es sino una derivación de ella. Ninguna visión beatífica o pacífica podrá quitar de tu corazón ese dolor. Considéralo más valioso que tu misma vida, porque es el don mayor que Dios puede conceder a quien cree en su Hijo. Además, mi visión, dentro de su paz, no desentona con las solemnidades de esta semana. Mi José sufrió también su Pasión, que comenzó en Jerusalén cuando notó mi estado; y duró algunos días, como en el caso de Jesús y mío. No fue, espiritualmente, poco dolorosa. Sólo fue la santidad de mi justo esposo lo que la contuvo, y en tal modo, tan digno y secreto, que ha pasado los siglos siendo poco notada. ¡Oh, nuestra primera Pasión! ¿Quién podrá referir su íntima y silenciosa intensidad, y mi dolor al constatar que aún no me había llegado del Cielo la ayuda que esperaba, de revelarle a José el Misterio? Comprendí que lo ignoraba al verlo conmigo con la misma actitud respetuosa que de costumbre. Si él hubiera sabido que llevaba en mí al Verbo de Dios, habría adorado a ese Verbo cerrado en mi seno con actos de veneración propios de Dios. Sí, José habría realizado esos actos, y yo no habría rehusado recibirlos, no por mí, sino por Aquel que estaba en mí y que yo llevaba, de la misma forma que el Arca de la alianza llevaba el código de piedra y los vasos de maná. ¿Quién podrá describir mi batalla contra el desánimo que pretendía subyugarme para persuadirme de que había esperado en vano en el Señor? ¡Oh, creo que fue la rabia de Satanás! Sentí surgirme la duda a las espaldas, y sentí cómo alargaba ésta sus gélidas zarpas para aprisionarme el alma y detener su oración. La duda... tan peligrosa, letal para el espíritu. Letal, porque es el primer elemento agente de la enfermedad mortal que tiene por nombre "desesperación"; contra él se debe reaccionar con todas las fuerzas, para no perecer en el alma y perder a Dios. ¿Quién podrá exponer con exacta verdad el dolor de José, sus pensamientos, la turbación de sus sentimientos? Él se encontraba, cual barquichuela en medio de una gran tempestad, en un remolino de ideas contrapuestas, en un torbellino de reflexiones a cuál más mordiente y penosa. Era un hombre aparentemente traicionado por su mujer. Veía que se derrumbaban juntos su buen nombre y la estima del mundo; por causa de Ella se veía ya señalado con el dedo y compadecido por el pueblo. Ante la evidencia de un hecho, veía caer muertos el afecto y la estima puestos en mí. Su santidad aquí resplandece aún más alta que la mía. De ello doy testimonio con afecto de esposa, porque quiero que améis a mi José, a este hombre sabio y prudente, a este hombre paciente y bueno, el cual no está desligado del misterio de la Redención, antes bien, está íntimamente relacionado con él, porque por este misterio apuró el dolor y se consumió, salvándoos al Salvador con su sacrificio y santidad. Si hubiera sido menos santo, hubiera actuado humanamente, denunciándome como adúltera para que me hubieran lapidado y pereciera conmigo el hijo de mi pecado. Si hubiera sido menos santo, Dios no le habría concedido la guía de su luz en tan ardua prueba. Pero José era santo. Su espíritu puro vivía en Dios, y tenía una caridad encendida y fuerte, y por la caridad os salvó al Salvador, tanto cuando no me acusó ante los ancianos, como cuando, dejándolo todo con diligente obediencia, salvó a Jesús en Egipto. Aunque breves numéricamente, los tres días de la Pasión de José fueron de tremenda intensidad; como también la mía, esta primera pasión mía. En efecto, yo comprendía su sufrimiento, y no podía aliviarlo en modo alguno, por obediencia al decreto de Dios que me había dicho: "¡Guarda silencio!". ¡Ay, y, llegados a Nazaret, cuando lo vi marcharse, tras un lacónico saludo, cabizbajo y como envejecido en poco tiempo, y no volver por la tarde como solía hacer, os digo, hijos, que mi corazón lloró con grandísima aflicción! Sola, cerrada en mi casa, en la casa en que todo me recordaba el Anuncio y la Encarnación, y donde todo me recordaba a José, desposado conmigo en intachable virginidad, tuve que resistir contra el abatimiento y las insinuaciones de Satanás, y esperar, esperar, tener esperanza, y orar, orar, orar, y perdonar, perdonar, perdonar la sospecha de José, su movimiento interior de justa indignación. Hijos, es necesario esperar, orar, perdonar, para obtener que Dios intervenga en favor nuestro. Vivid también vosotros vuestra pasión, merecida por vuestras culpas. Yo os enseño a superarla y convertirla en gozo. Esperad sin medida, orad con confianza, perdonad para ser perdonados; el perdón de Dios será, hijos, la paz que deseáis. Por ahora no os digo nada más. Hasta pasado el triunfo pascual, silencio. Es la Pasión (esta revelación se la dio Dios a María Valtorta en Semana Santa). Sed compasivos para con vuestro Redentor. Oíd sus quejidos, contad sus heridas y sus lágrimas, cada una de las cuales fue vertida por vosotros, fue padecida por vosotros. Desaparezca cualquier otra visión ante esta que os recuerda la Redención que por vosotros se ha cumplido. }

\chapter*{José pide perdón a María \\ \normalfont\normalsize\textit{Fe, caridad y humildad para recibir a Dios.}}
\addcontentsline{toc}{chapter}{\normalfont\scshape{José pide perdón a María}}

Después de 53 días, la Madre reanuda sus manifestaciones con esta visión, y me dice que la escriba en este libro. La alegría me invade. Ver a María, en efecto, es poseer la Alegría. 

Así, veo el huertecillo de Nazaret. María está hilando a la sombra de un tupidísimo manzano repleto de frutos, que ya empiezan a tomar color rojo y que parecen, con su redondez y color rosado, carrillos de niño. 

Sin embargo, María no tiene, de ninguna manera, ese color. Le ha desaparecido la linda coloración que, en Hebrón, avivaba su cara. En la palidez de marfil de su rostro, sólo los labios trazan una curva de pálido coral. Bajo los párpados semicerrados hay dos sombras oscuras y los bordes de los ojos están hinchados como en quien ha llorado. No veo los ojos, porque Ella está con la cabeza más bien agachada, pendiente de su trabajo y, sobre todo, de un pensamiento suyo, que debe afligirla, pues la oigo suspirar como quien tuviera un pesar en el corazón.

Está toda vestida de blanco, de lino blanco; es que hace mucho calor, a pesar de que la frescura todavía intacta de las flores me dice que es por la mañana. Tiene la cabeza descubierta, y el Sol, que juega con las frondas del manzano movidas por un ligerísimo viento, y se filtra con agujas de luz hasta tocar la tierra oscura de los parterres, deposita en su cabeza rubia aritos de luz en que los cabellos parecen de oro cobrizo. 

De la casa no viene ningún ruido, ni tampoco de los lugares cercanos. Se oye sólo el murmullo del regatillo que va a un pilón del fondo del huerto. 

María se estremece al oír un golpe dado con resolución a la puerta de la casa. Apoya rueca y huso y se levanta para ir a abrir. A pesar de que el vestido sea suelto y amplio, no llega a ocultar completamente la rotundidad de su pelvis. 

Se encuentra de frente a José. María palidece, hasta incluso en los labios. Ahora su rostro parece una hostia de lo pálido que está. María mira con ojos que escrutan tristemente; José, con ojos que parecen suplicar. Guardan silencio, mirándose. María rompe el silencio: 

- ¿A esta hora, José? ¿Necesitas algo? ¿Qué deseas decirme? Ven. 

José entra y cierra la puerta. Todavía guarda silencio. 

Habla, José. ¿Qué deseas de mí? 

Tu perdón - José se curva como si quisiera arrodillarse. Pero María, siempre tan reservada al tocarlo, lo agarra con resolución por los hombros y se lo impide. 

El color aparece y desaparece del rostro de María, ora completamente rojo, ora de nieve como antes. 

- ¿Mi perdón? No tengo nada que perdonarte, José. No debo sino agradecerte una vez más todo cuanto has hecho aquí dentro, en mi ausencia, y el amor que me tienes. 

José la mira. Veo formarse dos gruesas gotas en la cavidad de sus ojos profundos, permanecer como en el borde de un recipiente, y luego deslizarse por las mejillas y por la barba. 

Perdóname, María. He desconfiado de ti. Ahora veo. Soy indigno de poseer tan gran tesoro. He faltado a la caridad, te he acusado en mi corazón, te he acusado injustamente porque no te había preguntado la verdad. He faltado contra la ley de Dios no amándote como yo me habría amado a mí mismo... 

- ¡Oh, no, no has faltado! 

Sí, María. Si yo hubiera sido acusado de un delito así, me habría defendido. Tú... No te concedía defenderte, porque ya iba a tomar decisiones sin preguntarte primero. He faltado contra ti ofendiéndote con una sospecha. El solo hecho de sospechar ya es ofensa, María. El que sospecha no conoce. Yo no te he conocido como debía. Pero, por el dolor que he sufrido... tres días de suplicio... perdóname, María. 

No tengo nada que perdonarte. Es más, te pido yo perdón por el dolor que te he causado. 

- ¡Oh, dolor sí que fue! ¡Cuánto dolor! Fíjate, esta mañana me han dicho que tengo las sienes canosas y arrugas en la cara. ¡Estos días han significado más de diez años de vida! Pero, María, ¿por qué has sido tan humilde de celarme a mí, tu esposo, tu gloria, y permitirme que sospechara de ti? 

José no está de rodillas, pero sí tan curvado que es como si lo estuviera. María le pone su mano en la cabeza, y sonríe. 

Parece como si lo absolviera. Dice: 

Si no lo hubiera sido de modo perfecto, no habría merecido concebir al Esperado, que viene a anular la culpa de soberbia que ha destruido al hombre. Y además no he hecho sino obedecer... Dios me pidió esta obediencia... Me ha costado mucho,. por ti, por el dolor que te produciría... pero, tenía que obedecer. Soy la Esclava de Dios, y los siervos no discuten las órdenes que reciben; las ejecutan, José, aunque provoquen lágrimas de sangre. 

María, mientras dice esto, llora silenciosamente, tan silenciosamente que José, agachado como está, no lo advierte hasta que no cae una lágrima al suelo. Entonces, levanta la cabeza y — es la primera vez que le veo hacer este gesto — aprieta las manos de María entre las suyas, oscuras y fuertes, y besa la punta de sus rosados y delgados dedos, de esos dedos que sobresalen del anillo de sus manos como capullos de melocotonero. 

Ahora habrá que tomar las medidas necesarias para que... - José no sigue; mira al cuerpo de María, y Ella se pone como la púrpura, y se sienta de golpe para apartar sus formas de la mirada que la observa - Habrá que actuar rápidamente. Yo vendré aquí... Cumpliremos la ceremonia de la boda... La próxima semana. ¿Te parece bien? 

Todo lo que tú haces está bien, José. Tú eres el jefe de la casa; yo, tu sierva. 

No. Yo soy tu siervo. Yo soy el devoto siervo de mi Señor que crece en tu seno. Bendita tú entre todas las mujeres de Israel. Esta tarde aviso a los parientes. Y después... ya estando yo aquí, nos dedicaremos a preparar todo para recibir... ¡Oh, cómo podré recibir en mi casa a Dios; en mis brazos, a Dios? ¡Moriré de gozo!.. ¡Jamás podré osar tocarle!... 

Podrás, como yo, por gracia de Dios. 

Pero tú eres tú. ¡Yo soy un pobre hombre, el más pobre de los hijos de Dios!... 

Jesús viene por nosotros, pobres, para hacernos ricos en Dios; viene a nosotros dos porque somos los más pobres y reconocemos que lo somos. Exulta, José. La estirpe de David tiene a su Rey esperado, y nuestra casa va a ser más fastuosa que el palacio de Salomón, porque aquí estará el Cielo y compartiremos con Dios el secreto de paz que después conocerán los hombres. Crecerá entre nosotros dos. Nuestros brazos le servirán de cuna al Redentor durante su crecimiento, y nuestras fatigas le procurarán el pan... ¡Oh, José! Oiremos la voz de Dios llamándonos "¡Padre y Madre!" ¡Oh!... 

María llora de alegría; ¡un llanto tan feliz...! Y José, arrodillado ahora, a sus pies, llora, con su cabeza casi oculta en el amplio vestido de María que cae, formando pliegues, sobre las pobres baldosas de la reducida estancia. La visión termina en este momento. 

Dice María: 

\emph{Que nadie interprete erróneamente mi palidez. No provenía de miedo humano. Humanamente no podía esperar sino la lapidación. Pero no temía por eso. Sufría por el dolor de José. Y, en cuanto al pensamiento de que me acusara, no me turbaba tampoco por mí; lo único que me contrariaba era que él, insistiendo en acusarme, hubiera podido faltar a la caridad. Cuando le vi, por este motivo, la sangre se me fue toda al corazón; era el momento en que un justo, ofendiendo a la Caridad, habría podido ofender a la Justicia. Y el hecho de que un justo hubiera cometido una falta — él, que no la cometía nunca — me hubiera producido un dolor supremo. Si yo no hubiera sido humilde hasta el extremo límite — como he dicho a José — no habría merecido llevar en mí a Aquel que, para borrar la soberbia en la raza, siendo Dios, se anonadaba a sí mismo hasta la humillación de ser hombre. Te he mostrado esta escena, no recogida por ningún Evangelio, porque quiero atraer la atención, demasiado extraviada, de los hombres hacia las condiciones esenciales para agradar a Dios y para recibir su continuo hacerse presente en los corazones. Fe. José creyó ciegamente en las palabras del enviado celeste. No pedía otra cosa sino creer, porque tenía la convicción sincera de que Dios era bueno y de que el Señor no le depararía el dolor de ser un hombre traicionado, defraudado por su prójimo, un hombre de quien su prójimo se burlara, pues esperaba en el Señor. No pedía otra cosa sino creer en mí, porque, siendo honesto como era, sólo con dolor podía pensar que otro no lo fuera. Él vivía la Ley, y la Ley dice: 'Ama a tu prójimo como a ti mismo". Nuestro amor hacia nosotros mismos es tanto que nos creemos perfectos aun cuando no lo somos; y, ¿por qué, entonces, vamos a desamar al prójimo pensándole imperfecto? Caridad absoluta. Caridad que sabe perdonar, que quiere perdonar: perdonar de antemano, disculpando dentro del propio corazón las faltas del prójimo; perdonar en el momento, concediendo todos los atenuantes al culpable. Humildad tan absoluta como la caridad. Saber reconocer que se ha cometido falta incluso con el simple pensamiento, y no tener ese orgullo, que es más nocivo que la culpa antecedente, de no querer decir: "He cometido un error". Menos Dios, todos cometen errores. ¿Quién podrá decir: "Yo nunca cometo errores"? Y esa humildad aún más difícil de saber callar las maravillas de Dios en nosotros— cuando el darle gloria no requiera proclamarlas — para que el prójimo, que no tiene esos dones especiales de Dios, no se sienta menos. ¡Oh, si quiere Dios, si quiere, se manifestará en su siervo! Isabel me "vio" como yo era cuando llegó la hora, y mi esposo supo lo que yo realmente era cuando le llegó la hora de saberlo. Dejad que sea el Señor quien se preocupe de proclamaros siervos suyos. Él tiene amorosa prisa de hacerlo, porque toda criatura elevada a una misión especial es una nueva gloria que se añade a la suya, ya infinita, porque es testimonio de lo que el hombre es en el estado en que Dios lo quería: una perfección subordinada que refleja a su Autor. ¡Permaneced en la sombra y en el silencio, oh vosotros, predilectos de la Gracia, para poder oír las únicas palabras de "vida" que existen, para poder merecer el tener sobre vosotros y en vosotros el Sol que, eterno, resplandece! ¡Oh, Luz beatísima que eres Dios, que eres la alegría de tus siervos, resplandece sobre estos siervos tuyos y así exulten en su humildad, alabándote a ti, sólo a ti, que dispersas a los soberbios y en cambio elevas a los esplendores de tu Reino a los humildes que te aman. }

\chapter*{El edicto de empadronamiento. \\ \normalfont\normalsize\textit{Enseñanzas sobre el amor al esposo y la confianza en Dios.}}
\addcontentsline{toc}{chapter}{\normalfont\scshape{El edicto de empadronamiento.}}
 
De nuevo veo la casa de Nazaret, la pequeña habitación en que María habitualmente come. Ahora Ella está trabajando en una tela blanca. La deja para ir a encender una lámpara, pues está atardeciendo y no ve ya bien con la luz verdosa que entra por la puerta entornada que da al huerto. Cierra también la puerta. 

Observo que su cuerpo está ya muy engrosado, pero sigue viéndosele muy hermosa. Su paso continúa siendo ágil; todos sus movimientos, donosos. No se ve en Ella ninguna de esas sensaciones de peso que se notan en la mujer cuando está próxima a dar a luz a un niño. Sólo en el rostro ha cambiado. Ahora es "la mujer". Antes, cuando el Anuncio, era una jovencita de carita serena e ingenua (como de niño inocente). Luego, en la casa de Isabel, cuando el nacimiento del Bautista, su rostro se había perfeccionado, adquiriendo una gracia más madura. Ahora es el rostro sereno, pero dulcemente majestuoso, de la mujer que ha alcanzado su plena perfección en la maternidad. 

Ya no recuerda a esa "Virgen de la Anunciación" de Florencia. Cuando era niña, yo sí que la veía reflejada en ella. Ahora el rostro es más alargado y delgado; la mirada, más pensativa y grande. En pocas palabras: como es María actualmente en el Cielo. Porque ahora ha asumido el aspecto y la edad del momento en que nació el Salvador. 

Tiene la eterna juventud de quien no sólo no ha conocido corrupción de muerte, sino que ni siquiera ha conocido el marchitamiento de los años. El tiempo no ha tocado a esta Reina nuestra y Madre del Señor que ha creado el tiempo. Es verdad que en el suplicio de los días de la Pasión — suplicio que para Ella empezó muchísimo antes, podría decir que desde que Jesús comenzó la evangelización — se la vio envejecida, pero tal envejecimiento era sólo como un velo corrido por el dolor sobre su incorruptible cuerpo. Efectivamente, desde cuando Ella vuelve a ver a Jesús, resucitado, torna a ser la criatura fresca y perfecta de antes del suplicio: como si al besar las santísimas Llagas hubiera bebido un bálsamo de juventud que hubiese cancelado la obra del tiempo y, sobre todo, del dolor. También hace ocho días, cuando he visto la venida del Espíritu Santo el día de Pentecostés, veía a María "hermosísima y, en un instante, rejuvenecida", como escribía; ya antes había escrito: "Parece un ángel azul". Los ángeles no experimentan la vejez. Poseen eternamente la belleza de la eterna juventud, del eterno presente de Dios que en sí mismos reflejan. 

La juventud angélica de María, ángel azul, se completa y alcanza la edad perfecta — que se ha llevado consigo al Cielo y que conservará eternamente en su santo cuerpo glorificado, cuando el Espíritu pone el anillo nupcial a su Esposa y la corona en presencia de todos — ahora, y no ya en el secreto de una habitación ignorada por el mundo, con un arcángel como único testigo. 

He querido hacer esta digresión porque la consideraba necesaria. 

Ahora vuelvo a la descripción. 

María, pues, ahora ya es verdaderamente "mujer", llena de dignidad y donaire. Incluso su sonrisa se ha transformado, en dulzura y majestad. ¡Qué hermosa está María! 

Entra José. Da la impresión de que vuelve del pueblo, porque entra por la puerta de la casa y no por la del taller. María levanta la cabeza y le sonríe. También José le sonríe a Ella... no obstante, parece como si lo hiciera forzado, como quien estuviera preocupado. María lo observa escrutadora y se levanta para coger el manto que José se está quitando, para doblarlo y colocarlo encima de un arquibanco. 

José se sienta al lado de la mesa. Apoya en ella un codo y la cabeza en una mano mientras con la otra, absorto, se peina y despeina alternativamente la barba. 

- ¿Estás preocupado por algo? - pregunta María. - ¿Te puedo servir de consuelo? 

Tú siempre me confortas, María. Pero esta vez es una gran preocupación... por ti. 

- ¿Por mí, José? ¿Y qué es, pues? 

Han puesto un edicto en la puerta de la sinagoga. Ha sido ordenado el empadronamiento de todos los palestinos. Hay que ir a anotarse al lugar de origen. Nosotros tenemos que ir a Belén... 

- ¡Oh! - interrumpe María, llevándose una mano al pecho. 

- ¿Te preocupa, verdad? Es penoso. Lo sé. 

No, José, no es eso. Pienso... pienso en las Sagradas Escrituras: Raquel, madre de Benjamín y esposa de Jacob, del cual nacerá la Estrella, el Salvador. Raquel, que está sepultada en Belén; de la que se dijo: "Y tú, Belén Efratá, eres la más pequeña entre las tierras de Judá, mas de ti saldrá el Dominador", el Dominador prometido a la estirpe de David; Él nacerá allí... 

- ¿Piensas... piensas que ya ha llegado el momento? ¡Oh! ¿Qué podemos hacer? - José está enormemente preocupado y mira a María con ojos llenos de compasión. 

Ella lo percibe, y sonríe. Su sonrisa es más para sí que para él. Es una sonrisa que parece decir: «Es un hombre; justo, pero hombre. Y ve como hombre, piensa como hombre. Sé compasiva con él, alma mía, y guíalo a la visión de espíritu». Y su bondad la impulsa a tranquilizarlo. No mintiendo, sino tratando de quitarle la preocupación, le dice: 

No sé, José. El momento está muy cercano, pero, ¿no podría el Señor alargarlo para aliviarte esta preocupación? Él todo lo puede. No temas. 

- ¡Pero el viaje!.. Y además, ¡con la cantidad de gente que habrá!.. ¿Encontraremos un buen lugar para alojarnos? ¿Nos dará tiempo a volver? Y si... si eres Madre allí, ¿cómo nos las arreglaremos? No tenemos casa... No conocemos a nadie.... 

No temas. Todo saldrá bien. Dios provee para que encuentre un amparo el animal que procrea, ¿y piensas que no proveerá para su Mesías? Nosotros confiamos en Él, ¿no es verdad? Siempre confiamos en Él, Cuanto más fuerte es la prueba, más confiamos. Como dos niños, ponemos nuestra mano en su mano de Padre. Él nos guía. Estamos completamente abandonados en Él. Mira cómo nos ha conducido hasta aquí con amor. Ni el mejor de los padres podría haberlo hecho con más esmero. Somos sus hijos y sus siervos. Cumplimos su voluntad. Nada malo nos puede suceder. Este edicto también es voluntad suya. ¿Qué es César, sino un instrumento de Dios? Desde que el Padre decidió perdonar al hombre, ha predispuesto los hechos para que su Hijo naciera en Belén. Antes de que ella, la más pequeña de las ciudades de Judá, existiera, ya estaba designada su gloria. Para que esta gloria se cumpla y la palabra de Dios no quede en entredicho — y lo quedaría si el Mesías naciera en otro lugar — he aquí que ha surgido un poderoso, muy lejos de aquí, y nos ha dominado, y ahora quiere saber quiénes son sus súbditos, ahora, en un momento de paz para el mundo... ¡Qué es una pequeña molestia nuestra comparada con la belleza de este momento de paz! Fíjate, José, ¡un tiempo en que no hay odio en el mundo! ¿Existe, acaso, hora más feliz que ésta, para que surja la "Estrella" de luz divina y de influjo redentor? ¡Oh, no tengas miedo, José! Si inseguros son los caminos, si la muchedumbre dificulta la marcha, los ángeles serán nuestra defensa y nuestro parapeto; no de nosotros, sino de su Rey. Si no encontramos un lugar donde ampararnos, sus alas nos harán de tienda. Nada malo nos sucederá, no puede sucedemos: Dios está con nosotros. 

José la mira y la escucha con devoción.. Las arrugas de la frente se alisan, la sonrisa vuelve. Se pone en pie, ya sin cansancio y sin pena. Sonríe. 

- ¡Bendita tú, Sol del espíritu mío! ¡Bendita tú, que sábese ver todo a través de la Gracia que te llena! No perdamos tiempo, pues, porque hay que partir lo antes posible y... volver cuanto antes, para que aquí todo está preparado para el... para el.... 

Para el Hijo nuestro, José. Tal debe ser a los ojos del mundo, recuérdalo. El Padre ha velado de misterio esta venida suya, y nosotros no debemos descorrer el velo. Él, Jesús, lo hará, llegada la hora... 

La belleza del rostro, de la mirada, de la expresión, de la voz de María al decir este «Jesús» no es describible. Es ya el éxtasis, y con este éxtasis cesa la visión. 

Dice María: 

\emph{No añado mucho, porque mis palabras son ya enseñanza. Eso sí, reclamo la atención de las mujeres casadas sobre un punto. Demasiadas uniones se transforman en desuniones por culpa de las mujeres, las cuales no tienen hacia el marido ese amor que es todo (amabilidad, compasión, consuelo). Sobre el hombre no pesa el sufrimiento físico que oprime a la mujer, pero sí todas las preocupaciones morales: necesidad de trabajo, decisiones que hay que tomar, responsabilidades ante el poder establecido y ante la propia familia... ¡Oh, cuántas cosas pesan sobre el hombre, y cuánta necesidad tiene también él de consuelo! Pues bien, es tal el egoísmo, que la mujer le añade al marido cansado, desilusionado, abrumado, preocupado, el peso de inútiles quejas, e incluso a veces injustas. Y todo porque es egoísta; no ama. Amar no significa satisfacer los propios sentidos o la propia conveniencia. Amar es satisfacer a la persona amada, por encima de los sentidos y conveniencias, ofreciéndole a su espíritu esa ayuda que necesita para poder tener siempre abiertas las alas en el cielo de la esperanza y de la paz. Hay otro punto en el que querría que centrarais vuestra atención. Ya he hablado de ello; no obstante, insisto. Se trata de la confianza en Dios. La confianza compendia las virtudes teologales. Si uno tiene confianza, es señal de que tiene fe; si tiene confianza, es señal de que espera y de que ama. Cuando uno ama, espera y cree en una persona, tiene confianza. Si no, no. Dios merece esta confianza nuestra. Si se la damos a veces a pobres hombres capaces de cometer faltas, ¿por qué negársela a Dios, que no comete falta alguna? La confianza es también humildad. El soberbio dice: "Voy a actuar por mí mismo. No me fío de éste, que es un incapaz, un embustero y un avasallador". El humilde dice: "Me fío. ¿Por qué no me voy a fiar? ¿Por qué debo pensar que yo soy mejor que él?". Y así, con mayor razón, de Dios dice: "¿Por qué voy a tener que desconfiar de Aquel que es bueno? ¿Por qué voy a tener que pensar que me basto por mí mismo?". Dios se dona al humilde, del soberbio se retira. La confianza es, además, obediencia; y Dios ama al obediente. La obediencia es signo de que nos reconocemos hijos suyos, de que lo reconocemos como Padre; y un padre, cuando es verdadero padre, no puede hacer otra cosa sino amar. Dios es para nosotros Padre verdadero y perfecto. Hay un tercer punto que quiero que meditéis. Se funda también en la confianza. Ningún hecho puede acaecer si Dios no lo permite. Por lo cual, ya tengas poder, ya seas súbdito, será porque Dios lo ha permitido. Preocúpate, pues, ¡oh tú que tienes poder!, de no hacer de este poder tuyo tu mal. En cualquier caso sería "tu mal", aunque en principio pareciese que lo fuera de otros. En efecto, Dios permite, pero no sin medida; y, si sobrepasas el punto señalado, asesta el golpe y te hace pedazos. Preocúpate, pues, tú que eres súbdito, de hacer de esta condición tuya una calamita para atraer hacia ti la celeste protección. No maldigas nunca. Deja que Dios se ocupe de ello. A Él, Señor de todos, le corresponde bendecir o maldecir a los seres que ha creado. }

\chapter*{La llegada a Belén.}
\addcontentsline{toc}{chapter}{\normalfont\scshape{La llegada a Belén.}}

Veo una vía de primer orden muy transitada. Jumentos que van cargados de todo tipo de cosas y de personas. Jumentos que regresan. La gente, azuza a sus cabalgaduras. Otros, los que van a pie, caminan deprisa porque hace frío. 

Hay un aire terso y seco, el cielo está sereno; todo tiene, no obstante, ese filo neto de los días de pleno invierno. El campo, desnudo, parece más grande; está poco crecida y ya requemada por los vientos invernales la hierba de los pastos en que las ovejas buscan un poco de alimento, y también de sol, que está saliendo poco a poco. Están pegadas las unas a las otras, porque también ellas tienen frío; y balan, levantando el morro y mirando al Sol como diciendo: « ¡Ven pronto, que hace frío!». El terreno es ondoso. Las sinuosidades se hacen cada vez más netas; es propiamente una zona de colinas, con depresiones herbosas y laderas, con pequeños valles y cimas. El camino pasa por el medio en dirección sudeste. 

María va montada en un borriquillo pardo, toda arropada en su grueso manto. En la parte de adelante de la albardilla está ese arnés ya visto en el viaje hacia Hebrón; encima, el baulillo con las cosas más necesarias. 

José camina al lado llevando las riendas. De vez en cuando le pregunta a María si está cansada. Ella lo mira sonriendo y le responde que no; pero a la tercera vez añade: 

Tú sí que estarás cansado, que vas a pie. 

- ¡Oh!, ¿yo? Para mí no es nada. Lo que pienso es que si hubiera encontrado otro asno podrías ir más cómoda y además llegaríamos antes. Pero, me ha sido imposible encontrarlo; ahora todos necesitan una cabalgadura. ¡Ánimo de todas formas! Pronto llegaremos a Belén. Al otro lado de aquel monte está Efratá. 

Ahora guardan silencio. La Virgen cuando calla parece recogerse internamente en oración. Sonríe dulcemente por un pensamiento suyo, y, cuando mira a la gente, parece como si no viera en ella lo que es (un hombre, una mujer, un anciano, un pastor, un rico o un pobre), sino eso que sólo Ella ve. 

- ¿Tienes frío? - pregunta José, dado que empieza a levantarse viento. 

No, gracias». 

Pero José no se fía. Le toca los pies, que penden por el lado del borriquillo, los pies calzados en las sandalias y que apenas si se ven sobresalir del largo vestido; debe sentirlos fríos porque menea la cabeza y se quita una manta que llevaba en bandolera y arropa con ella las piernas de María, y se la extiende también sobre el regazo, de forma que sus manos, bajo la cobija y el manto, estén bien calientes. 

Encuentran a un pastor, que corta el camino con su rebaño, pasando de los pastos de la derecha a los de la izquierda. José se inclina hacia él para decirle algo. El pastor hace un gesto afirmativo. José toma el borriquillo y tira de él detrás del rebaño hasta el prado. El pastor saca de una alforja una tosca escudilla, ordeña a una gruesa oveja de ubres llenas, da la escudilla a José y éste a su vez se la ofrece a María. 

- ¡Que Dios os bendiga a los dos! — dice María —. A tí, por tu amor; y a tí por tu bondad. Oraré por ti. - ¿Venís de lejos? 

De Nazaret - responde José. 

- ¿Y vais hacia...? 

A Belén. 

Largo viaje para esta mujer en este estado. ¿Es tu esposa? 

Es mi esposa». 

- ¿Tenéis dónde ir? 

No. 

- ¡Mala cosa! Belén está llena de gente llegada de todas partes para inscribirse o para ir a otro lugar, No sé si encontraréis alojamiento. ¿Conoces bien este lugar? 

No mucho. 

Bueno, pues... yo te digo... por ella (y señala a María). Preguntad por la posada. Estará llena. Más que nada os lo digo como referencia. Está en una plaza, en la más grande. Se llega por este mismo camino, no hay pérdida posible. Delante hay una fuente. La posada es grande y baja y tiene un portal grande. Estará llena. De todas formas, si no encontráis nada en ella ni en las otras casas, id a la parte de atrás de la posada, hacia el campo. En el monte hay unos establos que algunas veces les sirven a los mercaderes que van a Jerusalén para meter a los animales que no tienen sitio en la posada. Son establos — ya sabéis — que están en el monte; por tanto, húmedos, fríos y sin puerta. Pero son al menos un refugio; esta mujer... no puede quedarse en la calle. Quizás allí encontréis un sitio... y heno para dormir y para el burro... ¡Y que Dios os acompañe! - ¡Y que alegre tus días! - responde María. José en cambio dice: 

La paz sea contigo. 

Vuelven al camino. Salvan una prominencia del terreno desde la que se ve una depresión más vasta limitada por delicadas pendientes. En la cuenca y arriba y abajo por las laderas hay casas y más casas: es Belén. 

Estamos en la tierra de David, María. Ahora podrás descansar. Te veo muy cansada... 

No. Estaba pensando... estoy pensando... 

María le coge la mano a José y, sonriendo con beatitud, le dice: 

Tengo la firme impresión de que ha llegado el momento. 

- ¡Dios de misericordia! ¿Qué hacemos? 

No te preocupes, José. Permanece firme. ¿No ves lo tranquila que estoy yo». - Pero estás sufriendo mucho. 

- ¡Oh! ¡No! Estoy llena de gozo. Siento un júbilo tal, tan fuerte, tan hermoso, tan incontenible, que mi corazón late fortísimamente y me dice: "¡Va a nacer! ¡Va a nacer!". Lo dice en cada latido. Es mi Niño, que llama a mi corazón y me dice: "Mamá, estoy aquí, vengo a darte el beso de Dios". ¡Oh, qué alegría, José mío! 

José, sin embargo, no está jubiloso. Piensa más bien en la urgencia de encontrar un lugar donde ampararse, y acelera el paso. Puerta por puerta lo solicita... Nada. Todo lleno. Llegan a la posada... Está llena, incluso con gente prácticamente al raso bajo el rústico pórtico que rodea el vasto patio interior. 

José deja a María montada en su burrito, dentro del patio, y sale para buscar en las otras casas. Vuelve desconsolado. No hay ningún sitio. El rápido crepúsculo invernal comienza a extender sus velos. José le suplica al posadero, suplica a los que han venido de fuera: ellos son hombres, y están sanos; aquí hay una mujer que está para dar a luz a un hijo; que tengan piedad... Nada. 

Un rico fariseo, que los está mirando con desprecio manifiesto, cuando María se acerca, se separa como si hubiera sido una leprosa la que se hubiera acercado. José le mira, y se le enciende de indignación el rostro. María le pone una mano en su muñeca, para calmarlo y le dice: 

- No insistas. Vamos. Dios proveerá, 

Salen. Siguen el muro de la posada. Tuercen por una callejuela encajonada entre aquélla y unas casas pobres. Giran 

hacia la parte de atrás de la posada. Buscan. Hay una especie de grutas. Por lo bajas que son y lo húmedas que están, diría que más que establos son bodegas. Las más lindas ya están ocupadas. José siente caérsele el alma a los pies. 

- ¡Eh! ¡Galileo! - le grita por detrás un viejo - Allí, en el fondo, bajo aquellas ruinas, hay una guarida. Quizás todavía no se ha metido nadie. 

Se apresuran hacia esa "guarida". Es realmente una guarida. Entre las ruinas de lo que sería un edificio, hay una abertura; dentro, una gruta, más que una gruta una cavidad excavada en el monte. 

Diríase que son los cimientos de la antigua construcción, cuyos restos derrumbados, apuntalados con troncos de árbol casi sin desbastar, hacen de techo. 

Para ver mejor, puesto que hay poquísima luz, José trae yesca y piedra de chispa, y enciende una lamparita que ha sacado del talego que lleva cruzado al pecho. Entra. Un mugido le saluda. 

- Ven, María; está vacía, sólo hay un buey - José sonríe - ¡Mejor que nada...! 

María baja del burrito y entra. 

José ha colgado la lamparita de un clavo que está hincado en uno de los troncos de sostén. Se ve la techumbre llena de telas de araña, y pajas esparcidas por todo el suelo (que es de tierra batida y su superficie es completamente irregular; con hoyos, guijarros, detritos y excrementos). En la parte del fondo, un buey, con heno colgándole de la boca, se vuelve y mira con ojos tranquilos. Hay un tosco taburete y dos piedras en un ángulo ennegrecido — señal de que en ese lugar se enciende fuego — que está junto a una tronera. 

María se acerca al buey. Tiene frío. Le pone las manos sobre el cuello para sentir su calorcillo. El buey muge; se deja. Parece como si hubiera comprendido. Se deja también cuando José lo separa un poco para coger abundante heno del pesebre para hacerle a María una yacija—el pesebre es doble: está el en que come el buey, y, encima, una especie de estante con heno de reserva; éste es el que coge José. Y le hace sitio al burrito, que, cansado y hambriento, enseguida se pone a comer. 

José encuentra también un cubo volcado y todo abollado. Sale — porque fuera había visto un regato — y vuelve con agua para el borriquillo. Luego se hace con un haz de ramajes que estaba en un rincón y trata de barrer un poco el suelo. Después extiende el heno, hace con él una yacija, junto al buey, en el ángulo más seco y resguardado; pero siente que este mísero heno está húmedo, y suspira. Enciende el fuego y, con una paciencia de cartujo, lo seca a manojos cerca del calor. 

María, sentada en el taburete, cansada, mira sonriente. Ya está. María se dispone mejor sobre el mullido heno, con los hombros apoyados en un tronco. José termina de... aparejar la estancia extendiendo su manto como si fuera una cortina en la apertura que hace de puerta. Una protección muy relativa. Luego le ofrece a la Virgen pan y queso, y le da a beber agua de un boto. 

Duerme ahora - le dice - Yo velaré, para que la lumbre no se apague. Menos mal que hay leña. Esperemos que dure y que arda. Así podré ahorrar aceite de la lámpara. 

María se echa obedientemente. José, con la manta que tenía en los pies y con el manto de la misma María, la tapa. 

- ¿Y tú?.. Vas a pasar frío. 

No, María. Estoy junto al fuego. Trata de descansar. Mañana irá mejor. 

María cierra los ojos sin insistir más. José se pone en su rinconcillo, sentado en el taburete, con unas — pocas — ramillas secas al lado; no creo que duren mucho. 

Están colocados así: María a la derecha, dando la espalda a la... puerta, semioculta por el tronco y por el cuerpo del buey, que está recostado ahora en la cama de paja; José a la izquierda y de cara a la puerta, en diagonal por tanto; estando frente al fuego, da la espalda a María, pero, de vez en cuando, se vuelve a mirarla, y la ve tranquila, como si durmiera. Rompe lentamente sus ramitas, y las va echando, una a una, en el débil fuego para que no se apague, para que dé luz, para que la poca leña dure. La única luz, ora más viva, ora mortecina, es la del fuego; la lámpara está ya apagada; en la penumbra resalta sólo el blancor del buey y del rostro y manos de José. Todo el resto es una masa que se confunde en la penumbra densa. 

No hay dictado - dice María \emph{- La visión habla por sí sola. Tarea vuestra es entender la lección de caridad, humildad y pureza que de ella emana. }

\chapter*{Nacimiento de Jesús. \\ \normalfont\normalsize\textit{La eficacia salvadora de la divina maternidad de María. \\ 6 de Junio de 1944.}}
\addcontentsline{toc}{chapter}{\normalfont\scshape{Nacimiento de Jesús.}}

Continúa mi visión del interior de este pobre refugio de piedra en que han encontrado amparo, unidos en la suerte a  unos animales, María y José. 

El fueguecillo se adormila junto con su guardián. María levanta lentamente la cabeza de su yacija y mira. Ve que José tiene la cabeza reclinada sobre el pecho como si estuviera meditando... será — piensa — que el cansancio ha sobrepujado su buena voluntad de permanecer despierto, y sonríe bondadosa; luego, con menos ruido del que puede hacer una mariposa posándose en una rosa, se sienta, para después arrodillarse. Ora con una sonrisa beatífica en su rostro. Ora con los brazos extendidos casi en cruz, con las palmas hacia arriba y hacia adelante... y no parece cansarse de esa posición molesta. Luego se postra con el rostro contra el heno, adentrándose aún más en su oración; y la oración es larga. 

José sale bruscamente de su sueño; ve mortecino el fuego y casi oscuro el establo. Echa un puñado de tamujo muy fino. La llama vuelve a chispear. Y va añadiendo ramitas cada vez más gruesas; en efecto, el frío debe ser punzante, el frío de esa noche invernal, serena, que penetra por todas las partes de esas ruinas. El pobre José, estando como está cerca de la puerta — llamemos así a la abertura a la que hace de cortina su manto —, debe estar congelado. Acerca las manos a la llama, se quita las sandalias, acerca también los pies; así se calienta. Luego, cuando el fuego ha adquirido ya viveza y su luz es segura, se vuelve; no ve nada, ni siquiera la blancura del velo de María que antes dibujaba una línea clara sobre el heno oscuro. Se pone en pie y se acerca despacio a la yacija. 

- ¿No duermes, María? - pregunta. 

Lo pregunta tres veces, hasta que Ella torna en sí y responde: 

Estoy orando. 

- ¿No necesitas nada? 

No, José. 

Trata de dormir un poco, de descansar al menos. 

Lo intentaré, pero la oración no me cansa. 

Hasta luego, María. 

Hasta luego, José. 

María vuelve a su posición de antes. José, para no ceder otra vez al sueño, se pone de rodillas junto al fuego, y ora. Ora con las manos unidas en el rostro; de vez en cuando las separa para alimentar el fuego, y luego vuelve a su ferviente oración. Menos el ruido del crepitar de la leña y el del asno, que de tanto en tanto pega con una pezuña en el suelo, no se oye nada. 

Un inicio de luna se insinúa a través de una grieta de la techumbre. Parece un filo de incorpórea plata que buscase a María. Se alarga a medida que la Luna va elevándose en el cielo y, por fin, la alcanza. Ya está sobre la cabeza de la orante, nimbándosela de candor. 

María levanta la cabeza como por una llamada celeste y se yergue hasta quedar de nuevo de rodillas. ¡Oh, qué hermoso es este momento! Ella levanta la cabeza, que parece resplandecer bajo la luz blanca de la Luna, y una sonrisa no humana la transfigura. ¿Qué ve? ¿Qué oye? ¿Qué siente? Sólo Ella podría decir lo que vio, oyó y sintió en la hora fúlgida de su Maternidad. Yo sólo veo que en torno a Ella la luz aumenta, aumenta, aumenta; parece descender del Cielo, parece provenir de las pobres cosas que están a su alrededor, parece, sobre todo, que proviene de Ella. 

Su vestido, azul oscuro, parece ahora de un delicado celeste de miosota; sus manos, su rostro, parecen volverse azulinas, como los de uno que estuviera puesto en el foco de un inmenso zafiro pálido. Este color, que me recuerda, a pesar de ser más tenue, el que veo en las visiones del santo Paraíso, y también el que vi en la visión de la venida de los Magos, se va extendiendo progresivamente sobre las cosas, y las viste, las purifica, las hace espléndidas. 

El cuerpo de María despide cada vez más luz, absorbe la de la luna, parece como si Ella atrajera hacia sí la que le puede venir del Cielo. Ahora ya es Ella la Depositaría de la Luz, la que debe dar esta Luz al mundo. Y esta beatífica, incontenible, inmensurable, eterna, divina Luz que de un momento a otro va a ser dada, se anuncia con una alba, un lucero de la mañana, un coro de átomos de luz que aumenta, aumenta como una marea, sube, sube como incienso, baja como una riada, se extiende como un velo... 

La techumbre, llena de grietas, de telas de araña, de cascotes que sobresalen y están en equilibrio por un milagro de estática, esa techumbre negra, ahumada, repelente, parece la bóveda de una sala regia. Los pedruscos son bloques de plata; las grietas, reflejos de ópalo; las telas de araña, preciosísimos baldaquinos engastados de plata y diamantes. Un voluminoso lagarto, aletargado entre dos bloques de piedra, parece un collar de esmeraldas olvidado allí por una reina; y un racimo de murciélagos en letargo, una lámpara de ónice de gran valor. Ya no es hierba el heno que cuelga del pesebre más alto, es una multitud de hilos de plata pura que oscilan temblorosos en el aire con la gracia de una cabellera suelta. 

La madera oscura del pesebre de abajo parece un bloque de plata bruñida. Las paredes están recubiertas de un brocado en que el recamo perlino del relieve oculta el candor de la seda. Y el suelo... ¿Qué es ahora el suelo? Es un cristal encendido por una luz blanca; los salientes parecen rosas de luz arrojadas al suelo como obsequio; los hoyos, cálices valiosos de cuyo interior ascenderían aromas y perfumes. 

La luz aumenta cada vez más. El ojo no la resiste. En ella desaparece, como absorbida por una cortina de incandescencia, la Virgen... y emerge la Madre. 

Sí. Cuando mi vista de nuevo puede resistir la luz, veo a María con su Hijo recién nacido en los brazos. Es un Niñito rosado y regordete, que gesticula, con unas manitas del tamaño de un capullo de rosa; que menea sus piececitos, tan pequeños que cabrían en el corazón de una rosa; que emite vagidos con su vocecita trémula, de corderito recién nacido, abriendo una boquita que parece una menudita fresa de bosque, y mostrando una lengüecita temblorosa contra el rosado paladar; que menea su cabecita, tan rubia que parece casi desprovista de cabellos, una cabecita redonda que su Mamá sostiene en la cavidad de una de sus manos, mirando a su Niño, adorándolo, llorando y riendo al mismo tiempo... Y se corva para besarlo, no en la inocente cabeza, sino en el centro del pecho, sobre ese corazoncito que palpita, que palpita por nosotros... en donde un día se abrirá la Herida. Su Mamá se la está curando anticipadamente, con su beso inmaculado. 

El buey se ha despertado por el resplandor, se levanta haciendo mucho ruido con las pezuñas, y muge. El asno vuelve la cabeza y rebuzna. Es la luz la que los saca del sueño, pero me seduce la idea de pensar que hayan querido saludar a su Creador, por ellos mismos y por todos los animales. 

Y José, que, casi en rapto, estaba orando tan intensamente que era ajeno a cuanto le rodeaba, también torna en sí, y por entre los dedos apretados contra el rostro ve filtrarse la extraña luz. Se descubre el rostro, levanta la cabeza, se vuelve. El buey, que está en pie, oculta a María, pero Ella le llama: «José, ven». 

José acude. Cuando ve, se detiene, como fulminado de reverencia, y está casi para caer de rodillas en ese mismo lugar; pero María insiste: 

Ven, José - y, apoyando la mano izquierda en el heno y teniendo con la derecha estrechado contra su corazón al Infante, se alza y se dirige hacia José, quien, por su parte, se mueve azarado por el contraste entre su deseo de ir y el temor a ser irreverente. 

Junto a la cama para el ganado los dos esposos se encuentran, y se miran llorando con beatitud. 

Ven, que ofrecemos a Jesús al Padre - dice María. José se pone de rodillas. Ella, erguida, entre dos troncos sustentantes, alza a su Criatura en sus brazos y dice: 

Heme aquí — por Él, ¡oh Dios!, te digo esto —, heme aquí para hacer tu voluntad. Y con Él yo, María, y José, mi esposo. He aquí a tus siervos, Señor, para hacer siempre, en todo momento y en todo lo que suceda, tu voluntad, para gloria tuya y por amor a Ti. 

Luego María se inclina hacia José y, ofreciéndole el Infante le dice: - Toma, José. 

- ¿Yo? ¿A mí? ¡Oh, no! ¡No soy digno! - José se siente profundamente turbado, anonadado ante la idea de deber tocar a Dios. 

Pero María insiste sonriendo: Bien digno eres de ello tú, y nadie lo es más que tú, y por eso el Altísimo te ha elegido. Toma, José, tenlo mientras yo busco su ropita. 

José, rojo como una púrpura, alarga los brazos y toma ese copito de carne que grita de frío; una vez que lo tiene entre sus brazos, no persiste en la intención de mantenerlo separado de sí por respeto, sino que lo estrecha contra su corazón rompiendo a llorar fuertemente: 

- ¡Oh! ¡Señor! ¡Dios mío! - y se inclina para besar los piececitos. Los siente fríos y entonces se sienta en el suelo y lo 

recoge en su regazo, y con su indumento marrón y con las manos trata de cubrirlo, calentarlo, defenderlo del cierzo de la noche. Quisiera acercarse al fuego, pero allí se siente esa corriente de aire que entra por la puerta. Mejor quedarse donde está, o, mejor todavía, entre los dos animales, que hacen de escudo al aire y dan calor. Y se pone entre el buey y el asno dando la espalda a la puerta, con su cuerpo hacia el Recién Nacido para hacer de su pecho una hornacina, cuyas paredes laterales son: una cabeza gris, con largas orejas; un hocico grande, blanco, con unos ojos húmedos buenos y un morro que exhala vapor. 

María ha abierto el baulillo y ha sacado unos pañales y unas fajas, ha ido al fuego y las ha calentado. Ahora se acerca a José y envuelve al Niño en esos paños calentitos, y con su velo le cubre la cabeza. 

- ¿Dónde le ponemos ahora? - pregunta. 

José mira alrededor, piensa... 

- Mira — dice —, corremos un poco más para acá a los dos animales y la paja, y bajamos ese heno de allí arriba y lo ponemos a Él aquí dentro. La madera del borde le resguardará del aire, el heno será su almohada, el buey con su aliento lo calentará un poquito. Mejor el buey. Es más paciente y tranquilo. 

Y se pone manos a la obra mientras María acuna a su Niño estrechándolo contra su corazón, con su carrillo sobre la cabecita para darle calor. 

José reaviva el fuego, sin ahorrar leña, para hacer una buena hoguera, y se pone a calentar el heno, de forma que según lo va secando, para que no se enfríe, se lo va metiendo en el pecho; luego, cuando ya tiene suficiente para un colchoncito para el Infante, va al pesebre y lo dispone como una cunita. 

Ya está - dice - Ahora sería necesaria una manta, porque el heno pica; y además para taparlo... 

Coge mi manto - dice María. 

Vas a tener frío. 

- ¡Oh, no tiene importancia! La manta es demasiado áspera; el manto, sin embargo, es suave y caliente. Yo no tengo frío 

en absoluto. ¡Lo importante es que Él no sufra más! 

José coge el amplio manto de suave lana azul oscura y lo dispone doblado encima de la paja, y deja un borde colgando fuera del pesebre. El primer lecho del Salvador está preparado. 

Su Madre, con dulce paso ondeante, lo lleva al pesebre, en él lo coloca, y lo tapa con la parte del manto que había quedado fuera y con ella arropa también la cabecita desnuda, que se hunde en el heno, protegida apenas por el fino velo de María. Queda sólo destapada la carita, del tamaño de un puño de hombre, y los dos, inclinados hacia el pesebre, lo miran con beatitud mientras duerme su primer sueño; en efecto, el calorcito de los paños y de la paja le ha calmado el llanto y le ha hecho conciliar el sueño al dulce Jesús. 

Dice María: \emph{- Te había prometido que Él vendría a traerte su paz. ¿Te acuerdas de la paz que tenías durante los días de Navidad, cuando me veías con mi Niño? Entonces era tu tiempo de paz, ahora es tu tiempo de sufrimiento. Pero ya sabes que es en el sufrimiento donde se conquista la paz y toda gracia para nosotros y para el prójimo. Jesús - Hombre tornó a ser Jesús - Dios después del tremendo sufrimiento de la Pasión; tornó a ser Paz, Paz en el Cielo del que había venido y desde el cual, ahora, derrama su paz sobre aquellos que en el mundo le aman. Mas durante las horas de la Pasión, Él, Paz del mundo, fue privado de esta paz. No habría sufrido si la hubiera tenido, y debía sufrir, sufrir plenamente. Yo, María, redimí a la mujer con mi Maternidad divina, mas se trataba sólo del comienzo de la redención de la mujer. Negándome, con el voto de virginidad, al desposorio humano, había rechazado toda satisfacción concupiscente, mereciendo gracia de parte de Dios. Pero no bastaba, porque el pecado de Eva era árbol de cuatro ramas: soberbia, avaricia, glotonería, lujuria. Y había que quebrar las cuatro antes de hacerlo estéril en sus raíces. Vencí la soberbia humillándome hasta el fondo. Me humillé delante de todos. No hablo ahora de mi humildad respecto a Dios; ésta deben tributársela al Altísimo todas las criaturas. La tuvo su Verbo. Yo, mujer, debía también tenerla. ¿Has reflexionado, más bien, alguna vez, en qué tipo de humillaciones tuve que sufrir de parte de los hombres y sin defenderme en manera alguna? Incluso José, que era justo, me había acusado en su corazón. Los demás, que no eran justos, habían pecado de murmuración sobre mi estado, y el rumor de sus palabras había venido, como ola amarga, a estrellarse contra mi humanidad. Y éstas fueron sólo las primeras de las infinitas humillaciones que mi vida de Madre de Jesús y del género humano me procuraron. Humillaciones de pobreza; la humillación de quien debe abandonar su tierra; humillaciones a causa de las reprensiones de los familiares y de las amistades, que, desconociendo la verdad, juzgaban débil mi forma de ser madre respecto a mi Jesús, cuando empezaba ya a ser un hombre; humillaciones durante los tres años de su ministerio; crueles humillaciones en el momento del Calvario; humillaciones hasta en el tener que reconocer que no tenía con qué comprar ni sitio ni perfumes para enterrar a mi Hijo. Vencí la avaricia de los Progenitores renunciando con antelación a mi Hijo. Una madre no renuncia nunca a su hijo, si no se ve obligada a ello. Ya sea la patria, o el amor de una esposa, o el mismo Dios quienes piden el hijo a su corazón, ella se resiste a la separación. Es natural que sea así. El hijo crece dentro de nosotras, y el vínculo de su persona con la nuestra jamás queda completamente roto. A pesar de que el conducto del vital ombligo haya sido cortado, siempre permanece un nervio que nace en el corazón de la madre (un nervio espiritual, más vivo y sensible que un nervio físico) y arraiga en el corazón del hijo, y que siente como si le estiraran hasta el límite de lo soportable, si el amor dé Dios o de una criatura, o las exigencias de la patria alejan al hijo de la madre; y que se rompe, lacerando el corazón si la muerte arranca un hijo a su madre. Yo renuncié, desde el momento en que lo tuve, a mi Hijo. A Dios se lo di, a vosotros os lo di. Me despojé del Fruto de mi vientre para dar reparación al hurto de Eva del fruto de Dios. Vencí la glotonería, tanto de saber como de gozar, aceptando sorber únicamente lo que Dios quería que supiera, sin preguntarme a mí misma, sin preguntarle a Él, más de cuanto se me dijera. Creí sin indagar. Vencí la gula de gozar porque me negué todo deleite del sentido. Mi carne la puse bajo las plantas de mis pies. Puse la carne, instrumento de Satanás, y con ella al mismo Satanás, bajo mi calcañar para hacerme así un escalón para acercarme al Cielo. ¡El Cielo!.. Mi meta. Donde estaba Dios. Mi única hambre. Hambre que no es gula sino necesidad bendecida por Dios, por este Dios que quiere que sintamos apetito de Él. Vencí la lujuria, que es la gula llevada a la exacerbación. En efecto, todo vicio no refrenado conduce a un vicio mayor. Y la gula de Eva, ya de por sí digna de condena, la condujo a la lujuria; efectivamente, no le bastó ya el satisfacerse sola sino que quiso portar su delito a una refinada intensidad; así conoció la lujuria y se hizo maestra de ella para su compañero. Yo invertí los términos y, en vez de descender, siempre subí; en vez de hacer bajar, atraí siempre hacia arriba; y de mi compañero, que era un hombre honesto, hice un ángel. Es ese momento en que poseía a Dios, y con El sus riquezas infinitas, me apresuré a despojarme de todo ello diciendo: "Que por Él se haga tu voluntad y que Él la haga". Casto es aquel que controla no sólo su carne, sino también los afectos y los pensamientos. Yo tenía que ser la Casta para anular a la Impúdica de la carne, del corazón y de la mente. Me mantuve comedida sin decir ni siquiera de mi Hijo, que en la tierra era sólo mío, como en el Cielo era solamente de Dios: "Es mío y para mí lo quiero". Y a pesar de todo no era suficiente para que la mujer pudiera poseer la paz que Eva había perdido. Esa paz os la procuré al pie de la Cruz, viendo morir a Aquel que tú has visto nacer. Y, cuando me sentí arrancar las entrañas ante el grito de mi Hijo, quedé vacía de toda feminidad de connotación humana: ya no carne sino ángel. María, la Virgen desposada con el Espíritu, murió en ese momento; quedó la Madre de la Gracia, la que os generó la Gracia desde su tormento y os la dio. La hembra, a la que había vuelto a consagrar mujer la noche de Navidad, a los pies de la Cruz conquistó los medios para venir a ser criatura del Cielo. Esto hice yo por vosotras, negándome toda satisfacción, incluso las satisfacciones santas. De vosotras, reducidas por Eva a hembras no superiores a las compañeras de los animales, he hecho — basta con que lo queráis — las santas de Dios. Por vosotras subí, y, como a José, os elevé. La roca del Calvario es mi Monte de los Olivos. Ése fue mi impulso para llevar al Cielo, santificada de nuevo, el alma de la mujer, junto con mi carne, glorificada por haber llevado al Verbo de Dios y anulado en mí hasta el último vestigio de Eva, la última raíz de aquel árbol de las cuatro ramas venenosas, aquel árbol que tenía hincada su raíz en el sentido y que había arrastrado a la caída a la Humanidad, y que hasta el final de los siglos y hasta la última mujer os morderá las entrañas. Desde allí, donde ahora resplandezco envuelta en el rayo del Amor, os llamo y os indico cuál es la Medicina para venceros a vosotras mismas: la Gracia de mi Señor y la Sangre de mi Hijo. Y tú, voz mía, haz descansar a tu alma con la luz de esta alborada de Jesús para tener fuerza en las futuras crucifixiones que no te van a ser evitadas, porque te queremos aquí, y aquí se viene a través del dolor; porque te queremos aquí, y más alto se viene cuanto mayor ha sido la pena sobrellevada para obtener Gracia para el mundo. Ve en paz. Yo estoy contigo .}

\chapter*{El anuncio a los pastores \\ \normalfont\normalsize\textit{que vienen a ser los primeros adoradores del Verbo hecho Hombre.\\7 de junio de 1944. Víspera del Corpus Christi. }}
\addcontentsline{toc}{chapter}{\normalfont\scshape{El anuncio a los pastores,}}

Y ahora veo extensos campos. La Luna está en su cénit surcando tranquila un cielo colmado de estrellas. Parecen bullones de diamante hincados en un enorme palio de terciopelo azul oscuro; la Luna ríe en medio con su carota blanquísima de la que descienden ríos de luz láctea que pone blanca la tierra. Los árboles, desnudos, sobre este suelo emblanquecido, parecen más altos y negros; y los muros bajos, que acá o allá se levantan como lindes, parecen de leche. Una casita lejana parece un bloque de mármol de Carrara. 

A mi derecha veo un recinto, dos de cuyos lados son un seto de espinos; los otros dos, una tapia baja y tosca. En ésta apoya la techumbre de una especie de cobertizo ancho y bajo, que en el interior del recinto está construido parte de fábrica y parte de madera: como si en verano las partes de madera se debieran quitar y se transformase así el cobertizo en un pórtico. De dentro del cercado viene, de tanto en tanto, un balar intermitente y breve. Deben ser ovejas que sueñan, o que quizás creen que pronto se hará de día, por la luz que da la Luna; una luz que es tan intensa que incluso es excesiva y que aumenta como si el astro se estuviera acercando a la Tierra o centellease debido a un misterioso incendio. 

Un pastor se asoma a la puerta, se lleva un brazo a la frente para proteger los ojos y mira hacia arriba. Parece imposible que uno tenga que proteger los ojos de la luz de la Luna, pero, en este caso es tan intensa que ciega, especialmente si uno sale de un lugar cerrado oscuro. Todo está en calma, pero esa luz produce estupor. 

El pastor llama a sus compañeros. Salen todos a la puerta: un grupo numeroso de hombres rudos, de distintas edades. 

Entre ellos hay algunos que apenas si han llegado a la adolescencia, otros ya tienen el pelo cano. Comentan este hecho extraño. Los más jóvenes tienen miedo, especialmente uno, un chiquillo de unos doce años, que se echa a llorar, con lo cual se hace objeto de las burlas de los más mayores. 

- ¿A qué le tienes miedo, tonto? - le dice el más viejo - ¿No ves qué serenidad en el ambiente? ¿No has visto nunca resplandecer la Luna? ¿Has estado siempre pegado a las faldas de tu madre, como un pollito a la gallina, no? ¡Pues anda que no tendrás que ver cosas! Una vez, yo había llegado hasta los montes del Líbano, e incluso los había sobrepasado, hacia arriba. Era joven, no me pesaba andar, incluso era rico entonces... Una noche vi una luz de tal intensidad que pensé que estuviera volviendo Elías en su carro de fuego. El cielo estaba todo de fuego. Un viejo — entonces el viejo era él — me dijo: "Un gran advenimiento está para llegar al mundo". Y para nosotros supuso una desventura, porque vinieron los soldados de Roma. ¡Oh, muchas cosas tendrás que ver, si la vida te da años!... 

Pero el pastorcillo ya no le está escuchando. Parece haber perdido incluso el miedo. De hecho, alejándose del umbral de la puerta, dejando a hurtadillas la espalda de un musculoso pastor, detrás del cual estaba refugiado, sale al redil herboso que está delante del cobertizo. Mira hacia arriba y se pone a caminar como un sonámbulo, o como uno que estuviera hipnotizado por algo que le embelesara. Llegado un momento grita: 

- ¡Oh! - y se queda como petrificado, con los brazos un poco abiertos. 

Los demás se miran estupefactos. 

Pero, ¿qué le pasa a ese tonto? - dice uno. 

Mañana lo mando con su madre. No quiero locos cuidando a las ovejas - dice otro. 

El anciano que estaba hablando poco antes dice: 

Vamos a ver antes de juzgar. Llamad también a los que están durmiendo y coged palos. No vaya a ser un animal malo o gente malintencionada... 

Entran llamando a los otros pastores, y salen con teas y garrotes. Llegan donde el muchacho. 

Allí, allí - susurra sonriendo - Más arriba del árbol, mirad esa luz que se está aproximando. Parece como si siguiera el rayo de la Luna. Mirad. Se acerca. ¡Qué bonita es! 

Yo lo único que veo es una luz más viva. 

Yo también. 

Yo también» dicen los otros. 

No. Yo veo como un cuerpo - dice uno. Lo reconozco: es el pastor que ofreció leche a María. 

- ¡Es un... es un ángel! - grita el niño - Mirad, está bajando, y se acerca... ¡De rodillas ante el ángel de Dios! 

Un « ¡oh!» largo y lleno de veneración se alza del grupo de los pastores, que caen rostro en tierra. Cuanto más ancianos son, más contra el suelo se les ve por la aparición fulgente. Los jovencitos están de rodillas, pero miran al ángel, que se aproxima cada vez más, hasta detenerse, candor de perla en el candor de luna que le circunda, suspendido en el aire, moviendo sus grandes alas, a la altura de la tapia del recinto. 

No temáis. No vengo como portador de desventura, sino que os traigo el anuncio de un gran gozo para el pueblo de Israel y para todo el pueblo de la tierra - La voz angélica es como una armonía de arpa acompañada del canto de gargantas de ruiseñores. 

Hoy en la ciudad de David ha nacido el Salvador - Al decir esto, el ángel abre más las alas, y las mueve como por un sobresalto de alegría, y una lluvia de chispas de oro y de piedras preciosas parece desprenderse de ellas. Un verdadero arco iris de triunfo sobre el pobre redil. - ... el Salvador, que es Cristo - El ángel resplandece con mayor luz. Sus dos alas, ahora ya detenidas, tendiendo su punta hacia el cielo, como dos velas inmóviles sobre el zafiro del mar, parecen dos llamas que suben ardiendo. - ... ¡Cristo, el Señor! - El ángel recoge sus dos fulgidas alas y con ellas se cubre — es como un manto de diamante sobre un vestido de perla —, se inclina como adorando, con las manos cruzadas sobre su corazón; su rostro, inclinado sobre su pecho, queda oculto entre la sombra de los vértices de las alas recogidas. No se ve sino una oblonga forma luminosa, inmóvil durante el tiempo que dura un "Gloria". 

Se mueve de nuevo. Vuelve a abrir las alas, levanta ese rostro suyo en que luz y sonrisa paradisíaca se funden, y dice: 

Lo reconoceréis por estas señales: en un pobre establo, detrás del Belén, encontraréis a un niño envuelto en pañales en un pesebre, pues para el Mesías no había un techo en la ciudad de David - El ángel se pone serio al decir esto; más que serio, triste. 

Y del Cielo vienen muchos — ¡oh, cuántos! — muchos ángeles semejantes a él, una escalera de ángeles que desciende exultando y anulando la Luna con su resplandor paradisíaco, y se reúnen en torno al ángel anunciador, batiendo las alas, emanando perfumes, con un arpegio de notas en que las más hermosas voces de la creación encuentran un recuerdo, alcanzada en este caso la perfección del sonido. Si la pintura es el esfuerzo de la materia para transformarse en luz, aquí la melodía es el esfuerzo de la música para hacer resplandecer ante los hombres la belleza de Dios; y oír esta melodía es conocer el Paraíso, donde todo es armonía de amor, que de Dios emana para hacer dichosos a los bienaventurados, y que de éstos va a Dios para decirle: «¡Te amamos!». 

El "Gloria" angélico se extiende en ondas cada vez más vastas por los campos tranquilos, y con él la luz. Las aves unen a ello un canto que es saludo a esta luz precoz, y las ovejas sus balidos por este sol anticipado. Mas a mí, como ya con el buey y el asno en la gruta, me place creer que es el saludo de los animales a su Creador, que viene a ellos para amarlos como Hombre además de como Dios. 

El canto se hace más tenue, y la luz, mientras los ángeles retornan al Cielo... ...Los pastores vuelven en sí. 

- ¿Has oído? 

- ¿Vamos a ver? 

- ¿Y las ovejas? 

- ¡No les sucederá nada! ¡Vamos para obedecer a la palabra de Dios!.. - Pero, ¿a dónde? 

- ¿Ha dicho que ha nacido hoy? ¿Y que no ha encontrado sitio en Belén? - El que habla ahora es el pastor que ofreció la leche - Venid, yo sé. He visto a la Mujer y me ha dado pena. He indicado un lugar para Ella, porque pensaba que no encontrarían hospedaje, y al hombre le he dado leche para Ella. Es muy joven y hermosa. Debe ser tan buena como el ángel que nos ha hablado. Venid. Venid. Vamos a coger leche, quesos, corderos y pieles curtidas. Deben ser muy pobres y... ¡quién sabe qué frío no tendrá Aquel a quien no oso nombrar! Y pensar que yo le he hablado a la Madre como si se tratara de una pobre esposa cualquiera!

Entran en el cobertizo y, al poco rato, salen; quién con unas pequeñas cantimploras de leche, quién con unos quesitos de forma redondeada dentro de unas rejillas de esparto entretejido, quién con cestas con un corderito balando, quién con pieles de oveja curtidas. 

Yo llevo una oveja. Ha parido hace un mes. Tiene la leche buena. Les puede venir bien, si la Mujer no tiene leche. Me parecía una niña, ¡y tan blanca!.. Un rostro de jazmín bajo la luna - dice el pastor que ofreció la leche. Y los guía. Caminan bajo la luz de la luna y de las teas, tras haber cerrado el cobertizo y el recinto. Van por senderos rurales, entre setos de espinos deshojados por el invierno. 

Van a la parte de atrás de Belén. Llegan al establo, yendo no por la parte por la que fue María, sino por la opuesta, de forma que no pasan por delante de los establos más lindos, y aquél es el primero que encuentran. Se acercan a la entrada. - ¡Entra! 

No me atrevo. 

Entra tú. 

No. 

Mira, al menos. 

Tú, Leví, mira tú que has sido el primero que ha visto al ángel, que es señal de que eres mejor que nosotros - La verdad es que antes lo han llamado loco... pero ahora les conviene que él se atreva a lo que ellos no tienen el valor de hacer. 

El muchacho vacila, pero luego se decide. Se acerca a la entrada, descorre un poquito el manto, mira, y... se queda extático. 

- ¿Qué ves? - le preguntan ansiosos en voz baja. 

Veo a una mujer, joven y hermosa, y a un hombre inclinados hacia un pesebre, y oigo..., oigo que llora un niñito, y la mujer le habla con una voz... ¡oh, qué voz! 

- ¿Qué dice? 

Dice: "¡Jesús, pequeñito! ¡Jesús, amor de tu Mamá! ¡No llores, Hijito!". Dice: "¡Ay, si pudiera decirte: 'Toma la leche, pequeñín! Pero no la tengo todavía". Dice: "¡Tienes mucho frío, amor mío! Y te pincha el heno. ¡Qué dolor para tu Mamá oírte llorar así, y no poderte aliviar!". Dice: "¡Duerme, alma mía! ¡Que se me rompe el corazón oyéndote llorar y viéndote verter lágrimas!", y lo besa y se ve que le está calentando los piececitos con sus manos, porque está inclinada con los brazos dentro del pesebre. 

- ¡Llama! ¡Que te oigan! 

Yo no. Tú, que nos has traído y que la conoces. 

El pastor abre la boca, pero se limita a farfullar unos sonidos. José se vuelve y va a la puerta. 

- ¿Quiénes sois? 

Pastores. Os traemos comida y lana. Venimos a adorar al Salvador. 

Entrad. 

Entran. Las teas iluminan el establo. Los viejos empujan a los niños delante de ellos. 

María se vuelve y sonríe. «Venid» dice. « ¡Venid!» y los invita con la mano y la sonrisa; toma al que había visto al ángel y lo acerca hacia sí, hasta el mismo pesebre. El niño mira con beatitud. 

Los otros, invitados también por José, se arriman con sus dones y los depositan, con breves y emocionadas palabras, a los pies de María. Luego miran al Niño, que está llorando quedo, y sonríen emocionados y dichosos. 

Uno de ellos, más intrépido, dice: 

Toma, Madre. Es suave y está limpia. La había preparado para mi hijo, que está para nacer. Yo te la doy. Arropa a tu Hijo en esta lana; la sentirá suave y caliente – Y le ofrece una piel de oveja, una piel preciosa de abundante lana blanca y larga. 

María alza a Jesús y lo envuelve en la piel. Luego se lo muestra a los pastores, los cuales, de rodillas sobre el heno del suelo, lo miran extasiados. 

Sintiéndose más valerosos, uno de ellos propone: 

Habría que darle un sorbo de leche, o mejor: agua y miel. Pero no tenemos miel Se les da a los niñitos. Yo tengo siete hijos y entiendo de ello... 

Aquí está la leche. Toma, Mujer. 

Pero está fría. Tiene que ser caliente. ¿Dónde está Elías? Él tiene la oveja. 

Elías debe ser el de la leche, pero no está; se había quedado afuera y ahora está mirando por el portillo, y en la oscuridad de la noche se difumina. 

- ¿Quién os ha conducido aquí? 

Un ángel nos ha dicho que viniéramos, luego Elías nos ha guiado hasta aquí. Pero, ¿dónde está ahora? La oveja lo delata con un balido. 

Ven. Se te requiere. 

Entra con su oveja, avergonzado por ser el más notado. 

- ¿Eres tú! - dice José habiéndolo reconocido; María, por su parte, le sonríe diciendo: «Eres bueno». 

Ordeñan a la oveja y, con la punta de un paño embebido de leche caliente y espumosa, María moja los labios del Niño, el cual absorbe ese dulzor cremoso. Todos sonríen, y más aún cuando, con la punta de tela todavía entre sus labiecitos, Jesús se duerme bajo el calor de la lana. 

Pero aquí no podéis quedaros. Hace frío y hay humedad. Y además... demasiado olor a animales. No es bueno... y... no está bien para el Salvador. 

Lo sé - dice María suspirando profundamente - pero, no hay sitio para nosotros en Belén. 

Ánimo, Mujer. Nosotros te buscaremos una casa. 

Se lo digo a mi ama - dice el de la leche, Elías - Es buena. Os recibirá, aunque tuviera que ceder su propia habitación. 

Nada más que amanezca se lo digo. Su casa está llena de gente, pero os dejará un sitio. 

Por lo menos para mi Niño. Yo y José podemos estar incluso en el suelo. Pero, para el Pequeñuelo... 

No te angusties, Mujer; yo me ocupo de eso. Y diremos a muchos lo que nos ha sido comunicado. No os faltará nada. Por el momento, recibid lo que nuestra pobreza os puede dar. Somos pastores... 

Nosotros también somos pobres, y no os podemos pagar – dice José. 

- ¡Oh... ni lo queremos! ¡Aunque pudierais, no querríamos! El Señor ya nos ha retribuido. Él ha prometido la paz a todos. Los ángeles decían esto: "Paz a los hombres de buena voluntad". Pero a nosotros nos la ha dado ya, porque el ángel ha dicho que este Niño es el Salvador, que es Cristo, el Señor. Somos pobres e ignorantes, pero sabemos que los Profetas dicen que el Salvador será el Príncipe de la Paz. Y a nosotros nos ha dicho que viniéramos a adorarle. Por eso nos ha dado su paz. ¡Gloria a Dios en el Cielo altísimo y gloria a este Cristo suyo, y bendita seas tú, Mujer, que lo has engendrado! Eres santa porque has merecido llevarlo en ti. Como Reina, mándanos; que servirte será para nosotros motivo de felicidad. ¿Qué podemos hacer por ti? 

Amar a mi Hijo y conservar siempre en el corazón estos pensamientos. 

- ¿Y para ti? ¿No deseas nada? ¿No tienes familiares a los que quieras comunicar que Él ha nacido? 

Sí, los tengo... pero no están cerca de aquí, están en Hebrón... 

Voy yo - dice Elías. «¿Quiénes son? 

Zacarías, el sacerdote, e Isabel, mi prima. 

- ¿Zacarías? ¡Lo conozco bien! En verano subo a esos montes porque tienen pastos abundantes y buenos, y soy amigo de 

su pastor. 

Después de que te vea establecida voy a donde Zacarías. 

Gracias, Elías. 

Nada de gracias. Es un gran honor para mí, que soy un pobre pastor, ir a hablar con el sacerdote y decirle que ha nacido el Salvador. 

No. Le dirás: "Ha dicho María de Nazaret, tu prima, que Jesús ha nacido y que vayas a Belén". 

Eso diré. 

Que Dios te lo pague. Me acordaré de ti, de todos vosotros... 

- ¿Le hablarás a tu Niño de nosotros? 

Lo haré. 

Yo soy Elías. 

Y yo, Leví. 

Y yo, Samuel. 

Y yo, Jonás. 

Y yo, Isaac. 

Y yo, Tobías. 

Y yo, Jonatán. 

Y yo, Daniel. 

Simeón, yo. 

Yo me llamo Juan. 

Yo, José; y mi hermano, Benjamín. Somos gemelos. 

Recordaré vuestros nombres. 

Tenemos que marchamos... pero volveremos... ¡Y te traeremos a otros para adorar!... 

- ¿Cómo volver al aprisco dejando a este Niño? 

- ¡Gloria a Dios que nos lo ha mostrado! 

Déjanos besar su vestido - dice Leví con una sonrisa de ángel. 

María alza despacio a Jesús y, sentada sobre el heno, ofrece los piececitos arropados para que los besen. Y los pastores se inclinan hasta el suelo y besan esos piececitos minúsculos cubiertos por la tela. Quien tiene barba primero se la adereza. Casi todos lloran y, cuando tienen que marcharse, salen caminando hacia atrás, dejando allí su corazón... 

La visión me termina así, con María sentada en la paja con el Niño en su regazo, y José mirando y adorando, apoyado con un codo en el pesebre. 

Dice Jesús: 
\emph{Hoy hablo Yo. Estás muy cansada, pero ten paciencia todavía durante un poco. Es la víspera del Corpus Christi. Podría hablarte de la Eucaristía y de los santos que se hicieron apóstoles de su culto, del mismo modo que te he hablado de los santos que fueron apóstoles del Sagrado Corazón. Pero quiero referirme a otra cosa y a una categoría de adoradores de mi Cuerpo, que son los precursores del culto al mismo, los pastores; ellos son los primeros adoradores de mi Cuerpo de Verbo hecho Hombre. Una vez te dije — y esto mismo lo dice también mi Iglesia — que los Santos Inocentes son los protomártires de Cristo. Ahora te digo que los pastores son los primeros adoradores del Cuerpo de Dios. En ellos se encuentran todos los requisitos que se necesitan para ser adoradores del Cuerpo mío, para ser almas eucarísticas. Fe segura: ellos creen pronta y ciegamente en el ángel. Generosidad: dan todo lo que poseen a su Señor. Humildad: se acercan a otros más pobres que ellos, humanamente, con una modestia de actos que hace que no se sientan rebajados; y se profesan siervos de ellos. Deseo: lo que no pueden dar por sí mismos, se las ingenian para procurarlo con apostolado y esfuerzo. Prontitud de obediencia: María desea que sea avisado Zacarías, y Elías va enseguida. No lo deja para otro momento. Amor, en fin: no saben irse de ese lugar. Tú dices: "dejan allí su corazón". Dices bien. ¿Y no habría que comportarse así también con mi Sacramento? Otra cosa. Ésta enteramente para ti. Observa a quién se revela el ángel en primer lugar, y quién es el que merece escuchar las efusiones del ánimo de María. Leví: el niño. A quien tiene alma de niño Dios se le manifiesta, y le muestra sus misterios y permite que escuche las palabras divinas y de María. Y quien tiene alma de niño tiene también la santa intrepidez de Leví y dice: "Déjame besar el vestido de Jesús". Se lo dice a María, porque es siempre María la que os da a Jesús. Ella es la Portadora de la Eucaristía. Ella es el Sagrario Vivo. Quien va a María me encuentra a mí. Quien me pide a Ella de Ella me recibe. La sonrisa de mi Madre, cuando una criatura le dice: "Dame a tu Jesús para que yo le ame" — tan feliz se siente —, hace que el color del Cielo se cambie en un esplendor más vivo de júbilo. Dile pues: "Déjame besar el vestido de Jesús, déjame besar sus llagas". Atrévete incluso a más. Di: "Déjame reclinar mi cabeza en el Corazón de tu Jesús para sentirme así bienaventurada". Ven. Descansa. Como Jesús en la cuna, entre Jesús y María.  }

\chapter*{Visita de Zacarías. \\ \normalfont\normalsize\textit{La santidad de José y la obediencia a los sacerdotes. \\ 8 de junio de 1944.}}
\addcontentsline{toc}{chapter}{\normalfont\scshape{Visita de Zacarías.}}

Veo la larga sala donde presencié el encuentro de los Magos con Jesús y su acto de adoración. Comprendo que me encuentro en la casa hospitalaria que ha acogido a la Sagrada Familia. Asisto a la llegada de Zacarías. Isabel no está. 

La dueña de la casa sale presurosa, por la terraza que circunda la casa, al encuentro del huésped que está llegando... Le acompaña hasta una puerta y llama; luego, discreta, se retira. 

José abre y, al ver a Zacarías, exulta de júbilo. Lo pasa a una habitacioncita pequeña, de las dimensiones de un pasillo. 

María está dándole la leche al Niño. Espera un poco. Siéntate, que estarás cansado - Y le deja sitio en su recostadero, sentándose a su lado. 

Oigo que José pregunta por el pequeño Juan, y que Zacarías responde: 

Crece vigoroso como un potrillo. De todas formas, ahora está sufriendo un poco por los dientes. Por eso no hemos querido traerlo. Hace mucho frío. Así que tampoco ha venido Isabel. No podía dejarlo sin la leche. Lo ha sentido mucho; pero, ¡está siendo una estación tan fría...! 

Sí, efectivamente, muy fría - responde José. 

Me dijo el hombre que me enviasteis que cuando nació el Niño estabais sin casa. ¡Lo que habréis tenido que pasar!.. 

Sí, verdaderamente lo hemos pasado muy mal; pero era mayor el miedo que la precariedad en que nos encontrábamos. Teníamos miedo de que esta precariedad le pudiera perjudicar al Niño. Y los primeros días tuvimos que pasarlos allí. A nosotros no nos faltaba nada, porque los pastores habían transmitido la buena nueva a los betlemitas y muchos vinieron con dones. Pero faltaba una casa, faltaba una habitación resguardada, un lecho... y Jesús lloraba mucho, especialmente por la noche, por el viento que entraba por todas partes. Yo encendía un poco de fuego, pero poco, porque el humo le hacía toser al Niño... y así el frío seguía. Dos animales calientan poco, ¡y menos todavía en un sitio donde el aire entra por todas partes! Faltaba agua caliente para lavarlo, faltaba ropa seca para cambiarlo... ¡Oh! ¡Ha sufrido mucho! Y María sufría al verlo sufrir. ¡Sufría yo... conque te puedes hacer una idea Ella, que es su Madre! Le daba leche y lágrimas, leche y amor... Ahora aquí estamos mejor. Yo había hecho una cuna muy cómoda y María había puesto un colchoncito blando. ¡Pero la tenemos en Nazaret! ¡Ah, si hubiera nacido allí, habría sido distinto! 

Pero el Cristo tenía que nacer en Belén. Así estaba profetizado. 

María ha oído que hablaban y entra. Está toda vestida de lana blanca. Ya no lleva el vestido oscuro que tenía durante el viaje y en la gruta. Con este de ahora está enteramente blanca, como ya la he visto otras veces; no lleva nada en la cabeza. En sus brazos sí, a Jesús, que está durmiendo, satisfecho de leche, envuelto en sus blancos pañales. 

Zacarías se alza reverente y se inclina con veneración. Luego se acerca y mira a Jesús dando señales de un grandísimo respeto. Está inclinado, no tanto para verlo mejor, cuanto para rendirle homenaje. María se lo ofrece. Zacarías lo toma con tal adoración que parece como si estuviera elevando un ostensorio. Efectivamente, está cogiendo en brazos la Hostia, la Hostia ya ofrecida, que será inmolada sólo cuando se haya dado a los hombres como alimento de amor y de redención. Zacarías devuelve Jesús a María. 

Se sientan. Zacarías refiere de nuevo — esta vez a María — el motivo por el cual Isabel no ha venido, y cómo ello la ha apenado. 

Durante estos meses ha estado preparando ropa para tu bendito Hijo. Te lo he traído. Está abajo, en el carro». 

Se levanta y va afuera. Vuelve con un paquete voluminoso y con otro más pequeño. De uno y de otro — José enseguida lo ha liberado del grande — saca inmediatamente los presentes: una suave colcha de lana tejida a mano, pañales y vestiditos. Del otro, miel, harina blanquísima, mantequilla y manzanas, para María, y tortas amasa- das y cocidas por Isabel y muchas otras cositas que manifiestan el afecto maternal de la agradecida prima hacia la joven Madre. 

Le dirás a Isabel que le quedo agradecida, como también a ti. Me habría gustado mucho verla, pero comprendo las razones. También me hubiera gustado ver de nuevo al pequeño Juan... 

Lo veréis para la primavera. Vendremos a veros. 

Nazaret está demasiado lejos - dice José. 

- ¿Nazaret? Pero si debéis quedaros aquí. El Mesías debe crecer en Belén. Es la ciudad de David. El Altísimo lo ha traído, a través de la voluntad de César, a nacer en la tierra de David, la tierra santa de Judea. ¿Por qué llevarlo a Nazaret? Ya sabéis qué es lo que piensan los judíos de los nazarenos. El día de mañana este Niño deberá ser el Salvador de su pueblo. La capital no debe despreciar a su Rey por el hecho de despreciar a su ciudad de proveniencia. Vosotros sabéis como yo lo insidioso que es en sus razonamientos el Sanedrín y lo desdeñosas que son las tres castas principales... Además aquí, no lejos de mí, podré ayudaros bastante, y podré poner todo lo que tengo no tanto de cosas materiales cuanto de dones morales — al servicio de este Recién Nacido. Y cuando esté en edad de entender me sentiré dichoso de ser maestro suyo, como de mi hijo, para que así, incluso, cuando sea mayor, me bendiga. Tenemos que pensar en el gran destino suyo, y que, por tanto, debe poderse presentar al mundo con todas las cartas para poder ganar fácilmente su partida. Está claro que Él poseerá la Sabiduría, pero el solo hecho de que haya tenido a un sacerdote por maestro le hará más acepto a los difíciles fariseos y a los escribas, y le facilitará la misión. 

María mira a José, José mira a María. Por encima de la cabeza inocente del Niño, que duerme rosado y ajeno a lo que le rodea, se entreteje un mudo intercambio de preguntas. Son preguntas veladas de tristeza. María piensa en su casita; José, en su trabajo. Aquí habría que partir de cero, en un lugar en que, apenas unos días antes, nadie los conocía. En este lugar no hay ninguna de esas cosas amadas dejadas allí, y que habían sido preparadas para el Niño con gran amor. 

Y María lo dice: 

- ¿Cómo hacemos? Allí hemos dejado todo. José ha trabajado para mi Jesús sin ahorrar esfuerzo ni dinero. Ha trabajado de noche, para trabajar durante el día para los demás y ganar así lo necesario para poder comprar las maderas más bonitas, la lana más esponjosa, el lino más cándido, para preparar todo para Jesús. Ha hecho colmenas, ha trabajado hasta de albañil para darle otra distribución a la casa, de forma que la cuna pudiera estar en mi habitación hasta que Jesús fuese más grande, y que luego pudiese dar espacio a la cama; porque Jesús estará conmigo hasta que sea un jovencito. 

José puede ir a recoger lo que habéis dejado. 

- ¿Y dónde lo metemos? Como tú sabes, Zacarías, nosotros somos pobres. No tenemos más que el trabajo y la casa. Y ambos nos dan para tirar adelante sin pasar hambre. Pero aquí... trabajo encontraremos, quizás, pero tendremos que pensar de todas formas en una casa. Esta buena mujer no nos puede hospedar permanentemente, y yo no puedo sacrificar a José más de lo que ya lo está por mí. 

- ¡Oh, yo! ¡Por mí no es nada! Me preocupa el dolor de María, el dolor de no vivir en su casa... 

Le brotan a María dos lagrimones. 

- Yo creo que debe amar esa casa como el Paraíso, por el prodigio; que allí tuvo lugar en Ella... Hablo poco, pero entiendo mucho. Si no fuera por este motivo, no me sentiría afligido. A fin de cuentas, lo único es que trabajaré el doble, pero soy fuerte y joven como para trabajar el doble de lo acostumbrado y cubrir todas las necesidades. Si María no sufre demasiado... si tú dices que se debe hacer así... por mí... aquí estoy. Haré lo que estiméis más justo. Basta con que le sea útil a Jesús. 

Ciertamente será útil. Pensad en ello y veréis los motivos. 

Se dice también que el Mesías será llamado Nazareno... – objeta María. 

Cierto. Pero, al menos hasta que se haga adulto, haced que crezca en Judea. Dice el Profeta: "Y tú, Belén Efratá, serás la más grande, porque de ti saldrá el Salvador". No habla de Nazaret. Quizás ese apelativo se le dará por un motivo que desconocemos. Pero su tierra es ésta. 

Tú lo dices, sacerdote, y nosotros... y nosotros con dolor te escuchamos... y seguimos tu consejo. ¡Y qué dolor!.. 

¿Cuándo veré aquella casa donde fui Madre?- María llora quedo. Y yo entiendo este llanto suyo... ¡Vaya que si lo entiendo! La visión me termina con este llanto de María. 

Dice luego María: 

\emph{Sé que comprendes mi llanto. De todas formas, me verás llorar más intensamente. Por el momento voy a aliviar tu espíritu mostrándote la santidad de José, que era hombre, o sea, que no tenía más ayuda de su espíritu que su santidad. Yo, en mi condición de Inmaculada, tenía todos los dones de Dios; no sabía que lo era, pero en mi alma éstos eran activos y me daban fuerza espiritual. Él, sin embargo, no era inmaculado. La humanidad estaba en él con todo su peso gravoso, y debía elevarse hacia la perfección con todo ese peso, a costa del esfuerzo continuo de todas sus facultades por querer alcanzar la perfección y ser agradable a Dios. ¡Oh, sí, verdaderamente santo era mi esposo! Santo en todo, incluso en las cosas más humildes de la vida: santo por su castidad de ángel, santo por su honestidad de hombre, santo por su paciencia, por su laboriosidad, por su serenidad siempre igual, por su modestia, por todo. Esa santidad brilla también en este hecho acaecido. Un sacerdote le dice: "Conviene que te establezcas aquí"; y él, aun sabiendo que su decisión le acarreará el tener que trabajar mucho más, dice: "Por mí no es nada. Lo que me preocupa es el sufrimiento de María. Si no fuera por esto, yo, por mí, no me afligiría; es suficiente con que le sea útil a Jesús". Jesús, María: sus angélicos amores. Mi santo esposo no tuvo otro amor en este mundo... y se hizo a sí mismo siervo de este amor. Lo han hecho protector de las familias cristianas, de los trabajadores, de muchas otras categorías (moribundos, esposos...); pues bien, a mayor razón, debería hacérsele protector de los consagrados. Entre los consagrados de este mundo al servicio de Dios, quienquiera que sea, ¿habrá alguno que se haya ofrecido como él al servicio de su Dios, aceptando todo, renunciando a todo, soportándolo todo, llevando todo a cabo con prontitud, con espíritu gozoso, con constancia de ánimo como él? No, no lo hay. Y observa otra cosa; o, mejor, dos. Zacarías es un sacerdote; José, no. Y, sin embargo, observa cómo él, que no lo es, tiene su espíritu en el Cielo más que quien lo es. Zacarías piensa humanamente, y humanamente interpreta las Escrituras, porque — no es la primera vez que lo hace — se deja guiar demasiado por su buen sentido humano. Ya fue castigado por ello, pero vuelve a caer en lo mismo, aunque menos gravemente. Ya respecto al nacimiento de Juan había dicho: "¿Cómo podrá ser esto, si yo soy viejo y mi mujer estéril?". Ahora dice: "Para allanarse el camino, el Cristo debe crecer aquí"; y piensa — con esa pequeña raíz de orgullo que persiste incluso en los mejores — que él le puede ser útil a Jesús — no útil como quiere serlo José (sirviéndole), sino útil siendo maestro suyo (!) —. Dios le perdonó de todas formas por la buena intención; pero, ¿necesitaba, acaso, maestros el "Maestro"? Traté de hacerle ver la luz en las profecías, mas él se sentía más docto que yo y usaba a su modo esta impresión suya. Yo habría podido insistir y vencer, pero — y ésta es la segunda observación que te presento — respeté al sacerdote; por su dignidad, no por su saber. Por lo general, Dios ilumina siempre al sacerdote. Digo "por lo general". Es iluminado cuando es un verdadero sacerdote. No es el hábito el que consagra; consagra el alma. Para juzgar si uno es un verdadero sacerdote, debe juzgarse lo que sale de su alma. Como dijo mi Jesús: del alma salen las cosas que santifican o que contaminan, las que informan todo el modo de actuar de un individuo. Pues bien, cuando uno es un verdadero sacerdote, generalmente siempre Dios le inspira. ¿Y los otros, que no son tales?: tener con ellos caridad sobrenatural, orar por ellos. Y mi Hijo te ha puesto ya al servicio de esta redención, y no digo más. Alégrate de sufrir porque aumenten los verdaderos sacerdotes. Descansa en la palabra de aquel que te guía. Cree y presta obediencia a su consejo. Obedecer salva siempre. Aunque no sea en todo perfecto el consejo que se recibe. Tú has visto que nosotros obedecimos, y el fruto fue bueno. Verdad es que Herodes se limitó a ordenar el exterminio de los niños de Belén y de los alrededores. Pero, ¿no habría podido, acaso, Satanás llevar estas ondas de odio, propagarlas, mucho más allá de Belén, y persuadir a un mismo delito a todos los poderosos de Palestina para lograr matar al futuro Rey de los judíos? Sí, habría podido. Y esto habría sucedido en los primeros tiempos del Cristo, cuando el repetirse de los prodigios ya había despertado la atención de las muchedumbres y el ojo de los poderosos. Y, si ello hubiera sucedido, ¿cómo habríamos podido atravesar toda Palestina para ir, desde la lejana Nazaret, a Egipto, tierra que daba asilo a los hebreos perseguidos, y, además, con un niño pequeño y en plena persecución? Más sencilla la fuga de Belén, aunque — eso sí — igualmente dolorosa. La obediencia salva siempre, recuérdalo; "y el respeto al sacerdote es siempre señal de formación cristiana. ¡Ay — y Jesús lo ha dicho — ay de los sacerdotes que pierden su llama apostólica! Pero también ¡ay de quien se cree autorizado a despreciarlos!, porque ellos consagran y distribuyen el Pan verdadero que del Cielo baja. Este contacto los hace santos cual cáliz sagrado, aunque no lo sean. De ello deberán responder a Dios. Vosotros consideradlos tales y no os preocupéis de más. No seáis más intransigentes que vuestro Señor Jesucristo, el cual, ante su imperativo, deja el Cielo y desciende para ser elevado por sus manos. Aprended de Él. Y, si están ciegos, o sordos, o si su alma está paralítica y su pensamiento enfermo, o si tienen la lepra de unas culpas que contrastan demasiado con su misión, si son Lázaros en un sepulcro, llamad a Jesús para que les devuelva la salud, para que los resucite. Llamadlo, almas víctimas, con vuestro orar y vuestro sufrir. Salvar un alma es predestinar al Cielo la propia. Pero salvar un alma sacerdotal es salvar un número grande de almas, porque todo sacerdote santo es una red que arrastra almas hacia Dios, y salvar a un sacerdote, o sea, santificar, santificar de nuevo, es crear esta mística red. Cada una de sus capturas es una luz que se añade a vuestra eterna corona. Vete en paz.}

\chapter*{Presentación de Jesús en el Templo. \\ \normalfont\normalsize\textit{La virtud de Simeón y la profecía de Ana. \\ 1 de Febrero de 1944.}}
\addcontentsline{toc}{chapter}{\normalfont\scshape{Presentación de Jesús en el Templo.}}

Veo que de una casita modestísima sale una pareja de personas. 

Por una escalerita externa baja una jovencísima madre con un niño en brazos envuelto en un lienzo blanco. 

Reconozco a esta Mamá nuestra. Es la misma de siempre: pálida y rubia, grácil y muy fina en todos sus movimientos. Va vestida de blanco y arropada con un manto azul pálido, cubre su cabeza un velo blanco. Lleva con mucho cuidado a su Niño. 

Al pie de la escalera la está aguardando José al lado de un burrito pardo. José, tanto por lo que se refiere a la túnica como al manto está vestido todo de color marrón claro. Mira a María y le sonríe. Cuando María llega hasta el burrito, José se pasa las riendas del borriquillo al brazo izquierdo y para que María pueda sentarse mejor en la albardilla del asno, toma un momento al Niño, que duerme tranquilo. Luego le vuelve a dar a Jesús y se ponen en camino. 

José va andando al lado de María, sujetando siempre por las riendas al jumento y poniendo cuidado en que éste vaya derecho y sin tropiezos. María tiene a Jesús en el regazo, y, como si tuviera miedo a que cogiese frío, le extiende encima un borde de su manto. Los dos esposos hablan poquísimo, pero se sonríen frecuentemente. 

El camino, que no es ningún modelo de vía, en una campiña desnuda por la estación que corre, se articula en varias direcciones. Algún que otro viajero se cruza con ellos dos, o los alcanza, pero son raros. 

Luego pueden verse algunas casas y unos muros que recintan una ciudad. Los dos esposos entran en ella por una puerta y comienzan el recorrido por la calzada urbana, hecha de adoquines muy separados. El camino es ahora mucho más difícil, ya porque haya un tráfico que en todo momento hace que el burro se detenga, ya porque éste, por las piedras y los agujeros de las piedras que faltan, haga continuamente movimientos bruscos, los cuales incomodan a María y al Niño. 

La calle no es horizontal; sube, aunque ligeramente; es estrecha, entre casas altas de puertecitas estrechas y bajas, de escasas ventanas que dan a la calle. Arriba el cielo se asoma en multitud de listas azules entre unas casas y otras, o más exactamente entre unas terrazas y otras; abajo, en la calle, hay gente y rumor de voces, y se cruzan otras personas a pie o en burros, o llevando jumentos cargados, y otras que van detrás de una caravana de camellos que dificulta el paso. En un momento dado, pasa, con gran ruido de cascos y de armas, una patrulla de legionarios romanos, que desaparece tras un arco que está a caballo de uno y otro lado de una vía muy estrecha y pedregosa. 

José gira a la izquierda y toma una calle más ancha y más bonita. Al fondo de la misma veo el muro almenado que ya conozco. 

María, al llegar a una puerta en que hay una especie de paradero para otros burros, baja del suyo. Digo "paradero" porque es una especie de cabaña grande, o, mejor, de cobertizo, donde hay paja esparcida por el suelo y unos palos con unas argollas para atar a los cuadrúpedos. 

José da algunas monedas a un hombre que ha venido. Con ellas se procura un poco de heno, luego saca un cubo de agua de un pozo tosco que hay en un ángulo y da las dos cosas al burrito. Después se llega de nuevo hasta donde María y ambos entran en el recinto del Templo. 

Se dirigen, primero, hacia un pórtico donde están aquellos a quienes Jesús, pasado el tiempo, pegará egregiamente con un azote, o sea, los vendedores de tórtolas y corderos y los cambistas. José compra dos pichones blancos. No cambia el dinero. Se entiende que tiene ya el que necesita. 

José y María se dirigen hacia una puerta lateral que tiene ocho escalones — creo que también las otras puertas; es como si el cubo del Templo estuviera elevado respecto al resto del suelo —. Ésta tiene un gran atrio, como los portales de nuestras casas de ciudad, pero más vasto y ornado. En él, a la derecha y a la izquierda, hay como dos altares, dos volúmenes rectangulares cuya finalidad de momento no entiendo bien (parecen pilas, poco profundas: la parte interna es más baja, en algunos centímetros, respecto al borde externo). 

Viene un sacerdote — no sé si motu propio o es que José lo ha llamado —. María ofrece los dos pobres pichones, y yo, que comprendo cuál será su suerte, dirijo la mirada a otra parte. Observo la decoración de la recargadísima puerta, del techo y del atrio. Me parece ver con el rabillo del ojo que el sacerdote asperja a María con agua. Debe ser agua porque no veo manchas en su vestido. Luego María, que junto con los dos pichones había dado un montoncillo de monedas al sacerdote — me había olvidado de decirlo —, entra con José en el Templo propiamente dicho, acompañada por el sacerdote. 

Miro a todas partes. Es un lugar decoradísimo. Cabezas de ángeles esculpidas y palmas y ornatos se extienden por las columnas, las paredes y el techo. La luz penetra por unas curiosas ventanas alargadas, estrechas, naturalmente sin cristales, y abiertas en diagonal con respecto a la pared. Supongo que será para impedir que entre el agua cuando llueve torrencialmente. 

María se adentra hasta un determinado punto en que se detiene. Unos metros más adelante hay otros escalones y encima hay otra especie de altar, tras el cual hay otra construcción. 

Ahora me doy cuenta de que no estaba en el Templo, como creía, sino en lo que rodea al Templo propiamente dicho, o sea, al Santo; traspasar su linde, aparte de los sacerdotes, parece que nadie puede hacerlo. Lo que yo creía que era el Templo, por tanto, no es sino un vestíbulo cerrado, que rodea por tres partes al Templo, que custodia el Tabernáculo. No sé si me he explicado bien; de todas formas, yo no soy ni arquitecta ni ingeniera. 

María ofrece el Niño — que se ha despertado y dirige a su alrededor sus ojitos inocentes, con esa mirada de asombro propia de los niños de pocos días — al sacerdote. Éste lo toma y lo eleva extendiendo los brazos, vuelto hacia el Templo, dando la espalda a esa especie de altar que está encima de aquellos escalones. El rito ha quedado cumplido. La Madre recibe de nuevo al Niño y el sacerdote se marcha. 

Algunos miran curiosos. Entre ellos se abre paso un viejecito que camina encorvado y renco apoyándose en un bastón. 

Debe ser muy anciano — para mí, sin duda, de más de ochenta años —. Se acerca a María y le solicita por un momento al Pequeñuelo. María, sonriendo, se lo concede, y Simeón — que yo siempre había creído que pertenecía a la casta sacerdotal y que, sin embargo, a juzgar al menos por el vestido, es un simple fiel — lo toma y lo besa. Jesús le sonríe con ese gesto mimoso, incierto, de los lactantes. Parece que lo observa curioso, porque el viejecillo llora y ríe al mismo tiempo, y sus lágrimas crean todo un bordado de destellos que se insinúa entre las arrugas y que perla su larga barba blanca hacia la cual Jesús tiende sus manitas. Es Jesús, pero es un niñito pequeñín, y todo lo que se mueve delante de Él atrae su atención, y se le antoja cogerlo para entender mejor lo que es. María y José sonríen, como también las otras personas que están presentes, que celebran la hermosura del Pequeñuelo. 

Oigo las palabras del santo anciano y veo la mirada de asombro de José, la mirada emocionada de María, y las de la pequeña multitud (quién se muestra asombrado y emocionado, quién, al oír las palabras del anciano, ríe irónicamente). Entre éstos hay algún barbudo y pomposo miembro del Sanedrín, y menean la cabeza mirando a Simeón con irónica piedad. Deben pensar que ha perdido la razón por la edad. 

La sonrisa de María se difumina en su avivada palidez cuando Simeón le anuncia el dolor. A pesar de que Ella ya lo sepa, esta palabra le traspasa el espíritu. Se acerca más a José, María, buscando consuelo; estrecha con pasión a su Niño contra su pecho, y bebe, como alma sedienta, las palabras de Ana, la cual, siendo mujer, siente compasión de su sufrimiento y le promete que el Eterno le mitigará con sobrenatural fuerza la hora del dolor. 

- Mujer, a Aquel que ha dado el Salvador a su pueblo no le faltará el poder de otorgar el don de su ángel para confortar tu llanto. Nunca les ha faltado la ayuda del Señor a las grandes mujeres de Israel, y tú eres mucho más que Judith y que Yael. Nuestro Dios te dará corazón de oro purísimo para aguantar el mar de dolor por el que serás la Mujer más grande de la creación, la Madre. Y tú, Niño, acuérdate de mí en la hora de tu misión. 

Y aquí me cesa la visión. 

\section*{\normalfont\normalsize\textit{2 de Febrero de 1944.}}
 
Dice Jesús: 
\emph{De la descripción que has hecho, brotan para todos dos enseñanzas. Primera: no se manifiesta la verdad a aquel sacerdote que, aun estando inmerso en los ritos, tiene su espíritu ausente; antes bien, se revela a un simple fiel. El sacerdote — siempre en contacto con la Divinidad, orientado al cuidado de cuanto concierne a Dios, dedicado a todo aquello que es superior a la carne — habría debido intuir enseguida quién era el Niño que ofrecían al Templo esa mañana. Mas, para poder intuir, necesitaba tener un espíritu vivo, y no solamente una vestidura externa de un espíritu que, si no estaba muerto, sí al menos muy soñoliento. El Espíritu de Dios puede, si quiere, tronar como un rayo y sacudir como un terremoto al espíritu más cerrado; puede hacerlo. Pero, generalmente — porque es Espíritu de orden como es Orden Dios en cada una de sus Personas y en su modo de actuar —, se efunde y habla, no digo donde existe mérito suficiente para recibir su manifestación — en ese caso, muy pocas veces se manifestaría, y tú no conocerías tampoco sus luces —, sino en donde ve la "buena voluntad" de merecer su manifestación. ¿Cómo se hace notoria esta buena voluntad? Con una vida hecha toda de Dios hasta donde os es posible. En la fe, en la obediencia, en la pureza, en la caridad, en la generosidad, en la oración. No en las prácticas. En la oración. Hay menos diferencia entre la noche y el día que entre las prácticas y la oración. Ésta es comunión de espíritu con Dios, de la cual salís con vigor nuevo y decididos a ser cada vez más de Dios. Aquéllas son una costumbre cualquiera, con objetivos diversos pero siempre egoístas, y que os deja como erais; es más, os agrava con culpa de embuste o de desidia. Simeón tenía esta buena voluntad. La vida no le había escatimado ni trabajos ni pruebas. Pero él no había perdido su buena voluntad. Los años y las vicisitudes no habían mellado, ni removido, su fe en el Señor, en sus promesas, como tampoco habían cansado su buena voluntad de ser cada vez más digno de Dios. Y Dios, antes de que los ojos de su siervo fiel se cerrasen a la luz del Sol — en espera de volver a abrirse al Sol de Dios rutilante desde los Cielos, abiertos a mi ascensión después del Martirio — le mandó el rayo de luz del Espíritu para que lo guiara al Templo y ver así la Luz que había venido al mundo. "Movido por el Espíritu Santo" dice el Evangelio. ¡Oh, si los hombres supieran qué perfecto Amigo es el Espíritu Santo!¡qué Guía, qué Maestro! ¡Oh, si amaran los hombres, e invocaran, a este Amor de la Santísima Trinidad, a esta Luz de la Luz, a este Fuego del Fuego, a esta Inteligencia, a esta Sabiduría! ¡Cuánto más sabrían de aquello que es necesario saber! Mira, María; mirad, hijos. Simeón esperó durante toda una vida "ver la Luz"; saber que se había cumplido la promesa de Dios. Pero no dudó nunca. Nunca se dijo a sí mismo: "Es inútil que persevere en esperar y en orar". Perseveró. Y obtuvo "ver" lo que no vieron ni el sacerdote ni los miembros del Sanedrín, que estaban llenos de soberbia y completamente ofuscados: al Hijo de Dios, al Mesías, al Salvador en esa carne infantil que le daba calor y sonrisas. Recibió a través de mis labios de Niño, la sonrisa de Dios, como primer premio por su vida honrada y pía. Segunda lección: las palabras de Ana. Ella, profetisa, también ve en mí, recién nacido, al Mesías. Esto, dada su capacidad de profecía, sería natural; pero, escucha, escuchad lo que, impulsada por la fe y la caridad, dice a mi Madre... e iluminad con ello vuestro espíritu, ese espíritu vuestro que tiembla en este tiempo de tinieblas y en esta Fiesta de la Luz. Dice: "A Aquel que ha otorgado un Salvador no le faltará el poder de enviar a su ángel para confortar tu llanto, el vuestro". Considerad que Dios se ha dado para cancelar la obra de Satanás en los espíritus. ¿No va a poder derrotar ahora a los diablos que os torturan? ¿No va a poder enjugar vuestro llanto, dispersando a estos diablos y volviendo a enviar de nuevo la paz de su Cristo? ¿Por qué no se lo pedís con fe? Pero con fe verdadera, impetuosa, una fe ante la cual el rigor de Dios — indignado por tantas culpas vuestras - caiga con una sonrisa, y llegue el perdón, que es ayuda, y venga su bendición, como arco iris, a esta tierra que se hunde en un diluvio de sangre querido por vosotros mismos. Considerad que el Padre, después de haber castigado a los hombres con el diluvio, se dijo a sí mismo y dijo a su Patriarca: "No volveré a maldecir la tierra a causa de los hombres, porque los sentidos y los pensamientos del corazón humano están inclinados al mal ya desde la adolescencia; por tanto no volveré a castigar a todo ser vivo, como he hecho". Y se ha mostrado fiel a su palabra; no ha vuelto a mandar el diluvio. Sin embargo, vosotros ¿cuántas veces os habéis dicho, y habéis dicho a Dios: "Si nos salvamos esta vez, si nos salvas, no volveremos jamás a hacer guerras, nunca jamás", para hacerlas luego y cada vez más tremendas? ¿Cuántas veces, ¡oh falsos!, y sin respeto hacia el Señor y hacia vuestra palabra? Y, no obstante, Dios os ayudaría una vez más si la gran masa de los fíeles lo llamase con fe y amor impetuoso. ¡Oh, vosotros — demasiado pocos para contrapesar a los muchos que mantienen vivo el rigor de Dios — vosotros, los que, a pesar del tremendo presente amenazador, que crece por momentos, permanecéis de todas formas devotos a Él, depositad vuestras fatigas a los pies de Dios! Él sabrá enviaros a su ángel, como envió al Salvador al mundo. No temáis. Estad unidos a la Cruz, que siempre ha vencido las insidias del demonio, el cual viene, con la crueldad de los hombres y con las tristezas de la vida, a tratar de reducir a la desesperación— o sea, a que queden separados de Dios — a los corazones a los que no puede atrapar de otra manera.}

\chapter*{Canción de cuna de la Virgen. \\ \normalfont\normalsize\textit{28 de Noviembre de 1944.}}
\addcontentsline{toc}{chapter}{\normalfont\scshape{Canción de cuna de la Virgen.}}

Esta mañana he tenido un suave despertar. Estando aún entre las nieblas del sopor, oía una voz purísima cantar dulcemente una calma canción de cuna. Parecía, por lo lenta y arcaica que era, una pastoral navideña. Yo seguía ese motivo y esa voz, gozándome en ella cada vez más, recobrando la lucidez bajo su onda. Y la he recobrado y he comprendido. He dicho: - ¡Te saludo, María, llena de Gracia! (porque quien cantaba era Mamá). Ella, por su parte, después de decirme: 
\emph{- Yo también te saludo. ¡Ven y alégrate! -} ha alzado la voz.

Y la he visto, en la casa de Belén, en la habitación que ocupa Ella, acunando a Jesús para dormirlo. En la estancia estaba el telar de María y unas labores de costura. Parecía que María hubiera dejado el trabajo para darle la leche al Niño, cambiarle los fajos, mejor, la ropa, porque era ya un niño de algunos meses, yo diría que seis u ocho al máximo; y parecía que tuviera intención de seguir trabajando una vez que el Niño se hubiera dormido. 

Caía la tarde. El ocaso, ya casi cumplido, había sembrado el cielo sereno de vedijas de oro. Había rebaños que, paciendo las últimas hierbas de un prado florido, regresaban al aprisco, y balaban alzando el morrito. 

El Niño tenía dificultad en dormirse; parecía un poco inquieto, como si estuviera incómodo por los dientes o por otra de esas cositas que dan molestias a los niños pequeños. 

Escribí, como pude, el canto, en la penumbra de esa hora del amanecer, sobre un pedazo de papel. Ahora lo transcribo aquí. 

\begin{verse}
Nubecitas todas de oro — cuales greyes del Señor.\\
En el prado florecido — un rebaño mira allá. \\
Aun teniendo los rebaños — todos los que hay sobre la tierra tú serías el corderito — que siempre querría más... \\
Duerme, duerme, duerme, duerme... No llores más... \\
Mil estrellas relucientes — contemplando desde el cielo. \\
Esas tus pupilas dulces — no las hagas más llorar. \\
Y tus ojos de zafiro — astros de mi pecho son. \\
¡Y tu llanto es mi dolor! — ¡Oh, no, no, no llores más!\\
Duerme, duerme, duerme, duerme... \\
No llores más... \\
Ángeles resplandecientes — todos los del Paraíso cual corona en torno a ti — por ver tu rostro, sonrientes. Y tú lloras, inocente — porque quieres a tu lado que te arrulle tu Mamá — Nana, nana, nana, na... \\
Duerme, duerme, duerme, duerme... No llores más... \\
Pintará el cielo de rosa — la alborada que retorna y Mamá aún no reposa — porque tú no llores más. Dirás "¡Mamá!" en despertando — "¡Hijo!" Ella te dirá; beso, amor y vida juntos — con la leche te dará... \\
Duerme, duerme, duerme, duerme... No llores más... \\
¿Cómo estar sin tu Mamá? — aunque soñaras el Cielo.\\ 
¡Ven! ¡Ven! ¡Ven! Bajo este velo — que dormir Ella te hará. \\
Y mí pecho por almohada — y mis brazos como cuna. ¡Y no temas cosa alguna — que contigo estoy aquí!.. \\
Duerme, duerme, duerme, duerme... No llores más... \\
Yo contigo estaré siempre — vida de mi corazón... \\
Ya duerme... Como una flor — reclinada sobre el pecho... \\
Ya duerme... ¡Chist! ¡Despacio! — Quizás ve a su Padre Santo... Su visión enjuga el llanto — de mi Jesús dulce amado... \\
Duerme ya, ya duerme, duerme y su llanto enjugado está...
\end{verse} 

Describir la gracia de la escena es imposible. Se trata sólo de una madre acunando a un pequeñuelo; ¡pero son esa Madre y ese Pequeñuelo! Por tanto, puede hacerse una idea de qué gracia, qué amor, qué pureza, qué Cielo hay en esta pequeña, grande, delicada escena que me regocija con su recuerdo, del cual, como confirmación, queda la melodía que repito para mis adentros, para podérsela cantar a usted; aunque yo no tengo la voz de plata purísima de María, la voz virginal de la Virgen... y pareceré un organillo que pierde aire. No importa, haré lo que pueda. ¡Qué hermosa pastoral para cantarla alrededor de la Cuna de Navidad! 

La Madre, primero, estaba meciendo suavemente la cuna de madera; mas luego, viendo que Jesús todavía rebullía, se lo ha puesto junto a su cuello, sentada cerca de la ventana abierta — al lado, la cunita — y, con un vaivén ligero al ritmo de la melodía, ha repetido dos veces la nana, hasta que el pequeño Jesús ha cerrado sus ojitos, ha vuelto la cabecita apoyándola sobre el pecho materno y se ha dormido así, con la carita aplastada contra el calorcito de ese pecho, con una manita apoyada sobre un seno de su Mamá junto a su carrillito rosado, y la otra cayendo sobre el regazo materno. El velo de María daba sombra a la Criaturita santa. 

Luego María se levantó con infinito cuidado y puso a su Jesús en la cunita, lo tapó con las sábanas, extendió un velo para protegerlo de las moscas y del aire, y se quedó contemplando a su Tesoro durmiente. Tenía una mano en el corazón; la otra, apoyada todavía en la cuna, preparada para mecerla si hubiera habido posibilidad de que se hubiera vuelto a despertar; y sonreía, dichosa, un poco inclinada hacia la cuna, mientras las sombras y el silencio descendían sobre la tierra e invadían la habitación virginal. 

¡Qué paz! ¡Qué belleza! ¡Y yo me siento dichosa! 

No es una visión grandiosa. Quizás, en el conjunto general de las otras, será considerada inútil, porque no revela nada de especial. Lo sé. Pero para mí es una auténtica gracia, y tal la considero porque hace apacible a mi espíritu; lo hace puro, amoroso, como si le hubieran recreado las manos de nuestra Madre. Somos "niños"... ¡Mejor así! Somos gratos a Jesús. Que la gente, las personas doctas y complicadas, piensen lo que quieran; que nos llamen incluso "pueriles". Nosotros no pensamos en eso, ¿verdad? 

\chapter*{Adoración de los Magos. \\ \normalfont\normalsize\textit{Es "evangelio de la fe". \\ 28 de Febrero de 1944.}}
\addcontentsline{toc}{chapter}{\normalfont\scshape{Adoración de los Magos.}}

Mi interno consejero me dice: 

\emph{«A estas contemplaciones que vas a tener, que Yo te voy a manifestar, llámalas "evangelios de la fe", porque vendrán a ilustrarte a tí y a los demás el poder de la fe y de sus frutos, así como a confirmaros en la fe en Dios».}

Veo Belén, pequeña y blanca, recogida como una parvada bajo la claridad de las estrellas. Dos calles principales la cortan en cruz: una, que llega desde fuera, y es la vía principal, que luego prosigue más allá del pueblo; la segunda va de un extremo a otro de éste, y ahí termina. Hay otras callecitas que dividen a este pueblecillo, pero sin la más mínima norma de planificación urbana como nosotros concebimos, sino adaptándose más bien al terreno sinuoso y a las casas que han ido surgiendo aquí o allá, según el capricho del suelo o del constructor. Estando unas hacia la derecha, otras hacia la izquierda, algunas formando arista con la calle que pasa por ellas, estas casas obligan a las calles a ser como una cinta que se desenrede tortuosamente, en vez de algo rectilíneo que vaya de una a otra parte sin desviarse. Una placita de vez en cuando, o bien por un mercado, o bien por una fuente, o porque se ha construido arbitrariamente sin criterio: restos de suelo al sesgo en que no es posible ya construir nada. 

En el punto en que de forma particular me parece estar, hay precisamente una de estas placitas irregulares. Debería haber sido cuadrada, o, al menos, rectangular; sin embargo ha resultado un trapecio tan extraño que parece un triángulo acutángulo con el vértice truncado. En el lado más largo — la base del triángulo — hay una construcción ancha y baja, la más grande del pueblo. La rodea un muro liso y desnudo, abierto sólo en dos puntos: dos puertas, que ahora están perfectamente cerradas. Al otro lado del muro, sin embargo, en su vasto cuadrado, se abren en el primer piso muchas ventanas; en la planta baja hay unos pórticos que rodean a unos patios que tienen paja y detritos en el suelo y sus correspondientes pilones para abrevar a los caballos o a otros animales. En las toscas columnas de las arcadas hay unas argollas para atar a los animales, y, en uno de los lados, existe un vasto cobertizo para cobijar a rebaños y cabalgaduras. Comprendo que se trata de la posada de Belén. 

En los otros dos lados iguales de la placita hay casas más o menos grandes, unas con un poco de huerto delante, otras no; efectivamente, algunas de ellas tienen la fachada hacia la plaza, mientras que otras, por el contrario, la parte de atrás. Finalmente, en el lado más corto, de frente a la posada, hay una única casita con una escalerita externa que introduce a mitad de la fachada en las habitaciones del piso habitado. Todas las casas están cerradas porque es de noche. No hay nadie por las calles, dada la hora. 

Veo intensificarse la luz nocturna que llueve del cielo lleno de estrellas, hermosísimas en el cielo oriental, tan vivas y grandes que parecen cercanas y se ve fácil llegarse a donde esas flores resplandecientes que están en el terciopelo del firmamento, y tocarlas. Levanto la mirada para tratar de comprender el origen de este aumento de luz... Una estrella, cuyo insólito tamaño le hace asemejarse a una pequeña Luna, avanza por el cielo de Belén. Las otras parecen eclipsarse y apartarse, cual siervas al paso de su reina, pues el resplandor es tan grande que las sumerge y las anula. Su globo, que parece un enorme zafiro pálido encendido internamente por un Sol, va dejando una estela en la que con el predominante color del zafiro claro se funden los amarillos de los topacios, los verdes de las esmeraldas, los opalescentes de los ópalos, los sanguíneos destellos de los rubíes y el delicado titilar de las amatistas. Todas las piedras preciosas de la Tierra están presentes en esa estela que barre el cielo con un movimiento veloz y ondulante, como si estuviera viva. El color que predomina, no obstante, es el que emana del globo de la estrella: el paradisíaco color de pálido zafiro que desciende a colorar de plata azul las casas, las calles, el suelo de Belén, cuna del Salvador. No es ya esa pobre villa que para nosotros no sería ni siquiera un pueblo; es una villa fantástica de fábula, en que todo es de plata, y el agua de las fuentes y de los pilones es de diamante líquido. 

El efluvio de resplandor se hace más vivo. La estrella se detiene encima de la casita que está situada en el lado más corto de la plazuela. Ni los que en aquélla habitan ni los betlemitas la ven, pues están durmiendo en sus casas cerradas. Pero la estrella acelera sus latidos de luz; su cola vibra y ondula más intensamente trazando ca- si semicírculos en el cielo, que se ilumina todo por la red de astros que la estrella arrastra, por esta red llena de joyas resplandecientes que tiñen de los más hermosos colores a las otras estrellas, casi como si les transmitieran una palabra de alegría. 

La casita ahora está toda bañada de este fuego líquido de gemas. El techo de la breve terraza, la escalerita de piedra oscura, la pequeña puerta... todo es como un bloque de pura plata sembrado todo de polvo de diamantes y perlas. Ningún palacio de la Tierra ha tenido jamás, ni la tendrá, una escalera como ésta, hecha para recibir el paso de los ángeles, para ser usada por la Madre que es Madre de Dios; sus pequeños pies de Virgen Inmaculada pueden apoyarse sobre ese cándido esplendor, esos sus pequeños pies destinados a descansar sobre los escalones del trono de Dios. Y, sin embargo, la Virgen está ajena de ello; Ella vela orante junto a la cuna de su Hijo. En su alma tiene resplandores que superan a éstos con que la estrella embellece las cosas. 

Por la calle principal avanza una caravana. Caballos enjaezados, caballos guiados de las riendas, dromedarios y camellos montados o que transportan su carga. El sonido de los cascos produce un rumor como el del agua de un torrente cuando roza las piedras y choca contra ellas. Llegados a la plaza, todos se detienen. La caravana, bajo la luz radiante de la estrella, tiene un esplendor fantástico. Los jaeces de las riquísimas cabalgaduras, los indumentos de sus jinetes, las caras, los equipajes... todo resplandece, uniendo y avivando su brillo de metal, de cuero, de seda, de piedras preciosas, de pelaje... con el brillo estelar. Y los ojos relucen, y ríen las bocas, porque en los corazones se ha encendido otro fulgor: el de una alegría sobrenatural. 

Mientras los siervos se encaminan hacia la posada con los animales, tres de la caravana se bajan de sus respectivas cabalgaduras; un siervo las conduce inmediatamente a otra parte, y ellos, a pie, se dirigen hacia la casa. Se postran, rostro en tierra, para besar el suelo. Son tres potentados, a juzgar por sus riquísimas vestiduras. Uno de ellos, de piel muy oscura, que se ha bajado de un camello, se arropa con una toga de cándida seda esplendente; ciñen su frente y su cintura preciosos aros; del de la cintura pende un puñal o una espada, cuya empuñadura está cuajada de gemas. Los otros dos, que montaban espléndidos caballos, están vestidos así: uno, de paño de rayas bellísimo en que predomina el color amarillo, elaborado a manera de dominó, largo, ornado con capucha y cordón, tan recamados que parecen una única labor de filigrana de oro; el otro lleva una camisa sedeña, que, formando bolsas, sobresale del pantalón amplio y largo ceñido a los pies, y va envuelto en un finísimo chal, tan ornado todo él de flores y tan vivas éstas, que asemeja a un jardín florido, y lleva en la cabeza un turbante sujetado por una cadenita, toda ella con engastes de diamantes. 

Tras haber venerado la casa en que está el Salvador, se ponen de nuevo en pie y se dirigen a la posada, ya abierta a los pajes que se habían adelantado para llamar a la puerta. 

Y aquí cesa la visión. ''Tres horas después vuelve: es la escena de la adoración de los Magos a Jesús. 

Ahora es de día. Un hermoso sol resplandece en el cielo de la tarde. Un paje de los tres Magos cruza la plaza y sube la escalerita de la casa. Entra. Vuelve a salir. Regresa a la posada. 

Salen los tres Sabios, cada uno seguido de su propio paje. Atraviesan la plaza. Los escasos transeúntes se vuelven a mirar a estos pomposos personajes que pasan muy lentamente, con solemnidad. Entre cuando el paje ha entrado y la entrada de éstos, ha transcurrido ampliamente un cuarto de hora; los habitantes de la casita así han podido prepararse para recibir a los que llegan. 

Los tres están vestidos aún más ricamente que la noche precedente. Las sedas resplandecen, las gemas brillan, un gran penacho de preciosas plumas, sembrado de escamas aún más preciosas, ondula trémulo e irradia destellos sobre la cabeza del que lleva el turbante. 

Los pajes llevan: uno, un cofre todo taraceado, cuyos refuerzos metálicos son de oro burilado; el segundo, una labradísima copa, cubierta por una aún más labrada tapa, toda de oro; el tercero, una especie de ánfora ancha y baja, también de oro, cubierta con una tapa en forma de pirámide en cuyo vértice hay un brillante. Debe pesar, pues los pajes lo llevan con esfuerzo, especialmente el del cofre. 

Suben por la escalera y entran. Entran en una habitación que va de la parte de la calle al dorso de la casa. Por una ventana abierta al sol, se ve el huertecillo posterior. Hay puertas en las otras dos paredes; desde ellas los propietarios curiosean. 

Éstos son: un hombre, una mujer y, entre jovencitos y niños, tres o cuatro. 

María está sentada con José, en pie, a su lado. Tiene al Niño en su regazo. No obstante, cuando ve entrar a los tres Magos, se levanta y hace una reverencia. Está toda vestida de blanco. ¡Qué hermosa, con su sencillo vestido blanco que la cubre desde la base del cuello hasta los pies, desde los hombros hasta sus delgadas muñecas; qué hermosa, con su cabeza pequeña coronada de trenzas rubias, con ese rostro suyo más vivamente rosado por la emoción, con esos ojos que sonríen dulcemente, con esa su boca que se abre para saludar diciendo: «Dios sea con vosotros»! Tanto es así, que los tres Magos, impresionados, se detienen un instante. Pero luego caminan otro poco y se postran a sus pies. Y le ruegan que se siente. 

Ellos no, no se sientan, a pesar de los ruegos de Ella; permanecen de rodillas, relajados sobre los talones. Detrás, también de rodillas, los tres pajes; se han detenido apenas traspasado el umbral de la puerta, han depositado delante de ellos los tres objetos que llevaban y están esperando. 

Los tres Sabios contemplan al Niño, que creo que puede tener de nueve meses a un año, pues su aspecto es muy vivaz y pujante; está sentado sobre el regazo de su Mamá, y sonríe y balbucea con una vocecita de pajarillo. Está vestido todo de blanco como su Mamá; en sus diminutos piececitos, unas pequeñas sandalias. Es un vestidito muy sencillo: una tuniquita de la que sobresalen los bonitos piececitos inquietos y las manitas gorditas que querrían agarrar todas las cosas, y, sobre todo, la lindísima carita en que brillan los ojos azul oscuros y la boca hace hoyitos a los lados riendo y descubriendo los primeros dientecitos diminutos. Los ricitos de Jesús son tan lúcidos y vaporosos, que parecen polvo de oro. 

El más anciano de los Sabios toma la palabra en nombre de los tres, para explicarle a María que durante una noche del pasado diciembre vieron encenderse una nueva estrella en el cielo, de inusitado esplendor. Jamás las cartas del cielo habían registrado ese astro, jamás lo habían mencionado. No se conocía su nombre, porque no lo tenía. Nacida, entonces, del seno de Dios, esa estrella había brillado para manifestar a los hombres una bendita verdad, un secreto de Dios. Pero los hombres no le habían prestado atención, porque tenían hundida el alma en el fango; no alzaban la mirada hacia Dios y no sabían leer las palabras que Él escribe — alabado sea eternamente por ello — con astros de fuego en la bóveda del cielo. 

Ellos la habían visto y se habían esforzado por entender su voz. Y, perdiendo contentos el poco sueño que concedían a sus miembros, y aun olvidándose del alimento, se habían sumido en el estudio del zodiaco; las conjunciones de los astros, el tiempo, la estación, el cálculo de las horas pasadas y de las combinaciones astronómicas les habían dicho el nombre y el secreto de la estrella. Su nombre: "Mesías"; su secreto: "ser el Mesías venido al mundo". Y se habían puesto en camino para adorarlo. Cada uno de ellos sin que los otros lo supieran. Por montes y desiertos, por valles y ríos, viajando incluso durante la noche, habían venido hacia Palestina, porque la estrella se movía en esa dirección. Para cada uno de ellos, desde tres puntos distintos de la tierra, se movía en esa dirección. Se habían encontrado después del Mar Muerto. La voluntad de Dios los había reunido allí, y juntos habían continuado, comprendiéndose a pesar de que cada uno hablaba su propia lengua, y comprendiendo y pudiendo hablar la lengua del país por un milagro del Eterno. 

Juntos se habían dirigido a Jerusalén, dado que el Mesías debía ser el Rey de esta ciudad, el Rey de los judíos; pero en el cielo de esa ciudad la estrella se había ocultado, sintiendo ellos rompérseles de dolor el corazón, y se habían examinado para saber si quizás se hubieran hecho indignos de Dios. Pero, habiéndolos tranquilizado su conciencia, fueron a donde el rey Herodes para preguntarle en qué palacio había nacido el Rey de los judíos que ellos habían venido a adorar. El rey, convocados los príncipes de los sacerdotes y los escribas, había interrogado acerca del lugar en que podía nacer el Mesías, a lo que éstos habían respondido: «En Belén de Judá». 

Y habían venido hacia Belén. La estrella, dejada ya la Ciudad santa, había aparecido de nuevo ante sus ojos, y, de noche, el día anterior había aumentado sus resplandores: el cielo todo era un fuego; luego se había parado sobre esta casa, reuniendo toda la luz de las otras estrellas en su haz luminoso. Así, habían comprendido que ahí estaba el Nacido divino. Y ahora lo estaban adorando, ofreciendo sus pobres presentes y, sobre todo, su propio corazón, el cual jamás cesaría de bendecir a Dios por la gracia concedida y de amar a su Hijo, cuya santa Humanidad estaban viendo. Luego volverían a informar al rey Herodes, pues también él deseaba adorarlo. 

- Este es el oro que a todo rey corresponde poseer; esto, el incienso, como corresponde a Dios; y esto, ¡oh Madre!, esto es la mirra, porque tu Hijo es, además de Dios, Hombre, y habrá de conocer, de la carne y de la vida humana, la amargura y la inevitable ley de la muerte. Nuestro amor quisiera no pronunciar estas palabras y concebirlo eterno también en la carne como eterno es su Espíritu. Pero, ¡oh Mujer!, si nuestros mapas, y, sobre todo, nuestras almas, no yerran, Él es, este Hijo tuyo, el Salvador, el Cristo de Dios, y, por tanto, deberá, para salvar a la Tierra, cargar sobre sí mismo el peso del mal de la Tierra, uno de cuyos castigos es la muerte. Esta resina es para esa hora, para que la carne santa no conozca la podredumbre de la corrupción y conserve la integridad hasta su resurrección. ¡Y que por este presente nuestro Él se acuerde de nosotros y salve a sus siervos dándoles su Reino! - De momento — añade — Ella, la Madre, para ser santificados por Él, dé su Niño a nuestro amor, para que, besando sus pies, descienda sobre nosotros la bendición celeste. 

María, que ha superado la turbación suscitada por las palabras del Sabio y ha celado la tristeza de la fúnebre evocación bajo una sonrisa, ofrece el Niño. Lo deposita en los brazos del más anciano, que lo besa — y Jesús lo acaricia — y luego lo pasa a los otros dos. 

Jesús sonríe y juguetea con las cadenitas y las cintas de los indumentos de los tres, y mira con curiosidad el cofre abierto, lleno de una cosa amarilla que brilla, y ríe al ver que el sol hace un arco iris al herir el brillante de la tapa de la mirra. 

Los tres Magos devuelven el Niño a María y se levantan. También se pone en pie María. Inclinan mutuamente la cabeza en gesto de reverencia. Antes el más joven había dado una orden al siervo y éste había salido. Los tres siguen hablando todavía un poco. No saben decidirse a separarse de esa casa. Lágrimas de emoción en sus ojos... Al final se dirigen hacia la salida acompañados por María y José. 

El Niño ha querido bajar y darle la manita al más anciano de los tres, y anda así, de la mano de María y del Sabio, los cuales se inclinan para tenerlo de la mano. Jesús, con su pasito todavía inseguro; de infante, ríe, golpeando con sus piececitos sobre la franja que el sol dibuja en el suelo. 

Llegados al umbral de la puerta — téngase presente que la habitación tenía la misma largura de la casa — los tres se despiden arrodillándose una vez más y besando los piececitos de Jesús. María, inclinada hacía el Pequeñuelo, le toma la manita y la guía y hace así ésta un gesto de bendición sobre la cabeza de cada uno de los Magos. Es éste ya un signo de cruz trazado por los pequeños dedos de Jesús, guiados por María. 

Tras ello, los tres bajan la escalera. La caravana ya está ahí esperando preparada. Los bullones de las cabalgaduras reflejan el Sol del ocaso. La gente se ha agolpado en la placita para ver este insólito espectáculo. 

Jesús ríe dando palmadas con sus manitas. Su Mamá lo ha alzado y lo ha apoyado en el ancho parapeto que limita el descansillo, y lo tiene con un brazo sujeto contra su pecho para que no se caiga. José, que ha bajado con los tres Magos, sujeta a cada uno de ellos el estribo al subirse éstos a los caballos o al camello. 

Ya todos, siervos y señores, están a caballo. Se da orden de marcha. Los tres, como último saludo, se inclinan hasta tocar el cuello de la cabalgadura. José hace una reverencia. María también, volviendo a guiar la manita de Jesús en un gesto de adiós y bendición. 

Dice Jesús: 
\emph{¿Y ahora? ¿Qué deciros ahora, almas que sentís morir la fe? Estos Sabios de Oriente no disponían de nada que los confirmara en la verdad; nada sobrenatural. Sólo tenían el cálculo astronómico y la propia reflexión perfeccionada por una vida íntegra. Y, con todo, tuvieron fe. Fe en todo: fe en la ciencia, fe en la conciencia, fe en la bondad divina. En la ciencia, en cuanto que creyeron en el signo de la estrella nueva, que no podía sino ser "ésa", la que la humanidad desde hacía siglos estaba esperando: el Mesías. En la conciencia, en cuanto que tuvieron fe en la voz de la misma, la cual, recibiendo "voces" celestes, les decía: "Esa estrella es la que signa la venida del Mesías". En la bondad, en cuanto que tuvieron fe en que Dios no los engañaría, y en que, dado que su intención era recta, los ayudaría en todos los modos para alcanzar el objetivo. Y lo lograron. Sólo ellos, entre tantos otros estudiosos de los signos, comprendieron ese signo, porque sólo ellos tenían en el alma el ansia de conocer las palabras de Dios con un fin recto, cuyo principal pensamiento consistía en dar enseguida a Dios honor y gloria. No buscaban el provecho personal. Antes bien, les esperaban dificultades y gastos, y no piden compensación humana alguna. Piden solamente que Dios se acuerde de ellos y los salve para la eternidad. De la misma forma que su pensamiento no está puesto en ninguna compensación humana posterior, tampoco tienen, cuando deciden el viaje, ninguna preocupación humana. Vosotros habríais hecho mil cavilaciones: "¿Cómo me las voy a arreglar para hacer un viaje tan largo por países y entre gentes de lenguas distintas? ¿Me van a creer, o, por el contrario, me encarcelarán por espía? ¿Qué ayuda me van a ofrecer cuando tenga que pasar desiertos, ríos, montes? ¿Y el calor? ¿Y el viento de los altiplanos? ¿Y las fiebres pantanosas de las zonas palúdicas? ¿Y las riadas dilatadas por las lluvias? ¿Y las comidas distintas? ¿Y el lenguaje distinto? Y... y.. y". Así razonáis vosotros. Ellos no razonan así. Dicen, con sincera y santa audacia: "Tú, ¡oh Dios!, lees nuestro corazón y ves qué fin perseguimos. Nos ponemos en tus manos. Concédenos la sobrehumana alegría de adorar a tu Segunda Persona hecha Carne para la salud del mundo". Ello es suficiente. Se ponen en camino desde las lejanas Indias. (Jesús me dice luego que con 'Indias" quiere decir Asia meridional, donde ahora están Turquestán, Afganistán y Persia). Se ponen en camino desde las cadenas montañosas mongólicas, en cuyo espacio se mueven, libérrimos, sólo águilas y buitres, donde Dios habla con el fragor de los vientos y de los torrentes y escribe palabras de misterio en las inmensas páginas de los neveros. Se ponen en camino desde las tierras en que nace el Nilo, y discurre, vena verde- azul, hacia el corazón azul del Mediterráneo. Ni picos, ni zonas selvosas, ni arenas — océanos secos y más peligrosos que los marinos — detienen su paso. Y la estrella brilla sobre sus noches, negándoles el sueño. Cuando se busca a Dios, los hábitos animales deben ceder ante los anhelos impacientes y las necesidades suprahumanas. Reciben la estrella desde septentrión, desde oriente y desde meridión, y, por un milagro de Dios, avanza para los tres hacia un punto; como también, por otro milagro, los reúne tras muchas millas en ese punto; y, por otro, les da, anticipando la sabiduría pentecostal, el don de entenderse y de hacerse entender como en el Paraíso, donde se habla una sola lengua: la de Dios. Sólo un momento de turbación los sobrecoge: cuando la estrella desaparece. Ellos — humildes porque eran realmente grandes — no piensan que ello sea debido a la maldad de los demás — no habiendo merecido ver la estrella de Dios los hombres corrompidos de Jerusalén —, sino que piensan que ellos son los que se han hecho indignos de Dios, y se examinan con temblor y con contrición ya preparada para pedir perdón. Mas su conciencia los tranquiliza. Habituadas sus almas a la meditación, tenían una conciencia sensibilísima, afinada por una atención constante, por una aguda introspección, que había hecho de su interior un espejo en que se reflejaban las más ligeras sombras de los hechos cotidianos. Habían hecho de su conciencia una maestra, una voz que los advertía y les gritaba ante la más pequeña, no digo falta, sino mirada a la falta, a lo que es humano, a la complacencia de lo que es yo. Y por eso, cuando se ponen frente a esta maestra, frente a este espejo severo y nítido, saben que no les mentirá. Los tranquiliza y recobran el vigor. "¡Oh, qué dulce el sentir que en nosotros no hay nada que sea contrario a Dios; sentir que Él mira con complacencia al corazón del hijo fiel y lo bendice! Este sentir produce aumento de fe y confianza, y esperanza y fortaleza y paciencia. Es momento de tempestad, mas ésta pasará, porque Dios me ama y sabe que le amo, y me seguirá ayudando": esto dicen quienes poseen esa paz que procede de una conciencia recta, reina de todas sus acciones. He dicho que eran "humildes porque eran realmente grandes". ¿En vuestras vidas, sin embargo, qué sucede? Que uno, no porque sea grande, sino por su mayor despotismo — y se hace poderoso por su despotismo y por vuestra necia idolatría —, no es jamás humilde. Existen pobres desgraciados que, por el solo hecho de ser mayordomos de un déspota, conserjes en algún organismo, funcionarios de un arrabal — a fin de cuentas al servicio de quien los ha hecho lo que son — se dan aires de semidioses. ¡Bueno, pues dan pena!.. Ellos, los tres Sabios, eran realmente grandes, en primer lugar por virtudes sobrenaturales, en segundo lugar, por ciencia, y, por último, por riqueza. Y no obstante se sienten nada: polvo sobre el polvo de la tierra, respecto al Dios altísimo, que crea los mundos con una sonrisa suya, y los esparce como granos de trigo para saciar los ojos de los ángeles con collares hechos de estrellas. Se sienten nada respecto al Dios altísimo que ha creado el planeta en que viven, y que lo ha hecho variado, colocando, cual Escultor infinito de obras inmensas, aquí, con un toque de su pulgar, una corona de suaves colinas, allá una cadena de cumbres y de picos semejantes a vértebras de la tierra; de este cuerpo desmesurado cuyas venas son los ríos; pelvis, los lagos; corazones, los océanos; vestiduras, los bosques; velos, las nubes; ornatos, los glaciares de cristal; gemas, las turquesas y las esmeraldas, los ópalos y los berilos de todas las aguas que cantan, con las selvas y los vientos, el gran coro de alabanzas a su Señor. Se sienten nada en su sabiduría respecto al Dios altísimo de quien les viene y que les ha dado ojos más potentes que esas dos pupilas por las que ven las cosas: ojos del alma que saben leer en las cosas esa palabra no escrita por mano humana, sino grabada por el pensamiento de Dios. Se sienten nada en su riqueza: átomo respecto a la riqueza del Posesor del universo, que disemina metales y gemas en los astros y planetas, y riquezas sobrenaturales, inagotables riquezas, en el corazón de aquel que le ama. Y, llegados ante una pobre casa de la más mísera de las ciudades de Judá, no menean la cabeza diciendo: "Imposible", sino que se inclinan reverentes, se arrodillan, sobre todo con el corazón, y adoran. Ahí, detrás de esas paredes, está Dios; ese Dios que siempre invocaron, sin atreverse, ni por asomo, a esperar que podrían verlo. Le invocaron, más bien, por el bien de toda la humanidad, por "su propio" bien eterno. ¡Ah, sólo esto soñaban para ellos: poder verlo, conocer, poseerlo en la vida que no conocerá ni alboradas ni ocasos! Él está ahí, tras esas pobres paredes. ¿Quién sabe si, quizás, su corazón de Niño, que es el corazón de un Dios, no siente estos tres corazones que vueltos hacia el polvo del camino tintinean: "Santo, Santo, Santo. Bendito el Señor, Dios nuestro. Gloria a Él en los Cielos altísimos, y paz a sus siervos. Gloria, gloria, gloria y bendición"? Ellos se lo preguntan con temblor de amor. Y, durante toda la noche y la mañana siguiente preparan, con la más viva oración, su espíritu para la comunión con el Dios- Niño. No se dirigen a este altar — regazo virginal sobre el que está la Hostia divina — como hacéis vosotros, o sea, con el alma llena de preocupaciones humanas. Se olvidan del sueño y de la comida, toman las vestiduras más bellas — no por humana ostentación, sino por honrar al Rey de los reyes —. En los palacios de los soberanos, los dignatarios entran con las vestiduras más bellas; ¿no debían, acaso, ellos ir a donde este Rey con sus indumentos de fiesta? ¿Y qué fiesta mayor que ésta para ellos? En sus lejanas patrias, muchas veces tuvieron que ataviarse elegantemente por otros hombres de su mismo rango; para festejarlos u honrarlos. Era justo, pues, humillar ante los pies del Rey supremo púrpuras y joyas, sedas y plumas preciosas. Era justo poner a sus pies, ante sus delicados piececitos, las telas de la Tierra, las gemas de la Tierra, plumajes, metales de la Tierra, para que estas cosas de la Tierra — son obras suyas — adorasen también a su Creador. Y se hubieran sentido felices si la Criaturita les hubiera ordenado que se extendieran en el suelo haciendo una alfombra viva para sus pasitos de Niño, y los hubiera pisado, Él, que había dejado las estrellas por ellos, que sólo eran polvo, polvo, polvo... Eran humildes y generosos, y obedientes a las "voces" que venían de lo Alto. Tales "voces" ordenan llevar presentes al Rey recién nacido. Y ellos llevan los presentes. No dicen: "Es rico y por tanto no lo necesita. Es Dios y por tanto no conocerá la muerte". Obedecen. Y son ellos los primeros en ayudar al Salvador en su pobreza. Y ¡qué providente era ese oro para quien en un futuro próximo sería un fugitivo!, ¡cuánto significado tenía esa resina para quien a no tardar sería matado!, ¡qué pío ese incienso para quien había de sentir el hedor de las lujurias humanas en ebullición en torno a su pureza infinita! Humildes, generosos, obedientes, respetuosos unos con otros. Las virtudes engendran siempre otras virtudes. De las virtudes orientadas a Dios proceden las virtudes orientadas al prójimo. Respeto, que a fin de cuentas es caridad. Defieren al más anciano hablar por los tres, y ser el primero en recibir el beso del Salvador y en llevarlo de la mano. Los otros podrán volverlo a ver, pero él no. Es viejo. Cercano está ya su día de regreso a Dios. A este Cristo lo verá, tras su espantosa muerte, y lo seguirá por la estela de los salvados en el regreso al Cielo, mas no lo volverá a ver en esta Tierra. Quédele, pues, como viático, el calorcito de esta diminuta mano que se abandona en la suya ya rugosa. Y los demás no tuvieron ninguna envidia del sabio anciano; antes bien, aumentó su veneración por él: en efecto, había merecido más que ellos y durante más tiempo. El Dios- Infante esto lo sabía. La Palabra del Padre todavía no hablaba, pero su acto era ya palabra. ¡Bendita sea esta palabra suya, inocente, que designa a éste como su predilecto! Mas hay, todavía, hijos, otras dos enseñanzas en esta visión. Cómo José sabe estar dignamente en "su" puesto. Está presente como custodio y tutor de la Pureza y de la Santidad, pero sin usurpar sus derechos. María, con su Jesús, es quien recibe dones y palabras; José exulta por Ella y no se siente herido de ser una figura secundaria. José es un justo, es el Justo, y es justo siempre, y en este momento también lo es. No se embriaga con los vapores de la fiesta. Permanece humilde, justo. Se alegra de esos regalos. No por él mismo, sino pensando que con ellos va a poder hacerles más cómoda la vida a su Esposa y a su dulce Niño. En José no hay avaricia. Es un trabajador y va a seguir trabajando; pero otra cosa es que "Ellos", sus dos amores, puedan vivir con desahogo y comodidad. Ni él ni los Magos saben que esos regalos van a ser útiles para una fuga, para una vida en el exilio (en las que los haberes se disipan como una nube bajo la acción del viento), y para regresar a la patria, tras haber perdido todo: clientes, mobiliario, enseres; sólo con las paredes de la casa, que Dios la protegería porque en ese lugar Él se había unido a la Virgen y se había hecho Carne. José es humilde — él, que es custodio de Dios y de la Madre de Dios y Esposa del Altísimo — hasta el punto de sujetar el estribo a estos vasallos de Dios. Es un pobre carpintero, debido a que el despotismo humano ha despojado a los herederos de David de sus regios haberes, pero sigue siendo de estirpe real y posee rasgos de rey. De él hay que decir también: "Era humilde porque era realmente grande". Ultima, delicada, indicativa enseñanza. Es María quien toma la mano de Jesús, que todavía no sabe bendecir, y la guía en el gesto santo. Es siempre María la que toma la mano de Jesús y la guía. Y ahora sucede lo mismo. Ahora Jesús sabe bendecir, pero a veces su mano traspasada cae cansada y desesperanzada porque sabe que es inútil bendecir. Vosotros destruís mi bendición. Cae también indignada, porque vosotros me maldecís. Y entonces es María la que retira el desdén de esta mano besándola. ¡Oh, el beso de mi Madre! ¿Quién podría resistir a ese beso? Luego toma con sus finos dedos — finos, pero ¡cuan amorosamente imperiosos! — mi muñeca, y me fuerza a bendecir. No puedo decir que no a mi Madre. Pero tenéis que ir a Ella para hacerla Abogada vuestra. Ella es mi Reina antes de ser vuestra Reina, y su amor por vosotros guarda indulgencias que ni siquiera el mío conoce. Y Ella, incluso sin palabras, sólo con las perlas de su llanto y con el recuerdo de mi Cruz — cuyo signo me hace trazar en el aire — toma la defensa de vuestra causa recordándome: "Eres el Salvador. Salva". He aquí, hijos, el "evangelio de la fe" en la aparición de la escena de los Magos. Meditad e imitad, para bien vuestro.}

 
\chapter*{Huida a Egipto. \\ \normalfont\normalsize\textit{Enseñanzas sobre la última visión relacionada con la llegada de Jesús. \\ 9 de Junio de 1944.}}
\addcontentsline{toc}{chapter}{\normalfont\scshape{Huida a Egipto.}}

Mi espíritu ve la siguiente escena. 

Es de noche. José está durmiendo en su modesto lecho, en su diminuta habitación. Su sueño es pacífico, como el de quien está descansando del mucho trabajo cumplido con honradez y diligencia. 

Lo veo en la oscuridad de la estancia, oscuridad apenas interrumpida por un hilo de luz lunar que penetra por una rendija de la hoja de la ventana, que está sólo entornada, no cerrada del todo, como si José tuviera calor en esta pequeña habitación, o como si quisiera tener ese hilo de luz para saberse medir al amanecer y levantarse diligentemente. Está girado sobre uno de los lados, y sonríe mientras duerme, quién sabe ante qué visión que está soñando. 

Pero su sonrisa se transforma en congoja. Emite el típico suspiro, profundo de quien está teniendo una pesadilla, y se despierta sobresaltado. Se sienta en la cama, se restriega los ojos, mira a su alrededor, y mira hacia la ventanita de la que proviene ese hilo de luz. Es plena noche; no obstante, coge la prenda de vestir que está extendida a los pies de la cama y, todavía sentado en el lecho, se la pone encima de la túnica blanca de manga corta que tenía sobre la piel. Levanta las mantas, pone los pies en el suelo y busca las sandalias. Se las pone y se las ata. Se pone en pie y se dirige hacia la puerta que está frente a su cama; no hacia la que está lateral a la misma y que conduce al salón en que fueron recibidos los Magos. 

Llama suavemente con la punta de los dedos: un casi insensible tic- tic. Debe haber oído que se le invita a entrar, pues abre con cuidado la puerta y la vuelve a entornar sin hacer ruido. Antes de ir a la puerta había encendido una lamparita de aceite, de una sola llama; por tanto, se ilumina con ella. Entra... En una habitacioncita sólo un poco más grande que la suya, con una cama pequeña y baja al lado de una cuna, ya ardía una lamparita: la llamita oscilante, en un rincón, parece una estrellita de luz tenue y dorada que permite ver sin molestar a quien esté dormido. 

Pero María no está dormida, está arrodillada junto a la cuna. Tiene un vestido claro y está orando, y velando a Jesús, que duerme tranquilo. Jesús tiene la edad de la visión de los Magos. Es un niño de un año aproximadamente, un niño guapo, rosado y rubio, y está durmiendo, con su cabecita ensortijada hundida en la almohada y una manita bien cerrada junto a la garganta. 

- ¿No duermes? - pregunta José en voz baja denotando asombro - ¿Por qué? ¿Jesús no está bien? 

- ¡Oh, no! Él está bien. Yo estoy rezando. Luego me echaré a dormir. ¿Por qué has venido, José? Mientras habla, María sigue arrodillada donde estaba antes. 

José, en voz bajísima para no despertar al Niño, pero en tono apremiante, dice: 

Tenemos que irnos de aquí enseguida, enseguida. Prepara el baulillo y un fardo con todo lo que puedas meter en ellos. Yo me encargo de preparar lo demás, llevaré lo más que pueda... Cuando empiece a clarear huimos. Lo haría incluso antes, pero tengo que hablar con la dueña de la casa.... 

- ¿Y por qué esta huida? 

Después te lo explico mejor. Es por Jesús. Un ángel me ha dicho: "Toma al Niño y a la Madre y huye a Egipto". No pierdas tiempo. Yo ya empiezo a preparar todo lo que pueda. 

No era necesario decirle a María que no perdiese tiempo. Apenas ha oído hablar de ángel, de Jesús y de huida, ha comprendido que un peligro se cierne sobre su Criatura, y de un salto se ha puesto en pie; su cara más blanca que un cirio, una mano contra el pecho, angustiada. Enseguida se ha puesto en movimiento, ágil, ligera, y ha empezado a colocar la ropa de vestir en el baulillo y en un fardo grande que ha extendido primero sobre su cama aún intacta. Sin duda está angustiada, pero no pierde las riendas; hace las cosas con rapidez pero no sin orden. De vez en cuando, pasando junto a la cuna, mira al Niño, que duerme ajeno a lo que está sucediendo. 

- ¿Necesitas ayuda? - pregunta cada cierto tiempo José, asomando la cabeza por la puerta entreabierta. 

No, gracias - responde siempre María. 

Hasta que el fardo — que debe pesar bastante — no está lleno, no llama a José para que la ayude a cerrarlo y a quitarlo de encima de la cama. No obstante, José quiere hacerlo solo; coge el largo fardo y se lo lleva a su cuarto. 

- ¿Cojo también las mantas de lana? - pregunta María. 

Coge todo lo más que puedas; todo el resto lo perderemos. Toma todo lo que puedas. Nos servirá porque... ¡porque tendremos que estar fuera mucho tiempo, María!.. - José está muy apenado al decir esto, y María... se puede uno hacer idea de cómo está; suspirando, dobla las colchas suyas y las de José, y éste las ata con una cuerda. 

Dejamos los bordados y las esterillas» dice mientras está atando las colchas - A pesar de que voy a tomar tres burros, no puedo cargarlos demasiado, pues el camino será largo e incómodo, parte entre montañas y parte por el desierto. Tapa bien a Jesús. Las noches serán frías, tanto en las montañas como en el desierto. He cogido los regalos de los Magos, porque en aquella tierra nos vendrán bien. Todo lo que tengo lo gasto para comprar los dos burros. Debo comprarlos, porque no podemos devolverlos. Voy ahora, antes de que amanezca. Sé dónde buscarlos. Tú termina de prepararlo todo – Y se marcha. 

María recoge todavía algunos objetos. Observa a Jesús y sale, para volver con unos vestiditos que parecen todavía húmedos — quizás se lavaron el día antes —; los dobla y los envuelve en un pedazo de tela y los coloca junto con las otras cosas. Ya no queda nada más. 

Se vuelve mirando a su alrededor y ve, en un rincón, un juguete de Jesús: una ovejita tallada en madera. La toma en sus manos... un sollozo entrecortado... un beso: la madera conserva las huellas de los dientecitos de Jesús, y las orejas de la ovejita están del todo llenas de mordisquitos. María acaricia ese objeto sin valor en sí, de una pobre madera clara, pero de mucho valor para Ella, ya que le habla del afecto de José por Jesús, y de su Niño. Lo pone también con las otras cosas encima del baulillo cerrado,. 

Ahora ya sí que no queda nada. Sólo Jesús, que está en su cunita. María piensa que sería conveniente también preparar al Niño. Va donde la cuna y la mueve un poco para despertar al Pequeñuelo. Mas Él solamente refunfuña un poco; se da la vuelta y sigue durmiendo. María le acaricia delicadamente los ricitos. Jesús, bostezando, abre la boquita. María se inclina hacia Él y leo besa en la mejilla. Jesús termina de despertarse. Abre los ojos. Ve a su Mamá y sonríe, y tiende las manitas hacia su pecho. 

- Sí, amor de tu Mamá. Sí, la leche. Antes que de costumbre... ¡De todas formas, Tú siempre estás preparado para mamar, corderito mío santo! 

Jesús ríe y juguetea, agitando los piececitos por fuera de las mantas, y los brazos, con una de esas manifestaciones de alegría de los niños pequeños que tan bonitas son de ver. Hinca los piececitos contra el estómago de su Mamá, se curva en forma de arco y apoya su cabecita rubia en el pecho de Ella, y luego se echa bruscamente para atrás y se ríe agarrando con sus manitas las cintas que ciñen al cuello el vestido de María tratando de abrirlo. Con su camisita de lino, se le ve a Jesús guapísimo, regordete, rosado como una flor. 

María se inclina. Así, inclinada, sobre la cuna como protección, llora y sonríe al mismo tiempo, mientras el Niño balbucea esas palabras, que no son palabras, de todos los niños pequeños, entre las cuales se oye nítida y repetidamente la palabra "mamá". La mira, asombrado de verla llorar. Alarga una manita hacia los brillantes hilos de llanto, que se la mojan al hacer la caricia. Primorosamente, vuelve a apoyarse en el pecho materno y en él se recoge enteramente, acariciándoselo con su manita. 

María lo besa por entre el pelo y lo toma en brazos, se sienta y se pone a vestirlo: ya tiene el vestidito de lana, ya las diminutas sandalitas. Le da la leche. Jesús mama con avidez la leche buena de su Mamá, y, cuando ya le parece que por la parte derecha viene menos, va a buscar a la izquierda, y ríe al hacerlo, mirando a su Mamá de abajo arriba, para luego dormirse de nuevo — apoyado aún el carrillo rosado y redondo en el seno blanco y redondo — sobre el pecho de Ella. 

María se levanta muy despacito y lo coloca sobre la manta acolchada de su cama. Lo tapa con su manto. Vuelve a la cuna y dobla las mantitas. Piensa en si conviene o no coger también el colchoncito. ¡Tan pequeño como es... se puede llevar! Lo pone, junto con la almohada, con las cosas que ya estaban encima del baulito. Y llora ante la cuna vacía. ¡Pobre Madre, perseguida en su Criatura! 

José regresa. 

- ¿Estás preparada? ¿Está preparado Jesús? ¿Has cogido sus mantas y su camita? No podemos llevarnos la cuna, pero por lo menos que tenga su colchoncito. ¡Oh, pobre Pequeñuelo, perseguido a muerte! 

- ¡José! - grita María agarrándose al brazo de José. 

Sí, María, a muerte. Herodes lo quiere muerto... porque tiene miedo de Él... Esa fiera inmunda tiene miedo de este Inocente, por su reino humano. No sé lo que hará cuando comprenda que ha huido; pero para entonces nosotros ya estaremos lejos. No creo que se vengue buscándolo incluso en Galilea. Ya sería difícil para él descubrir que somos galileos; más difícil aún, saber que somos de Nazaret y quiénes somos exactamente. A no ser que Satanás le eche una mano en agradecimiento de sus fieles servicios. Mas... si eso sucede... Dios nos ayudará igualmente. No llores, María, que el verte llorar es para mí un dolor mucho mayor que el de tener que marchar al exilio. 

- ¡Perdóname, José! No lloro por mí, ni por los pocos bienes que pierdo. Lloro por ti... ¡Ya mucho te has tenido que sacrificar! Ahora, otra vez, te quedas sin clientes, sin casa... ¡Cuánto te cuesto, José! 

- ¿Cuánto? No, María. No me cuestas nada. Me consuelas. Siempre me consuelas. No pienses en el mañana. Tenemos el caudal que nos han dado los Magos. Nos servirán de ayuda al principio. Luego me buscaré un trabajo. Un obrero honrado y competente se abre camino enseguida. Ya lo has visto aquí. No me da abasto el tiempo para el cúmulo de trabajo. 

Sí, lo sé. Pero, ¿quién te va a aliviar tu nostalgia? 

- ¿Y a ti? ¿Quién te va a aliviar la nostalgia de esa casa que tanto amas? 

- Jesús. Teniéndolo a Él, tengo todo lo que allí tenía. 

- Y yo también teniendo a Jesús tengo ya esa patria que he esperado hasta hace pocos meses, y... tengo a mi Dios. Ya ves que no pierdo nada de lo que más amo. Basta con salvar a Jesús; si es así, todo nos queda. Aunque no volviéramos a ver este cielo, estos campos, o los aún más amados campos de Galilea, siempre tendremos todo porque lo tendremos a Él. Ven, María, que empieza a clarear. Llega el momento de saludar a la huésped y de cargar nuestras cosas. Todo irá bien. 

María se pone en pie, obediente. Se arropa en su manto; mientras tanto, José prepara un último bulto, se lo carga y sale. 

María levanta delicadamente al Niño, lo arropa en un mantón y lo aprieta contra su pecho. Mira las paredes que durante meses la han hospedado y, rozándolas apenas, las toca con una mano. ¡Bendita esa casa, que ha merecido ser amada y bendecida por María! 

Sale. Cruza la habitacioncita que era de José, entra en la estancia grande. La dueña de la casa, en lágrimas, la besa y se despide de Ella, y, levantando un borde del mantón, besa al Niño en la frente. Él duerme tranquilo. Bajan por la escalerita exterior. 

Hay un primer claror de alborada que apenas permite ver. En la escasa luz se ven tres burros. El más fuerte lleva los enseres. Los otros van sólo con la albarda. José está manos a la obra para asegurar bien el baulillo y los paquetes en la albarda del primero. Veo, atados en un haz, y colocados encima del fardo, sus utensilios de carpintero. 

Nuevos saludos y nuevas lágrimas. María se monta en su burrillo, mientras la patrona tiene a Jesús en brazos y lo besa una vez más; luego se lo devuelve a María. Monta también José, el cual ha atado su asno al que lleva los equipajes, para estar libre y poder así controlar el de María. 

La huida comienza mientras Belén, que sueña todavía la fantasmagórica escena de los Magos, duerme tranquila, sin saber lo que le espera. 

Y la visión cesa así. 

Dice Jesús: 
\emph{Y también esta serie de visiones terminan así. Hemos ido mostrándote las escenas que precedieron, acompañaron y siguieron a mi Llegada; no por ellas mismas, que son muy conocidas, sino para aplicación, en ti y en los demás, del sentido sobrenatural que de ellas deriva, y dároslo como norma de vida. Estas escenas son muy conocidas, aunque haya que decir que han sido alteradas por elementos que han ido superponiéndose con los siglos, debido siempre a ese modo de ver, humano, que, pretendiendo dar mayor gloria a Dios — y por ello queda perdonado —transforma en irreal lo que sería tan bonito dejar real. Porque ello no disminuye mi Humanidad ni la de María, de la misma manera que este ver las cosas en su realidad no ofende ni a mi Divinidad ni a la Majestad del Padre ni al Amor de la Trinidad santísima; antes bien, con ello resplandecen los méritos de mi Madre y mi perfecta humildad, y refulge la bondad omnipotente del eterno Señor. El Decálogo es la Ley; mi Evangelio, la doctrina que os la hace más clara y más atractiva de seguirse. Serían suficientes esta Ley y esta Doctrina para obtener, de los hombres, santos. Pero vuestra humanidad os pone tantas dificultades — humanidad que, verdaderamente, en vosotros sobrepuja demasiado al espíritu — que no podéis seguir estos caminos, y caéis, u os detenéis descorazonados. Os decís a vosotros mismos, y a quienes quisieran haceros caminar citándoos los ejemplos del Evangelio: "Pero Jesús, María, José... (y así todos los santos) no eran como nosotros. Eran fuertes; han sufrido, pero han sido inmediatamente consolados; fueron aliviados incluso de ese poco dolor que sufrieron; no sentían las pasiones... Eran seres que ya estaban fuera de la tierra". ¡Ese poco dolor!.. ¡No sentían las pasiones!.. El dolor fue amigo fiel nuestro, con los más variados aspectos y nombres. Las pasiones... No uséis mal la palabra, llamando "pasiones" a los vicios que os sacan del camino recto. Llamadlos sinceramente "vicios", y, además, capitales. No es que nosotros ignorásemos los vicios. Teníamos ojos y oídos, y Satanás hacía danzar ante nosotros y a nuestro alrededor estos vicios, mostrándonoslos en los viciosos con toda su carga de suciedad, o tentándonos con insinuaciones. Mas estas porquerías y estas insinuaciones, tendida como estaba la voluntad a querer agradar a Dios, en vez de producir lo que se había propuesto Satanás, producían lo contrario. Y cuanto más insistía él, más nos refugiábamos nosotros en la luz de Dios, por asco hacia las tinieblas fangosas que nos ponía ante los ojos del cuerpo y del espíritu. Pero no hemos ignorado las pasiones en sentido filosófico entre nosotros. Amamos la patria, y con ella a nuestra pequeña Nazaret, más que a cualquier otra ciudad de Palestina. Tuvimos afectos hacia nuestra casa, hacia los parientes y los amigos. ¿Por qué no íbamos a haberlos tenido? Pero no nos hicimos esclavos de los afectos, porque nada sino Dios debe ser señor; antes bien hicimos de ellos buenos compañeros nuestros. Mi Madre gritó de alegría cuando, pasados aproximadamente cuatro años, volvió a Nazaret y puso pie en su casa, y besó esas paredes entre las cuales su "Sí" abrió su seno para recibir la Semilla de Dios. José saludó con alegría a los parientes, a los sobrinitos, crecidos en número y en edad. Gozó al verse recordado por sus conciudadanos y al ver que por sus dotes en el oficio lo buscaron enseguida. Yo fui sensible a la amistad. Sufrí por la traición de Judas como por una crucifixión moral. ¿Y qué?: ni mi Madre ni José antepusieron su amor a la casa, o a los familiares, a la voluntad de Dios. Y Yo no escatimé palabras — si había que decirlas — que me habrían de acarrear el rencor de los hebreos o la animadversión de Judas. Yo sabía — y podría haberlo hecho — que bastaba el dinero para sujetarlo a mí; pero hubiera sido no a mí como Redentor sino a mí como rico. Yo, que multipliqué los panes, si hubiera querido, habría podido multiplicar el dinero; pero no había venido para proporcionar satisfacciones humanas. A nadie. Mucho menos a los que había llamado. Yo había predicado sacrificio, desapego, vida casta, puestos humildes. ¿Qué Maestro habría sido Yo, qué Justo, si hubiese dado dinero a uno para su sensualismo mental y físico, sólo porque ése hubiera sido el modo de sujetarlo a mí? Para ser grandes en mi Reino hay que hacerse "pequeños". Quien quiera ser "grande" a los ojos del mundo no es apto para reinar en mi Reino; paja es para el lecho de los demonios. Porque la grandeza del mundo está en antítesis con la Ley de Dios. El mundo llama "grandes" a quienes — con medios casi siempre ilícitos — saben conseguir los mejores puestos y, para hacerlo, hacen del prójimo escabel, y ponen su pie encima y lo aplastan; llama "grandes" a los que saben matar para reinar — matar moral o materialmente — y arrebatan puestos o se enseñorean de las naciones y se enriquecen desangrando a los demás, arrebatándoles la riqueza individual o colectiva. El mundo llama frecuentemente "grandes" a los delincuentes. No. La "grandeza" no está en la delincuencia, está en la bondad, la honradez, el amor, la justicia. ¡Observad qué venenosos frutos — recogidos en su malvado, demoníaco jardín interior — vuestros "grandes" os ofrecen! Deseo hablar de la última visión, dejando de lado otras cosas, total, sería inútil, porque el mundo no quiere oír la verdad que le concierne. Esta visión da luz acerca de un detalle citado dos veces en el Evangelio de Mateo, una frase repetida dos veces: "¡Levántate, toma al Niño y a su Madre y huye a Egipto!"; "¡Levántate, toma al Niño y a su Madre y vuelve a la tierra de Israel!". Y has podido ver cómo en la habitación estaba María sola con el Niño. La virginidad de María después del parto y la castidad de José sufren muchas agresiones por parte de quienes, siendo sólo lodo putrefacto, no admiten que uno pueda ser ala y luz. Desdichados, cuyo fauno está tan corrompido y cuya mente está tan prostituida a la carne, que son incapaces de pensar que uno como ellos pueda respetar a una mujer, viendo en ella el alma y no la carne; incapaces de elevarse a sí mismos viviendo en una atmósfera sobrenatural, tendiendo no a las cosas carnales, sino a las divinas. Pues bien, a estos que combaten contra la suprema belleza, a estos gusanos incapaces de transformarse en mariposa, a estos reptiles cubiertos por la baba de su lujuria, incapaces de comprender la belleza de una azucena, Yo les digo que María fue virgen y siguió siéndolo, y que solo su alma se desposó con José, como también su espíritu únicamente se unió al Espíritu de Dios, y por obra de Éste concibió al Único que llevó en su seno: a mí, a Jesucristo, Unigénito de Dios y de María. No se trata de una tradición que haya florecido después, por un amoroso respeto hacia mi Bienaventurada Madre; se trata de una verdad conocida ya desde los primeros tiempos. Mateo no nació siglos más tarde; era contemporáneo de María. Mateo no era un pobre ignorante que hubiera vivido en los bosques y que fuera propenso a creerse cualquier patraña. Era un funcionario de hacienda, como diríais ahora vosotros (nosotros entonces decíamos recaudador). Sabía ver, oír, entender, escoger entre la verdad y la falsedad. Mateo no oyó las cosas por referencias de terceros, sino que las recogió de labios de María, preguntándole a Ella, llevado de su amor hacia el Maestro y hacia la verdad. Y no quiero pensar que estos que niegan la inviolabilidad de María piensen que Ella quizás pudo mentir. Mis propios parientes, si hubiera habido otros hijos, hubieran podido desmentir su testimonio: Santiago, Judas, Simón y José eran condiscípulos de Mateo. Por tanto éste hubiera podido fácilmente confrontar las versiones, si hubiese habido otras versiones. Y sin embargo Mateo nunca dice: "¡Levántate y toma contigo a tu mujer!". Dice: "¡Toma contigo a la Madre de Él!". Y antes dice: "Virgen desposada con José"; 'José, su esposo". Y que éstos no objeten que se trataba de un modo de hablar de los hebreos, como si decir "la mujer de" fuera una infamia. No, negadores de la Pureza. Ya desde las primeras palabras del Libro se lee: "... y se unirá a su mujer". Se la llama "compañera" hasta el momento de la consumación física del vínculo matrimonial, y luego se la llama "la mujer de" en distintos momentos y en distintos capítulos. Así se les llama a las esposas de los hijos de Adán; y a Sara, llamada "mujer de" Abraham: "Sara, tu mujer". Y también: "Toma contigo a tu mujer y a tus dos hijas", a Lot. Y en el libro de Rut está escrito: "La Moabita, mujer de Majlón". Y en el primer libro de los Reyes se dice: "Elcana tuvo dos mujeres"; y luego: "Elcana después conoció a su mujer Ana"; y también: "Elí bendijo a Elcana y a la mujer de éste". Y también en el libro de los Reyes está escrito: "Betsabé, mujer de Urías Eteo, vino a ser mujer de David y le dio a luz un hijo". Y ¿qué se lee en el libro azul de Tobías, lo que la Iglesia os canta en vuestras bodas, para aconsejaros que seáis santos en el matrimonio? Se lee: "Llegado Tobit con su mujer y con su hijo..."; y también: "Tobit logró huir con su hijo y con su mujer". Y en los Evangelios, o sea, en tiempos contemporáneos a Cristo, en que, por tanto, se escribía con lenguaje moderno respecto a aquellos tiempos — por lo que no pueden sospecharse errores de trascripción — se dice, y precisamente lo dice Mateo en el capítulo 22: "...y el primero, habiendo tomado mujer, murió y dejó su mujer a su hermano". Y Marcos en el capítulo 10: "Quien repudia a su mujer...". Y Lucas llama a Isabel mujer de Zacarías, cuatro veces seguidas; y en el capítulo 8 dice: 'Juana, mujer de Cusa". Como podéis ver, este nombre no era un vocablo proscrito por quien estaba en las vías del Señor, un vocablo inmundo, no digno de ser proferido, y mucho menos escrito, donde se tratara de Dios y de sus obras admirables. Y el ángel, diciendo: "el Niño y su Madre", os demuestra que María fue verdadera Madre suya, pero no fue la mujer de José; siempre fue: la Virgen desposada con José. Y ésta es la última enseñanza de estas visiones. Y es una aureola que resplandece sobre las cabezas de María y de José. La Virgen inviolada. El hombre justo y casto. Las dos azucenas entre las que crecí oyendo sólo fragancias de pureza. A ti, pequeño Juan, te podría hablar sobre el dolor de María por su doble, brusca separación de la casa y de la patria. Pero no hay necesidad de palabras. Tú lo comprendes y ello te hace morir. Dame tu dolor. Sólo quiero esto. Es más que cualquier otra cosa que puedas darme. Es viernes, María. Piensa en mi dolor y en el de María en el Gólgota para poder soportar tu cruz. Nuestra paz y nuestro amor quedan contigo.}

\chapter*{La Sagrada Familia en Egipto. \\ \normalfont\normalsize\textit{Una lección para las familias. \\ 25 de Enero de 1944 (12 de la noche).}}
\addcontentsline{toc}{chapter}{\normalfont\scshape{La Sagrada Familia en Egipto.}}

La suave visión de la Sagrada Familia. El lugar está en Egipto. No tengo dudas de ello porque veo el desierto y una pirámide. 

Veo una casucha de un solo piso, el bajo, toda blanca. Una pobre casa de una muy pobre gente. Las paredes están apenas revocadas y cubiertas de una mano de cal. La casita tiene dos puertas, una junto a la otra, que introducen en sus dos únicas habitaciones, en las que, por ahora, no entro. La casita está en medio de un pedazo de tierra arenosa rodeada por una protección de cañas hincadas en el suelo: una protección muy débil contra los ladrones; puede servir sólo como defensa contra algún perro o gato vagabundo. Claro, ¿a quién le van a venir ganas de robar donde se ve que no hay ni sombra de riqueza? 

Esta poca tierra que el seto de cañas limita ha sido cultivada pacientemente como una pequeña huerta, a pesar de ser árida y poco fértil. Para hacer más tupido y menos escuálido el seto, han traído unas plantas trepadoras, que me parecen modestos convólvulos. Sólo en uno de los lados, hay un arbusto de jazmines en flor y una mata de rosas de las más comunes. En la huertecilla, en los pocos cuadros del centro, noto que hay unas modestísimas verduras, bajo un árbol dejado crecer libremente, que no sé qué clase de árbol es, y que da un poco de sombra al terreno soleado y a la casita. A este árbol está atada una cabrita blanca y negra, que está comiendo y rumiando las hojas de algunas ramas dejadas caer al suelo. 

Allí cerca, sobre una estera extendida en el suelo, está el Niño Jesús. Me da la impresión de que tiene unos dos años, o dos años y medio como mucho. Está jugando con unos pedacitos de madera tallados, que parecen ovejitas o caballitos, y con unas virutas de madera de color claro, menos rizadas que sus bucles de oro. Con sus manitas regordetas está tratando de poner estos collares de madera en el cuello de sus animalitos. 

Está tranquilo y sonriente. Muy guapo. Una cabecita toda de bucles de oro muy tupidos; piel clara y delicadamente rosácea; ojitos vivos, brillantes, de color azul intenso. La expresión, naturalmente, es distinta, pero reconozco el color de los ojos de mi Jesús (dos zafiros oscuros y bellísimos). 

Viste una especie de larga camisita blanca, que será, sin duda, su túnica; con las mangas hasta el codo. Los pies, en este momento, al desnudo. Las diminutas sandalias están sobre la estera y juega también con ellas el Niño: mete en la suela sus animalitos, y tira de la correa de la sandalia, como si fuera un carrito. Son unas sandalias muy sencillas: una suela y dos correas, que salen: una, de la puntera; otra, del talón; la de la puntera tiene un punto en que se bifurca y una parte pasa por el ojo de la correa del talón para anudarse luego con la otra parte, formando un anillo en la garganta del pie. 

Un poco separada — también a la sombra del árbol — está la Virgen. Está tejiendo en un tosco telar; mientras, vigila al Niño. Veo que las finas y blancas manos van y vienen entramando, y el pie, calzado con sandalia, mueve el pedal. La viste una túnica de color flor de malva, un violeta rosáceo, como el de ciertas amatistas. Tiene la cabeza descubierta, con lo cual puedo ver cómo sus cabellos rubios están separados en dos en la cabeza y peinados sencillamente con dos trenzas que a la altura de la nuca le forman un bonito moño. Las mangas de la túnica son largas y más bien estrechas. No lleva ningún adorno, aparte de su belleza y de su expresión dulcísima. El color del rostro, del pelo y de los ojos, la forma de la cara, son como siempre que la veo. Aquí parece jovencísima. Aparenta apenas veinte años. 

En un momento dado se levanta; se inclina hacia el Niño y, cuidadosamente, le pone otra vez las sandalias y se las ata; lo acaricia y lo besa en la cabecita y en los ojitos. El Niño farfulla unas palabras y Ella responde, pero no entiendo las palabras. Luego vuelve a su telar, extiende sobre la tela y sobre la trama un paño, coge la banqueta en que estaba sentada y se la lleva a la casa. El Niño la sigue con la mirada, sin importunarla cuando Ella lo deja solo. 

Se ve que el trabajo ha terminado y que empieza a caer la tarde. En efecto, el Sol baja hacia las arenas desnudas y un verdadero fuego invade el cielo detrás de la pirámide lejana. 

María vuelve. Coge de la mano a Jesús para que se levante de la esterilla. El Niño obedece sin resistencia. Mientras su Mamá está re- cogiendo los juguetes y la estera y llevando esas cosas a casa, Él corre hacia la cabrita con un trotecillo de sus bien torneadas piernecitas, y le echa los bracitos al cuello. La cabrita bala y frota su morrito en los hombros de Jesús. 

María vuelve. Tiene ahora un largo velo sobre la cabeza y una ánfora en la mano. Coge a Jesús de la manita y se encaminan los dos, rodeando la casa, hacia la otra fachada. 

Yo los sigo, admirando la gracia de la escena: la 'Virgen conformando su paso al del Niño, y el Niño a su lado dando saltitos o pasitos rápidos. Veo cómo se alzan y se posan los rosados talones, con la gracia propia de los pasos de los niños, sobre la arena del senderillo. Me doy cuenta de que su túnica no le llega a los pies, sino sólo hasta la mitad del muslo. Es primorosa, sencillísima, y está sujeta a la cintura por un cordoncito también blanco. 

Veo que en la parte delantera de la casa el seto está interrumpido por una tosca cancilla; María la abre para salir al camino (un mísero camino al extremo de una ciudad — o pueblo —, donde el centro habitado termina en el campo abierto, que aquí está constituido de arena y alguna que otra casita, pobre como ésta, con alguna que otra mísera huerta). 

No veo a nadie. María mira hacia el centro, no hacia el campo, como si esperara a alguien, luego se dirige a un pilón — o pozo — que está a unos cuantos metros más arriba, sombreado en círculo por palmeras. Y veo que el terreno en ese lugar tiene hierba verde. 

Veo que se acerca por el camino un hombre; no demasiado alto, pero robusto. Reconozco en él a José. Viene sonriente. Es más joven que cuando lo vi en la visión del Paraíso. Aparenta como mucho cuarenta años. Su pelo y barba son tupidos y negros; la piel, más bien tostada; los ojos, oscuros. Un rostro honesto y agradable, un rostro que inspira confianza. 

Al ver a Jesús y a María acelera el paso. Trae sobre el hombro izquierdo una especie de sierra y una especie de cepillo de carpintero, y en la mano otras herramientas del oficio, no iguales que las de ahora, pero sí muy parecidas. Parece como si estuviera regresando de haber hecho algún trabajo en casa de alguno. Su vestido es de un color entre avellana y marrón; no muy largo — le llega sólo hasta un buen trozo por encima del tobillo —, con las mangas cortas, hasta el codo. Lleva a la cintura una correa de cuero — me parece —. Se trata de un vestido típicamente de trabajo. Calzan sus pies unas sandalias cruzadas a la altura del tobillo. 

María sonríe y el Niño emite unos grititos de alegría mientras tiende hacia adelante su bracito libre. Cuando se encuentran los tres, José se inclina para ofrecerle al Niño un fruto — por el color y la forma, creo que es una manzana —. Luego le tiende los brazos y el Niño deja a su Mamá y se acurruca entre los brazos de José, e inclina su cabecita para apoyarla en la cavidad que forma el cuello de él. José besa a Jesús y Jesús besa a José. Una acción llena de afectuosa gracia. 

Me he olvidado de decir que María, diligentemente, había cogido las herramientas de trabajo de José para que pudiera abrazar al Niño sin ningún estorbo. 

Luego José, que se había acuclillado para ponerse a la altura de Jesús, se alza de nuevo. Coge sus herramientas con la mano izquierda y mantiene al pequeño Jesús estrechado contra su robusto pecho con la derecha; así, se encamina hacia la casa mientras María va a la fuente a llenar su ánfora. 

Entrado en el recinto de la casa, José baja al suelo al Niño, coge el telar de María y lo lleva a casa; luego ordeña a la cabrita. Jesús observa atentamente estas operaciones, como también la de encerrar a la cabrita en un cuartito hecho en uno de los lados de la casa. 

Se pone la tarde. Veo el rojo del ocaso hacerse violáceo sobre la arena que parece temblar por el calor; y la pirámide parece más oscura. 

José entra en la casa, en una habitación que debe ser taller, cocina y comedor al mismo tiempo. Se ve que el otro cuarto es el destinado al descanso; pero en él yo no entro. Hay una tenue lumbre encendida. Hay un banco de carpintero, una pequeña mesa, unas banquetas, unas repisas donde están los pocos platos y vasos que tienen y también dos lámparas de aceite. En uno de los rincones, el telar de María. Y... mucho, mucho orden y limpieza; es una morada pobrísima, pero está limpísima. 

Quisiera hacer esta observación: en todas las visiones que tienen por objeto la vida humana de Jesús, he notado que, tanto El, como María, como José, como Juan, tienen siempre en orden y limpios el vestido y la cabeza; vestidos modestos, peinados sencillos pero de una limpieza que les hace aparecer señoriales. 

María vuelve con el ánfora. Ha llegado rápido el crepúsculo. Cierran la puerta. Una lamparita, que José ha encendido y colocado sobre su banco, da claridad a la habitación; encorvado hacia éste, él sigue trabajando, en unas pequeñas tablas. Mientras tanto María prepara la cena. También la lumbre da claridad a la habitación. Jesús, con sus manitas apoyadas en el banco y con la cabecita mirando hacia arriba, observa lo que hace José. 

Luego se sientan a la mesa después de haber rezado. No se hacen — es natural — el signo de la cruz, pero rezan; José dirige la oración, María responde. No entiendo las palabras. Debe ser un salmo. Lo dicen en una lengua que me es totalmente desconocida. 

Se sientan a cenar. Ahora la lamparita está encima de la mesa. María tiene a Jesús en su regazo y le da a beber la leche de la cabrita y moja en la leche unas rebanadas de un pan pequeño y de forma redondeada, de corteza y miga duras. Parece un pan hecho con centeno y cebada. Tiene mucho salvado, claro, porque es pan moreno. Entre tanto, José come pan y queso: una raja delgada de queso y mucho pan. Luego María sienta a Jesús en una banquetita que está a su lado y trae a la mesa unas verduras cocidas — creo que están hervidas y condimentadas en la forma en que normalmente hacemos nosotros — y, después de servirse José, también las come Ella. Jesús mordisquea tranquilo su manzana, y descubre sonriendo sus dientecitos blancos. La cena termina con unas aceitunas o dátiles. No sé bien, porque, para ser aceitunas, son demasiado claras, pero, para ser dátiles, son demasiado duros. Vino, nada. Es una cena de gente pobre. 

Pero tanta es la paz que se respira en esta habitación, que no podría dármela igual la visión de ningún pomposo palacio. 

¡Y cuánta armonía! 

Dice Jesús: 
\emph{La lección, para ti y para los demás, está en las cosas que has visto. Es una lección de humildad, de resignación y de armonía. Sirva de ejemplo a todas las familias cristianas, y, de forma particular, a las que viven en este peculiar y doloroso momento. Has visto una casa pobre; una casa pobre — y esto es lo doloroso — en un país extranjero. Muchos, sólo por el hecho de ser unos fieles "pasables", que rezan y me reciben a mí bajo las especies eucarísticas, que rezan y comulgan por "sus" necesidades, no por las necesidades de las almas y para la gloria de Dios — porque es muy raro el que al orar no sea egoísta —, muchos, sólo por este hecho, esperan poder disfrutar de una vida material fácil al amparo del más mínimo dolor, de una vida próspera y feliz. José y María me tenían a mí, Dios verdadero, como Hijo suyo, y, no obstante, no tuvieron ni siquiera ese mínimo bien de ser pobres en su patria, en el país donde se los conocía; donde, por lo menos, tenían una casita "suya" y al menos la preocupación del alojamiento no añadía angustia a las muchas otras, en el país en que, por ser conocidos, habría sido más fácil encontrar trabajo y proveer a las necesidades de la vida. Son dos expatriados precisamente por tenerme a mí. Un clima distinto, un país distinto — ¡y tan triste respecto a los dulces campos de Galilea! —, lengua distinta, costumbres distintas, allí, entre una gente que no los conocía y que, como es normal entre los pueblos, desconfiaban de expatriados y desconocidos. Les faltaban los queridos y cómodos muebles de "su" casita, y esas otras muchas cosas, humildes pero necesarias, que allí había y que entonces no parecían tan necesarias, mientras que aquí, rodeados de esta nada, habrían parecido incluso bonitas (como lo superfluo que hace deliciosas las casas de los ricos). Sentían la nostalgia de la tierra y de la casa, y la preocupación de esas pobres cosas dejadas allí, de la huertecita que quizás ninguno cuidaría, de la vid y de la higuera y de las otras plantas útiles. Les apremiaba la necesidad de conseguir el alimento cotidiano, el vestido, el fuego todos los días; y la necesidad de atenderme a mí, un Niño, al cual no se le podía dar la comida que a sí mismo uno puede darse. Y tenían el corazón lleno de pesares: por las nostalgias, la incógnita del mañana, la desconfianza de la gente, reacia como es, especialmente en los primeros momentos, a acoger ofertas de trabajo de dos desconocidos. Y a pesar de todo, ya has visto cómo en esta morada se respira serenidad, sonrisa, concordia; y cómo, de común acuerdo, se trata de embellecerla — incluso la mísera huertecita — para que se asemeje más a la que han dejado y para hacerla más confortable. Y cómo en ellos hay un solo pensamiento: hacerme esa tierra menos hostil, a mí, Santo; hacerme esa tierra menos mísera, a mí, que vengo de Dios. Es un amor de creyentes y de padres, que se manifiesta en mil cuidados, que van desde la cabrita — comprada con muchas horas extra de trabajo — hasta los juguetitos tallados en la madera que sobraba, o hasta esa fruta cogida sólo para mí, negándose a sí mismos un bocado. ¡Oh, amado padre mío de la Tierra, cuánto te ha querido Dios, Dios Padre en las Alturas; Dios Hijo, que se ha hecho Salvador, en la Tierra! En esta casa no hay nerviosismos, caras largas o sombrías, como no hay tampoco el echarse en cara recíprocamente nada, y mucho menos a Dios, que no los ha colmado de bienestar material. José no acusa a María de ser causa de su incomodidad, como tampoco María acusa a José de no saberle dar un mayor bienestar. Se aman santamente, eso es todo, y, por tanto, su preocupación no es el propio bienestar, sino el del cónyuge. El verdadero amor no conoce egoísmo. El verdadero amor es siempre casto, aunque no sea perfecto en la castidad como el de los dos esposos vírgenes. La castidad unida a la caridad conlleva todo un bagaje de otras virtudes y, por tanto, hace, de dos que se aman castamente, dos cónyuges perfectos. El amor de mi Madre y de José era perfecto. Por tanto era impulso de todas las virtudes, especialmente de la caridad para con Dios, que en todo momento era bendecido, a pesar de que su santa voluntad resultase penosa para la carne y para el corazón; era bendecido porque por encima de la carne y del corazón, en estos dos santos, vivía y dominaba más intensamente el espíritu, el cual magnificaba agradecido al Señor por haberlos elegido para ser los custodios de su eterno Hijo. En aquella casa se hacía oración. Demasiado poco se reza en las casas ahora. Se levanta el día y desciende la noche, empezáis a trabajar y os sentáis a la mesa... sin un pensamiento para el Señor, que os ha permitido ver un nuevo día, que os ha permitido llegar a una nueva noche, que ha bendecido vuestros esfuerzos y ha concedido que éstos os fueran medio para obtener ese alimento, ese fuego, esos vestidos, ese techo que, sí, también le son necesarios a vuestra condición humana. Siempre es "bueno" lo que viene de Dios, que es bueno. Aunque ello sea pobre y escaso, el amor le da sabor y sustancia; ese amor que os hace ver en el eterno Creador al Padre que os ama. En aquella casa había frugalidad. La habría habido aunque el dinero no hubiera faltado. Se comía para vivir, no para gozo de la gula con la insaciabilidad de los comilones y los caprichos de los glotones, que se llenan hasta rebosar o desperdician dinero en alimentos caros sin pensar siquiera en quien escasea de comida o no la tiene, sin reflexionar en que si fueran moderados ellos muchos podrían ser aliviados de las dentelladas del hambre. En aquella casa había amor por el trabajo. Este amor hubiera existido aunque el dinero hubiera abundado; porque, trabajando, el hombre obedece al mandato de Dios y se libera del vicio que, cual tenaz hiedra, aprieta y ahoga a los ociosos, que son como bloques de piedra inmóviles. Bueno es el alimento, sereno es el descanso, contento se siente el corazón, cuando uno ha trabajado bien y disfruta de su tiempo de reposo entre un trabajo y otro. El vicio, con sus múltiples facetas, no arraiga ni en la casa ni en la mente de quien ama el trabajo; al no arraigar el vicio, prospera el afecto, la estima, el respeto mutuo, y crecen los tiernos vástagos en un ambiente puro, viniendo a ser así a su vez origen de futuras familias santas. En aquella casa reinaba la humildad. ¡Cuán vasta lección de humildad para vosotros, soberbios! María habría tenido, humanamente, miles de motivos para ensoberbecerse y para obtener que el cónyuge la adorase. Muchas mujeres lo hacen, y sólo por ser un poco más cultas, o de ascendencia más noble, o más acaudaladas que el marido. María es Esposa y Madre de Dios, y, sin embargo, sirve — no se hace servir — al cónyuge, y es toda amor para con él. José es la cabeza en esa casa; ha sido juzgado por Dios digno de ser cabeza de familia, de recibir de Dios al Verbo encarnado y a la Esposa del Espíritu Santo para custodiarlos. Y, con todo, se muestra solícito en aligerar a María de esfuerzos y labores, y se ocupa de los más humildes quehaceres que puede haber en una casa, para que María no se fatigue; y no sólo esto, sino que, como puede, en la medida de sus posibilidades, la alivia y se las ingenia para hacerle cómoda la casa y alegre de flores la pequeña huerta. En aquella casa se respetaba el orden: sobrenatural, moral y material. Dios, como Señor supremo que es, recibe culto y amor: éste es el orden sobrenatural. José es el cabeza de familia, y recibe afecto, respeto y obediencia: orden moral. La casa es un don de Dios, como también el vestido y los enseres; en todas las cosas se manifiesta la Providencia de Dios, de ese Dios que proporciona la lana a las ovejas, plumas a los pájaros, hierba a los prados, heno a los animales, semillas y ramas a las aves; de ese Dios que teje el vestido del lirio de los valles. Casa, vestido, enseres: estas cosas hay que recibirlas con gratitud, bendiciendo la mano divina que las otorga, tratándolas con respeto, como don del Señor; no mirándolas, porque sean pobres, con enfado; y sin maltratarlas abusando de la Providencia: éste es el orden material. No has comprendido la conversación en dialecto nazareno, ni tampoco las palabras de la oración, pero las cosas que has visto han servido de gran lección. ¡Meditadla, vosotros, los que tanto sufrís ahora por haber faltado en tantas cosas a Dios, incluso en aquellas en que jamás faltaron los santos Esposos que me fueron Madre y padre! Y tú regocíjate con el recuerdo del pequeño Jesús; sonríe pensando en sus pasitos infantiles. Dentro de poco le verás caminar bajo una cruz; entonces será una visión de llanto.}

\chapter*{Primera lección de trabajo a Jesús, \\ \normalfont\normalsize\textit{que se sujetó a la regla de la edad. \\ 21 de Marzo de 1944.}}
\addcontentsline{toc}{chapter}{\normalfont\scshape{Primera lección de trabajo a Jesús}}

Veo aparecer, dulce como un rayo de sol en día lluvioso, a mi Jesús, pequeñuelo de unos cinco años aproximadamente, todo rubio y todo lindo con un sencillo vestidito azul celeste que le llega hasta la mitad de sus bien contorneados muslos. 

Está jugando con la tierra en el pequeño huerto. Está haciendo montoncillos de tierra, y plantando encima ramitas, como si fueran bosques en miniatura; con piedrecitas marca los senderos. Luego intenta hacer un pequeño lago en la base de sus minúsculas colinas. Para ello coge un fondo de alguna pieza vieja de loza y lo entierra, hasta el borde; luego lo llena de agua con una botija que zambulle en un pilón usado como lavadero o para regar el huerto. Pero lo único que consigue es mojarse el vestido, sobre todo las mangas. El agua se sale del plato desportillado, y, tal vez, rajado, y... el lago se seca. 

José ha salido a la puerta y, silencioso, se queda un tiempo mirando todo ese trabajo que está haciendo el Niño, y sonríe. En efecto, es un espectáculo que hace sonreír de alegría. Luego, para impedir que 

Jesús se moje más, le llama. Jesús se vuelve sonriendo, y, viendo a José, corre hacia él con sus bracitos tendidos hacia adelante. José, con el borde de su indumento corto de trabajo, le seca las manitas llenas de tierra y se las besa. Y comienza un dulce diálogo entre los dos. 

Jesús explica su trabajo y su juego, así como las dificultades que había encontrado para llevarlo a cabo. Quería hacer un lago como el de Genesaret (por ello supongo que le habían hablado de él o que lo habían llevado a verlo). Quería hacerlo en pequeño, como entretenimiento. Aquí estaba Tiberíades, allí Magdala, allí Cafarnaúm. Esta era la vía que llevaba, pasando por Caná, a Nazaret. Quería botar al lago unas barquitas — estas hojas son barcas — e ir a la otra orilla. Pero, el agua se sale... 

José observa y se interesa tomándolo todo con seriedad. Luego propone hacer él "mañana" un pequeño lago, no con el plato desportillado, sino con un pequeño recipiente de madera, bien estucado y empecinado, en el que Jesús podrá botar verdaderas barquitas de madera que José le va a enseñar a hacer. Precisamente en este momento le iba a traer unas pequeñas herramientas de trabajo, adecuadas para Él; para que pudiera aprender, sin mayor esfuerzo, a usarlas. 

- ¡Así te podré ayudar! - dice Jesús con una sonrisa. 

- Así me podrás ayudar, y te harás un hábil carpintero. Ven a verlas. 

Y entran en el taller. Y José le muestra un pequeño martillo, una sierra pequeña, unos minúsculos destornilladores, una garlopa como de juguete; todo ello puesto encima de un banco de carpintero recién hecho: un banco adecuado a la estatura del pequeño Jesús. 

- ¿Ves cómo se sierra? Se apoya este pedazo de madera así. Se coge la sierra así, y, con cuidado de no ir a los dedos, se sierra. Prueba tú... 

Y empieza la lección. Y Jesús, rojo del esfuerzo y apretando los labios, sierra con cuidado, y luego alisa la tablita con la garlopa, y, a pesar de que esté no poco torcida, le parece bonita, y José le alaba y le enseña a trabajar, con paciencia y amor. 

María regresa — estaba fuera de casa —, se asoma a la puerta y mira. Ninguno de los dos la ve porque están vueltos de espaldas. La Madre sonríe al ver el interés con que Jesús usa la garlopa, y el afecto con que José le enseña. 

Pero Jesús debe sentir esa sonrisa. Se vuelve. Ve a su Mamá y corre hacia Ella con su tablita medio cepillada y se la enseña. María observa con admiración y se inclina hacia Jesús para darle un beso. Le pone en orden los ricitos despeinados, le seca el sudor de su cara acalorada, y, afectuosa, le escucha cuando Jesús le promete que le va a hacer una banquetita para que trabaje más cómoda. 

José, erguido junto al minúsculo banco, apoyada su mano en uno de los lados, mira y sonríe. 

He presenciado la primera lección de trabajo a mi Jesús. Y toda la paz de esta Familia santa está en mí. 

Dice Jesús: 

\emph{- Te he confortado, alma mía, con una visión de mi niñez, feliz dentro de su pobreza por haber estado rodeada del afecto de dos santos mayores cuales el mundo no tiene ninguno. Se dice que José fue el padre nutricio mío. ¡Cierto es que, si bien no pudo, como hombre, darme la leche con que me nutrió María, sí se quebrantó a sí mismo trabajando para darme pan y confortación, y tuvo una dulzura de sentimientos de verdadera madre! De él aprendí — y jamás alumno alguno tuvo un maestro mejor — todo aquello que hace del niño un hombre; un hombre, además, que ha de ganarse el pan. Si bien mi inteligencia de Hijo de Dios era perfecta, hay que reflexionar y creer que Yo no quise saltarme sin más la regla de la edad. Por eso, humillando mi perfección intelectiva de Dios hasta el nivel de una perfección intelectiva humana, me sujeté a tener como maestro a un hombre, a tener necesidad de un maestro. Y el hecho de haber aprendido con rapidez y buena voluntad no me quita el mérito de haberme sujetado a un hombre, como tampoco le quita a este hombre justo el de haber sido él quien nutrió mi pequeña mente con las nociones necesarias para la vida. Esas gratas horas pasadas al lado de José (quien, como a través de un juego, me puso en condiciones de ser capaz de trabajar), esas horas, no las olvido ni siquiera ahora que estoy en el Cielo. Y cuando miro a mi padre putativo, veo nuevamente el huertecito y el humoso taller, y me parece ver a mi Madre asomándose con esa sonrisa suya que hacía de oro el lugar y dichosos a nosotros. ¡Cuánto deberían las familias aprender de estos esposos perfectos, que se amaron como ningunos otros lo hicieran! José era la cabeza. Clara e indiscutible era su autoridad familiar; ante ella se plegaba reverente la de la Esposa y Madre de Dios; a ella se sujetaba el Hijo de Dios. Todo lo que José decidía, bien hecho estaba; sin discusiones, sin obstinaciones, sin resistencia alguna. Su palabra era nuestra pequeña ley. ¡Y, a pesar de ello, cuánta humildad tuvo! Jamás abusó de su poder, jamás dictaminó cosa alguna contra todo canon, simplemente por ser el jefe. La Esposa era su dulce consejera, y aunque Ella, en su profunda humildad, se considerase la sierva de su consorte, éste extraía, de su sabiduría de Llena de Gracia, la luz para conducirse en todo lo que acaecía. Y Yo así fui creciendo, cual flor protegida por dos vigorosos árboles, entre estos dos amores que se entrelazaban por encima de mí para protegerme y amarme. No. Mientras la edad me hizo ignorar el mundo, Yo no sentí nostalgia del Paraíso. Presentes estaban Dios Padre y el Divino Espíritu, pues María estaba llena de Ellos. Y los ángeles allí moraban, porque nada les hacía alejarse de esa casa. Y hasta podría decir que uno de ellos se había revestido de carne y era José, alma angélica liberada del peso de la carne, dedicada sólo a servir a Dios y a su causa y a amarlo como le aman los serafines. ¡Oh, la mirada de José!: pacífica y pura como la de una estrella ajena a toda concupiscencia terrena. Era nuestro descanso y nuestra fuerza. Hay muchos que piensan que Yo no sufrí humanamente cuando la muerte apagó esa mirada de santo, esa mirada celadora presente en nuestra casa. Si bien, siendo Dios — y, como tal, conociendo la feliz ventura de José — no me apenó su partida (que tras breve estancia en el Limbo le había de abrir el Cielo), como Hombre sí lloré en esa casa privada de su amorosa presencia. Lloré por el amigo desaparecido. ¿Y es que, acaso, no debía haber llorado por este santo mío, en cuyo pecho, de pequeño, yo había dormido, y del cual había recibido amor durante tantos años? Finalmente, pongo ante la consideración de los padres cómo sin contar con una erudición pedagógica José supo hacer de mí un hábil artesano. Apenas llegado Yo a la edad que me permitía manejar las herramientas, no dejándome saborear la ociosidad, me encaminó al trabajo, y se sirvió sobre todo de mi amor por María para estimularme a trabajar: hacer aquellos objetos que le fueran útiles a Mamá. Y así se inculcaba el debido respeto que todo hijo debería tener hacia su madre, y sobre este respetuoso y amoroso fulcro apoyaba la formación del futuro carpintero. ¿Dónde están ahora las familias en que a los pequeños se les haga amar el trabajo como medio para realizar algo grato a los padres? Los hijos, actualmente, son los déspotas de la casa. Se desarrollan indiferentes, duros, mezquinos para con sus padres, a quienes consideran a su servicio, como si fueran sus esclavos; no los aman, y de ellos reciben a su vez poco amor. En efecto, al mismo tiempo que hacéis de vuestros hijos unos déspotas caprichosos, os separáis de ellos desentendiéndoos vergonzosamente. Padres del siglo veinte (ya veintiuno), vuestros hijos son de todos menos vuestros: son de la nodriza, de la institutriz, del colegio, si sois ricos; de los compañeros, de la calle, de las escuelas, si sois pobres. No son vuestros. Vosotras, madres, los generáis, nada más; vosotros, padres hacéis lo mismo. Y, sin embargo, un hijo no es sólo carne; es mente, es corazón, es espíritu. Creed, pues, que nadie tiene más deber y derecho que un padre y una madre de formar esta mente, este corazón, este espíritu. La familia existe, debe existir. No hay teoría o progreso alguno que pueda válidamente demoler esta verdad sin provocar un desastre. Una institución familiar desmoronada sólo puede dar futuros hombres y mujeres cada vez más depravados, causa a su vez de calamidades crecientes. En verdad os digo que sería preferible que no os casarais más, que no engendrarais más sobre esta tierra, en lugar de tener estas familias menos unidas que un clan de monos, estas familias que no son escuela de virtud, de trabajo, de amor, de religión, sino un caos en que todos viven autónomamente, como engranajes desengranados que al final terminan por romperse. Seguid, seguid destruyendo. Ya estáis viendo y sufriendo los frutos de vuestra acción quebrantadora de la forma más santa de la vida social. Seguid, seguid, si queréis. Pero luego no os quejéis de que este mundo sea cada vez más infernal, morada de monstruos devoradores de familias y naciones. ¿Así lo queréis? Pues sea así..."}

Esto lo decía Jesús en 1944... ¿Qué diría ahora, en 2005, con tantísima corrupción, con tantísimos devaneos y divorcios en los matrimonios, que ya ni siquiera se casan sino que se juntan como los animales, y encima en uniones brutales de hombres con hombres y de mujeres con mujeres, queriendo incluso adoptar hijos, en estas uniones abominables (ante los ojos de Dios) para que éstos vivan la corrupción desde pequeños?

 
\chapter*{María, maestra de Jesús, Judas y Santiago.}
\addcontentsline{toc}{chapter}{\normalfont\scshape{María, maestra de Jesús, Judas y Santiago.}}

Veo la habitación (ya en Nazaret) que habitualmente usan como comedor, la misma en que María teje o cose. Es la habitación contigua al taller de José, cuyo diligente trabajar se siente; aquí hay, por el contrario, silencio. María está cosiendo unas piezas de lana alargadas, ciertamente tejidas por Ella, que tienen aproximadamente medio metro de anchas y un poco más del doble de largas; creo entender que están destinadas a ser un manto para José. 

Por la puerta abierta de la parte del huerto- jardín se ve el seto formado por unas matas de enredado ramaje de esas margaritas pequeñas de color azul- violeta que comúnmente se llaman "Marías" o "Cielo estrellado". Desconozco su exacto nombre botánico. Están florecidas. Por tanto, debe ser otoño. De todas formas, los árboles tienen todavía un follaje verde tupido y hermoso, y las abejas, desde dos colmenas adosadas a una pared soleada, vuelan zumbando, danzando y brillando al sol, de una higuera a la vid, de ésta a un granado lleno de redondos frutos, algunos de los cuales han estallado ya por exceso de vigor y muestran sus collares de jugosos rubíes, alineados en el interior de su verde- rojo cofre, de compartimentos amarillos. 

Bajo los árboles. Jesús está jugando con otros dos niños de más o menos su misma edad. Son de pelo rizado, no rubios. Es más, uno de ellos es intensamente moreno: una cabecita de corderito negro que hace resaltar aún más la blancura de la piel de su carita redonda en que se abren dos ojazos de un azul tendente al violáceo; bellísimos. El otro es menos rizado y de un color castaño oscuro, tiene ojos castaños y coloración más morena, aunque con una tonalidad rosácea en los carrillos. Jesús, con su cabecita rubia, entre los otros dos, oscuros, parece ya aureolado de fulgor. Están jugando en concordia con unos pequeños carritos en los que hay... distintas mercancías:: piedrecitas, virutas, pedacitos de madera. Eran mercaderes, sin duda, y Jesús era el que compraba para su Mamá, a la que le lleva ora una cosa, ora otra; María, sonriendo, acepta los objetos comprados. 

Pero después de un poco el juego cambia. Uno de los dos niños propone: 

- ¿Por qué no hacemos el Éxodo a través de Egipto? Jesús es Moisés; yo, Aarón; tú... María. 

- ¡Pero si yo soy chico! 

- ¡No importa! ¿Qué más da? Tú eres María y bailas ante el becerro de oro, que será aquella colmena. 

Yo no bailo. Soy un hombre y no quiero ser una mujer; soy un fiel, y no quiero bailar ante el ídolo. 

Jesús interviene diciendo: 

Pues no hacemos este pasaje. Podemos hacer ese otro de cuando le eligen a Josué sucesor de Moisés. Así no está ese feo pecado de idolatría y Judas estará contento de ser hombre y sucesor mío. ¿Verdad que estás contento? 

Sí, Jesús. Pero entonces Tú tienes que morir, porque Moisés muere después. No quiero que Tú mueras; Tú, que siempre me quieres tanto». 

Todos morimos... Pero Yo antes de morir bendeciré a Israel, y, dado que aquí sólo estáis vosotros, en vosotros bendeciré a todo Israel. 

Es aceptada la propuesta. Pero luego surge una cuestión: si el pueblo de Israel, después de tanto caminar; llevaba o no los carros que tenía al salir de Egipto. Hay disparidad de ideas. Se recurre a María. 

Mamá, Yo digo que los israelitas tenían todavía los carros. Santiago dice que no. Judas no sabe a quién de los dos dar la razón. ¿Tú sabes si los tenían? 

Sí, Hijo. El pueblo nómada tenía todavía sus carros. En los descansos los reparaban. Montaban en ellos los más débiles. Se cargaba en ellos aquellos víveres o cosas que un pueblo tan numeroso necesitaba. Todas las demás cosas iban en los carros, menos el Arca, que la llevaban a mano. 

La cuestión está resuelta. 

Los niños van al final del huerto y, desde allí, entonando salmos, vienen hacia la casa. Jesús viene delante cantando salmos con su vocecita de plata. Detrás de Él vienen Judas y Santiago portando un pequeño carrito elevado al rango de Tabernáculo. Pero, dado que además de a Aarón y a Josué tienen que representar también al pueblo, se han quitado los cinturones y se han atado al pie los otros carros en miniatura, y así caminan, serios como si fueran verdaderos actores. 

Hacen el recorrido de la pérgola, pasan por delante de la puerta de la habitación donde está María, y Jesús dice: 

Mamá, pasa el Arca, salúdala. 

María se levanta sonriendo y se inclina ante su Hijo que, radiante, pasa, aureolado de sol. 

Acto seguido Jesús trepa un poco por el lado del monte que limita la casa, o mejor, el huerto. Arriba de la gruta, erguido, dirige unas palabras a... Israel. Manifiesta los preceptos y las promesas de Dios, señala a Josué como caudillo, le llama a sí — Judas también sube arriba de la peña —, le anima y le bendice. Luego pide una... tabla (es la hoja ancha de una higuera) y escribe el cántico, y lo lee; no todo, pero sí una buena parte de él, y al hacerlo da la impresión de que realmente lo estuviera leyendo en la hoja. A continuación se despide de Josué, el cual le abraza llorando, y sube más arriba, justo hasta el borde de la peña. Allí bendice a todo Israel, es decir, a los dos niños que están prosternados en tierra, y luego se acuesta sobre la corta hierbecilla, cierra los ojos y... muere. 

María se había quedado, sonriente, a la puerta, y, cuando lo ve echado en el suelo, rígido, grita: 

- ¡Jesús! ¡Jesús! ¡Levántate! ¡No estés así! ¡Mamá no quiere verte muerto! 

Jesús se levanta del suelo, sonríe, y va hacia Ella corriendo, y la besa. Se acercan lo mismo Santiago y Judas, y María los acaricia también. 

- ¿Cómo puede acordarse Jesús de ese cántico tan largo y difícil y de todas esas bendiciones? – pregunta Santiago. 

María sonríe y responde sencillamente: 

Tiene una memoria muy buena y está muy atento cuando yo leo. 

Yo, en la escuela, estoy atento, pero con tanta lamentación me viene el sueño... Entonces, ¿no voy a aprender nunca? 

Aprenderás. Tranquilo. 

Llaman a la puerta. José atraviesa con paso rápido huerto y habitación, y abre. 

- ¡La paz sea con vosotros, Alfeo y María! 

Y con vosotros. Paz y bendición. 

Es el hermano de José con su mujer. Un rústico carro tirado por un robusto burro está parado en la calle. - ¿Habéis tenido buen viaje? 

Sí, bueno. ¿Y los niños? 

Están en el huerto con María. 

Ya los niños venían corriendo a saludar a su mamá. También María está viniendo, trayendo a Jesús de la mano. Las dos cuñadas se besan. 

- ¿Se han portado bien? 

Sí, muy bien, y han sido muy cariñosos. ¿La familia está toda bien? 

Todos están bien. Nos han dado recuerdos para vosotros. De Caná os mandan muchos regalos: uvas, manzanas, queso, huevos, miel. Y... José, he encontrado exactamente lo que tú querías para Jesús. Está en el carro, en aquella cesta redonda. 

La mujer de Alfeo, sonriendo, se curva hacia Jesús, que la está mirando con unos ojos maravillados, abiertísimos; y le besa en esos dos pedacitos de azul y dice: 

- ¿Sabes lo que he traído para ti? Adivina. 

Jesús piensa, pero no adivina. Probablemente lo hace a propósito, para que José tenga la alegría de dar una sorpresa. En efecto, José entra trayendo consigo una cesta redonda. La deposita en el suelo a los pies de Jesús, desata la cuerda que está sujetando la tapadera, la levanta... y una ovejita toda blanca, un verdadero copo de espuma, aparece, dormida sobre un heno muy limpio. 

- ¡Oh! - exclama Jesús con estupor y beatitud, mientras hace ademán de echarse hacia el animalito, pero... no, se vuelve y corre a donde José, que aún está agachado, y lo abraza y lo besa dándole las gracias. 

Los primitos miran con admiración al animalito, que ahora está despierto y alza su rosado morrito y bala buscando a su mamá. Sacan de la cesta a la ovejita y le ofrecen un manojo de tréboles. Ella come, mirando a su alrededor con sus mansos ojos. Jesús repite una y otra vez: - ¡Para mí! ¡Para mí! ¡Padre, gracias! 

- ¿Te gusta mucho? 

- ¡Oh, mucho! Blanca, limpia... una cordera... ¡oh! - y le echa sus bracitos al cuello a la ovejita, pone su cabeza rubia sobre la cabecita, y se queda así, satisfecho. 

También os he traído a vosotros otras dos - dice Alfeo a sus hijos - Pero son de color oscuro. Vosotros no sois ordenados como lo es Jesús y, si hubieran sido blancas, las tendríais mal. Serán vuestro rebaño, las tendréis juntas, y así vosotros dos, golfos, no estaréis ya más por ahí por las calles tirando piedras. 

Los dos niños van corriendo al carro para ver a estas otras dos ovejas, más negras que blancas. 

Jesús por su parte se ha quedado con la suya. La lleva al huerto, le da de beber, y el animalito le sigue como si lo conociera desde siempre. Jesús la llama. Le pone por nombre «Nieve». Ella responde balando jubilosa. 

Los llegados ya están sentados a la mesa. María les sirve pan, aceitunas y queso. Trae también un ánfora de sidra o de agua de manzanas, no lo sé; veo que es de un color dorado muy claro. 

Los niños juegan con los tres animales y ellos se ponen a conversar. Jesús quiere que estén las tres ovejas, para darles a las otras también agua y un nombre: 

La tuya, Judas, se llamará "Estrella", por el signo ese que tiene en la frente; y la tuya "Llama", porque tiene un color como el de ciertas llamas de brezo lánguido. 

De acuerdo. 

Espero haber resuelto así la historia de las peleas entre muchachos – dice Alfeo - Tu idea, José, ha sido la que me ha iluminado. Dije: "Mi hermano quiere una cordera para Jesús, para que juegue un poco. Yo me llevo dos para esos golfos, para que estén un poco tranquilos y no tener siempre problemas con otros padres por cabezas o rodillas rotas. Un poco la escuela y un poco las ovejas, lograré tenerlos quietos. Por cierto, este año tendrás que mandar tú también a Jesús a la escuela. Ya es tiempo. 

Yo no voy a mandarlo jamás a Jesús a la escuela - dice María con tono resoluto. Resulta insólito oírla hablar así, y además antes que José (!). 

- ¿Por qué? El Niño tiene que aprender, para que a su debido tiempo sea capaz de afrontar el examen de la mayoría de edad...». 

El Niño sabrá; pero no irá a la escuela. Está decidido. 

Pues serías la única que actuara así en Israel. 

Pues seré la única, pero actuaré así. ¿No es verdad, José? 

Así es; Jesús no tiene necesidad de ir a la escuela. María se ha formado en el Templo y es una verdadera doctora en el conocimiento de la Ley. Será su Maestra. Es también mi deseo. 

Le estáis mimando demasiado al muchacho. 

Eso no puedes decirlo. Es el mejor de Nazaret. ¿Lo has visto alguna vez llorar o cogerse alguna pataleta o negarse a obedecer o faltar al respeto? 

No. Pero un día será así si lo seguís mimando. 

Tener al lado a los hijos no es mimarlos; es quererlos, con mente cabal y buen corazón. Nosotros amamos así a nuestro Jesús, y, dado que María es una mujer más instruida que el maestro, será Ella la Maestra de Jesús. 

Y cuando sea hombre, tu Jesús será una mujercita temerosa hasta de las moscas. 

No lo será. María es una mujer fuerte y sabe educarle virilmente; y yo no soy ningún mezquino, y sé dar ejemplos viriles. Jesús es un niño sin defectos físicos ni morales. Por tanto se desarrollará recto y fuerte en el cuerpo y en el espíritu. Estate seguro de esto, Alfeo. No dejará mal a la familia. Y, además, ya lo he decidido y es suficiente. 

Lo habrá decidido María. Tú sólo.... 

- ¿Y si así fuera? ¿No es acaso bonito que dos personas que se aman estén en la disposición de tener el mismo pensamiento y la misma voluntad, porque mutuamente abrazan el deseo del otro y lo hacen propio? Si María desease estupideces, yo le diría que no, pero lo que pide son cosas llenas de sabiduría, y yo las apruebo y hago mías. Nosotros nos amamos como el primer día... y lo seguiremos haciendo mientras vivamos, ¿verdad, María? 

Sí, José. Y aún en el caso — y ojalá no suceda jamás — de que uno de los dos muriese y el otro no, nos seguiríamos amando. 

José le acaricia a María la cabeza, como si fuera una hija pequeña, y Ella a su vez lo mira con ojos serenos y amorosos. 

La cuñada interviene diciendo: 

Tenéis realmente razón. ¡Si yo fuera capaz de enseñar!.. En la escuela nuestros hijos aprenden el bien y el mal; en casa, sólo el bien. Pero yo no sé hacerlo... Si María... 

- ¿Qué quieres, cuñada? Habla libremente. Tú sabes que te quiero y que me siento contenta cada vez que puedo satisfacerte en algo. 

No, yo lo que pensaba... era... Santiago y Judas son sólo un poco mayores que Jesús. Ya van a la escuela... ¡pero, para lo que saben!.. Por el contrario, Jesús ya sabe muy bien la Ley... Yo quisiera... bueno, ¿si te pidiera que los tuvieras también a ellos cuando enseñas a Jesús? Creo que ganarían en bondad y en conocimientos. Al fin y al cabo son primos y sería justo que se quisieran como hermanos.., ¡Qué feliz me sentiría! 

Si José y tu marido quieren, yo por mí estoy dispuesta. Hablar para uno o para tres es igual. Repasar la Escritura es motivo de gozo. Que vengan. 

Los tres niños, que habían entrado despacito, han oído estas palabras y están a la espera del veredicto. 

Te harán desesperar, María - dice Alfeo. 

- ¡No! Conmigo siempre se portan bien. ¿Verdad que os vais a portar bien si yo os enseño? 

Los dos niños acuden a su lado corriendo, uno a la derecha, el otro a la izquierda. Le ponen los brazos en torno a los hombros apoyando en ellos sus cabecitas, y hacen promesas de todo el bien posible. 

Déjalos que prueben, Alfeo, y déjame probar también a mí. Yo creo que no quedarás descontento de la prueba. Que vengan todos los días desde la hora sexta hasta la tarde. Será suficiente, créelo. Conozco el arte de enseñar sin cansar. A los niños hay que tenerlos cautivados y distraídos al mismo tiempo. Hay que comprenderlos, amarlos y ser amados para conseguir de ellos. Y vosotros me queréis, ¿no? 

La respuesta es dos fuertes besos. 

- ¿Lo ves? 

Ya lo veo. Sólo me queda decirte: "Gracias". Y Jesús ¿qué va a decir cuando vea a su mamá entretenida en otros? ¿Tú qué dices, Jesús? 

Yo digo: "Bienaventurados los que le prestan atención y levantan su morada junto a la de Ella". Como con la Sabiduría, dichoso aquel que es amigo de mi Madre. Me gozo viendo que aquéllos a quienes amo son sus amigos. - ¿Quién pone tales palabras en labios de este Niño? – pregunta Alfeo asombrado. 

Nadie, hermano, nadie de este mundo. 

La visión cesa en este momento. 

Dice Jesús: 
\emph{- Y María fue Maestra mía, de Santiago y de Judas. Y éste es el motivo por el cual hubo entre nosotros amor fraternal, además de por el parentesco; por la ciencia y por haber crecido juntos, como tres sarmientos con un único palo como soporte: la Madre mía. Que en verdad mi dulce Madre era doctora como nadie en Israel. Sede de la Sabiduría, de la verdadera Sabiduría, Ella nos instruyó para el mundo y para el Cielo. Digo que "nos instruyó", porque yo fui alumno suyo no en modo distinto de mis primos. Y el "sello" colocado sobre el misterio de Dios fue mantenido contra las pesquisas de Satanás, mantenido bajo la apariencia de una vida común.}

\chapter*{Preparativos para la mayoría de edad de Jesús \\ \normalfont\normalsize\textit{y salida de Nazaret.}}
\addcontentsline{toc}{chapter}{\normalfont\scshape{Preparativos para la mayoría de edad de Jesús}}
  
Veo a María encorvada hacia una batea, o, mejor, un barreño de barro, mezclando algo que despide vapor en el aire frío y sereno que llena el huerto de Nazaret. 

Debe ser pleno invierno. Lo deduzco del hecho de que, menos olivos, todos los árboles están deshojados y exhaustos. Arriba, un cielo tersísimo y un sol que aun siendo radiante no logra templar la tramontana que hay, que sopla y hace chocar unas con otras las desnudas ramas u ondular las ramitas entre grises y verdes de los olivos. 

La Virgen María lleva un vestido tupido de color marrón casi negro, que la cubre enteramente. Se ha colocado delante una tela basta, a manera de mandil, para protegerlo. Saca de la tina el palo conque estaba removiendo el contenido. Veo que del palo caen gotas de un bonito color bermejo. María observa, se moja un dedo con las gotas que caen, y prueba el color en el mandil. Parece satisfecha. 

Entra en la casa y vuelve a salir con muchas madejas de blanquísima lana, y las echa, una a una, en la tina, con paciencia y cautela. 

Mientras está haciendo esto, entra su cuñada — que viene del taller de José — María de Alfeo. Se saludan. Se hablan. - ¿Queda bien? - pregunta María de Alfeo. 

Espero que sí. 

Me aseguró esa gentil que se trata de la misma tinta y del mismo sistema de teñir que utilizan en Roma. Si me lo dio es porque se trataba de ti y por haber hecho aquellas labores. Ella dice que no hay quien borde como tú, ni siquiera en Roma. Debes haber perdido la vista haciéndolas... 

María sonríe y hace un movimiento de cabeza como diciendo: 

- ¡Son cosas sin importancia! 

La cuñada mira las últimas madejas de lana antes de pasárselas a María, y exclama: 

- ¡Qué bien las has hilado! Son hilos tan finos y uniformes que parecen cabellos. Tú todo lo haces bien... y ¡qué rápida! 

¿Estas últimas serán más claras? 

Sí, para la túnica; el manto es más oscuro. 

Las dos mujeres se ponen a trabajar juntas: primero, en la tina; luego sacan las madejas, ya de un lindo color purpúreo, y corren veloces a sumergirlas en el agua helada que llena el pilón, colocado bajo la fina vena que mana y cae produciendo notas de risitas apenas perceptibles. Aclaran una y otra vez y luego extienden las madejas sobre unas cañas aseguradas a los árboles de unas ramas a otras. 

Con este viento se secarán bien y rápido - dice la cuñada. 

Vamos donde José. Hay lumbre. Debes estar helada - dice María Stma. - Has sido buena conmigo ayudándome. He acabado pronto y con menos esfuerzo. Gracias. 

- ¡Oh! ¡María! ¿Qué no haría yo por ti! Estar a tu lado es motivo siempre de gozo. Además... todo este trabajo es por Jesús. Y, ¡es tan encantador tu Hijo!.. Ayudándote a ti para la celebración de su mayoría de edad, me parecerá sentirme yo también madre suya. 

Y las dos mujeres entran en el taller, lleno de ese olor a madera cepillada que es típico de los talleres de carpintero. 

Y la visión sufre una interrupción... para continuar después, en el momento de la partida de Jesús para Jerusalén a los doce años. 

Su figura es bellísima. Está tan desarrollado, que parece un hermano menor de su joven Madre (ya le llega a María a los hombros); su cabeza, rubia y ensortijada, de melena hasta más abajo de las orejas — ya no tiene el pelo corto, como en los primeros años de su vida — parece un casco de oro repleto de relucientes bucles laborados. 

Va vestido de rojo, un bonito rojo de rubí claro: una túnica que le llega hasta los tobillos dejando ver sólo los pies, calzados con sandalias; es una túnica suelta, de mangas largas y amplias. En el cuello, en los bordes de las mangas y en la base, grecas tejidas con colores sobrepuestos, muy bonitas... 

Veo el momento en que Jesús entra, acompañado de su Madre, en el — digámoslo así — comedor de la casa de Nazaret. 

Jesús tiene doce años. Es un muchacho alto, bien formado, fuerte, aunque no gordo; parece, por su complexión, más adulto de lo que realmente es; le llega ya a su Madre a la altura de los hombros. Su rostro es todavía redondeado y rosado, es todavía el rostro de Jesús niño, rostro que, con el paso del tiempo, con la edad juvenil y viril, se habrá de alargar, y tomará un cromatismo indefinido, una tonalidad como la de ciertos alabastros delicados que tienden apenas al amarillo- rosa. 

Sus ojos — también sus ojos — son todavía ojos de niño. Son grandes y miran bien abiertos, con una chispa de alegría perdida en la seriedad de la mirada. Pasado el tiempo, ya no estarán tan abiertos... Los párpados descenderán hasta medio cerrar los ojos, para velarle al Puro y Santo el exceso de mal que hay en el mundo. Solamente en los momentos de los milagros, o cuando ponga en fuga a los demonios o a la muerte, o para curar las enfermedades y los pecados; solamente entonces los abrirá, y centellearán, aún más que ahora. Pero, ni siquiera entonces tendrán esta chispa de alegría mezclada con la seriedad... La muerte y el pecado estarán cada vez más cerca y más presentes, y, con ambos, el conocimiento — con su faceta humana — de la inutilidad del sacrificio a causa de la voluntad contraria del hombre. Sólo en rarísimos momentos de alegría, por estar con los redimidos, y especialmente con los puros — generalmente niños — brillarán de júbilo estos ojos santos y buenos. 

Ahora, estando con su Madre, en su casa, y con San José frente a Él, sonriéndole con amor, y con esos primitos suyos que le admiran, y con su tía, María de Alfeo, que le está acariciando, se siente feliz. Mi Jesús tiene necesidad de amor para sentirse feliz, y en este momento lo tiene. 

Está vestido con una túnica suelta, de lana, de color rojo rubí claro, suave, perfectamente tejida, fina y compacta al mismo tiempo. En el cuello, por la parte de delante, en la base de las mangas largas y amplias, y en la base de la túnica, que llega hasta abajo dejando apenas ver los pies calzados con sandalias nuevas y bien hechas — no las usuales suelas sujetas al pie con unas correas —, tiene una greca, no bordada, sino tejida en un color más oscuro sobre el color rubí de la túnica. Deduzco que debe ser obra de su Madre, porque la cuñada la admira y alaba. 

Su bonito pelo rubio tiene ya una tonalidad más cargada que cuando era un niño pequeño, con reflejos cobrizos en los aros de los bucles que terminan bajo las orejas; ya no son esos ricitos cortos y vaporosos de la infancia, pero tampoco es la melena de la edad adulta, ondulada, que termina a la altura de los hombros en delicada forma tubular; de todas maneras ya tiende a ésta, en color y forma. 

He aquí a nuestro Hijo - dice María levantando con su mano derecha la izquierda de Jesús. Parece como si se lo quisiera presentar a todos y confirmar la paternidad del Justo, que sonríe. Y añade: - Bendícelo, José, antes de partir para Jerusalén. No fue necesaria la bendición para su inicio en la escuela, primer paso en la vida; hazlo ahora que Él va al Templo para ser declarado mayor de edad. Y bendíceme también a mí. Tu bendición... (María contiene el llanto) lo fortalecerá a Él y me dará fuerza a mí para separarme de Él un poco más... 

María, Jesús será siempre tuyo. La fórmula no lesionará nuestras mutuas relaciones. Yo no te voy a disputar a este Hijo, amado nuestro. Ninguno merece como tú el guiarlo en la vida, ¡oh Santa mía! 

María se inclina, toma la mano de José y la besa: es la esposa, y ¡qué respetuosa y amante de su consorte! 

José acoge este signo de respeto y de amor con dignidad, mas luego alza esa misma mano y la deposita sobre la cabeza de su Esposa diciéndole: 

- Sí. Te bendigo, Bendita, y a Jesús contigo. Venid, mis únicos tesoros, honor y finalidad míos - José se muestra solemne: con los brazos extendidos y las palmas vueltas hacia abajo sobre las dos cabezas inclinadas, igualmente rubias y santas, pronuncia la bendición: «El Señor os guarde y os bendiga, tenga misericordia de vosotros y os dé paz. El Señor os dé su bendición». Y luego dice: - En marcha. La hora es propicia para el viaje. 

María coge un manto, amplio, de color granate oscuro, y en elegantes pliegues lo dispone sobre el cuerpo de su Hijo. ¡Y cómo lo acaricia al hacerlo! 

Salen. Cierran. Se ponen en marcha. Otros peregrinos van en la misma dirección. Fuera del pueblo, las mujeres se separan de los hombres. Los niños van con quien quieren. Jesús se queda con su Madre. 

Los peregrinos caminan — la mayoría entonando salmos — por las campiñas llenas de hermosura en el más jubiloso tiempo de primavera. Frescos prados, tiernos cereales, frescos follajes en los árboles poco ha florecidos; hombres cantando por los campos y por los caminos, cantos de pájaros en celo entre las frondas; límpidos arroyos, espejo de las flores de las orillas; corderitos saltarines al lado de sus madres... Paz y alegría bajo el más hermoso cielo de abril. La visión cesa así. 

\tableofcontents

\end{document}