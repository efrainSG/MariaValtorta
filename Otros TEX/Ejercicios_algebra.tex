\documentclass[11pt,twocolumn]{article} % use larger type; default would be 10pt
\usepackage[utf8]{inputenc} % set input encoding (not needed with XeLaTeX)
%%% PAGE DIMENSIONS
\usepackage{geometry} % to change the page dimensions
\geometry{letterpaper} % or letterpaper (US) or a5paper or....
\geometry{margin=2cm} % for example, change the margins to 2 inches all round
%   read geometry.pdf for detailed page layout information
\usepackage{graphicx} % support the \includegraphics command and options
% \usepackage[parfill]{parskip} % Activate to begin paragraphs with an empty line rather than an indent

%%% PACKAGES
\usepackage{booktabs} % for much better looking tables
\usepackage{array} % for better arrays (eg matrices) in maths
\usepackage{paralist} % very flexible & customisable lists (eg. enumerate/itemize, etc.)
\usepackage{verbatim} % adds environment for commenting out blocks of text & for better verbatim
\usepackage[spanish,activeacute]{babel}
\usepackage{subfig} % make it possible to include more than one captioned figure/table in a single float
% These packages are all incorporated in the memoir class to one degree or another...

%%% HEADERS & FOOTERS
\usepackage{fancyhdr} % This should be set AFTER setting up the page geometry
\pagestyle{fancy} % options: empty , plain , fancy
\renewcommand{\headrulewidth}{0pt} % customise the layout...
\lhead{}\chead{}\rhead{}
\lfoot{}\cfoot{\thepage}\rfoot{}

%%% SECTION TITLE APPEARANCE
\usepackage{sectsty}
\allsectionsfont{\sffamily\mdseries\upshape} % (See the fntguide.pdf for font help)
% (This matches ConTeXt defaults)

%%% ToC (table of contents) APPEARANCE
\usepackage[nottoc,notlof,notlot]{tocbibind} % Put the bibliography in the ToC
\usepackage[titles,subfigure]{tocloft} % Alter the style of the Table of Contents
\renewcommand{\cftsecfont}{\rmfamily\mdseries\upshape}
\renewcommand{\cftsecpagefont}{\rmfamily\mdseries\upshape} % No bold!

%%% The "real" document content comes below...
\title{Primera colección de ejercicios algebraicos}
\author{M. C. Y T. E. Efraín Serna Gracia}
%\date{} % Activate to display a given date or no date (if empty),

\begin{document}
\maketitle
\tableofcontents
\section*{Introducción}
Estos ejercicios son bastante simples, por lo que sirven bien para asimilar los procesos que te ayudan a resolverlos. Procura hacerlos tomando completa conciencia sobre la secuencia de pasos que realizas para resolverlos.\\

Cuando hayas concluido la colección, intenta resolverlos busando un camino alterno, recuerda que las matemáticas son un juego y el secreto es saber aplicar las reglas para hacer tus jugadas y ganar. 

\section{Fundamentos}
\begin{description}
\item[Valor Absoluto] Existe una definición matemática, pero nosotros vamos a conceptualizarlo de forma más simple: \emph{El valor absoluto de un número es ese mismo número pero SIN signo} y se representa como el número entre dos barras. Por ejemplo:
\begin{itemize}
\item $|1| = 1$
\item $|-4| = 4$
\end{itemize}
\end{description}
Ahora determina el valor absoluto de los siguientes números:
\begin{enumerate}
\item $|-3|=$
\item $|15|=$
\item $|-150|=$
\item $|-25|=$
\item $|45|=$
\end{enumerate}

\subsection{Leyes}
Estas leyes memorízalas, asimílalas hasta que rápidamente puedas responder correctamente. A veces un error de signo puede costarte muy caro, tal vez tengas que repetir desde el inicio. En algunos de mis videos verás a lo que me refiero. A propósito no los re-edité, para que puedas ver esos efectos.

\subsubsection{De los signos}
\begin{itemize}
\item En sumas y restas, si ambos términos tienen el mismo signo, este prevalece
\item En sumas y restas, si ambos tienen signo distinto, al restar del valor absoluto más alto el valor absoluto más pequeño, el signo que prevalece es el que corresponde al término con valor absoluto más grande.
\item En productos y cocientes, si ambos signos son iguales, el resultado es positivo, mientras que si son distintos el resultado es negativo.
\end{itemize}
\begin{tabular}{cc}
\begin{tabular}{|c|c|c|}
\hline
$\times$ & $(+)$ & $(-)$\\
\hline
$(+)$ & $(+)$ & $(-)$\\
\hline
$(-)$ & $(-)$ & $(+)$\\
\hline
\end{tabular} &
\begin{tabular}{|c|c|c|}
\hline
$\div$ & $(+)$ & $(-)$\\
\hline
$(+)$ & $(+)$ & $(-)$\\
\hline
$(-)$ & $(-)$ & $(+)$\\
\hline
\end{tabular}\end{tabular}\\

Ahora resuelve los siguientes ejercicios de multiplicación:
\begin{enumerate}
\item $(-4)(3)=$
\item $(5)(8)=$
\item $(-14)(13)=$
\item $(43)(-3)=$
\item $(-54)(-12)=$
\item $(-4)(3)(-11)=$
\item $(20)(3)(-7)=$
\item $(-4)(-3)(-5)=$
\item $(4)(-32)(5)=$
\item $(-4)(3)(-4)=$
\end{enumerate}

Ahora resuelve los siguientes ejercicios de división:
\begin{enumerate}
\item $(-24)\div(3)=$
\item $(56)\div(8)=$
\item $(-143)\div(13)=$
\item $(42)\div(-3)=$
\item $(-84)\div(-12)=$
\item $(-45)\div(3)=$
\item $(28)\div(-7)=$
\item $(-45)\div(-5)=$
\end{enumerate}

\subsubsection{De los exponentes}
Para las siguientes leyes, la $x$ representa a un número cualquiera, positivo o negativo, entero o fraccionario, racional o irracional, es decir: \emph{Real}, mientras que las letras $n$ y $m$ representan a números enteros, positivos o negativos.
\begin{tabular}{|lcl|c|lcl|}
\hline
$x^0$     & $=$ & $1$   & & $x^1$ & $=$ & $x$ \\
$x \cdot x$ & $=$ & $x^2$ & & $x^n$ & $=$ & $\underbrace{x\cdots x}_{n-veces}$ \\
$(x^m)(x^n)$ & $=$ & $x^{m+n}$ & & $\frac{1}{x}$ & $=$ & $x^{-1}$ \\
$\frac{x^m}{x^n}$ & $=$ & $x^{m-n}$ & & $(x^m)^n$ & $=$ & $x^{mn}$ \\
$\sqrt{x}$ & $=$ & $x^{1/2}$ & & $\sqrt[n]{x}$ & $=$ & $x^{1/n}$ \\
$\sqrt{x^n}$ & $=$ & $x^{n/2}$ & & $\sqrt[n]{x^m}$ & $=$ & $x^{m/n}$ \\
\hline
\end{tabular}\\

Ahora resuelve los siguientes ejercicios de exponentes escribiéndo los resultados en términos de las leyes de exponentes y luego su resultado final:
\begin{enumerate}
\item $4^0=$
\item $10^2=$
\item $(3^2)(3^7)=$
\item $\frac{10^5}{10^3}=$
\item $\sqrt{144}=$
\item $\sqrt{2^20}=$
\item $5^4=$
\item $8^-1=$
\item $(3^2)^3=$
\item $\sqrt[3]{1000}=$
\item $\sqrt[4]{144^2}=$
\item $4^0=$
\item $10^2=$
\item $(3^2)(3^7)=$
\item $\frac{10^5}{10^3}=$
\item $\sqrt{144}=$
\item $\sqrt{2^20}=$
\item $5^4=$
\item $8^-1=$
\item $(3^2)^3=$
\item $\sqrt[3]{1000}=$
\item $\sqrt[4]{144^2}=$
\end{enumerate}

Antes de proceder con las operaciones, debes recordar que existe una regla para cuando se combinan diferentes operaciones en una misma expresión matemática, a esta regla le llamamos \emph{Jerarquía Operacional} o \emph{Jerarquía de Operaciones}, y establece el orden en el que se resuelven las operaciones dentro de las expresiones a como sigue:
\begin{enumerate}
\item Resuelve funciones
\item Resuelve lo que está dentro de los paréntesis
\item Resuelve potencias y raíces
\item Resuelve Productos y cocientes
\item Resuelve Sumas y restas
\end{enumerate}
\subsection{Suma y resta}
\begin{enumerate}
\item $3x^2 + 2y^3 + 4x^2 + 7y^3 =$
\item $15ab^2 + 3bc - 8ab^2 -bc=$
\item $-8x^2y - 6xy^2 - 10x^2y - 4xy^2 =$
\item $ 10d^3e - 5mn^2 - 12d^3e + 7mn^2 =$
\end{enumerate}
\subsection{Producto}
\begin{enumerate}
\item $(3xy)(5x) =$
\item $(2ab)(4a + 5b)=$
\item $(3a^2b)(3b + 7ab^2 - 5ab) =$
\item $(5x + 3w)(4x - 5w) =$
\item $(4a^2b - 2ab^2)(3a -2ab +4b)=$
\end{enumerate}
\subsection{Cociente}
\begin{enumerate}
\item $15x^2y \div 5xy=$
\item $(8a^2b+10ab^2)\div(2b)=$
\item $(9a^2b^2+21a^3b^3-15a^3b^2)\div(3a^2b^2)=$
\item $(20x^2-13xw-15w^2)\div(4x-5w)=$
\item $(-8a^3b^2+12a^3b-4a^2b^3+10a^2b^2-8ab^3)\div(4a^2b-2ab^2)=$
\end{enumerate}
\section{Productos Notables}
Recuerda esto hacerlo por \emph{simple observación}
\begin{enumerate}
\item $(3a^2b)(4ab^2+2ab-5a^2b)=$
\item $(x+4)(x+4)=$
\item $(x+4y)(x+3y)=$
\item $(3x^2y-4y^2)(3x^2y-4y^2)=$
\item $(x^2+3y^2)(x^2-2y^2)=$
\item $(3a^2+4b^2)(3a^2-4b^2)=$
\end{enumerate}
\section{Factorización}
Estos también se hacen por \emph{simple inspección}
\begin{enumerate}
\item $14x^2y^3+8x^3y^2-28x^2y^2=$
\item $x^2 + 6x + 9 =$
\item $4a^4 - 20a^2b + 25b^2 =$
\item $x^2 +7x + 12=$
\item $25x^4 - 15x^2 - 28=$
\item $25x^2-16y^2=$
\end{enumerate}
\section{Ecuaciones de primer grado}
Recuerda que para verificar la solución de una ecuación tienes que sustituir la incógnita de la ecuación donde aparezca por el valor de la solución que encontraste.
\begin{enumerate}
\item $4x+5=8x-7$
\item $5x-4=4x$
\item $8y-3=6y+9$
\end{enumerate}
\section{Sistemas de ecuaciones de primer grado}
Recuerda que las soluciones deben satisfacer todas las ecuaciones del sistema.
\subsection{Por Sustitución}
\begin{equation}
\left\lbrace
\begin{array}{rcrcr}
3x & + & 2y & = & 22\\
4x & - & 4y & = & -4
\end{array}
\right.
\end{equation}

\begin{equation}
\left\lbrace
\begin{array}{rcrcr}
5x & - & 3y & = & -1\\
8x & + & 4y & = & 16
\end{array}
\right.
\end{equation}
\subsection{Por Igualación}
\begin{equation}
\left\lbrace
\begin{array}{rcrcr}
-6x & - & 6y & = & -48\\
3x & - & 8y & = & -31
\end{array}
\right.
\end{equation}

\begin{equation}
\left\lbrace
\begin{array}{rcrcr}
14x & + & 2y & = & 24\\
-8x & - & 3y & = & -10
\end{array}
\right.
\end{equation}
\subsection{Por Eliminación}
\begin{equation}
\left\lbrace
\begin{array}{rcrcr}
3x & + & 2y & = & -13\\
4x & - & 4y & = & -4
\end{array}
\right.
\end{equation}

\begin{equation}
\left\lbrace
\begin{array}{rcrcr}
-6x & - & 6y & = & -18\\
3x & - & 8y & = & -13
\end{array}
\right.
\end{equation}
\section{Ecuaciones de segundo grado}
Puedes resolver por lo menos de dos formas: Factorizando y por fórmula general. Intenta resolverlas de todas las maneras posibles.
\begin{enumerate}
\item $12x^2-10x=-6$
\item $5x^2=-11x+12$
\item $9x^2+2x-10=-x+10$
\end{enumerate}
\section{Inecuaciones}
\begin{enumerate}
\item $4x+3\leq 5$
\item $6x-4 < 4x+3$
\item $10>2x-4>x-5$
\end{enumerate}
\end{document}
