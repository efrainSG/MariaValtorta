\documentclass[12pt,spanish,lettersize,twocolumn]{article}
\usepackage[latin1]{inputenc}
\usepackage[spanish]{babel}
\usepackage[dvips]{graphicx}
\usepackage[usenames,dvipsnames]{xcolor}
\usepackage{mathrsfs}
\usepackage{amssymb}
\usepackage{amsmath}
\usepackage[makeroom]{cancel}
\usepackage{vmargin}
\usepackage{chngcntr}
\usepackage{graphics}
\usepackage{epstopdf}
\counterwithin*{equation}{section}
\counterwithin*{equation}{subsection}

\title{%
\color{Maroon}Instrumentaci\'on\\
       \large Convertidores A-D y D-A\\
}
\author{Efra\'in Serna Gracia}
\date{\color{gray}\today}
\begin{document}
\maketitle
\section{Convertidor A-D}
Existen diversas formas de hacer un convertidor anal\'ogico-digital. Unos m\'as veloces que otros.\\
\subsection{Rampa}
Recibe una entrada anal\'ogica que llega a un comparador. El valor de referencia del comparador viene de un DAC que tiene como valor de entrada una se\~nal proveniente de un contador binario habilitado por un reloj. Cuando el valor de la se\~nal anal\'ogica es menor que el valor de referencia, detinene el contador que en ese momento tiene el valor en forma digital, lo env\'ia a una salida, un arreglo, y reinicia el contador. 
\includegraphics[scale=0.75]{./ADC}
\section{Convertidor D-A}
Permite realizar la conversi\'on de una se\~nal digital en anal\'ogica. Una de las formas es mediante el uso de resistencias ponderadas m\'as un Amplificador operacional que sumar\'ia las diferentes corrientes para obtener una sola de la que resultar\'ia un voltaje. Dado que la se\~nal se hace pasar por diferentes resistencias seg\'un se requiera, se hace necesario utilizar alg\'un tipo de interruptor, y como solo puede o no pasar se\~nal por cada resistencia, y para mantener la ponderaci\'on, \'esta se har\'a en  potencias de 2.\\
El siguiente diagrama sirve de ejemplo para un convertidor D-A sencillo.
\includegraphics[scale=1]{./CDA}
\subsection{An\'alisis}
Consideramos todos los interruptores cerrados, con lo cual la se\~nal de salida ser\'a el m\'aximo.
\begin{eqnarray}
i_1+i_2+i_3=i_f\\
\left.
\begin{array}{c}
i_1=\frac{V_{cc}-V_{-}}{4R}; i_2=\frac{V_{cc}-V_-}{2R}\\
i_3=\frac{V_{cc}-V_-}{R}; i_1=\frac{V_-V_s}{R_F}\end{array}\right\}\Rightarrow\\
\nonumber\\
\nonumber\textnormal{Como $V_-=V_+=0v$}\Rightarrow\\
\nonumber\\
\frac{V_{cc}}{4R}+\frac{V_{cc}}{2R}+\frac{V_{cc}}{R}=\frac{-V_s}{R_F}\\
\nonumber\frac{V_{cc}}{R}\left(\frac{1}{4}+\frac{1}{2}+\frac{1}{1}\right)=\frac{-V_s}{R_F}\\
\frac{V_{cc}}{R}\left(\frac{1+2+4}{4}\right)=\frac{V_{cc}}{R}\left(\frac{7}{4}\right)=\frac{-V_s}{R_F}
\end{eqnarray}
POr un an\'alisis similar, para 4 y 5 resitencias se puede encontrar que la f\'ormula resulta para cada caso en\\
\begin{equation}
\begin{array}{|c|c|c|cr|}
\hline
 & \textnormal{4 Res.}& \textnormal{5 Res.} & & \\
\hline
 & & & & \\
\frac{V_{cc}}{R} & (\frac{15}{8}) & (\frac{31}{16}) & = & -\frac{V_S}{R_F}\\
 & & & & \\
\hline
\end{array}
\end{equation}
Comparando y observando los numeradores y denminadores, y cnsiderando que los interruptores tienen 2 estados (son binarios) ,y de ah\'i la ponderaci\'on de las resitencias en potencias de 2.
\begin{equation*}
\begin{array}{|c|c|c|}
\hline &       & \\
\textnormal{Resistencias} & \textnormal{Denominador} & \textnormal{Numerador}\\
\hline & &\\
2 &  2=2^1     &  3=2^2-1\\
3 &  4=2^2     &  7=2^3-1\\
4 &  8=2^3     & 15=2^4-1\\
5 & 16=2^4     & 31=2^5-1\\
\vdots &       & \\
n &  m=2^{n-1} &  p=2^n-1\\
\hline
\end{array}
\end{equation*}
Por lo que la f\'ormula gener\'al quedar\'ia
\begin{equation}
\frac{V_{cc}}{R}\left(\frac{2^n-1}{2^{n-1}}\right)=\frac{-V_s}{R_F}
\end{equation}
\end{document}