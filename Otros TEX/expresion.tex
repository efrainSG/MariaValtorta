\documentclass[12pt,spanish,lettersize,twocolumn]{article}
\usepackage[utf8]{inputenc}
%\usepackage[latin1]{inputenc}
\usepackage[spanish]{babel}
\usepackage[dvips]{graphicx}
\usepackage[usenames,dvipsnames]{xcolor}
\usepackage{mathrsfs}
\title{\color{Maroon}Ejercicios de matem\'aticas}
\author{L.C.C. Efra\'in Serna Gracia}
\date{\color{gray}\today}

\begin{document}
\maketitle
\section{Operaciones básicas}
\begin{eqnarray}
12+3\times 4+25/5=\\
(3+4)^{3} \times 5 - 45 / 3 =\\
\sqrt{6543}=\\
\sqrt{845698}=
\end{eqnarray}
\begin{array}
\
\end{array}
\section{Regla de tres directa e inversa}
Un kilo de manzanas se forma con 7 manzanas. \\
\begin{eqnarray}
\frac{34}{7} = \frac{46}{x}
\end{eqnarray}
Un viaje de 154 Kms lo realizas en 180 min. a 51.3 Km/h. ¿A qué velocidad deberás ir para realizarlo en 120 min.?
\begin{eqnarray}
\frac{154}{51.3} = \frac{x}{180}
\end{eqnarray}

\end{document}