\documentclass[12pt,spanish,lettersize]{article}
\usepackage[dvips]{graphicx}
\usepackage[latin1]{inputenc}
\usepackage[makeroom]{cancel}
\usepackage[spanish]{babel}
\usepackage[usenames,dvipsnames]{xcolor}
\usepackage{amssymb}
\usepackage{chngcntr}
\usepackage{epstopdf}
\usepackage{graphics}
\usepackage{mathrsfs}
\usepackage{multicol}
\usepackage{subcaption}
\usepackage{vmargin}
\counterwithin*{equation}{section}
\counterwithin*{equation}{subsection}
\title{\color{Maroon}Sistemas masa-resorte-amortiguador}
\author{Efra\'in Serna Gracia}
\date{\color{gray}\today}
\begin{document}
\maketitle
\tableofcontents
\tableofimages
\section{Sistema de dos masas, dos resortes y un amortiguador}
\begin{eqnarray}
\left.\begin{array}{rcl}
m_1\ddot{x}&=&-k_1(x-y)-c_1(\dot{x}-\dot{y})-k_2(x-u)\\
m_2\ddot{y}&=&-k_1(y-x)-c_1(\dot{y}-\dot{x})
\end{array}\right\}\Rightarrow\\
\left.\begin{array}{rcl}
m_1S^2X&=&-k_1(X-Y)-c_1(SX-SY)-k_2(X-U)\\
m_2S^2Y&=&-k_1(Y-X)-c_1(SY-SX)
\end{array}\right\}
\end{eqnarray}
Variaci\'on de valores para $m_i$, $k_i$, $c_i$ entre 1 y 5 con incrementos de 2.\\
\begin{figure}[h]
\begin{subfigure}{0.45\textwidth}
\includegraphics[width=0.9\linewidth]{./graficas_01_YU}
\caption{Utilizando $S=x$ y $\frac{Y}{U}$}
\end{subfigure}
\begin{subfigure}{0.45\textwidth}
\includegraphics[width=0.9\linewidth]{./graficas_01_UY}
\caption{Utilizando $S=x$ y $\frac{U}{Y}$}
\end{subfigure}
\end{figure}
\begin{figure}[h]
\begin{subfigure}{0.45\textwidth}
\includegraphics[width=0.9\linewidth]{./graficas_02_YU}
\caption{Utilizando $S=sen(x)$ y $\frac{Y}{U}$}
\end{subfigure}
\begin{subfigure}{0.45\textwidth}
\includegraphics[width=0.9\linewidth]{./graficas_02_UY}
\caption{Utilizando $S=sen(x)$ y $\frac{U}{Y}$}
\end{subfigure}
\end{figure}
\begin{figure}[h]
\begin{subfigure}{0.45\textwidth}
\includegraphics[width=0.9\linewidth]{./graficas_03_YU}
\caption{Utilizando $S=\frac{(x+|x|)}{2}$ y $\frac{Y}{U}$}
\end{subfigure}
\begin{subfigure}{0.45\textwidth}
\includegraphics[width=0.9\linewidth]{./graficas_03_UY}
\caption{Utilizando $S=\frac{(x+|x|)}{2}$ y $\frac{U}{Y}$}
\end{subfigure}
\end{figure}
\begin{figure}[h]
\begin{subfigure}{0.45\textwidth}
\includegraphics[width=0.9\linewidth]{./graficas_07_UX}
\caption{Utilizando $S=x$ y $\frac{U}{X}$}
\end{subfigure}
\begin{subfigure}{0.45\textwidth}
\includegraphics[width=0.9\linewidth]{./graficas_07_XU}
\caption{Utilizando $S=x$ y $\frac{X}{U}$}
\end{subfigure}
\end{figure}
\begin{figure}[h]
\begin{subfigure}{0.45\textwidth}
\includegraphics[width=0.9\linewidth]{./graficas_08_UX}
\caption{Utilizando $S=sen(x)$ y $\frac{U}{X}$}
\end{subfigure}
\begin{subfigure}{0.45\textwidth}
\includegraphics[width=0.9\linewidth]{./graficas_08_XU}
\caption{Utilizando $S=sen(x)$ y $\frac{X}{U}$}
\end{subfigure}
\end{figure}
\begin{figure}[h]
\begin{subfigure}{0.45\textwidth}
\includegraphics[width=0.9\linewidth]{./graficas_09_UX}
\caption{Utilizando $S=\frac{x+|x|}{2}$ y $\frac{U}{X}$}
\end{subfigure}
\begin{subfigure}{0.45\textwidth}
\includegraphics[width=0.9\linewidth]{./graficas_09_XU}
\caption{Utilizando $S=\frac{x+|x|}{2}$ y $\frac{X}{U}$}
\end{subfigure}
\end{figure}
\section{Sistema de una masa con un resorte y un amortiguador en paralelo}
\begin{eqnarray}
m\ddot{x_0}+k(x_0-x_1)+c(\dot{x_0}-\dot{x_1})-\ddot{x_1}&=&0\\
\nonumber mS^2X_0+k(X_0-X_1)+c(SX_0-SX_1)-S^2X_1&=&0\\
\nonumber mS^2X_0+kX_0+cSX_0&=&kX_1+cSX_1+S^2X_1\\
(mS^2+k+cS)X_0&=&(k+cS+S^2)X_1\\
\frac{mS^2+k+cS}{k+cS+S^2}&=&\frac{X_1}{X_0}
\end{eqnarray}\\
\begin{figure}[h]
\begin{subfigure}{0.45\textwidth}
\includegraphics[width=0.9\linewidth]{./graficas_04_X1X0}
\caption{Utilizando $S=x$ y $\frac{X_1}{X_0}$}
\end{subfigure}
\begin{subfigure}{0.45\textwidth}
\includegraphics[width=0.9\linewidth]{./graficas_04_X0X1}
\caption{Utilizando $S=x$ y $\frac{X_0}{X_1}$}
\end{subfigure}
\end{figure}
\begin{figure}[h]
\begin{subfigure}{0.45\textwidth}
\includegraphics[width=0.9\linewidth]{./graficas_05_X1X0}
\caption{Utilizando $S=sen(x)$ y $\frac{X_1}{X_0}$}
\end{subfigure}
\begin{subfigure}{0.45\textwidth}
\includegraphics[width=0.9\linewidth]{./graficas_05_X0X1}
\caption{Utilizando $S=sen(x)$ y $\frac{X_0}{X_1}$}
\end{subfigure}
\end{figure}
\begin{figure}[h]
\begin{subfigure}{0.45\textwidth}
\includegraphics[width=0.9\linewidth]{./graficas_06_X1X0}
\caption{Utilizando $S=\frac{(x+|x|)}{2}$ y $\frac{X_1}{X_0}$}
\end{subfigure}
\begin{subfigure}{0.45\textwidth}
\includegraphics[width=0.9\linewidth]{./graficas_06_X0X1}
\caption{Utilizando $S=\frac{(x+|x|)}{2}$ y $\frac{X_0}{X_1}$}
\end{subfigure}
\end{figure}
\end{document}