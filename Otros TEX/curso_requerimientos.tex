% !TEX TS-program = pdflatex
% !TEX encoding = UTF-8 Unicode

% This is a simple template for a LaTeX document using the "article" class.
% See "book", "report", "letter" for other types of document.

\documentclass[12pt]{article} % use larger type; default would be 10pt

\usepackage[utf8]{inputenc} % set input encoding (not needed with XeLaTeX)

%%% Examples of Article customizations
% These packages are optional, depending whether you want the features they provide.
% See the LaTeX Companion or other references for full information.

%%% PAGE DIMENSIONS
\usepackage{geometry} % to change the page dimensions
\geometry{letterpaper} % or letterpaper (US) or a5paper or....
\geometry{margin=1.5in} % for example, change the margins to 2 inches all round
% \geometry{landscape} % set up the page for landscape
%   read geometry.pdf for detailed page layout information

\usepackage{graphicx} % support the \includegraphics command and options

% \usepackage[parfill]{parskip} % Activate to begin paragraphs with an empty line rather than an indent

%%% PACKAGES
\usepackage{booktabs} % for much better looking tables
\usepackage{array} % for better arrays (eg matrices) in maths
\usepackage{paralist} % very flexible & customisable lists (eg. enumerate/itemize, etc.)
\usepackage{verbatim} % adds environment for commenting out blocks of text & for better verbatim
\usepackage{subfig} % make it possible to include more than one captioned figure/table in a single float
% These packages are all incorporated in the memoir class to one degree or another...

%%% HEADERS & FOOTERS
\usepackage{fancyhdr} % This should be set AFTER setting up the page geometry
\pagestyle{fancy} % options: empty , plain , fancy
\renewcommand{\headrulewidth}{0pt} % customise the layout...
\lhead{}\chead{}\rhead{}
\lfoot{}\cfoot{\thepage}\rfoot{}

%%% SECTION TITLE APPEARANCE
\usepackage{sectsty}
\allsectionsfont{\sffamily\mdseries\upshape} % (See the fntguide.pdf for font help)
% (This matches ConTeXt defaults)

%%% ToC (table of contents) APPEARANCE
\usepackage[nottoc,notlof,notlot]{tocbibind} % Put the bibliography in the ToC
\usepackage[titles,subfigure]{tocloft} % Alter the style of the Table of Contents
\renewcommand{\cftsecfont}{\rmfamily\mdseries\upshape}
\renewcommand{\cftsecpagefont}{\rmfamily\mdseries\upshape} % No bold!

%%% END Article customizations

%%% The "real" document content comes below...

\title{Curso de Sygno}
\author{E. S. G.}
%\date{} % Activate to display a given date or no date (if empty),
         % otherwise the current date is printed 

\begin{document}
\maketitle

\section{levantamiento y análisis de requerimientos}
Cosas del curso:
estandar ISO\\
Matriz de trazabilidad que suena a matríz de trancisión. Sirve para definir prioridades y secuencia de los requerimientos, ayuda a encontrar posibles dependencias y costos (económicos / de tiempo)\\
Lista de requerimientos\\
Lista de productos\\
Ciclo de desarrollo de Sw\footnote{Revisar Procesos de Ingeniería del Software}\\
\begin{itemize}
\item Los requerimientos definen el \textbf{qué} del sistema. Son la principal entrada para las actividades de análisis y diseño, que define el \textbf{cómo}.
\item Las actividaddes de prueba validan el sistema a trqavés de requerimientos
\item Los requerimientos son usados en la definición de pruebas y en las pruebas subsecuentes de evaluación
\item Los requerimientos son importantes entradas para las actividades de la administración del proyecto
\end{itemize}
\section{dos}
\begin{enumerate}
\item Descripción de requerimientos
\item Listado de productos
\item Hacer documentación final de cada producto ya realizado
\item Contactar clientes para obtener resultados
\end{enumerate}

Metodologías a usar: en cascada, espiral, lineal, prototpios y ágiles. Depende de la mecánica del proyecto se elige la más adecuada.

\section{Técnicas de levantamiento}
El proceso de obtención de requerimientos, no es un proceso técnico, también lo es \emph{social}\footnote{El libro de \emph{Cómo ganar amigos e influir en las personas}} que involucra distintas personas, lo que agrega dificultades a su realización.\\

No se requieren técnicas, sino \emph{estrategias}.\\

Mantener la atención del intorlocutor para recuperar las necesidades del cliente. Herramientas a utilizar: las ya mencionadas en el curso de Ingeniería del Software de la carrera.\\

\begin{description}
\item[Análisis de documentación.]\footnote{Revisar mis \emph{Memorias en BI}} Integra varios tipos de documentos: manuales, reportes, entre otros que proporcionan al analista lainfo importante relacionada con la organización y sus operaciones. La documentación dificilmente refleja la forma en que se desarrollan las actividades, donde se encuetra el poder de la toma de decisiones. Pero puede ser de granimportacioa para introducir al analista en el domicio de la operación y el vocabulario que utiliza.
\begin{itemize}
\item Pro: ahorro de tiempo, puede ejecutarse fuera de la organización.
\item Contra: Desactualizado, genérico.
\end{itemize}
\item[Observación.] Se obtiene información de primera mano sobre la forma en que se efectúan las actividades. Se ve la forma en que se realizan los procesos y,verificar que realmente se sigan todos los pasos. En muchos casos, sabemos que, los procesos difieren entre el papel y la práctica. Con experiencia, los analistas saben qué buscar y cómo evaluar la relevancia de lo que observan.
\begin{itemize}
\item Pro: Certero para conocer el flujo \emph{real} del proceso.
\item Contra: tiempo.
\end{itemize}
\item[Entrevista individual / grupal] Es la más utilizada, se obtiene una comprensión general de lo que hacen los futuros usuarios del sistema, como podrían interactuar con el sistema y las dificultades a las que se enfrentan con los sistemas actuales. Exige del analista una mayor experiencia y preparación. El éxito depende de varios factores: entendimiento del entrevistador y su habilidad, así como preparación del entrevistado, química, etc.
\begin{itemize}
\item Pro: Orientado a las personas, interactivo y flexible.
\item Contra: costoso, depende de habildiades interpersonales.
\end{itemize}
\begin{enumerate}
\item Preparación. documentación, investigación, definición de objetivo y contenido de la entrevista, planificación de lugar y hora. \emph{propuesta de enviar previamente al entrevistado un cuestionario o documento}.
\item Realización: Apertura (presentación personal y proposito), Desarrollo (Registro de info) y Terminación (Recapitulación de lo revisado y dejando la puerta abierta)
\item Análisis. Lectura de notas, pasar en limpio, contraste con otroas entrevistas, organización y evaluación de fuentes de info.
\end{enumerate}
\item[Encuesta / Cuestionario] No sustituye a la entrevista. Su principal uso es validar asunciones y obtener datos estadíticos. Suelen contar con un formato establecido, los cuestionarios son propios para una ámplia distribución.
\begin{itemize}
\item Pro: Respuestas anónimas, conveniente para quien contesta.
\item Contra: Problemas por no tener respuestas, menos rico en info, esfuerzo al no contar con respuestas ámplias.
\end{itemize}
\item[Mesas de trabajo.] Técnica efectiva para obtener información rápidamente de varias personas. Se recomienda tener una agenda predefinida y preseleccionada a los participantes, siguiendo \textbf{buenas prácticas para reuniones efectivas}. Se puede utilizar un facilitador neutral y un transcriptor (distinto del facilitador). Se puede utilizar un material con el desglose del proceso o flujograma. Se pueden combinar  con otras técnicas como entrevistas y cuestionarios. \emph{REQUIEREN DE UN TIEMPO ESTABLECIDO}.
\item[Lluvia de ideas.] Consiste en reuniones de cuatro a diez personas dond, como primer paso, sugieren toda clase de ideas sin juzgar su validez -por muy disparatadas que parezcan-, y después de recopilar todas las ideas se realiza un análisis detallado de cada propuesta buscando el consenso sobre los requerimientos. Fomenta la participación de todos lo sinvolucrados, no permite debatir ni criticar.
\item[Historias de usuario / Casos de uso / Prototipos.] Las HU son forma rápida de admiistrar los requisitos de los usuarios sin tener que elaborar gran candidad de documentos formales y sn requerir mucho tiempo para administrarlos. Permiten responder a requisitos cambiantes. Los CU son un formato simple y estructurado para desarrolladores y usuarios. ayudan a identificar aspectos \emph{funcionales}. A la par del detalle del CU se puede ir integrando prototipos. Son input para el desarrollo y testing. No utilizar cuando no hay participación de usuario o pocar interfaces. Los prototipos permite a los desarrolladores comprender mejor los requerimientos,los necesarios  y deseables.
\end{description}
Toda info recuperada puede ser incluida en una matriz de trazabilidad. Al levantamiento de requerimientos le sigue el análisis de los mismos, por medio de técnicas como la descomposición funcional, modelado de procesos, inspecciones y prototipos.


\section{Pendientes}
Muchas cosas para Chubb ya están en su \emph{Cadena de Valor}
\begin{itemize}
\item Documentar tipos de correo con descripción de cada parte.
\end{itemize}

\section{Presentaciones de alto impacto}
\emph{¿Qué es lo que quiero lograr con la presentación?} considera que al final de la presentación algo debe pasar con lo expuesto: toma de acción, reflexión... ¿qué cosa?
\begin{itemize}
\item Conectar los intereses e ideas el orador con los de la audiencia. 
\item Pasar de lo que quiero como orador a lo que quiero como espectador.
\item ¿Qué quiero que haga la audiencia?
\item Intereses comunes
\item ¿quiénes son más influyentes para transmitir el mensaje y lograr mi objetivo?
\item ¡CONTENIDO! El mensaje involucra MI IDEA $+$ ESPECTATIVAS DE LA AUDIENCIA.
\begin{itemize}
\item ¿Cómo defino mi mensaje principal de forma tal que incorpore los intereses de la audiencia.
\item Si la audiencia pudiera recordad una sola cosa ¿qué quisiera que fuera?
\item Dar enfoque de problema o resaltar el problema. Es decir, plantear el mensaje principal como una dualidad problema-solución para generar interés.
\item Pruebas que dan sooporte al mensaje principal. Contundentes, expertos, irrefutables.
\begin{itemize}
\item Demostaración
\item Ejemplo
\item Simulación
\item Datos contundentes
\item Estadística
\item Imagen
\item Historias
\item Testimonio: Autoridad en materia, Gente como uno.
\item Casos
\end{itemize}
Las pruebas han de ser\ldots
\begin{itemize}
\item Lógicas
\item Emocionales
\item Concretas e inmediatas
\end{itemize}
¿Cómo presentar pruebas para que resulten demostrativas? Usando el recurso de \textbf{urgencia} combinando aspectos visuales para el hemisferio derecho y datos duros para el lado izquierdo.
\end{itemize}
\end{itemize}
El problema: es lento y tedioso hacer pruebas de varios escenarios durante el desarrollo.
\subsection{Apertura y cierre}
Los puntos claves de la presentación. El primero dice si es buena o no la presentación y el segundo dice qué se lleva la gente al terminar.\\
¿Qué buscamos de la apertura? Generar confianza, que crean lo que voy a decir. Cautivar la atención. Recursos para la apertura: preguntas, humor, enigma, cita, dinámicas lúdicas, clip.\\
El cierre refuerza el mensaje principal e invita a la acción a tomar.\\
La apertura y el cierre se construyen al final.
\subsection{La conducción}
¿cómo manejamos el tiempo? Si falta tiempo, recortando. Si sobra tiempo haces uso de historias para ajustar tiempo.
\begin{itemize}
\item Contacto visual
\item Postura
\item Tono de voz
\item Ensayar
\end{itemize}
\section{Comunicación efectiva$\rightarrow$ Comunicación de la experiencia}
NO LIMITARSE POR PREJUICIOS. ``Volver a ser niños''\\

\end{document}
