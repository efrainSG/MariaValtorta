\documentclass[12pt,spanish,twocolumn,lettersize]{article}

\title{FLEX (Man)}
\author{L.C.C. Efra\'in Serna Gracia}
\date{15 IV 2013}

\begin{document}
\maketitle

NAME
       flex - fast lexical analyzer generator

SYNOPSIS
       flex  [-bcdfhilnpstvwBFILTV78+? -C[aefFmr] -ooutput -Ppre-
       fix -Sskeleton] [--help --version] [filename ...]

OVERVIEW
       This manual describes flex, a tool for generating programs that perform pattern-matching on text. The manual includes both tutorial and reference sections:

	   Description
	       a brief overview of the tool

	   Some Simple Examples

	   Format Of The Input File

	   Patterns
	       the extended regular expressions used by flex

	   How The Input Is Matched
	       the rules for determining what has been matched

	   Actions
	       how to specify what to do when a pattern is matched

	   The Generated Scanner
	       details regarding the scanner that flex produces; how to control the input source

	   Start Conditions
	       introducing context into your scanners, and managing "mini-scanners"

	   Multiple Input Buffers
	       how to manipulate multiple input sources; how to scan from strings instead of files

	   End-of-file Rules
	       special rules for matching the end of the input

	   Miscellaneous Macros
	       a summary of macros available to the actions

	   Values Available To The User
	       a summary of values available to the actions

	   Interfacing With Yacc
	       connecting flex scanners together with yacc parsers

	   Options
	       flex command-line options, and the "%option"
	       directive

	   Performance Considerations
	       how to make your scanner go as fast as possible

	   Generating C++ Scanners
	       the (experimental) facility for generating C++ scanner classes

	   Incompatibilities With Lex And POSIX
	       how flex differs from AT&T lex and the POSIX lex standard

	   Diagnostics
	       those error messages produced by flex (or scanners it generates) whose meanings might not be apparent

	   Files
	       files used by flex

	   Deficiencies / Bugs
	       known problems with flex

	   See Also
	       other documentation, related tools

	   Author
	       includes contact information

DESCRIPTION
       flex is a tool for  generating  scanners:  programs  which recognized lexical patterns in text.  flex reads the given input files, or its standard input if no file  names  are given,  for  a  description of a scanner to generate.  The description is in the form of pairs of regular expressions and  C  code,  called  rules. flex generates as output a C source file, lex.yy.c, which defines  a	routine	 yylex().
       This  file is compiled and linked with the -lfl library to produce an executable.  When the	 executable  is	 run,  it
       analyzes	 its input for occurrences of the regular expressions.  Whenever it finds one, it executes the corresponding C code.

SOME SIMPLE EXAMPLES
       First  some  simple  examples to get the flavor of how one
       uses flex.  The following flex input specifies  a  scanner
       which  whenever	it  encounters the string "username" will
       replace it with the user's login name:

	   \%\%
	   username    printf( "\%s", getlogin() );

       By default, any text not matched	 by  a	flex  scanner  is
       copied to the output, so the net effect of this scanner is
       to copy its input file to its output with each  occurrence
       of  "username" expanded.	 In this input, there is just one
       rule.  "username" is the pattern and the "printf"  is  the
       action.	The "%%" marks the beginning of the rules.

       Here's another simple example:

		   int num_lines = 0, num_chars = 0;

	   %%
	   \n	   ++num_lines; ++num_chars;
	   .	   ++num_chars;

	   %%
	   main()
		   {
		   yylex();
		   printf( "# of lines = %d, # of chars = %d\n",
			   num_lines, num_chars );
		   }

       This  scanner counts the number of characters and the num-
       ber of lines in its input (it  produces	no  output  other
       than  the  final	 report	 on  the counts).  The first line
       declares two globals, "num_lines" and  "num_chars",  which
       are  accessible both inside yylex() and in the main() rou-
       tine declared after the second "%%".  There are two rules,
       one which matches a newline ("\n") and increments both the
       line count and the character count, and one which  matches
       any  character  other than a newline (indicated by the "."
       regular expression).

       A somewhat more complicated example:

	   /* scanner for a toy Pascal-like language */

	   %{
	   /* need this for the call to atof() below */
	   #include <math.h>
	   %}

	   DIGIT    [0-9]
	   ID	    [a-z][a-z0-9]*

	   %%

	   {DIGIT}+    {
		       printf( "An integer: %s (%d)\n", yytext,
			       atoi( yytext ) );
		       }

	   {DIGIT}+"."{DIGIT}*	      {
		       printf( "A float: %s (%g)\n", yytext,
			       atof( yytext ) );
		       }

	   if|then|begin|end|procedure|function	       {
		       printf( "A keyword: %s\n", yytext );
		       }

	   {ID}	       printf( "An identifier: %s\n", yytext );

	   "+"|"-"|"*"|"/"   printf( "An operator: %s\n", yytext );

	   "{"[^}\n]*"}"     /* eat up one-line comments */

	   [ \t\n]+	     /* eat up whitespace */

	   .	       printf( "Unrecognized character: %s\n", yytext );

	   %%

	   main( argc, argv )
	   int argc;
	   char **argv;
	       {
	       ++argv, --argc;	/* skip over program name */
	       if ( argc > 0 )
		       yyin = fopen( argv[0], "r" );
	       else
		       yyin = stdin;

	       yylex();
	       }

       This is the beginnings of a simple scanner for a	 language
       like  Pascal.  It identifies different types of tokens and
       reports on what it has seen.

       The details of this example will be explained in the  fol-
       lowing sections.

FORMAT OF THE INPUT FILE
       The  flex input file consists of three sections, separated
       by a line with just %% in it:

	   definitions
	   %%
	   rules
	   %%
	   user code

       The definitions section contains	 declarations  of  simple
       name  definitions  to  simplify the scanner specification,
       and declarations of start conditions, which are	explained
       in a later section.

       Name definitions have the form:

	   name definition

       The  "name" is a word beginning with a letter or an under-
       score ('_') followed by zero or more letters, digits, '_',
       or  '-'	(dash).	  The definition is taken to begin at the
       first non-white-space character	following  the	name  and
       continuing  to  the  end	 of the line.  The definition can
       subsequently be referred to  using  "{name}",  which  will
       expand to "(definition)".  For example,

	   DIGIT    [0-9]
	   ID	    [a-z][a-z0-9]*

       defines "DIGIT" to be a regular expression which matches a
       single digit, and "ID" to be a  regular	expression  which
       matches	a letter followed by zero-or-more letters-or-dig-
       its.  A subsequent reference to

	   {DIGIT}+"."{DIGIT}*

       is identical to

	   ([0-9])+"."([0-9])*

       and matches one-or-more digits followed by a '.'	 followed
       by zero-or-more digits.

       The  rules  section of the flex input contains a series of
       rules of the form:

	   pattern   action

       where the pattern must be unindented and the  action  must
       begin on the same line.

       See  below  for	a  further  description	 of  patterns and
       actions.

       Finally,	 the  user  code  section  is  simply  copied  to
       lex.yy.c	 verbatim.   It	 is  used  for companion routines
       which call or are called by the scanner.	 The presence  of
       this  section is optional; if it is missing, the second %%
       in the input file may be skipped, too.

       In the definitions and rules sections, any  indented  text
       or  text	 enclosed  in %{ and %} is copied verbatim to the
       output (with the %{}'s removed).	 The  %{}'s  must  appear
       unindented on lines by themselves.

       In  the	rules section, any indented or %{} text appearing
       before the first rule may be  used  to  declare	variables
       which  are  local  to  the scanning routine and (after the
       declarations) code which is to be  executed  whenever  the
       scanning	 routine  is entered.  Other indented or %{} text
       in the rule section is still copied to the output, but its
       meaning is not well-defined and it may well cause compile-
       time errors (this feature is present for POSIX compliance;
       see below for other such features).

       In the definitions section (but not in the rules section),
       an unindented comment (i.e., a line beginning  with  "/*")
       is also copied verbatim to the output up to the next "*/".

PATTERNS
       The patterns in the input are written  using  an	 extended
       set of regular expressions.  These are:

	   x	      match the character 'x'
	   .	      any character (byte) except newline
	   [xyz]      a "character class"; in this case, the pattern
			matches either an 'x', a 'y', or a 'z'
	   [abj-oZ]   a "character class" with a range in it; matches
			an 'a', a 'b', any letter from 'j' through 'o',
			or a 'Z'
	   [^A-Z]     a "negated character class", i.e., any character
			but those in the class.	 In this case, any
			character EXCEPT an uppercase letter.
	   [^A-Z\n]   any character EXCEPT an uppercase letter or
			a newline
	   r*	      zero or more r's, where r is any regular expression
	   r+	      one or more r's
	   r?	      zero or one r's (that is, "an optional r")
	   r{2,5}     anywhere from two to five r's
	   r{2,}      two or more r's
	   r{4}	      exactly 4 r's
	   {name}     the expansion of the "name" definition
		      (see above)
	   "[xyz]\"foo"
		      the literal string: [xyz]"foo
	   \X	      if X is an 'a', 'b', 'f', 'n', 'r', 't', or 'v',
			then the ANSI-C interpretation of \x.
			Otherwise, a literal 'X' (used to escape
			operators such as '*')
	   \0	      a NUL character (ASCII code 0)
	   \123	      the character with octal value 123
	   \x2a	      the character with hexadecimal value 2a
	   (r)	      match an r; parentheses are used to override
			precedence (see below)

	   rs	      the regular expression r followed by the
			regular expression s; called "concatenation"

	   r|s	      either an r or an s

	   r/s	      an r but only if it is followed by an s.	The
			text matched by s is included when determining
			whether this rule is the "longest match",
			but is then returned to the input before
			the action is executed.	 So the action only
			sees the text matched by r.  This type
			of pattern is called trailing context".
			(There are some combinations of r/s that flex
			cannot match correctly; see notes in the
			Deficiencies / Bugs section below regarding
			"dangerous trailing context".)
	   ^r	      an r, but only at the beginning of a line (i.e.,
			which just starting to scan, or right after a
			newline has been scanned).
	   r$	      an r, but only at the end of a line (i.e., just
			before a newline).  Equivalent to "r/\n".

		      Note that flex's notion of "newline" is exactly
		      whatever the C compiler used to compile flex
		      interprets '\n' as; in particular, on some DOS
		      systems you must either filter out \r's in the
		      input yourself, or explicitly use r/\r\n for "r$".

	   <s>r	      an r, but only in start condition s (see
			below for discussion of start conditions)
	   <s1,s2,s3>r
		      same, but in any of start conditions s1,
			s2, or s3
	   <*>r	      an r in any start condition, even an exclusive one.

	   <<EOF>>    an end-of-file
	   <s1,s2><<EOF>>
		      an end-of-file when in start condition s1 or s2

       Note that inside of a character class, all regular expres-
       sion operators lose their special  meaning  except  escape
       ('\') and the character class operators, '-', ']', and, at
       the beginning of the class, '^'.

       The regular expressions listed above are grouped according
       to  precedence, from highest precedence at the top to low-
       est at the bottom.   Those  grouped  together  have  equal
       precedence.  For example,

	   foo|bar*

       is the same as

	   (foo)|(ba(r*))

       since the '*' operator has higher precedence than concate-
       nation, and concatenation higher than  alternation  ('|').
       This  pattern therefore matches either the string "foo" or
       the string "ba" followed by zero-or-more	 r's.	To  match
       "foo" or zero-or-more "bar"'s, use:

	   foo|(bar)*

       and to match zero-or-more "foo"'s-or-"bar"'s:

	   (foo|bar)*

       In  addition to characters and ranges of characters, char-
       acter classes can also  contain	character  class  expres-
       sions.	These  are  expressions enclosed inside [: and :]
       delimiters (which themselves must appear between	 the  '['
       and  ']'	 of the character class; other elements may occur
       inside the character class, too).  The  valid  expressions
       are:

	   [:alnum:] [:alpha:] [:blank:]
	   [:cntrl:] [:digit:] [:graph:]
	   [:lower:] [:print:] [:punct:]
	   [:space:] [:upper:] [:xdigit:]

       These expressions all designate a set of characters equiv-
       alent to the corresponding standard C isXXX function.  For
       example,	 [:alnum:]  designates those characters for which
       isalnum() returns true - i.e., any alphabetic or	 numeric.
       Some  systems  don't  provide  isblank(),  so flex defines
       [:blank:] as a blank or a tab.

       For example,  the  following  character	classes	 are  all
       equivalent:

	   [[:alnum:]]
	   [[:alpha:][:digit:]
	   [[:alpha:]0-9]
	   [a-zA-Z0-9]

       If  your	 scanner  is case-insensitive (the -i flag), then
       [:upper:] and [:lower:] are equivalent to [:alpha:].

       Some notes on patterns:

       -      A negated character class such as the example "[^A-
	      Z]"  above  will match a newline unless "\n" (or an
	      equivalent escape sequence) is one of  the  charac-
	      ters  explicitly	present	 in the negated character
	      class (e.g., "[^A-Z\n]").	 This is unlike how  many
	      other  regular expression tools treat negated char-
	      acter classes, but unfortunately the  inconsistency
	      is   historically	 entrenched.   Matching	 newlines
	      means that a  pattern  like  [^"]*  can  match  the
	      entire  input  unless  there's another quote in the
	      input.

       -      A rule can have at most one  instance  of	 trailing
	      context  (the  '/'  operator  or the '$' operator).
	      The start condition, '^',	and "<<EOF>>"
	      patterns  can  only  occur  at  the  beginning of a
	      pattern,  and,  as  well  as  with   '/'  and  '$', 
	      cannot  be  grouped  inside   parentheses.  A '^'
	      which  does not occur at the beginning of a rule or
	      a  '$'	 which	does  not  occur  at the end of a
	      rule loses its special properties and is treated as
	      a normal character.

	      The following are illegal:

		  foo/bar$
		  <sc1>foo<sc2>bar

	      Note that	 the  first  of	 these,	 can  be  written
	      "foo/bar\n".

	      The  following  will  result  in	'$'  or '^' being
	      treated as a normal character:

		  foo|(bar$)
		  foo|^bar

	      If what's wanted is a "foo" or a bar-followed-by-a-
	      newline,	the  following could be used (the special
	      '|' action is explained below):

		  foo	   |
		  bar$	   /* action goes here */

	      A similar trick will work for matching a foo  or	a
	      bar-at-the-beginning-of-a-line.

HOW THE INPUT IS MATCHED
       When  the  generated scanner is run, it analyzes its input
       looking for strings which match any of its  patterns.   If
       it  finds  more	than one match, it takes the one matching
       the most text (for trailing context rules,  this	 includes
       the  length of the trailing part, even though it will then
       be returned to the  input).   If	 it  finds  two	 or  more
       matches	of  the same length, the rule listed first in the
       flex input file is chosen.

       Once the match is determined, the  text	corresponding  to
       the  match  (called  the	 token)	 is made available in the
       global character pointer yytext, and  its  length  in  the
       global  integer	yyleng.	  The action corresponding to the
       matched pattern is then executed (a more detailed descrip-
       tion  of actions follows), and then the remaining input is
       scanned for another match.

       If no match is found, then the default rule  is	executed:
       the  next character in the input is considered matched and
       copied to the standard output.  Thus, the  simplest  legal
       flex input is:

	   %%

       which  generates	 a  scanner  that simply copies its input
       (one character at a time) to its output.

       Note that yytext can be defined	in  two	 different  ways:
       either  as  a  character	 pointer or as a character array.
       You can control which definition flex  uses  by	including
       one  of	the  special directives %pointer or %array in the
       first (definitions)  section  of	 your  flex  input.   The
       default is %pointer, unless you use the -l lex compatibil-
       ity option, in which case yytext will be	 an  array.   The
       advantage  of using %pointer is substantially faster scan-
       ning and no  buffer  overflow  when  matching  very  large
       tokens (unless you run out of dynamic memory).  The disad-
       vantage is that you are restricted in how your actions can
       modify  yytext  (see  the  next section), and calls to the
       unput() function destroys the present contents of  yytext,
       which  can  be a considerable porting headache when moving
       between different lex versions.

       The advantage of %array is that you can then modify yytext
       to  your	 heart's  content,  and	 calls	to unput() do not
       destroy yytext (see  below).   Furthermore,  existing  lex
       programs sometimes access yytext externally using declara-
       tions of the form:
	   extern char yytext[];
       This definition is erroneous when used with %pointer,  but
       correct for %array.

       %array defines yytext to be an array of YYLMAX characters,
       which defaults to a fairly large value.	 You  can  change
       the size by simply #define'ing YYLMAX to a different value
       in the first section of your  flex  input.   As	mentioned
       above,  with %pointer yytext grows dynamically to accommo-
       date large tokens.  While this means your %pointer scanner
       can accommodate very large tokens (such as matching entire
       blocks of comments), bear in mind that each time the scan-
       ner  must  resize  yytext  it  also must rescan the entire
       token from the beginning,  so  matching	such  tokens  can
       prove slow.  yytext presently does not dynamically grow if
       a call to unput() results in too much  text  being  pushed
       back; instead, a run-time error results.

       Also  note  that	 you  cannot  use %array with C++ scanner
       classes (the c++ option; see below).

ACTIONS
       Each pattern in a rule has a corresponding  action,  which
       can be any arbitrary C statement.  The pattern ends at the
       first non-escaped whitespace character; the  remainder  of
       the line is its action.	If the action is empty, then when
       the pattern is matched the  input  token	 is  simply  dis-
       carded.	For example, here is the specification for a pro-
       gram which deletes all occurrences of "zap  me"	from  its
       input:

	   %%
	   "zap me"

       (It  will  copy	all  other characters in the input to the
       output since they will be matched by the default rule.)

       Here is a program which	compresses  multiple  blanks  and
       tabs  down  to  a single blank, and throws away whitespace
       found at the end of a line:

	   %%
	   [ \t]+	 putchar( ' ' );
	   [ \t]+$	 /* ignore this token */

       If the action contains a '{', then the action  spans  till
       the  balancing '}' is found, and the action may cross mul-
       tiple lines.  flex knows about C strings and comments  and
       won't  be  fooled  by  braces  found within them, but also
       allows actions to begin with  %{	 and  will  consider  the
       action to be all the text up to the next %} (regardless of
       ordinary braces inside the action).

       An action consisting solely of a vertical bar ('|')  means
       "same  as the action for the next rule."	 See below for an
       illustration.

       Actions can include arbitrary  C	 code,	including  return
       statements  to  return  a value to whatever routine called
       yylex().	 Each time yylex() is called  it  continues  pro-
       cessing tokens from where it last left off until it either
       reaches the end of the file or executes a return.

       Actions are free to modify yytext except	 for  lengthening
       it  (adding  characters	to  its end--these will overwrite
       later characters in the input stream).  This however  does
       not  apply  when	 using	%array (see above); in that case,
       yytext may be freely modified in any way.

       Actions are free to modify yyleng except they  should  not
       do  so  if  the	action also includes use of yymore() (see
       below).

       There are a number of  special  directives  which  can  be
       included within an action:

       -      ECHO copies yytext to the scanner's output.

       -      BEGIN  followed  by  the	name of a start condition
	      places the scanner in the corresponding start  con-
	      dition (see below).

       -      REJECT  directs  the  scanner  to proceed on to the
	      "second best" rule which matched the  input  (or	a
	      prefix  of  the  input).	 The  rule  is	chosen as
	      described above in "How the Input is Matched",  and
	      yytext  and  yyleng  set	up appropriately.  It may
	      either be one which matched as  much  text  as  the
	      originally  chosen  rule but came later in the flex
	      input file, or one which matched	less  text.   For
	      example, the following will both count the words in
	      the input and call the routine  special()	 whenever
	      "frob" is seen:

			  int word_count = 0;
		  %%

		  frob	      special(); REJECT;
		  [^ \t\n]+   ++word_count;

	      Without the REJECT, any "frob"'s in the input would
	      not be counted as words, since the scanner normally
	      executes	only  one  action  per	token.	 Multiple
	      REJECT's are allowed, each  one  finding	the  next
	      best  choice  to	the  currently	active rule.  For
	      example, when the following scanner scans the token
	      "abcd", it will write "abcdabcaba" to the output:

		  %%
		  a	   |
		  ab	   |
		  abc	   |
		  abcd	   ECHO; REJECT;
		  .|\n	   /* eat up any unmatched character */

	      (The  first  three  rules share the fourth's action
	      since they use the special '|' action.)  REJECT  is
	      a	 particularly expensive feature in terms of scan-
	      ner performance; if it is used in any of the  scan-
	      ner's  actions  it  will slow down all of the scan-
	      ner's matching.  Furthermore, REJECT cannot be used
	      with the -Cf or -CF options (see below).

	      Note  also  that	unlike the other special actions,
	      REJECT is a branch; code immediately  following  it
	      in the action will not be executed.

       -      yymore()	tells  the  scanner that the next time it
	      matches a rule, the corresponding token  should  be
	      appended	onto  the  current value of yytext rather
	      than replacing it.  For example,	given  the  input
	      "mega-kludge"  the following will write "mega-mega-
	      kludge" to the output:

		  %%
		  mega-	   ECHO; yymore();
		  kludge   ECHO;

	      First "mega-" is matched and echoed to the  output.
	      Then  "kludge" is matched, but the previous "mega-"
	      is still hanging around at the beginning of  yytext
	      so  the  ECHO  for  the "kludge" rule will actually
	      write "mega-kludge".

       Two notes regarding  use	 of  yymore().	 First,	 yymore()
       depends	on  the	 value of yyleng correctly reflecting the
       size of the current token, so you must not  modify  yyleng
       if  you	are  using  yymore().	Second,	 the  presence of
       yymore() in the scanner's action entails a  minor  perfor-
       mance penalty in the scanner's matching speed.

       -      yyless(n) returns all but the first n characters of
	      the current token back to the input  stream,  where
	      they  will  be rescanned when the scanner looks for
	      the next match.  yytext  and  yyleng  are	 adjusted
	      appropriately  (e.g., yyleng will now be equal to n
	      ).  For example, on the input "foobar" the  follow-
	      ing will write out "foobarbar":

		  %%
		  foobar    ECHO; yyless(3);
		  [a-z]+    ECHO;

	      An  argument  of	0 to yyless will cause the entire
	      current input string to be scanned  again.   Unless
	      you've  changed  how  the scanner will subsequently
	      process its input (using BEGIN, for example),  this
	      will result in an endless loop.

       Note  that  yyless  is a macro and can only be used in the
       flex input file, not from other source files.

       -      unput(c) puts the character c back onto  the  input
	      stream.	It  will  be  the next character scanned.
	      The following action will take  the  current  token
	      and  cause it to be rescanned enclosed in parenthe-
	      ses.

		  {
		  int i;
		  /* Copy yytext because unput() trashes yytext */
		  char *yycopy = strdup( yytext );
		  unput( ')' );
		  for ( i = yyleng - 1; i >= 0; --i )
		      unput( yycopy[i] );
		  unput( '(' );
		  free( yycopy );
		  }

	      Note that since each unput() puts the given charac-
	      ter  back	 at  the  beginning  of the input stream,
	      pushing back strings must be done back-to-front.

       An important potential problem when using unput() is  that
       if you are using %pointer (the default), a call to unput()
       destroys the contents of yytext, starting with its  right-
       most  character	and  devouring	one character to the left
       with each call.	If you need the value of yytext preserved
       after  a	 call  to  unput() (as in the above example), you
       must either first copy it elsewhere, or build your scanner
       using %array instead (see How The Input Is Matched).

       Finally,	 note  that you cannot put back EOF to attempt to
       mark the input stream with an end-of-file.

       -      input() reads the next  character	 from  the  input
	      stream.	For  example, the following is one way to
	      eat up C comments:

		  %%
		  "/*"	      {
			      register int c;

			      for ( ; ; )
				  {
				  while ( (c = input()) != '*' &&
					  c != EOF )
				      ;	   /* eat up text of comment */

				  if ( c == '*' )
				      {
				      while ( (c = input()) == '*' )
					  ;
				      if ( c == '/' )
					  break;    /* found the end */
				      }

				  if ( c == EOF )
				      {
				      error( "EOF in comment" );
				      break;
				      }
				  }
			      }

	      (Note that if the scanner is  compiled  using  C++,
	      then  input()  is instead referred to as yyinput(),
	      in order to avoid a name clash with the C++  stream
	      by the name of input.)

       -      YY_FLUSH_BUFFER	flushes	 the  scanner's	 internal
	      buffer so that the next time the	scanner	 attempts
	      to  match	 a token, it will first refill the buffer
	      using YY_INPUT (see The Generated Scanner,  below).
	      This  action  is a special case of the more general
	      yy_flush_buffer() function, described below in  the
	      section Multiple Input Buffers.

       -      yyterminate()  can  be  used  in	lieu  of a return
	      statement in an action.  It terminates the  scanner
	      and returns a 0 to the scanner's caller, indicating
	      "all done".   By	default,  yyterminate()	 is  also
	      called when an end-of-file is encountered.  It is a
	      macro and may be redefined.

THE GENERATED SCANNER
       The output of flex is the file  lex.yy.c,  which	 contains
       the  scanning  routine yylex(), a number of tables used by
       it for matching tokens, and a number of auxiliary routines
       and macros.  By default, yylex() is declared as follows:

	   int yylex()
	       {
	       ... various definitions and the actions in here ...
	       }

       (If your environment supports function prototypes, then it
       will be "int yylex( void	 )".)	This  definition  may  be
       changed by defining the "YY_DECL" macro.	 For example, you
       could use:

	   #define YY_DECL float lexscan( a, b ) float a, b;

       to give the scanning routine the name lexscan, returning a
       float,  and  taking two floats as arguments.  Note that if
       you give arguments to the scanning routine  using  a  K&R-
       style/non-prototyped function declaration, you must termi-
       nate the definition with a semi-colon (;).

       Whenever yylex() is  called,  it	 scans	tokens	from  the
       global input file yyin (which defaults to stdin).  It con-
       tinues until it either reaches an  end-of-file  (at  which
       point  it  returns the value 0) or one of its actions exe-
       cutes a return statement.

       If the scanner reaches an  end-of-file,	subsequent  calls
       are undefined unless either yyin is pointed at a new input
       file (in which case scanning continues from that file), or
       yyrestart()  is called.	yyrestart() takes one argument, a
       FILE * pointer  (which  can  be	nil,  if  you've  set  up
       YY_INPUT	 to scan from a source other than yyin), and ini-
       tializes yyin for scanning from	that  file.   Essentially
       there  is  no  difference between just assigning yyin to a
       new input file or using yyrestart() to do so;  the  latter
       is  available  for compatibility with previous versions of
       flex, and because it can be used to switch input files  in
       the middle of scanning.	It can also be used to throw away
       the current input buffer, by calling it with  an	 argument
       of yyin; but better is to use YY_FLUSH_BUFFER (see above).
       Note that yyrestart() does not reset the	 start	condition
       to INITIAL (see Start Conditions, below).

       If yylex() stops scanning due to executing a return state-
       ment in one of the actions, the scanner may then be called
       again and it will resume scanning where it left off.

       By  default  (and for purposes of efficiency), the scanner
       uses block-reads rather than simple getc() calls	 to  read
       characters from yyin.  The nature of how it gets its input
       can  be	controlled  by	defining  the	YY_INPUT   macro.
       YY_INPUT's	    calling	     sequence	       is
       "YY_INPUT(buf,result,max_size)".	 Its action is	to  place
       up  to  max_size characters in the character array buf and
       return in the integer variable result either the number of
       characters  read	 or  the constant YY_NULL (0 on Unix sys-
       tems) to indicate EOF.  The default  YY_INPUT  reads  from
       the global file-pointer "yyin".

       A  sample  definition of YY_INPUT (in the definitions sec-
       tion of the input file):

	   %{
	   #define YY_INPUT(buf,result,max_size) \
	       { \
	       int c = getchar(); \
	       result = (c == EOF) ? YY_NULL : (buf[0] = c, 1); \
	       }
	   %}

       This definition will change the input processing to  occur
       one character at a time.

       When  the  scanner receives an end-of-file indication from
       YY_INPUT,  it  then  checks  the	 yywrap()  function.   If
       yywrap() returns false (zero), then it is assumed that the
       function has gone ahead	and  set  up  yyin  to	point  to
       another input file, and scanning continues.  If it returns
       true (non-zero), then the scanner terminates, returning	0
       to its caller.  Note that in either case, the start condi-
       tion remains unchanged; it does not revert to INITIAL.

       If you do not supply your own version  of  yywrap(),  then
       you  must  either  use %option noyywrap (in which case the
       scanner behaves as though yywrap()  returned  1),  or  you
       must  link  with -lfl to obtain the default version of the
       routine, which always returns 1.

       Three routines are available for scanning  from	in-memory
       buffers	   rather     than    files:	yy_scan_string(),
       yy_scan_bytes(), and yy_scan_buffer().  See the discussion
       of them below in the section Multiple Input Buffers.

       The  scanner  writes  its  ECHO output to the yyout global
       (default, stdout), which may be redefined by the user sim-
       ply by assigning it to some other FILE pointer.

START CONDITIONS
       flex  provides  a  mechanism  for conditionally activating
       rules.  Any rule whose pattern  is  prefixed  with  "<sc>"
       will  only be active when the scanner is in the start con-
       dition named "sc".  For example,

	   <STRING>[^"]*	{ /* eat up the string body ... */
		       ...
		       }

       will be active only when the scanner is	in  the	 "STRING"
       start condition, and

	   <INITIAL,STRING,QUOTE>\.	   { /* handle an escape ... */
		       ...
		       }

       will  be	 active	 only when the current start condition is
       either "INITIAL", "STRING", or "QUOTE".

       Start conditions are declared in the  definitions  (first)
       section of the input using unindented lines beginning with
       either %s or %x followed by a list of names.   The  former
       declares	 inclusive start conditions, the latter exclusive
       start conditions.  A start condition  is	 activated  using
       the  BEGIN  action.   Until  the next BEGIN action is exe-
       cuted, rules with the given start condition will be active
       and  rules  with	 other start conditions will be inactive.
       If the start condition is inclusive, then  rules	 with  no
       start  conditions  at  all  will also be active.	 If it is
       exclusive, then only rules qualified with the start condi-
       tion  will  be  active.	 A set of rules contingent on the
       same exclusive start condition describe a scanner which is
       independent  of	any of the other rules in the flex input.
       Because of this, exclusive start conditions make	 it  easy
       to  specify  "mini-scanners"  which  scan  portions of the
       input that  are	syntactically  different  from	the  rest
       (e.g., comments).

       If  the	distinction between inclusive and exclusive start
       conditions is still a little vague, here's a simple  exam-
       ple  illustrating the connection between the two.  The set
       of rules:

	   %s example

	   %%

	   <example>foo	  do_something();

	   bar		  something_else();

       is equivalent to

	   %x example
	   %%

	   <example>foo	  do_something();

	   <INITIAL,example>bar	   something_else();

       Without the <INITIAL,example> qualifier, the  bar  pattern
       in  the	second example wouldn't be active (i.e., couldn't
       match) when in start condition example.	If we  just  used
       <example>  to  qualify  bar, though, then it would only be
       active in example and not in INITIAL, while in  the  first
       example	it's active in both, because in the first example
       the example startion condition is an inclusive (%s)  start
       condition.

       Also  note  that the special start-condition specifier <*>
       matches every start condition.  Thus,  the  above  example
       could also have been written;

	   %x example
	   %%

	   <example>foo	  do_something();

	   <*>bar    something_else();

       The default rule (to ECHO any unmatched character) remains
       active in start conditions.  It is equivalent to:

	   <*>.|\n     ECHO;

       BEGIN(0) returns to the	original  state	 where	only  the
       rules with no start conditions are active.  This state can
       also be referred to as the start-condition  "INITIAL",  so
       BEGIN(INITIAL)  is equivalent to BEGIN(0).  (The parenthe-
       ses around the start condition name are not  required  but
       are considered good style.)

       BEGIN  actions  can  also be given as indented code at the
       beginning of the rules section.	For example, the  follow-
       ing  will  cause	 the scanner to enter the "SPECIAL" start
       condition whenever yylex() is called and the global  vari-
       able enter_special is true:

		   int enter_special;

	   %x SPECIAL
	   %%
		   if ( enter_special )
		       BEGIN(SPECIAL);

	   <SPECIAL>blahblahblah
	   ...more rules follow...

       To  illustrate  the  uses  of  start conditions, here is a
       scanner which provides two different interpretations of	a
       string  like  "123.456".	  By  default it will treat it as
       three tokens, the integer "123",	 a  dot	 ('.'),	 and  the
       integer	"456".	 But if the string is preceded earlier in
       the line by the string "expect-floats" it will treat it as
       a single token, the floating-point number 123.456:

	   %{
	   #include <math.h>
	   %}
	   %s expect

	   %%
	   expect-floats	BEGIN(expect);

	   <expect>[0-9]+"."[0-9]+	{
		       printf( "found a float, = %f\n",
			       atof( yytext ) );
		       }
	   <expect>\n		{
		       /* that's the end of the line, so
			* we need another "expect-number"
			* before we'll recognize any more
			* numbers
			*/
		       BEGIN(INITIAL);
		       }

	   [0-9]+      {
		       printf( "found an integer, = %d\n",
			       atoi( yytext ) );
		       }

	   "."	       printf( "found a dot\n" );

       Here  is	 a scanner which recognizes (and discards) C com-
       ments while maintaining a count of the current input line.

	   %x comment
	   %%
		   int line_num = 1;

	   "/*"		BEGIN(comment);

	   <comment>[^*\n]*	   /* eat anything that's not a '*' */
	   <comment>"*"+[^*/\n]*   /* eat up '*'s not followed by '/'s */
	   <comment>\n		   ++line_num;
	   <comment>"*"+"/"	   BEGIN(INITIAL);

       This  scanner  goes  to	a bit of trouble to match as much
       text  as	 possible  with	 each  rule.   In  general,  when
       attempting  to  write a high-speed scanner try to match as
       much possible in each rule, as it's a big win.

       Note that start-conditions names are really integer values
       and  can	 be  stored  as	 such.	 Thus, the above could be
       extended in the following fashion:

	   %x comment foo
	   %%
		   int line_num = 1;
		   int comment_caller;

	   "/*"		{
			comment_caller = INITIAL;
			BEGIN(comment);
			}

	   ...

	   <foo>"/*"	{
			comment_caller = foo;
			BEGIN(comment);
			}

	   <comment>[^*\n]*	   /* eat anything that's not a '*' */
	   <comment>"*"+[^*/\n]*   /* eat up '*'s not followed by '/'s */
	   <comment>\n		   ++line_num;
	   <comment>"*"+"/"	   BEGIN(comment_caller);

       Furthermore, you can access the	current	 start	condition
       using the integer-valued YY_START macro.	 For example, the
       above assignments to comment_caller could instead be writ-
       ten

	   comment_caller = YY_START;

       Flex provides YYSTATE as an alias for YY_START (since that
       is what's used by AT&T lex).

       Note that start conditions do not  have	their  own  name-
       space;  %s's and %x's declare names in the same fashion as
       #define's.

       Finally, here's an example of how to match C-style  quoted
       strings	 using	 exclusive  start  conditions,	including
       expanded escape sequences (but not including checking  for
       a string that's too long):

	   %x str

	   %%
		   char string_buf[MAX_STR_CONST];
		   char *string_buf_ptr;

	   \"	   string_buf_ptr = string_buf; BEGIN(str);

	   <str>\"	  { /* saw closing quote - all done */
		   BEGIN(INITIAL);
		   *string_buf_ptr = '\0';
		   /* return string constant token type and
		    * value to parser
		    */
		   }

	   <str>\n	  {
		   /* error - unterminated string constant */
		   /* generate error message */
		   }

	   <str>\\[0-7]{1,3} {
		   /* octal escape sequence */
		   int result;

		   (void) sscanf( yytext + 1, "%o", &result );

		   if ( result > 0xff )
			   /* error, constant is out-of-bounds */

		   *string_buf_ptr++ = result;
		   }

	   <str>\\[0-9]+ {
		   /* generate error - bad escape sequence; something
		    * like '\48' or '\0777777'
		    */
		   }

	   <str>\\n  *string_buf_ptr++ = '\n';
	   <str>\\t  *string_buf_ptr++ = '\t';
	   <str>\\r  *string_buf_ptr++ = '\r';
	   <str>\\b  *string_buf_ptr++ = '\b';
	   <str>\\f  *string_buf_ptr++ = '\f';

	   <str>\\(.|\n)  *string_buf_ptr++ = yytext[1];

	   <str>[^\\\n\"]+	  {
		   char *yptr = yytext;
		   while ( *yptr )
			   *string_buf_ptr++ = *yptr++;
		   }

       Often,  such as in some of the examples above, you wind up
       writing a whole bunch of rules all preceded  by	the  same
       start  condition(s).   Flex makes this a little easier and
       cleaner by introducing a notion of start condition  scope.
       A start condition scope is begun with:

	   <SCs>{

       where  SCs  is  a  list	of  one or more start conditions.
       Inside the start condition scope, every rule automatically
       has  the	 prefix	 <SCs>	applied	 to it, until a '}' which
       matches the initial '{'.	 So, for example,

	   <ESC>{
	       "\\n"   return '\n';
	       "\\r"   return '\r';
	       "\\f"   return '\f';
	       "\\0"   return '\0';
	   }

       is equivalent to:

	   <ESC>"\\n"  return '\n';
	   <ESC>"\\r"  return '\r';
	   <ESC>"\\f"  return '\f';
	   <ESC>"\\0"  return '\0';

       Start condition scopes may be nested.

       Three routines are available for	 manipulating  stacks  of
       start conditions:

       void yy_push_state(int new_state)
	      pushes  the current start condition onto the top of
	      the start condition stack and switches to new_state
	      as though you had used BEGIN new_state (recall that
	      start condition names are also integers).

       void yy_pop_state()
	      pops the top of the stack and switches  to  it  via
	      BEGIN.

       int yy_top_state()
	      returns  the  top of the stack without altering the
	      stack's contents.

       The start condition stack grows dynamically and so has  no
       built-in size limitation.  If memory is exhausted, program
       execution aborts.

       To use start condition stacks, your scanner must include a
       %option stack directive (see Options below).

MULTIPLE INPUT BUFFERS
       Some  scanners  (such  as  those	 which	support "include"
       files) require reading from  several  input  streams.   As
       flex  scanners  do a large amount of buffering, one cannot
       control where the next input will be read from  by  simply
       writing a YY_INPUT which is sensitive to the scanning con-
       text.  YY_INPUT is only called when  the	 scanner  reaches
       the  end	 of  its  buffer,  which may be a long time after
       scanning a statement such as an "include"  which	 requires
       switching the input source.

       To  negotiate  these  sorts  of	problems, flex provides a
       mechanism for  creating	and  switching	between	 multiple
       input buffers.  An input buffer is created by using:

	   YY_BUFFER_STATE yy_create_buffer( FILE *file, int size )

       which takes a FILE pointer and a size and creates a buffer
       associated with the given file and large	 enough	 to  hold
       size  characters	 (when	in doubt, use YY_BUF_SIZE for the
       size).  It returns a  YY_BUFFER_STATE  handle,  which  may
       then  be	 passed	 to  other  routines  (see  below).   The
       YY_BUFFER_STATE type is a  pointer  to  an  opaque  struct
       yy_buffer_state	structure,  so	you may safely initialize
       YY_BUFFER_STATE variables to ((YY_BUFFER_STATE) 0) if  you
       wish,  and  also refer to the opaque structure in order to
       correctly declare input buffers in source files other than
       that  of	 your scanner.	Note that the FILE pointer in the
       call to yy_create_buffer is only used as the value of yyin
       seen by YY_INPUT; if you redefine YY_INPUT so it no longer
       uses yyin, then you can safely pass a nil FILE pointer  to
       yy_create_buffer.   You select a particular buffer to scan
       from using:

	   void yy_switch_to_buffer( YY_BUFFER_STATE new_buffer )

       switches the scanner's input buffer so  subsequent  tokens
       will	 come	  from	   new_buffer.	    Note     that
       yy_switch_to_buffer() may  be  used  by	yywrap()  to  set
       things up for continued scanning, instead of opening a new
       file and pointing yyin at it.  Note  also  that	switching
       input sources via either yy_switch_to_buffer() or yywrap()
       does not change the start condition.

	   void yy_delete_buffer( YY_BUFFER_STATE buffer )

       is used to reclaim the storage associated with  a  buffer.
       (  buffer can be nil, in which case the routine does noth-
       ing.)  You can also clear the current contents of a buffer
       using:

	   void yy_flush_buffer( YY_BUFFER_STATE buffer )

       This  function discards the buffer's contents, so the next
       time the scanner	 attempts  to  match  a	 token	from  the
       buffer, it will first fill the buffer anew using YY_INPUT.

       yy_new_buffer() is an alias for	yy_create_buffer(),  pro-
       vided for compatibility with the C++ use of new and delete
       for creating and destroying dynamic objects.

       Finally,	  the	YY_CURRENT_BUFFER   macro    returns	a
       YY_BUFFER_STATE handle to the current buffer.

       Here  is	 an example of using these features for writing a
       scanner which expands include files (the	 <<EOF>>  feature
       is discussed below):

	   /* the "incl" state is used for picking up the name
	    * of an include file
	    */
	   %x incl

	   %{
	   #define MAX_INCLUDE_DEPTH 10
	   YY_BUFFER_STATE include_stack[MAX_INCLUDE_DEPTH];
	   int include_stack_ptr = 0;
	   %}

	   %%
	   include	       BEGIN(incl);

	   [a-z]+	       ECHO;
	   [^a-z\n]*\n?	       ECHO;

	   <incl>[ \t]*	     /* eat the whitespace */
	   <incl>[^ \t\n]+   { /* got the include file name */
		   if ( include_stack_ptr >= MAX_INCLUDE_DEPTH )
		       {
		       fprintf( stderr, "Includes nested too deeply" );
		       exit( 1 );
		       }

		   include_stack[include_stack_ptr++] =
		       YY_CURRENT_BUFFER;

		   yyin = fopen( yytext, "r" );

		   if ( ! yyin )
		       error( ... );

		   yy_switch_to_buffer(
		       yy_create_buffer( yyin, YY_BUF_SIZE ) );

		   BEGIN(INITIAL);

		   }

	   <<EOF>> {
		   if ( --include_stack_ptr < 0 )
		       {
		       yyterminate();
		       }

		   else
		       {
		       yy_delete_buffer( YY_CURRENT_BUFFER );
		       yy_switch_to_buffer(
			    include_stack[include_stack_ptr] );
		       }
		   }

       Three  routines are available for setting up input buffers
       for scanning in-memory strings instead of files.	  All  of
       them  create  a	new input buffer for scanning the string,
       and return a corresponding YY_BUFFER_STATE  handle  (which
       you  should  delete with yy_delete_buffer() when done with
       it).   They  also  switch  to   the   new   buffer   using
       yy_switch_to_buffer(),  so  the	next call to yylex() will
       start scanning the string.

       yy_scan_string(const char *str)
	      scans a NUL-terminated string.

       yy_scan_bytes(const char *bytes, int len)
	      scans len bytes (including possibly NUL's) starting
	      at location bytes.

       Note  that  both of these functions create and scan a copy
       of the string or bytes.	(This  may  be	desirable,  since
       yylex()	modifies  the  contents of the buffer it is scan-
       ning.)  You can avoid the copy by using:

       yy_scan_buffer(char *base, yy_size_t size)
	      which scans in place the buffer starting	at  base,
	      consisting  of  size  bytes,  the last two bytes of
	      which must be  YY_END_OF_BUFFER_CHAR  (ASCII  NUL).
	      These  last  two bytes are not scanned; thus, scan-
	      ning  consists  of  base[0]  through  base[size-2],
	      inclusive.

	      If  you  fail  to set up base in this manner (i.e.,
	      forget the final two YY_END_OF_BUFFER_CHAR  bytes),
	      then yy_scan_buffer() returns a nil pointer instead
	      of creating a new input buffer.

	      The type yy_size_t is an integral type to which you
	      can  cast an integer expression reflecting the size
	      of the buffer.

END-OF-FILE RULES
       The special rule "<<EOF>>" indicates actions which are  to
       be  taken  when an end-of-file is encountered and yywrap()
       returns non-zero (i.e., indicates no further files to pro-
       cess).	The  action  must  finish  by  doing  one of four
       things:

       -      assigning yyin to a new  input  file  (in	 previous
	      versions	of  flex,  after doing the assignment you
	      had to call the special action YY_NEW_FILE; this is
	      no longer necessary);

       -      executing a return statement;

       -      executing the special yyterminate() action;

       -      or,    switching	  to	a    new   buffer   using
	      yy_switch_to_buffer()  as	 shown	in  the	  example
	      above.

       <<EOF>>	rules  may  not be used with other patterns; they
       may only be qualified with a list of start conditions.  If
       an  unqualified	<<EOF>>	 rule is given, it applies to all
       start  conditions  which	 do  not  already  have	  <<EOF>>
       actions.	  To specify an <<EOF>> rule for only the initial
       start condition, use

	   <INITIAL><<EOF>>

       These rules are useful for catching things  like	 unclosed
       comments.  An example:

	   %x quote
	   %%

	   ...other rules for dealing with quotes...

	   <quote><<EOF>>   {
		    error( "unterminated quote" );
		    yyterminate();
		    }
	   <<EOF>>  {
		    if ( *++filelist )
			yyin = fopen( *filelist, "r" );
		    else
		       yyterminate();
		    }

MISCELLANEOUS MACROS
       The  macro  YY_USER_ACTION  can	be  defined to provide an
       action which is	always	executed  prior	 to  the  matched
       rule's action.  For example, it could be #define'd to call
       a  routine  to  convert	yytext	 to   lower-case.    When
       YY_USER_ACTION  is  invoked, the variable yy_act gives the
       number of the matched rule (rules  are  numbered	 starting
       with  1).   Suppose  you want to profile how often each of
       your rules is matched.  The following would do the trick:

	   #define YY_USER_ACTION ++ctr[yy_act]

       where ctr is an array to hold the counts for the different
       rules.	Note  that the macro YY_NUM_RULES gives the total
       number of rules (including the default rule, even  if  you
       use -s), so a correct declaration for ctr is:

	   int ctr[YY_NUM_RULES];

       The macro YY_USER_INIT may be defined to provide an action
       which is always executed before the first scan (and before
       the  scanner's  internal	 initializations  are done).  For
       example, it could be used to call a routine to read  in	a
       data table or open a logging file.

       The  macro  yy_set_interactive(is_interactive) can be used
       to control whether the current buffer is considered inter-
       active.	 An  interactive buffer is processed more slowly,
       but must be used when the scanner's input source is indeed
       interactive  to	avoid  problems	 due  to  waiting to fill
       buffers (see the discussion of the -I flag below).  A non-
       zero  value  in	the  macro invocation marks the buffer as
       interactive, a zero value as non-interactive.   Note  that
       use  of this macro overrides %option always-interactive or
       %option	  never-interactive    (see    Options	  below).
       yy_set_interactive() must be invoked prior to beginning to
       scan the buffer that is	(or  is	 not)  to  be  considered
       interactive.

       The  macro  yy_set_bol(at_bol)  can  be	used  to  control
       whether the current buffer's scanning context for the next
       token  match is done as though at the beginning of a line.
       A non-zero macro argument makes rules anchored with

       The macro YY_AT_BOL()  returns  true  if	 the  next  token
       scanned	from  the  current  buffer  will  have	'^' rules
       active, false otherwise.

       In the generated scanner, the actions are all gathered  in
       one  large  switch statement and separated using YY_BREAK,
       which may be  redefined.	  By  default,	it  is	simply	a
       "break", to separate each rule's action from the following
       rule's.	Redefining  YY_BREAK  allows,  for  example,  C++
       users  to #define YY_BREAK to do nothing (while being very
       careful	that  every  rule  ends	 with  a  "break"  or	a
       "return"!)  to  avoid suffering from unreachable statement
       warnings where because a rule's action ends with "return",
       the YY_BREAK is inaccessible.

VALUES AVAILABLE TO THE USER
       This  section  summarizes  the various values available to
       the user in the rule actions.

       -      char *yytext holds the text of the  current  token.
	      It  may  be modified but not lengthened (you cannot
	      append characters to the end).

	      If the special  directive	 %array	 appears  in  the
	      first  section  of  the  scanner	description, then
	      yytext is	 instead  declared  char  yytext[YYLMAX],
	      where  YYLMAX  is	 a  macro definition that you can
	      redefine in the first section if you don't like the
	      default	value	(generally  8KB).   Using  %array
	      results in somewhat slower scanners, but the  value
	      of  yytext  becomes  immune to calls to input() and
	      unput(), which potentially destroy its  value  when
	      yytext  is  a  character	pointer.  The opposite of
	      %array is %pointer, which is the default.

	      You cannot use %array when generating  C++  scanner
	      classes (the -+ flag).

       -      int yyleng holds the length of the current token.

       -      FILE  *yyin is the file which by default flex reads
	      from.  It may be redefined but doing so only  makes
	      sense  before  scanning  begins or after an EOF has
	      been encountered.	 Changing  it  in  the	midst  of
	      scanning	will  have  unexpected results since flex
	      buffers its input; use yyrestart()  instead.   Once
	      scanning terminates because an end-of-file has been
	      seen, you can assign yyin at the new input file and
	      then call the scanner again to continue scanning.

       -      void  yyrestart(	FILE *new_file ) may be called to
	      point yyin at the new input file.	 The  switch-over
	      to  the  new  file  is  immediate	 (any  previously
	      buffered-up input	 is  lost).   Note  that  calling
	      yyrestart()  with	 yyin  as an argument thus throws
	      away the current input buffer and	 continues  scan-
	      ning the same input file.

       -      FILE  *yyout  is the file to which ECHO actions are
	      done.  It can be reassigned by the user.

       -      YY_CURRENT_BUFFER returns a YY_BUFFER_STATE  handle
	      to the current buffer.

       -      YY_START	returns an integer value corresponding to
	      the current start condition.  You can  subsequently
	      use  this	 value with BEGIN to return to that start
	      condition.

INTERFACING WITH YACC
       One of the main uses of flex is as a companion to the yacc
       parser-generator.   yacc	 parsers expect to call a routine
       named yylex() to find the next input token.   The  routine
       is  supposed  to return the type of the next token as well
       as putting any associated value in the global yylval.   To
       use flex with yacc, one specifies the -d option to yacc to
       instruct it to generate the file y.tab.h containing  defi-
       nitions	of  all	 the %tokens appearing in the yacc input.
       This file is then included in the flex scanner.	For exam-
       ple,  if	 one  of  the tokens is "TOK_NUMBER", part of the
       scanner might look like:

	   %{
	   #include "y.tab.h"
	   %}

	   %%

	   [0-9]+	 yylval = atoi( yytext ); return TOK_NUMBER;

OPTIONS
       flex has the following options:

       -b     Generate	backing-up  information	 to   lex.backup.
	      This  is	a  list	 of  scanner states which require
	      backing up and the input characters on  which  they
	      do  so.	By adding rules one can remove backing-up
	      states.  If all backing-up  states  are  eliminated
	      and  -Cf or -CF is used, the generated scanner will
	      run faster (see the -p flag).  Only users who  wish
	      to  squeeze  every last cycle out of their scanners
	      need worry about this option.  (See the section  on
	      Performance Considerations below.)

       -c     is  a  do-nothing,  deprecated  option included for
	      POSIX compliance.

       -d     makes the generated  scanner  run	 in  debug  mode.
	      Whenever	a  pattern  is	recognized and the global
	      yy_flex_debug is non-zero (which is  the	default),
	      the  scanner  will  write	 to  stderr a line of the
	      form:

		  --accepting rule at line 53 ("the matched text")

	      The line number refers to the location of the  rule
	      in  the  file  defining the scanner (i.e., the file
	      that was fed to flex).  Messages are also generated
	      when  the	 scanner  backs	 up,  accepts the default
	      rule, reaches the	 end  of  its  input  buffer  (or
	      encounters  a  NUL; at this point, the two look the
	      same as far as the scanner's concerned), or reaches
	      an end-of-file.

       -f     specifies	 fast  scanner.	  No table compression is
	      done and stdio is bypassed.  The	result	is  large
	      but  fast.   This option is equivalent to -Cfr (see
	      below).

       -h     generates a "help" summary  of  flex's  options  to
	      stdout and then exits.  -?  and --help are synonyms
	      for -h.

       -i     instructs flex to generate a case-insensitive scan-
	      ner.   The  case of letters given in the flex input
	      patterns will be ignored, and tokens in  the  input
	      will  be	matched	 regardless of case.  The matched
	      text given in yytext will have the  preserved  case
	      (i.e., it will not be folded).

       -l     turns  on	 maximum  compatibility with the original
	      AT&T lex implementation.	Note that this	does  not
	      mean  full compatibility.	 Use of this option costs
	      a considerable amount of performance, and it cannot
	      be  used	with the -+, -f, -F, -Cf, or -CF options.
	      For details on the compatibilities it provides, see
	      the  section "Incompatibilities With Lex And POSIX"
	      below.   This  option  also  results  in	the  name
	      YY_FLEX_LEX_COMPAT being #define'd in the generated
	      scanner.

       -n     is another do-nothing, deprecated	 option	 included
	      only for POSIX compliance.

       -p     generates	 a  performance	 report	 to  stderr.  The
	      report consists of comments regarding  features  of
	      the flex input file which will cause a serious loss
	      of performance in the resulting  scanner.	  If  you
	      give  the	 flag  twice,  you will also get comments
	      regarding features that lead to  minor  performance
	      losses.

	      Note  that the use of REJECT, %option yylineno, and
	      variable trailing context (see the  Deficiencies	/
	      Bugs  section  below) entails a substantial perfor-
	      mance penalty; use of yymore(), the ^ operator, and
	      the -I flag entail minor performance penalties.

       -s     causes  the  default  rule  (that unmatched scanner
	      input is echoed to stdout) to  be	 suppressed.   If
	      the  scanner  encounters	input that does not match
	      any of its rules, it aborts with	an  error.   This
	      option  is  useful for finding holes in a scanner's
	      rule set.

       -t     instructs flex to write the scanner it generates to
	      standard output instead of lex.yy.c.

       -v     specifies	 that  flex should write to stderr a sum-
	      mary of statistics regarding the scanner it  gener-
	      ates.   Most  of	the statistics are meaningless to
	      the casual flex user, but the first line identifies
	      the  version  of flex (same as reported by -V), and
	      the next line the flags used  when  generating  the
	      scanner, including those that are on by default.

       -w     suppresses warning messages.

       -B     instructs	 flex  to  generate  a batch scanner, the
	      opposite of interactive scanners	generated  by  -I
	      (see  below).   In general, you use -B when you are
	      certain that your scanner will never be used inter-
	      actively,	 and  you  want	 to squeeze a little more
	      performance out of it.  If your goal is instead  to
	      squeeze  out a lot more performance, you should  be
	      using the -Cf or	-CF  options  (discussed  below),
	      which turn on -B automatically anyway.

       -F     specifies	 that  the fast scanner table representa-
	      tion should be used  (and	 stdio	bypassed).   This
	      representation  is  about as fast as the full table
	      representation (-f), and for some sets of	 patterns
	      will  be	considerably  smaller  (and  for  others,
	      larger).	In general, if the pattern  set	 contains
	      both "keywords" and a catch-all, "identifier" rule,
	      such as in the set:

		  "case"    return TOK_CASE;
		  "switch"  return TOK_SWITCH;
		  ...
		  "default" return TOK_DEFAULT;
		  [a-z]+    return TOK_ID;

	      then you're better off using the full table  repre-
	      sentation.   If  only the "identifier" rule is pre-
	      sent and you then use a hash table or some such  to
	      detect the keywords, you're better off using -F.

	      This  option is equivalent to -CFr (see below).  It
	      cannot be used with -+.

       -I     instructs flex to generate an interactive	 scanner.
	      An interactive scanner is one that only looks ahead
	      to decide what token has been matched if	it  abso-
	      lutely  must.  It turns out that always looking one
	      extra character ahead,  even  if	the  scanner  has
	      already  seen  enough text to disambiguate the cur-
	      rent token, is a bit faster than only looking ahead
	      when  necessary.	 But  scanners	that  always look
	      ahead give dreadful  interactive	performance;  for
	      example,	when  a	 user  types a newline, it is not
	      recognized as a  newline	token  until  they  enter
	      another  token, which often means typing in another
	      whole line.

	      Flex scanners default to interactive unless you use
	      the  -Cf	or  -CF	 table-compression  options  (see
	      below).  That's because if you're looking for high-
	      performance  you	should	be  using  one	of  these
	      options, so  if  you  didn't,  flex  assumes  you'd
	      rather  trade off a bit of run-time performance for
	      intuitive interactive behavior.  Note also that you
	      cannot  use  -I  in  conjunction	with  -Cf or -CF.
	      Thus, this option is not really needed; it is on by
	      default for all those cases in which it is allowed.

	      You can force a scanner to not  be  interactive  by
	      using -B (see above).

       -L     instructs	 flex  not  to generate #line directives.
	      Without this option,  flex  peppers  the	generated
	      scanner  with #line directives so error messages in
	      the actions will be correctly located with  respect
	      to  either  the  original	 flex  input file (if the
	      errors are due to	 code  in  the	input  file),  or
	      lex.yy.c	(if  the  errors  are flex's fault -- you
	      should report these sorts of errors  to  the  email
	      address given below).

       -T     makes  flex  run in trace mode.  It will generate a
	      lot of messages to stderr concerning  the	 form  of
	      the  input  and the resultant non-deterministic and
	      deterministic  finite  automata.	 This  option  is
	      mostly for use in maintaining flex.

       -V     prints  the  version  number  to	stdout and exits.
	      --version is a synonym for -V.

       -7     instructs flex to generate a 7-bit  scanner,  i.e.,
	      one  which  can only recognized 7-bit characters in
	      its input.  The advantage of using -7 is	that  the
	      scanner's	 tables	 can  be  up  to half the size of
	      those generated using the -8  option  (see  below).
	      The  disadvantage	 is that such scanners often hang
	      or crash if their input contains an  8-bit  charac-
	      ter.

	      Note,  however, that unless you generate your scan-
	      ner using the -Cf or -CF table compression options,
	      use  of  -7  will save only a small amount of table
	      space, and  make	your  scanner  considerably  less
	      portable.	  Flex's  default behavior is to generate
	      an 8-bit scanner unless you use the -Cf or -CF,  in
	      which  case flex defaults to generating 7-bit scan-
	      ners unless your site was always configured to gen-
	      erate  8-bit  scanners  (as  will often be the case
	      with non-USA sites).  You	 can  tell  whether  flex
	      generated a 7-bit or an 8-bit scanner by inspecting
	      the flag summary in  the	-v  output  as	described
	      above.

	      Note that if you use -Cfe or -CFe (those table com-
	      pression	options,  but  also   using   equivalence
	      classes	as   discussed	see  below),  flex  still
	      defaults to generating an 8-bit scanner, since usu-
	      ally  with  these	 compression  options  full 8-bit
	      tables are  not  much  more  expensive  than  7-bit
	      tables.

       -8     instructs	 flex to generate an 8-bit scanner, i.e.,
	      one which can  recognize	8-bit  characters.   This
	      flag  is	only  needed for scanners generated using
	      -Cf or -CF, as otherwise flex defaults to	 generat-
	      ing an 8-bit scanner anyway.

	      See  the	discussion of -7 above for flex's default
	      behavior and the tradeoffs between 7-bit and  8-bit
	      scanners.

       -+     specifies	 that  you  want  flex	to generate a C++
	      scanner class.  See the section on  Generating  C++
	      Scanners below for details.

       -C[aefFmr]
	      controls	the degree of table compression and, more
	      generally, trade-offs between  small  scanners  and
	      fast scanners.

	      -Ca  ("align")  instructs	 flex to trade off larger
	      tables in the generated scanner for faster  perfor-
	      mance because the elements of the tables are better
	      aligned for memory access and computation.  On some
	      RISC architectures, fetching and manipulating long-
	      words is more  efficient	than  with  smaller-sized
	      units  such  as shortwords.  This option can double
	      the size of the tables used by your scanner.

	      -Ce directs flex to construct equivalence	 classes,
	      i.e., sets of characters which have identical lexi-
	      cal properties (for example, if the only appearance
	      of  digits  in  the  flex input is in the character
	      class "[0-9]" then the digits '0',  '1',	...,  '9'
	      will  all	 be  put  in the same equivalence class).
	      Equivalence classes usually  give	 dramatic  reduc-
	      tions  in	 the final table/object file sizes (typi-
	      cally a factor of 2-5) and are pretty cheap perfor-
	      mance-wise   (one	  array	  look-up  per	character
	      scanned).

	      -Cf specifies that the full scanner  tables  should
	      be  generated - flex should not compress the tables
	      by taking advantages of  similar	transition  func-
	      tions for different states.

	      -CF  specifies that the alternate fast scanner rep-
	      resentation (described above  under  the	-F  flag)
	      should  be  used.	  This option cannot be used with
	      -+.

	      -Cm  directs  flex  to  construct	 meta-equivalence
	      classes,	which are sets of equivalence classes (or
	      characters, if equivalence classes  are  not  being
	      used) that are commonly used together.  Meta-equiv-
	      alence classes are often a big win when using  com-
	      pressed  tables,	but  they have a moderate perfor-
	      mance impact (one or two "if" tests and  one  array
	      look-up per character scanned).

	      -Cr  causes  the generated scanner to bypass use of
	      the  standard  I/O  library  (stdio)   for   input.
	      Instead  of  calling fread() or getc(), the scanner
	      will use the read() system  call,	 resulting  in	a
	      performance  gain	 which varies from system to sys-
	      tem, but in general is probably  negligible  unless
	      you are also using -Cf or -CF.  Using -Cr can cause
	      strange behavior if, for	example,  you  read  from
	      yyin  using  stdio  prior	 to  calling  the scanner
	      (because the scanner will miss whatever  text  your
	      previous reads left in the stdio input buffer).

	      -Cr  has	no effect if you define YY_INPUT (see The
	      Generated Scanner above).

	      A lone -C specifies that the scanner tables  should
	      be  compressed  but neither equivalence classes nor
	      meta-equivalence classes should be used.

	      The options -Cf or -CF and -Cm do	 not  make  sense
	      together - there is no opportunity for meta-equiva-
	      lence classes if the table is not being compressed.
	      Otherwise	 the options may be freely mixed, and are
	      cumulative.

	      The default setting is -Cem, which  specifies  that
	      flex  should generate equivalence classes and meta-
	      equivalence classes.   This  setting  provides  the
	      highest degree of table compression.  You can trade
	      off faster-executing scanners at the cost of larger
	      tables with the following generally being true:

		  slowest & smallest
			-Cem
			-Cm
			-Ce
			-C
			-C{f,F}e
			-C{f,F}
			-C{f,F}a
		  fastest & largest

	      Note  that  scanners  with  the smallest tables are
	      usually generated and  compiled  the  quickest,  so
	      during development you will usually want to use the
	      default, maximal compression.

	      -Cfe is often a good compromise between  speed  and
	      size for production scanners.

       -ooutput
	      directs  flex to write the scanner to the file out-
	      put instead of lex.yy.c.	If you	combine	 -o  with
	      the  -t option, then the scanner is written to std-
	      out but its #line directives  (see  the  -L  option
	      above) refer to the file output.

       -Pprefix
	      changes  the default yy prefix used by flex for all
	      globally-visible variable	 and  function	names  to
	      instead  be prefix.  For example, -Pfoo changes the
	      name of yytext to footext.   It  also  changes  the
	      name  of	the  default output file from lex.yy.c to
	      lex.foo.c.  Here are all of the names affected:

		  yy_create_buffer
		  yy_delete_buffer
		  yy_flex_debug
		  yy_init_buffer
		  yy_flush_buffer
		  yy_load_buffer_state
		  yy_switch_to_buffer
		  yyin
		  yyleng
		  yylex
		  yylineno
		  yyout
		  yyrestart
		  yytext
		  yywrap

	      (If you are using a C++ scanner, then  only  yywrap
	      and yyFlexLexer are affected.)  Within your scanner
	      itself, you can still refer to the global variables
	      and  functions  using either version of their name;
	      but externally, they have the modified name.
	      This option lets you easily link together	 multiple
	      flex  programs  into  the	 same  executable.  Note,
	      though,  that  using  this  option   also	  renames
	      yywrap(),	 so  you now must either provide your own
	      (appropriately-named) version of	the  routine  for
	      your  scanner,  or use %option noyywrap, as linking
	      with  -lfl  no  longer  provides	one  for  you  by
	      default.

       -Sskeleton_file
	      overrides the default skeleton file from which flex
	      constructs its scanners.	You'll	never  need  this
	      option  unless  you  are	doing flex maintenance or
	      development.

       flex also provides a  mechanism	for  controlling  options
       within  the scanner specification itself, rather than from
       the flex command-line.  This is done by including  %option
       directives  in the first section of the scanner specifica-
       tion.  You can specify  multiple	 options  with	a  single
       %option	directive,  and	 multiple directives in the first
       section of your flex input file.

       Most options are given simply as	 names,	 optionally  pre-
       ceded by the word "no" (with no intervening whitespace) to
       negate their meaning.  A number	are  equivalent	 to  flex
       flags or their negation:

	   7bit		   -7 option
	   8bit		   -8 option
	   align	   -Ca option
	   backup	   -b option
	   batch	   -B option
	   c++		   -+ option

	   caseful or
	   case-sensitive  opposite of -i (default)

	   case-insensitive or
	   caseless	   -i option

	   debug	   -d option
	   default	   opposite of -s option
	   ecs		   -Ce option
	   fast		   -F option
	   full		   -f option
	   interactive	   -I option
	   lex-compat	   -l option
	   meta-ecs	   -Cm option
	   perf-report	   -p option
	   read		   -Cr option
	   stdout	   -t option
	   verbose	   -v option
	   warn		   opposite of -w option

			   (use "%option nowarn" for -w)

	   array	   equivalent to "%array"
	   pointer	   equivalent to "%pointer" (default)

       Some %option's provide features otherwise not available:

       always-interactive
	      instructs	 flex  to generate a scanner which always
	      considers its input  "interactive".   Normally,  on
	      each  new	 input file the scanner calls isatty() in
	      an attempt to determine whether the scanner's input
	      source  is  interactive  and  thus should be read a
	      character at a time.  When  this	option	is  used,
	      however, then no such call is made.

       main   directs  flex  to	 provide a default main() program
	      for the scanner, which simply calls yylex().   This
	      option implies noyywrap (see below).

       never-interactive
	      instructs	 flex  to  generate a scanner which never
	      considers its input "interactive" (again,	 no  call
	      made to isatty()).  This is the opposite of always-
	      interactive.

       stack  enables the use  of  start  condition  stacks  (see
	      Start Conditions above).

       stdinit
	      if set (i.e., %option stdinit) initializes yyin and
	      yyout to stdin and stdout, instead of  the  default
	      of  nil.	Some existing lex programs depend on this
	      behavior, even though it is not compliant with ANSI
	      C,  which	 does  not require stdin and stdout to be
	      compile-time constant.

       yylineno
	      directs flex to generate a scanner  that	maintains
	      the  number of the current line read from its input
	      in the global variable yylineno.	 This  option  is
	      implied by %option lex-compat.

       yywrap if  unset (i.e., %option noyywrap), makes the scan-
	      ner not call yywrap() upon an end-of-file, but sim-
	      ply  assume  that	 there	are no more files to scan
	      (until the user points yyin at a new file and calls
	      yylex() again).

       flex  scans your rule actions to determine whether you use
       the REJECT or yymore() features.	 The  reject  and  yymore
       options	are  available	to  override  its  decision as to
       whether you use the options, either by setting them (e.g.,
       %option reject) to indicate the feature is indeed used, or
       unsetting them to indicate it actually is not used  (e.g.,
       %option noyymore).

       Three  options  take  string-delimited values, offset with
       '=':

	   %option outfile="ABC"

       is equivalent to -oABC, and

	   %option prefix="XYZ"

       is equivalent to -PXYZ.	Finally,

	   %option yyclass="foo"

       only applies when generating a C++ scanner (  -+	 option).
       It informs flex that you have derived foo as a subclass of
       yyFlexLexer, so flex will place your actions in the member
       function foo::yylex() instead of yyFlexLexer::yylex().  It
       also generates a yyFlexLexer::yylex() member function that
       emits  a	 run-time  error (by invoking yyFlexLexer::Lexer-
       Error()) if called.  See Generating C++	Scanners,  below,
       for additional information.

       A  number  of  options  are available for lint purists who
       want to suppress the appearance of  unneeded  routines  in
       the  generated  scanner.	  Each of the following, if unset
       (e.g., %option nounput ),  results  in  the  corresponding
       routine not appearing in the generated scanner:

	   input, unput
	   yy_push_state, yy_pop_state, yy_top_state
	   yy_scan_buffer, yy_scan_bytes, yy_scan_string

       (though	yy_push_state()	 and  friends won't appear anyway
       unless you use %option stack).

PERFORMANCE CONSIDERATIONS
       The main design goal of flex is that it generate high-per-
       formance scanners.  It has been optimized for dealing well
       with large sets of rules.  Aside from the effects on scan-
       ner  speed  of  the  table compression -C options outlined
       above, there are a number of options/actions which degrade
       performance.  These are, from most expensive to least:

	   REJECT
	   %option yylineno
	   arbitrary trailing context

	   pattern sets that require backing up
	   %array
	   %option interactive
	   %option always-interactive

	   '^' beginning-of-line operator
	   yymore()

       with  the  first	 three	all being quite expensive and the
       last two being quite cheap.  Note  also	that  unput()  is
       implemented  as a routine call that potentially does quite
       a bit of work, while yyless() is a quite-cheap  macro;  so
       if  just	 putting  back	some excess text you scanned, use
       yyless().

       REJECT should be avoided at all costs when performance  is
       important.  It is a particularly expensive option.

       Getting	rid  of	 backing  up is messy and often may be an
       enormous amount of work for  a  complicated  scanner.   In
       principal,  one	begins by using the -b flag to generate a
       lex.backup file.	 For example, on the input

	   %%
	   foo	      return TOK_KEYWORD;
	   foobar     return TOK_KEYWORD;

       the file looks like:

	   State #6 is non-accepting -
	    associated rule line numbers:
		  2	  3
	    out-transitions: [ o ]
	    jam-transitions: EOF [ \001-n  p-\177 ]

	   State #8 is non-accepting -
	    associated rule line numbers:
		  3
	    out-transitions: [ a ]
	    jam-transitions: EOF [ \001-`  b-\177 ]

	   State #9 is non-accepting -
	    associated rule line numbers:
		  3
	    out-transitions: [ r ]
	    jam-transitions: EOF [ \001-q  s-\177 ]

	   Compressed tables always back up.

       The first few lines tell us that there's a  scanner  state
       in which it can make a transition on an 'o' but not on any
       other character, and that  in  that  state  the	currently
       scanned	text  does  not match any rule.	 The state occurs
       when trying to match the rules found at lines 2 and  3  in
       the  input file.	 If the scanner is in that state and then
       reads something other than an 'o', it will have to back up
       to  find	 a  rule  which	 is matched.  With a bit of head-
       scratching one can see that this must be the state it's in
       when  it	 has  seen  "fo".   When  this	has  happened, if
       anything other than another 'o' is seen, the scanner  will
       have  to	 back  up to simply match the 'f' (by the default
       rule).

       The comment regarding State #8 indicates there's a problem
       when  "foob"  has  been scanned.	 Indeed, on any character
       other than an 'a', the scanner will have	 to  back  up  to
       accept  "foo".	Similarly,  the comment for State #9 con-
       cerns when "fooba" has been scanned and an  'r'	does  not
       follow.

       The  final  comment reminds us that there's no point going
       to all the trouble of removing backing up from  the  rules
       unless  we're  using  -Cf or -CF, since there's no perfor-
       mance gain doing so with compressed scanners.

       The way to remove the backing up is to add "error" rules:

	   %%
	   foo	       return TOK_KEYWORD;
	   foobar      return TOK_KEYWORD;

	   fooba       |
	   foob	       |
	   fo	       {
		       /* false alarm, not really a keyword */
		       return TOK_ID;
		       }

       Eliminating backing up among a list of keywords	can  also
       be done using a "catch-all" rule:

	   %%
	   foo	       return TOK_KEYWORD;
	   foobar      return TOK_KEYWORD;

	   [a-z]+      return TOK_ID;

       This is usually the best solution when appropriate.

       Backing	up  messages tend to cascade.  With a complicated
       set of rules it's not uncommon to  get  hundreds	 of  mes-
       sages.	If  one	 can decipher them, though, it often only
       takes a dozen or so rules  to  eliminate	 the  backing  up
       (though it's easy to make a mistake and have an error rule
       accidentally match a valid token.  A possible future  flex
       feature	will  be  to automatically add rules to eliminate
       backing up).

       It's important to keep in mind that you gain the	 benefits
       of  eliminating	backing	 up  only  if you eliminate every
       instance of backing up.	Leaving just one means	you  gain
       nothing.

       Variable	 trailing  context  (where  both  the leading and
       trailing parts do not have a fixed length) entails  almost
       the  same  performance loss as REJECT (i.e., substantial).
       So when possible a rule like:

	   %%
	   mouse|rat/(cat|dog)	 run();

       is better written:

	   %%
	   mouse/cat|dog	 run();
	   rat/cat|dog		 run();

       or as

	   %%
	   mouse|rat/cat	 run();
	   mouse|rat/dog	 run();

       Note that here the special '|' action does not provide any
       savings,	 and can even make things worse (see Deficiencies
       / Bugs below).

       Another area where the user can increase a scanner's  per-
       formance	 (and one that's easier to implement) arises from
       the fact that the longer the tokens  matched,  the  faster
       the  scanner  will  run.	 This is because with long tokens
       the processing of most input characters takes place in the
       (short) inner scanning loop, and does not often have to go
       through the additional work of  setting	up  the	 scanning
       environment  (e.g.,  yytext)  for  the action.  Recall the
       scanner for C comments:

	   %x comment
	   %%
		   int line_num = 1;

	   "/*"		BEGIN(comment);

	   <comment>[^*\n]*
	   <comment>"*"+[^*/\n]*
	   <comment>\n		   ++line_num;
	   <comment>"*"+"/"	   BEGIN(INITIAL);

       This could be sped up by writing it as:

	   %x comment
	   %%
		   int line_num = 1;

	   "/*"		BEGIN(comment);

	   <comment>[^*\n]*
	   <comment>[^*\n]*\n	   ++line_num;
	   <comment>"*"+[^*/\n]*
	   <comment>"*"+[^*/\n]*\n ++line_num;
	   <comment>"*"+"/"	   BEGIN(INITIAL);

       Now instead of each newline requiring  the  processing  of
       another	action, recognizing the newlines is "distributed"
       over the other rules to keep the matched text as	 long  as
       possible.   Note	 that adding rules does not slow down the
       scanner!	 The speed of the scanner is independent  of  the
       number of rules or (modulo the considerations given at the
       beginning of this section) how complicated the  rules  are
       with regard to operators such as '*' and '|'.

       A final example in speeding up a scanner: suppose you want
       to scan through a file  containing  identifiers	and  key-
       words,  one  per line and with no other extraneous charac-
       ters, and recognize all the  keywords.	A  natural  first
       approach is:

	   %%
	   asm	    |
	   auto	    |
	   break    |
	   ... etc ...
	   volatile |
	   while    /* it's a keyword */

	   .|\n	    /* it's not a keyword */

       To  eliminate  the  back-tracking,  introduce  a catch-all
       rule:

	   %%
	   asm	    |
	   auto	    |
	   break    |
	   ... etc ...
	   volatile |
	   while    /* it's a keyword */

	   [a-z]+   |
	   .|\n	    /* it's not a keyword */

       Now, if it's guaranteed that there's exactly one word  per
       line,  then we can reduce the total number of matches by a
       half by merging in the recognition of newlines  with  that
       of the other tokens:

	   %%
	   asm\n    |
	   auto\n   |
	   break\n  |
	   ... etc ...
	   volatile\n |
	   while\n  /* it's a keyword */

	   [a-z]+\n |
	   .|\n	    /* it's not a keyword */

       One  has	 to  be careful here, as we have now reintroduced
       backing up into the scanner.  In particular, while we know
       that  there  will  never	 be  any  characters in the input
       stream other than letters or newlines, flex  can't  figure
       this out, and it will plan for possibly needing to back up
       when it has scanned a token like "auto" and then the  next
       character  is  something other than a newline or a letter.
       Previously it would then just match the "auto" rule and be
       done, but now it has no "auto" rule, only a "auto\n" rule.
       To eliminate the	 possibility  of  backing  up,	we  could
       either duplicate all rules but without final newlines, or,
       since we never expect  to  encounter  such  an  input  and
       therefore  don't how it's classified, we can introduce one
       more catch-all rule, this one which doesn't include a new-
       line:

	   %%
	   asm\n    |
	   auto\n   |
	   break\n  |
	   ... etc ...
	   volatile\n |
	   while\n  /* it's a keyword */

	   [a-z]+\n |
	   [a-z]+   |
	   .|\n	    /* it's not a keyword */

       Compiled	 with -Cf, this is about as fast as one can get a
       flex scanner to go for this particular problem.

       A final note: flex is slow when matching	 NUL's,	 particu-
       larly  when a token contains multiple NUL's.  It's best to
       write rules which match short  amounts  of  text	 if  it's
       anticipated that the text will often include NUL's.

       Another	final  note  regarding	performance: as mentioned
       above in the section How the Input is Matched, dynamically
       resizing	 yytext to accommodate huge tokens is a slow pro-
       cess because it presently requires that the  (huge)  token
       be  rescanned  from the beginning.  Thus if performance is
       vital, you should attempt to match "large"  quantities  of
       text  but  not "huge" quantities, where the cutoff between
       the two is at about 8K characters/token.

GENERATING C++ SCANNERS
       flex provides two different ways to generate scanners  for
       use  with  C++.	 The  first  way  is  to simply compile a
       scanner generated by flex using a C++ compiler instead  of
       a  C  compiler.	You should not encounter any compilations
       errors (please report any you find to  the  email  address
       given  in the Author section below).  You can then use C++
       code in your rule actions instead of C  code.   Note  that
       the  default  input  source for your scanner remains yyin,
       and default echoing is still done to yyout.  Both of these
       remain FILE * variables and not C++ streams.

       You  can	 also  use  flex to generate a C++ scanner class,
       using the -+ option (or, equivalently, %option c++), which
       is  automatically  specified  if the name of the flex exe-
       cutable ends in a '+', such as flex++.	When  using  this
       option,	flex  defaults	to  generating the scanner to the
       file lex.yy.cc instead of lex.yy.c.  The generated scanner
       includes	 the  header  file FlexLexer.h, which defines the
       interface to two C++ classes.

       The first class,	 FlexLexer,  provides  an  abstract  base
       class  defining	the  general scanner class interface.  It
       provides the following member functions:

       const char* YYText()
	      returns the  text	 of  the  most	recently  matched
	      token, the equivalent of yytext.

       int YYLeng()
	      returns  the  length  of	the most recently matched
	      token, the equivalent of yyleng.

       int lineno() const
	      returns the current input line number (see  %option
	      yylineno), or 1 if %option yylineno was not used.

       void set_debug( int flag )
	      sets the debugging flag for the scanner, equivalent
	      to assigning to yy_flex_debug (see the Options sec-
	      tion  above).  Note that you must build the scanner
	      using %option debug to include  debugging	 informa-
	      tion in it.

       int debug() const
	      returns  the current setting of the debugging flag.

       Also  provided  are   member   functions	  equivalent   to
       yy_switch_to_buffer(),	yy_create_buffer()   (though  the
       first argument is an istream* object  pointer  and  not	a
       FILE*),	  yy_flush_buffer(),	yy_delete_buffer(),   and
       yyrestart() (again,  the	 first	argument  is  a	 istream*
       object pointer).

       The  second  class  defined in FlexLexer.h is yyFlexLexer,
       which is derived from FlexLexer.	 It defines the following
       additional member functions:

       yyFlexLexer( istream* arg_yyin = 0, ostream* arg_yyout = 0
	      )
	      constructs a yyFlexLexer	object	using  the  given
	      streams  for  input  and output.	If not specified,
	      the streams default to cin and cout,  respectively.

       virtual int yylex()
	      performs the same role is yylex() does for ordinary
	      flex scanners: it scans the input stream, consuming
	      tokens,  until a rule's action returns a value.  If
	      you derive a subclass S from yyFlexLexer	and  want
	      to  access  the member functions and variables of S
	      inside  yylex(),	then  you  need	 to  use  %option
	      yyclass="S"  to  inform flex that you will be using
	      that subclass  instead  of  yyFlexLexer.	 In  this
	      case,  rather than generating yyFlexLexer::yylex(),
	      flex generates S::yylex()	 (and  also  generates	a
	      dummy	 yyFlexLexer::yylex()	   that	    calls
	      yyFlexLexer::LexerError() if called).

       virtual void switch_streams(istream* new_in = 0,
	      ostream* new_out = 0) reassigns yyin to new_in  (if
	      non-nil) and yyout to new_out (ditto), deleting the
	      previous input buffer if yyin is reassigned.

       int yylex( istream* new_in, ostream* new_out = 0 )
	      first    switches	   the	  input	   streams    via
	      switch_streams(  new_in, new_out ) and then returns
	      the value of yylex().

       In addition, yyFlexLexer defines the  following	protected
       virtual	functions  which  you  can  redefine  in  derived
       classes to tailor the scanner:

       virtual int LexerInput( char* buf, int max_size )
	      reads  up	 to  max_size  characters  into	 buf  and
	      returns the number of characters read.  To indicate
	      end-of-input,  return  0	characters.   Note   that
	      "interactive"  scanners  (see  the -B and -I flags)
	      define the macro YY_INTERACTIVE.	If  you	 redefine
	      LexerInput()  and	 need  to  take different actions
	      depending on whether or not the  scanner	might  be
	      scanning	an interactive input source, you can test
	      for the presence of this name via #ifdef.

       virtual void LexerOutput( const char* buf, int size )
	      writes out size characters  from	the  buffer  buf,
	      which,   while  NUL-terminated,  may  also  contain
	      "internal" NUL's if the scanner's rules  can  match
	      text with NUL's in them.

       virtual void LexerError( const char* msg )
	      reports a fatal error message.  The default version
	      of this function writes the message to  the  stream
	      cerr and exits.

       Note  that  a yyFlexLexer object contains its entire scan-
       ning state.  Thus you can use such objects to create reen-
       trant scanners.	You can instantiate multiple instances of
       the same yyFlexLexer class, and you can also combine  mul-
       tiple  C++  scanner  classes  together in the same program
       using the -P option discussed above.

       Finally, note that the %array feature is not available  to
       C++  scanner classes; you must use %pointer (the default).

       Here is an example of a simple C++ scanner:

	       // An example of using the flex C++ scanner class.

	   %{
	   int mylineno = 0;
	   %}

	   string  \"[^\n"]+\"

	   ws	   [ \t]+

	   alpha   [A-Za-z]
	   dig	   [0-9]
	   name	   ({alpha}|{dig}|\$)({alpha}|{dig}|[_.\-/$])*
	   num1	   [-+]?{dig}+\.?([eE][-+]?{dig}+)?
	   num2	   [-+]?{dig}*\.{dig}+([eE][-+]?{dig}+)?
	   number  {num1}|{num2}

	   %%

	   {ws}	   /* skip blanks and tabs */

	   "/*"	   {
		   int c;

		   while((c = yyinput()) != 0)
		       {
		       if(c == '\n')
			   ++mylineno;

		       else if(c == '*')
			   {
			   if((c = yyinput()) == '/')
			       break;
			   else
			       unput(c);
			   }
		       }
		   }

	   {number}  cout << "number " << YYText() << '\n';

	   \n	     mylineno++;

	   {name}    cout << "name " << YYText() << '\n';

	   {string}  cout << "string " << YYText() << '\n';

	   %%

	   int main( int /* argc */, char** /* argv */ )
	       {
	       FlexLexer* lexer = new yyFlexLexer;
	       while(lexer->yylex() != 0)
		   ;
	       return 0;
	       }
       If you want to create multiple (different) lexer	 classes,
       you use the -P flag (or the prefix= option) to rename each
       yyFlexLexer to  some  other  xxFlexLexer.   You	then  can
       include <FlexLexer.h> in your other sources once per lexer
       class, first renaming yyFlexLexer as follows:

	   #undef yyFlexLexer
	   #define yyFlexLexer xxFlexLexer
	   #include <FlexLexer.h>

	   #undef yyFlexLexer
	   #define yyFlexLexer zzFlexLexer
	   #include <FlexLexer.h>

       if, for example, you used %option prefix="xx" for  one  of
       your scanners and %option prefix="zz" for the other.

       IMPORTANT:  the	present	 form  of  the	scanning class is
       experimental and may  change  considerably  between  major
       releases.

INCOMPATIBILITIES WITH LEX AND POSIX
       flex  is	 a  rewrite  of	 the  AT&T Unix lex tool (the two
       implementations do not share any code, though), with  some
       extensions  and	incompatibilities,  both  of which are of
       concern to those who wish to write scanners acceptable  to
       either  implementation.	 Flex is fully compliant with the
       POSIX lex specification, except that when  using	 %pointer
       (the  default), a call to unput() destroys the contents of
       yytext, which is counter to the POSIX specification.

       In this section we discuss  all	of  the	 known	areas  of
       incompatibility	between	 flex,	AT&T  lex,  and the POSIX
       specification.

       flex's -l option turns on maximum compatibility	with  the
       original	 AT&T  lex implementation, at the cost of a major
       loss in the  generated  scanner's  performance.	 We  note
       below which incompatibilities can be overcome using the -l
       option.

       flex is fully  compatible  with	lex  with  the	following
       exceptions:

       -      The  undocumented	 lex  scanner  internal	 variable
	      yylineno is not  supported  unless  -l  or  %option
	      yylineno is used.

	      yylineno	should	be  maintained	on  a  per-buffer
	      basis, rather than  a  per-scanner  (single  global
	      variable) basis.

	      yylineno is not part of the POSIX specification.

       -      The  input()  routine is not redefinable, though it
	      may be called to read characters following whatever
	      has  been matched by a rule.  If input() encounters
	      an end-of-file the normal	 yywrap()  processing  is
	      done.    A  ``real''  end-of-file	 is  returned  by
	      input() as EOF.

	      Input  is	 instead  controlled  by   defining   the
	      YY_INPUT macro.

	      The  flex	 restriction that input() cannot be rede-
	      fined is in accordance with  the	POSIX  specifica-
	      tion, which simply does not specify any way of con-
	      trolling the scanner's input other than  by  making
	      an initial assignment to yyin.

       -      The  unput()  routine  is	 not  redefinable.   This
	      restriction is in accordance with POSIX.

       -      flex scanners are not as reentrant as lex scanners.
	      In  particular,  if you have an interactive scanner
	      and an interrupt handler which  long-jumps  out  of
	      the scanner, and the scanner is subsequently called
	      again, you may get the following message:

		  fatal flex scanner internal error--end of buffer missed

	      To reenter the scanner, first use

		  yyrestart( yyin );

	      Note that this call will throw  away  any	 buffered
	      input;  usually this isn't a problem with an inter-
	      active scanner.

	      Also note that flex C++ scanner classes  are  reen-
	      trant,  so  if  using C++ is an option for you, you
	      should use them instead.	See "Generating C++ Scan-
	      ners" above for details.

       -      output()	is  not	 supported.  Output from the ECHO
	      macro is done to the  file-pointer  yyout	 (default
	      stdout).

	      output() is not part of the POSIX specification.

       -      lex  does	 not  support  exclusive start conditions
	      (%x), though they are in the POSIX specification.

       -      When definitions are expanded, flex  encloses  them
	      in parentheses.  With lex, the following:

		  NAME	  [A-Z][A-Z0-9]*
		  %%
		  foo{NAME}?	  printf( "Found it\n" );
		  %%

	      will  not	 match	the string "foo" because when the
	      macro is expanded the rule is equivalent to "foo[A-
	      Z][A-Z0-9]*?"   and the precedence is such that the
	      '?' is associated with "[A-Z0-9]*".  With flex, the
	      rule will be expanded to "foo([A-Z][A-Z0-9]*)?" and
	      so the string "foo" will match.

	      Note that if the definition begins with ^	 or  ends
	      with $ then it is not expanded with parentheses, to
	      allow these  operators  to  appear  in  definitions
	      without  losing  their  special  meanings.  But the
	      <s>, /, and <<EOF>> operators cannot be used  in	a
	      flex definition.

	      Using  -l	 results in the lex behavior of no paren-
	      theses around the definition.

	      The POSIX specification is that the  definition  be
	      enclosed in parentheses.

       -      Some  implementations  of lex allow a rule's action
	      to begin on a separate line, if the rule's  pattern
	      has trailing whitespace:

		  %%
		  foo|bar<space here>
		    { foobar_action(); }

	      flex does not support this feature.

       -      The  lex	%r  (generate a Ratfor scanner) option is
	      not supported.  It is not part of the POSIX  speci-
	      fication.

       -      After  a call to unput(), yytext is undefined until
	      the next token is matched, unless the  scanner  was
	      built  using %array.  This is not the case with lex
	      or the POSIX specification.   The	 -l  option  does
	      away with this incompatibility.

       -      The  precedence  of the {} (numeric range) operator
	      is different.  lex interprets "abc{1,3}" as  "match
	      one,  two,  or three occurrences of 'abc'", whereas
	      flex interprets it as "match 'ab' followed by  one,
	      two,  or	three occurrences of 'c'".  The latter is
	      in agreement with the POSIX specification.

       -      The precedence of the ^ operator is different.  lex
	      interprets "^foo|bar" as "match either 'foo' at the
	      beginning of a line, or  'bar'  anywhere",  whereas
	      flex  interprets it as "match either 'foo' or 'bar'
	      if they come at the beginning of a line".	 The lat-
	      ter is in agreement with the POSIX specification.

       -      The special table-size declarations such as %a sup-
	      ported by lex are not required  by  flex	scanners;
	      flex ignores them.

       -      The  name FLEX_SCANNER is #define'd so scanners may
	      be written for use with either flex or lex.   Scan-
	      ners   also   include   YY_FLEX_MAJOR_VERSION   and
	      YY_FLEX_MINOR_VERSION indicating which  version  of
	      flex  generated  the  scanner (for example, for the
	      2.5 release, these defines would be 2 and 5 respec-
	      tively).

       The following flex features are not included in lex or the
       POSIX specification:

	   C++ scanners
	   %option
	   start condition scopes
	   start condition stacks
	   interactive/non-interactive scanners
	   yy_scan_string() and friends
	   yyterminate()
	   yy_set_interactive()
	   yy_set_bol()
	   YY_AT_BOL()
	   <<EOF>>
	   <*>
	   YY_DECL
	   YY_START
	   YY_USER_ACTION
	   YY_USER_INIT
	   #line directives
	   %{}'s around actions
	   multiple actions on a line

       plus almost all of the flex flags.  The	last  feature  in
       the  list  refers  to  the fact that with flex you can put
       multiple actions on the same line,  separated  with  semi-
       colons, while with lex, the following

	   foo	  handle_foo(); ++num_foos_seen;

       is (rather surprisingly) truncated to

	   foo	  handle_foo();

       flex  does  not truncate the action.  Actions that are not
       enclosed in braces are simply terminated at the end of the
       line.

DIAGNOSTICS
       warning,	 rule  cannot be matched indicates that the given
       rule cannot be matched because it follows other rules that
       will  always  match  the same text as it.  For example, in
       the following "foo" cannot be  matched  because	it  comes
       after an identifier "catch-all" rule:

	   [a-z]+    got_identifier();
	   foo	     got_foo();

       Using REJECT in a scanner suppresses this warning.

       warning,	 -s  option given but default rule can be matched
       means that it is possible (perhaps only	in  a  particular
       start  condition)  that the default rule (match any single
       character) is the only one that will  match  a  particular
       input.	Since  -s  was	given,	presumably  this  is  not
       intended.

       reject_used_but_not_detected	    undefined	       or
       yymore_used_but_not_detected  undefined - These errors can
       occur at compile time.  They  indicate  that  the  scanner
       uses REJECT or yymore() but that flex failed to notice the
       fact, meaning that flex scanned	the  first  two	 sections
       looking	for  occurrences  of  these actions and failed to
       find any, but somehow you snuck some in	(via  a	 #include
       file,  for example).  Use %option reject or %option yymore
       to indicate to flex that you really do use these features.

       flex  scanner  jammed  -	 a  scanner  compiled with -s has
       encountered an input string which wasn't matched by any of
       its  rules.   This  error  can  also occur due to internal
       problems.

       token too large, exceeds YYLMAX - your scanner uses %array
       and one of its rules matched a string longer than the YYL-
       MAX constant (8K bytes by default).  You can increase  the
       value  by #define'ing YYLMAX in the definitions section of
       your flex input.

       scanner requires -8 flag to use the character 'x'  -  Your
       scanner specification includes recognizing the 8-bit char-
       acter 'x' and you did not specify the -8	 flag,	and  your
       scanner defaulted to 7-bit because you used the -Cf or -CF
       table compression options.  See the discussion of  the  -7
       flag for details.

       flex scanner push-back overflow - you used unput() to push
       back so much text that the scanner's buffer could not hold
       both the pushed-back text and the current token in yytext.
       Ideally the scanner should dynamically resize  the  buffer
       in this case, but at present it does not.

       input  buffer overflow, can't enlarge buffer because scan-
       ner uses REJECT - the scanner was working on  matching  an
       extremely  large	 token	and  needed  to	 expand the input
       buffer.	This doesn't work with scanners that use  REJECT.

       fatal  flex scanner internal error--end of buffer missed -
       This can occur in an scanner which is  reentered	 after	a
       long-jump  has  jumped out (or over) the scanner's activa-
       tion frame.  Before reentering the scanner, use:

	   yyrestart( yyin );

       or, as noted above, switch to using the C++ scanner class.

       too  many  start	 conditions in <> construct! - you listed
       more start conditions in a <> construct than exist (so you
       must have listed at least one of them twice).

FILES
       -lfl   library with which scanners must be linked.

       lex.yy.c
	      generated scanner (called lexyy.c on some systems).

       lex.yy.cc
	      generated C++ scanner class, when using -+.

       <FlexLexer.h>
	      header file defining the C++  scanner  base  class,
	      FlexLexer, and its derived class, yyFlexLexer.

       flex.skl
	      skeleton	scanner.   This	 file  is  only used when
	      building flex, not when flex executes.

       lex.backup
	      backing-up information for -b flag (called  lex.bck
	      on some systems).

DEFICIENCIES / BUGS
       Some  trailing context patterns cannot be properly matched
       and  generate  warning	messages   ("dangerous	 trailing
       context").   These  are	patterns  where the ending of the
       first part of the rule matches the beginning of the second
       part, such as "zx*/xy*", where the 'x*' matches the 'x' at
       the beginning of the trailing  context.	 (Note	that  the
       POSIX  draft states that the text matched by such patterns
       is undefined.)

       For some trailing context rules, parts which are	 actually
       fixed-length  are  not  recognized as such, leading to the
       abovementioned performance  loss.   In  particular,  parts
       using  '|' or {n} (such as "foo{3}") are always considered
       variable-length.

       Combining trailing context with the special '|' action can
       result  in  fixed  trailing  context being turned into the
       more expensive variable trailing context.  For example, in
       the following:

	   %%
	   abc	    |
	   xyz/def

       Use  of	unput() invalidates yytext and yyleng, unless the
       %array directive or the -l option has been used.

       Pattern-matching of NUL's  is  substantially  slower  than
       matching other characters.

       Dynamic	resizing  of  the  input  buffer  is  slow, as it
       entails rescanning all the text matched so far by the cur-
       rent (generally huge) token.

       Due  to both buffering of input and read-ahead, you cannot
       intermix calls to <stdio.h> routines, such as,  for  exam-
       ple,  getchar(),	 with  flex  rules and expect it to work.
       Call input() instead.

       The total table entries listed by the -v flag excludes the
       number  of table entries needed to determine what rule has
       been matched.  The number of entries is equal to the  num-
       ber  of DFA states if the scanner does not use REJECT, and
       somewhat greater than the number of states if it does.

       REJECT cannot be used with the -f or -F options.

       The flex internal algorithms need documentation.

SEE ALSO
       lex(1), yacc(1), sed(1), awk(1).

       John Levine, Tony Mason,	 and  Doug  Brown,  Lex	 &  Yacc,
       O'Reilly	 and Associates.  Be sure to get the 2nd edition.
       M. E. Lesk and E. Schmidt, LEX - Lexical Analyzer  Genera-
       tor

       Alfred  Aho,  Ravi  Sethi  and  Jeffrey Ullman, Compilers:
       Principles, Techniques and Tools,  Addison-Wesley  (1986).
       Describes  the  pattern-matching	 techniques  used by flex
       (deterministic finite automata).

AUTHOR
       Vern Paxson, with the help of many ideas and much inspira-
       tion   from   Van   Jacobson.   Original	 version  by  Jef
       Poskanzer.  The fast table  representation  is  a  partial
       implementation  of  a  design  done  by Van Jacobson.  The
       implementation was done by Kevin Gong and Vern Paxson.

       Thanks to the many  flex	 beta-testers,	feedbackers,  and
       contributors,  especially  Francois  Pinard, Casey Leedom,
       Robert  Abramovitz,  Stan  Adermann,  Terry  Allen,  David
       Barker-Plummer,	John  Basrai,  Neal  Becker,  Nelson H.F.
       Beebe, benson@odi.com, Karl Berry, Peter A.  Bigot,  Simon
       Blanchard,  Keith  Bostic,  Frederic Brehm, Ian Brockbank,
       Kin Cho, Nick Christopher, Brian	 Clapper,  J.T.	 Conklin,
       Jason Coughlin, Bill Cox, Nick Cropper, Dave Curtis, Scott
       David Daniels, Chris G. Demetriou, Theo Deraadt, Mike Don-
       ahue,  Chuck  Doucette, Tom Epperly, Leo Eskin, Chris Fay-
       lor, Chris Flatters,  Jon  Forrest,  Jeffrey  Friedl,  Joe
       Gayda,  Kaveh  R.  Ghazi,  Wolfgang  Glunz,  Eric Goldman,
       Christopher M. Gould, Ulrich  Grepel,  Peer  Griebel,  Jan
       Hajic,  Charles	Hemphill,  NORO Hideo, Jarkko Hietaniemi,
       Scott Hofmann, Jeff Honig, Dana Hudes, Eric  Hughes,  John
       Interrante, Ceriel Jacobs, Michal Jaegermann, Sakari Jalo-
       vaara, Jeffrey R.  Jones,  Henry	 Juengst,  Klaus  Kaempf,
       Jonathan	  I.   Kamens,	 Terrence   O  Kane,  Amir  Katz,
       ken@ken.hilco.com, Kevin B. Kenny, Steve Kirsch,	 Winfried
       Koenig,	Marq  Kole,  Ronald  Lamprecht,	 Greg  Lee, Rohan
       Lenard, Craig Leres, John Levine, Steve Liddle, David Lof-
       fredo,  Mike  Long,  Mohamed el Lozy, Brian Madsen, Malte,
       Joe Marshall, Bengt Martensson, Chris Metcalf,  Luke  Mew-
       burn,  Jim  Meyering,  R. Alexander Milowski, Erik Naggum,
       G.T.  Nicol,  Landon  Noll,  James  Nordby,  Marc  Nozell,
       Richard Ohnemus, Karsten Pahnke, Sven Panne, Roland Pesch,
       Walter  Pelissero,  Gaumond  Pierre,  Esmond   Pitt,   Jef
       Poskanzer,  Joe	Rahmeh,	 Jarmo Raiha, Frederic Raimbault,
       Pat Rankin, Rick Richardson, Kevin Rodgers, Kai	Uwe  Rom-
       mel,  Jim  Roskind, Alberto Santini, Andreas Scherer, Dar-
       rell Schiebel, Raf Schietekat, Doug Schmidt, Philippe Sch-
       noebelen,  Andreas  Schwab,  Larry Schwimmer, Alex Siegel,
       Eckehard Stolz, Jan-Erik Strvmquist, Mike Stump, Paul Stu-
       art,  Dave  Tallman,  Ian  Lance	 Taylor,  Chris	 Thewalt,
       Richard M. Timoney, Jodi Tsai, Paul Tuinenga,  Gary  Weik,
       Frank  Whaley,  Gerhard	Wilhelms, Kent Williams, Ken Yap,
       Ron Zellar, Nathan Zelle,  David	 Zuhn,	and  those  whose
       names  have  slipped my marginal mail-archiving skills but
       whose contributions are appreciated all the same.

       Thanks to Keith Bostic, Jon Forrest, Noah  Friedman,  John
       Gilmore,	 Craig	Leres,	John  Levine,  Bob  Mulcahy, G.T.
       Nicol, Francois Pinard, Rich Salz,  and	Richard	 Stallman
       for help with various distribution headaches.

       Thanks to Esmond Pitt and Earle Horton for 8-bit character
       support; to Benson Margulies and Fred Burke for	C++  sup-
       port;  to Kent Williams and Tom Epperly for C++ class sup-
       port; to Ove Ewerlid for support of  NUL's;  and	 to  Eric
       Hughes for support of multiple buffers.

       This work was primarily done when I was with the Real Time
       Systems Group  at  the  Lawrence	 Berkeley  Laboratory  in
       Berkeley,  CA.  Many thanks to all there for the support I
       received.

       Send comments to vern@ee.lbl.gov.
\end{document}