\documentclass[12pt,spanish,lettersize]{article}
\usepackage[latin1]{inputenc}
\usepackage[spanish]{babel}
\usepackage[dvips]{graphicx}
\usepackage[usenames,dvipsnames]{xcolor}
\usepackage{mathrsfs}
\usepackage{multicol}
\usepackage{vmargin}
\setmargins{2.5cm}
{1.5cm}
{16.5cm}
{23.42cm}
{10pt}
{1cm}
{0pt}
{2cm}
\title{\color{Maroon}Leyes de Maxwell}
\author{L.C.C. Efra\'in Serna Gracia}
\date{\color{gray}\today}

\begin{document}
\maketitle
%% \tableofcontents
\section{Significado de variales y constantes}
\begin{multicols}{4}
\begin{description}
\item[Z] Impedancia
\item[V] Voltaje
\item[R] Resistencia
\item[I] Corriente
\item[t] tiempo
\item[$\phi$] Flujo magn\'etico
\item[$\mu_{0}$] Permeabilidad magn\'etica en el vac\'io
\item[$\overrightarrow{E}$] Campo el\'ectrico
\item[$\overrightarrow{dl}$] Elemento infinitesimal del contorno $c$
\item[$\overrightarrow{B}$] Densidad del campo magn\'etico
\item[$\overrightarrow{D}$] Densidad del flujo el\'ectrico
\item[$\overrightarrow{J}$] Densidad de la corriente
\end{description}
\end{multicols}

\section{Leyes de electromagnetismo}

\subsection{Gauss}

\subsubsection{Para la electricidad}
\begin{eqnarray}
\oint_{s}\overrightarrow{E}\cdot d\overrightarrow{S}=\frac{q}{\epsilon_0}\label{GaussInt}\\
\overrightarrow{\nabla}\cdot\overrightarrow{E}=\frac{\rho}{\epsilon_0}\label{GaussDif}\\
\nonumber \overrightarrow{\nabla}\cdot\overrightarrow{D}= \rho & \textrm{con} & \overrightarrow{D} = \overrightarrow{E}\epsilon_0
\end{eqnarray}
La ecuaci\'on (\ref{GaussInt}) dice que el flujo del campo e\'ectrico a trav\'es de una superficie cerrada es directaente proporcional a la carga (o suma de ellas) e inversamente proporcional a la permitividad el\'ectrica en el vac\'io.\\
Mientras que la ecuaci\'on (\ref{GaussDif}) dice que la divergencia del campo el\'ectrico es proporcional a la densidad de la carga el\'ectrica. Para valores positivos de $\nabla\overrightarrow{E}$ el campo diverge o sale desde una carga $\frac{\rho}{\epsilon_0}$, y para valores negativos, converge o entra a la carga

\subsubsection{Para el magnetismo}
\begin{eqnarray}
\overrightarrow{\nabla}\cdot\overrightarrow{B} = 0\label{GaussMagDif}\\
\oint\overrightarrow{B}\cdot d\overrightarrow{S}=0\label{GaussMagInt}
\end{eqnarray}
La ecuaci\'on (\ref{GaussMagDif}) explica que dentro de una superficie cerrada, no entra ni sale flujo magn\'etico, es decir, el valor de su devergencia es cero.\'
La ecuaci\'on (\ref{GaussMagInt}) dice que el flujo magn\'etico, a trav\'es de una superficie cerrada es cero.

\subsection{Faraday. \emph{para la inducci\'on}}
\begin{eqnarray}
\oint_c \overrightarrow{E}\cdot \overrightarrow{dl} = -\frac{d}{dt}\int_s\overrightarrow{B}\cdot \overrightarrow{ds}\label{FaradayInt}\\
\nabla\times\overrightarrow{E}=\frac{\partial \overrightarrow{B}}{\partial t}\label{FaradayDif}\\
V_c=N\frac{d\phi}{dt}
\end{eqnarray}
Esto significa, para la ecuaci\'on (\ref{FaradayInt}) que un campo el\'ectrico es equivalente a la variaci\'on del flujo magn\'etico dentro de una superficie cerrada a trav\'es del tiempo.\\
Con la ecuaci\'on (\ref{FaradayDif}) se entiende que si un campo magn\'etico var\'ia a lo largo del tiempo, esto provoca un campo el\'ectrico que circula a lo largo de l\'ineas cerradas, las cuales, en presencia de cargas libres puede mover \'estas y generar una corriente el\'ectrica.

\subsection{Amp\`ere}
\begin{eqnarray}
\oint_c\overrightarrow{B}\cdot\overrightarrow{cl}=\mu\int_s\overrightarrow{J}\cdot\overrightarrow{dS}+\mu_0\epsilon_0\int_s\overrightarrow{E}\cdot\overrightarrow{dS}\label{AmpereInt}\\
\nabla\times\overrightarrow{B}=\mu_0\overrightarrow{J}+\mu_0\epsilon_0\frac{\partial\overrightarrow{E}}{\partial t}\\
\overrightarrow{\nabla}\times\overrightarrow{H}=\overrightarrow{J}+\frac{\partial\overrightarrow{D}}{\partial t}
\end{eqnarray}
La ecuaci\'on (\ref{AmpereInt}) dice que cuando cun campo magn\'etico circula sobre una superficie cerrada equivale a densidad de la corriente sobre toda la superficie cerrada multiplicada por la permeabilidad magn\'etica del material, y considerando adicionalmente, la variaci\'on del campo el\'ectrico sobre la superficie a trav\'es del tiempo por la permeabilidad magn\'etica de la superficie multiplicada por la \emph{fem} inducida (equivalente a la derivada, con signo negativo, respecto al tiempo del flujo magn\'etico).

\end{document}