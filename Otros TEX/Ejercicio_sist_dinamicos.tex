\documentclass[12pt,spanish,lettersize]{article}
\usepackage[latin1]{inputenc}
\usepackage[spanish]{babel}
\usepackage[dvips]{graphicx}
\usepackage[usenames,dvipsnames]{xcolor}
\usepackage{mathrsfs}
\usepackage{multicol}
\usepackage{vmargin}
\usepackage[makeroom]{cancel}
\setmargins{2.5cm}
{1.5cm}
{16.5cm}
{23.42cm}
{10pt}
{1cm}
{0pt}
{2cm}
\title{\color{Maroon}Sistema din\'amico}
\author{L.C.C. Efra\'in Serna Gracia}
\date{\color{gray}\today}

\begin{document}
\maketitle
%% \tableofcontents
\begin{eqnarray}
my''(t)+cy'(t)+ky(t)=u(t)\\
\nonumber \textrm{Consideramos los siguientes valores:}\\
\nonumber m=\frac{1}{2}; c=\frac{1}{2}; k=1; u(t)=\frac{sin(\pi t)}{2}  \Rightarrow\\
\frac{1}{2}y''(t)+\frac{1}{2}y'(t)+y(t)=\frac{sin(\pi t)}{2}  \Rightarrow\\
\nonumber \textrm{Multiplicando por 2}\\
y''(t)+y'(t)+2y(t)= sin(\pi t)\label{y(t)}  \Rightarrow\\
\nonumber \textrm{Siendo que la soluci\'on general para \emph{y(t)} es } y_c(t)+y_p(t) \\
\nonumber \textrm{Tomamos } y(t) = e^{rt} \Rightarrow y'(t) = re^{rt} \Rightarrow y''(t) = r^2e^{rt} \Rightarrow \\
\nonumber \textrm{La ecuaci\'on (\ref{y(t)}) resulta en} \\
r^2e^{rt}+re^{rt}+2e^{rt}=sin(\pi t) \Rightarrow\\
\nonumber \textrm{Para \emph{y(t)} con \emph{t=0}} \\
\nonumber r^2\cancelto{1}{e^{r(0)}}+r\cancelto{1}{e^{r(0)}}+2\cancelto{1}{e^{r(0)}}=\cancelto{0}{sin(\pi (0))} \Rightarrow\\
r^2+r+2=0 \Rightarrow\\
\nonumber \textrm{Por f\'ormula general, las ra\'ices son:}\\
r_{1,2} = \frac{-1\pm\sqrt{1^2-4(1)(2)}}{2(1)} = \frac{-1\pm\sqrt{1-8}}{2} = \frac{-1\pm\sqrt{-7}}{2} \Rightarrow \\
\tilde{y}_1(t) = e^{-\frac{1}{2}t}\cdot e^{j\frac{\sqrt{7}}{2}t} \Rightarrow \\
\tilde{y}_1(t) = e^{-\frac{1}{2}}[cos(\frac{\sqrt{7}}{2})t+jsin(\frac{\sqrt{7}}{2})t] \Rightarrow \\
y_c(t) = C_1e^{-\frac{1}{2}}cos(\frac{\sqrt{7}}{2})t + C_2e^{-\frac{1}{2}}sin(\frac{\sqrt{7}}{2})t \Rightarrow  \\
\{e^{-\frac{1}{2}}cos(\frac{\sqrt{7}}{2})t,e^{-\frac{1}{2}}sin(\frac{\sqrt{7}}{2})t\}
\end{eqnarray}

Ahora, dado que\\
\begin{equation}ay''(t)+by'(t)+cy(t) = P_m(t)e^{mt}sin(\mu t)
\end{equation} con $m=0, \mu=\pi$ y sabiendo que \\

\begin{equation}
y_p(t) = t^s[(A_0t^m+\cdots +A_mt^0)cos(\mu t)+(B_0t^m+\cdots +B_mt^0)sin(\mu t)]e^{\alpha t}
\end{equation}

\begin{eqnarray}
ay''(t)+by'(t)+cy(t) = P_0(t)\cancelto{1}{e^{0t}}sin(\pi t)  \Rightarrow \\
\nonumber y_p(t) = \cancelto{1}{t^0}[(A_0\cancelto{1}{t^0})cos(\pi t)+(B_0\cancelto{1}{t^0})sin(\pi t)]\cancelto{1}{e^{(0)t}} \Rightarrow \\
y_p(t) = A\cdot cos(\pi t)+B\cdot sin(\pi t) \Rightarrow \\
y_p'(t) = -A\pi sin(\pi t)+B\pi cos(\pi t) \Rightarrow \\
y_p''(t) = -A\pi^2 cos(\pi t)-B\pi^2 sin(\pi t) \Rightarrow \\
\frac{1}{2}y''(t)+\frac{1}{2}y'(t)+2y(t) = sin(\pi t) \Rightarrow \\
\nonumber -\frac{1}{2}A\pi^2 cos(\pi t)-\frac{1}{2}B\pi^2 sin(\pi t) \\
\nonumber -\frac{1}{2}A\pi sin(\pi t)-\frac{1}{2}B\pi cos(\pi t) \\
+2A\cdot cos(\pi t)+2B\cdot sin(\pi t) = sin(\pi t)\\
t=0; -\frac{1}{2}A\pi^2-\frac{1}{2}B\pi+2A = 0\\
t=1; -\frac{1}{2}B\pi^2-\frac{1}{2}A\pi+2B = 1 \\
\left. 
\begin{array}{l}
-\frac{\pi^2}{2}A-\frac{\pi}{2}B+2A = 0 \\
-\frac{\pi^2}{2}B-\frac{\pi}{2}A+2B = 1 \\
\end{array}
\right\rbrace\Rightarrow\\
\left.
\begin{array}{rc}
(2-\frac{\pi^2}{2})A-\frac{\pi}{2}B = 0 & (\frac{\pi}{2})\\
-\frac{\pi}{2}A+(2-\frac{\pi^2}{2})B= 1 & (2-\frac{\pi^2}{2}) \\
\end{array}
\right\rbrace \Rightarrow\\
\left.
\begin{array}{l}
\cancel{(2-\frac{\pi^2}{2})(\frac{\pi}{2})A}-\frac{\pi}{2}(\frac{\pi}{2})B = 0 \\
\cancel{-(2-\frac{\pi^2}{2})\frac{\pi}{2}A}+(2-\frac{\pi^2}{2})^2B= (2-\frac{\pi^2}{2}) \\
\end{array}
\right\rbrace \Rightarrow\\
-\frac{\pi^2}{4}B+(2-\frac{\pi^2}{2})^2B=(2-\frac{\pi^2}{2})\Rightarrow\\
((2-\frac{\pi^2}{2})^2-\frac{\pi^2}{4})B=(2-\frac{\pi^2}{2})\Rightarrow\\
B=\frac{(2-\frac{\pi^2}{2})}{(2-\frac{\pi^2}{2})^2-\frac{\pi^2}{4}}\Rightarrow\\
B=\frac{(2-\frac{\pi^2}{2})}{((2-\frac{\pi^2}{2})-\frac{\pi}{2})((2-\frac{\pi^2}{2})+\frac{\pi}{2})}\Rightarrow\\
B=\frac{(2-\frac{\pi^2}{2})}{(2-\frac{\pi^2}{2}-\frac{\pi}{2})(2-\frac{\pi^2}{2}+\frac{\pi}{2})}\Rightarrow\\
\left.
\begin{array}{l}
\end{array}
\right\rbrace\Rightarrow\\
\left.
\begin{array}{l}
\end{array}
\right\rbrace\Rightarrow\\
\left.
\begin{array}{l}
\end{array}
\right\rbrace\Rightarrow\\
\left.
\begin{array}{l}
\end{array}
\right\rbrace\Rightarrow\\
\end{eqnarray}
\end{document}