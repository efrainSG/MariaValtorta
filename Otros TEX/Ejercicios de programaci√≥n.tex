\documentclass[12pt,spanish,lettersize,twocolumn]{article}
\usepackage[latin1]{inputenc}
\usepackage[spanish]{babel}
\usepackage[dvips]{graphicx}
\usepackage[usenames,dvipsnames]{xcolor}

\title{\color{Maroon}Ejercicios para programar con C/C++}
\author{L.C.C. Efra\'in Serna Gracia}
\date{\color{gray}\today}

\begin{document}
\maketitle
\section{Introducci\'on}
Los siguientes son ejercicios voluntarios para resolver en ratos libres. No afectan a ninguna calificaci\'on m\'as que de forma positiva al darte m\'as pr\'actica para resolver los problemas.\\
Recuerda que cada problema puede tener m\'as de una soluci\'on. Basta con que lo analices, siguas avanzando y luego puede que encuentres otra mejor, m\'as rebuscada, complicada o sencilla.\\
Experimenta con cada ejercicio, bien puedes preguntarte \emph{Y si hago que haga esto?\ldots} o \emph{Y si agrego aquello?\ldots}
\section{Preliminares}
La mejor forma de resolver un problema es dise\~nar la serie de pasos que deber\'a seguir el programa para realizar la tarea (\emph{Algoritmo}), luego realizar el diagramado para detectar posibles problemas desde otra perspectiva (\emph{Diagrama de flujo}) y finalmente codificarlo. En este \'ultimo punto, siguiendo los pasos, la mayor\'ia de los errores ser\'an de sintaxis (la forma en que se escribe seg\'un el lenguaje).
\subsection{Compilado, Enlazado y ejecutado}
Para compilar un programa utiliza la instrucci\'on\\
\emph{gcc -c $< destino>$ $<origen>$}\\

Si vas a compilar y enlazar (crear ejecutable), puedes usar la instrucci\'on\\
\emph{gcc -o $<destino>$ $<fuente>$}\\

Para ejecutar el resultado del compilado y enlazado (usando la opci\'on \emph{-o}), escribe la instrucci\'on siguiente:\\
\emph{./$<destino>$}\\

Donde, en las instrucciones anteriores, $<destino>$ es el archivo objeto (para la primera instrucci\'on) o el binario (para la segunda y tercera instrucciones) y $<fuente>$ es el c\'odigo fuente que capturaste.
\section{Ejercicios}
\begin{enumerate}
\item Construye un programa que solicite al usuario dos n\'umeros y luego muestre el resultado de las cuatro operaciones fundamentales con n\'umeros enteros y con n\'umeros de punto flotante. \emph{Recomendaci\'on: Utiliza el tipo de dato m\'as grande para los n\'umeros solicitados al usuario.}
\item Elabora un programa que reciba del usuario tres n\'umeros y regrese cu\'al es el menor, el mayor y el promedio.
\item El programa a realizar deber\'a solicitar al usuario tres n\'umeros y determinar si forman un tri\'angulo o no. \emph{En todo tri\'angulo, la suma de dos de sus lados es siempre mayor a la longitud del tercer lado}.
\item Elabora un programa que solicite un n\'umero entre 1 y 10 y luego muestre las tablas de multiplicar desde el 1 hasta el n\'umero que indic\'o el usuario. Si el usuario ingresa un n\'umero fuera de rango, el sistema deber\'a decirle al usuario que no se permiten n\'umeros menores a 1 ni mayores a 10.
\end{enumerate}
\end{document}