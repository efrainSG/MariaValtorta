\documentclass[12pt,spanish,lettersize,twocolumn]{article}
\usepackage[latin1]{inputenc}
\usepackage[spanish]{babel}
\usepackage[dvips]{graphicx}
\usepackage[usenames,dvipsnames]{xcolor}

\title{%
  \color{Maroon}C\'alculo diferencial e integral \\
\large Apuntes para bachillerato}
\author{L.C.C. Efra\'in Serna Gracia}
\date{\color{gray}\today}

\begin{document}
\maketitle
\section{Preliminares de \'Algebra}
Comenzamos con algunas declaraciones por norma:\\
\begin{enumerate}
\item Todo n\'umero precedido de un signo $(-)$ es negativo.\\
\item Todo n\'umero precedido de un signo $(+)$ es positivo.\\
\item Todo n\'umero sin un signo es positivo.\\
\item Dos expresiones encerradas entre par\'entesis cada una y colocadas una al lado de la otra, est\'an multiplicandose
\item Un n\'umero que precede a una letra est\'a multiplicando a esa letra
\end{enumerate}
\subsection{Leyes de los signos}
Son leyes muy importantes y \'amplamente utilizadas en matem\'aticas. Por ello es esencial memorizarlas hasta que sea algo mec\'anico su manejo.\\
\begin{tabular}{c|c|c}
$(+)$ & $(+)$ & $(+)$\\
\hline
$(-)$ & $(-)$ & $(+)$\\
\hline
$(+)$ & $(-)$ & $(-)$\\
\hline
$(-)$ & $(+)$ & $(-)$\\
\end{tabular}\\
Ejemplos:\\
\begin{itemize}
\item $(3)(4) = 12$
\item $(-3)(-4) = 12$
\item $(-3)(4) = -12$
\item $(3)(-4) = -12$
\item $-(3) = -3$
\item $-(-3) = 3$
\item $(3x)(4) = 12x$
\item $-(3x+4) = -3x-4$
\end{itemize}
\subsection{Leyes de los exponentes}
Otro conjunto de leyes muy importante y de uso general, es el conjunto de leyes de los exponentes, mismos que a continuaci\'on veremos:\\
\begin{tabular}{c|c}
$x^0 = 1$ & $x^1 = x$ \\
\hline
$(x^a)(x^b) = x^{a+b}$ & $(x^a)^b = x^{ab}$ \\
\hline
$x^{-1} = \frac{1}{x}$ & $\frac{x^a}{x^b} = x^{a-b}$\\
\hline
$\sqrt[a]{x} = x^{\frac{1}{a}}$ & $\sqrt[a]{x^b} = x^{\frac{b}{a}}$\\
\hline
$\$x^{-a} = \frac{1}{x^a}$ & \\
\end{tabular}\\
Ejemplos:
\begin{itemize}
\item $x^2x^3 = x^{2+3} = x^5$
\item $(x^2)^3 = x^{(2)(3)} = x^6$
\item $\frac{x^4}{x^2} = x^{4-2} = x^2$
\item $\sqrt[3]{x^6} = x^\frac{6}{3} = x^2$
\item $\$x^{-2} = \frac{1}{x^2}$
\end{itemize}
\section{M\'etodos de factoriaci\'on}
\subsection{Factor com\'un}
\subsection{Trinomio cuadrado perfecto}
\subsection{Trinomio $x^2+bx+c$}
\subsection{Trinomio $ax^2+bx+c$}
\subsection{Diferencia de cuadrados}
\section{Racionalizaci\'on}
\end{document}