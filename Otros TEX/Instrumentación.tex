\documentclass[12pt,spanish,lettersize]{article}
\usepackage[latin1]{inputenc}
\usepackage[spanish]{babel}
\usepackage[dvips]{graphicx}
\usepackage[usenames,dvipsnames]{xcolor}
\usepackage{mathrsfs}
\usepackage{vmargin}
\setmargins{2.5cm}
{1.5cm}
{16.5cm}
{23.42cm}
{10pt}
{1cm}
{0pt}
{2cm}
\title{\color{Maroon}Investigaciones de Instrumentaci\'on}
\author{L.C.C. Efra\'in Serna Gracia}
\date{\color{gray}\today}

\begin{document}
\maketitle
%% \tableofcontents
\section{Capacitores}
Es un elemento de dos terminales, donde la relaci\'on entre voltaje y corriente se da por \\
\begin{equation}
v(t) = \frac{1}{C}\int\limits_{-\infty}^{t}i(\tau)d\tau
\end{equation}
\begin{equation}\label{corriente_cap}
i(t) = C\frac{dv(t)}{dt}
\end{equation}
Donde $C$ es la capacidad medida en \emph{Faradios (F)}. Obs\'ervese que el voltaje depente de intervalos de tiempo previos, por lo que se dice que tiene "memoria".\\
La ecuaci\'on  de voltaje es la siguiente:
\begin{equation}
v(t) = \frac{1}{C}\int\limits_{0}^{t}i(\tau)d\tau + v(0)
\end{equation}
Siendo $v(0)$ la condici\'on inicial. Y dada la relaci\'on existente entre $i(t)$ y $q(t)$, esta se puede expresar como
\begin{equation}
q(t)=Cv(t) \Rightarrow C = \frac{q(t)}{v(t)}
\end{equation}
Lo que significa que la capacidad (C) es la relaci\'on entre la carga ($q(t)$) y el voltaje($i(t)$) en sus terminales.
\subsection{Potencia}
Siendo la potencia en un dipolo $p(t) = v(t)i(t)$ y considerando la ecuaci\'on (\ref{corriente_cap}), tenemos que
\begin{equation}\label{eq_pow}
p(t)=v(t)i(t) \Rightarrow p(t)=v(t)C\frac{dv(t)}{dt}
\end{equation}
Con $C$ positiva, pero como $v(t)\frac{dv(t)}{dt}$ puede ser positivo o negativo, $p(t)$ es tambi\'en positiva o negativa.
\subsection{Energ\'ia}
La energ\'ia puede expresarse como
\begin{equation}
w(t) = \int\limits_{-\infty}^{t}p(\tau)d\tau
\end{equation}
Tomando en cuenta a (\ref{eq_pow}), tenemos que
\begin{equation}\label{eq_pow2}
w(t) = \int\limits_{-\infty}^{t}Cv(\tau)\frac{dv(\tau)}{d\tau}d\tau = \int\limits_{v(-\infty)}^{v(t)}Cv(t)dv = \frac{Cv^2(t)}{2}-\frac{Cv^(-\infty)}{2}
\end{equation}
Considerando que $v(-\infty)=\frac{1}{C}\int\limits_{-\infty}^{-\infty}i(t)dt = 0$
\begin{equation}\label{eq_powf}
w(t) = \frac{Cv^2(t)}{2} \geq 0
\end{equation}
Lo que quiere decir que la energ\'ia del capacitor siempre es positiva o nula, aunque la potencia \emph{instantanea} pueda ser negativa, por lo que se define al capacitor como un elemento \emph{pasivo}.\\
Los valores manejados pueden ser pF o $\mu$F
\section{Inductores}
Es un componente de dos terminales, donde el voltaje y la corriente se relacionan por
\begin{equation}\label{eq_induct0}
i(t) = \frac{1}{L}\int\limits_{-\infty}^{t}v(\tau)d\tau \rightarrow v(t)= L\frac{di}{dt}
\end{equation}
Donde $L$ corresponde a la inductancia medida en \emph{Henrios} (H). Teniendo la condici\'on inicial $i(0)$, su ecuaci\'on para la corriente es:
\begin{equation}\label{eq_Induct1}
i(t) = \frac{1}{L}\int\limits_{t_{0}}^{t}v(\tau)d\tau + i(0)
\end{equation}
Mientras que los condensadores mantienen su carga en circuito abierto, los inductores (idealmente) mantienen sus cargas en cortocircuito, y en temperaturas cercanas al \emph{cero absoluto} mantienen la corriente por a\~nos.\\
Por la ley de Faraday tenemos $v(t) = \frac{d\phi}{dt}$ donde $\phi$ es el flujo magn\'etico.\\
La inductancia es la relaci\'on entre el flujo magn\'etico producido y la corriente que lo atraviesa:
\begin{equation}\label{eq_inductancia}
\phi(t) = L\cdot i(t) \rightarrow L=\frac{\phi(t)}{i(t)}
\end{equation}
\subsection{Potencia}
Expresada como $p(t)=v(t)i(t)$, y considerando (\ref{eq_induct0}), tenemos la siguiente ecuaci\'on:
\begin{equation}\label{eq_pot_induct}
p(t) = i(t)L\frac{di(t)}{dt}; L \geq 0 ; (i)\frac{di(t)}{dt} \textup{ positivo o negativo}
\end{equation}
\subsection{Energ\'ia}
La energ\'ia suministrada se expresa como
\begin{equation}\label{eq_ener_ind1}
w(t) = \int\limits_{-\infty}^{t}p(\tau)d\tau
\end{equation}
Aplicando (\ref{eq_pot_induct}) resulta en
\begin{equation}\label{eq_ener_indt2}
w(t) = \int\limits_{-\infty}^{t}L\cdot i(\tau)\frac{di(\tau)}{d\tau}d\tau = \int\limits_{i(-\infty)}^{i(t)}L\cdot i \cdot di
\end{equation}
Siendo que 
\begin{equation}
i(-\infty) = \int\limits_{-\infty}^{-\infty} = \frac{1}{L} \int\limits_{-\infty}^{-\infty}v(\tau)d\tau = 0
\end{equation}
Entonces tenemos
\begin{equation}\label{eq_ener_indf}
w(t) = \frac{L \cdot i^{2}(t)}{2}
\end{equation}
\section{Operacionales}
Un amplificador operacional es b\'asicamente una fuente de voltaje controlada por voltaje de ganancia (idealmente) infinita. En un amplificador operacinal de \emph{entrada diferencial}, la salida depende de la diferencia de voltajes entre sus terminales positiva y negativa
\\ \\ \\ \\ \\ \\ \\
Como se puede observar en el circuito equivalente (derecha), la resistencia de entrada es infinita (no hay conexi\'on entre $V_{+}$ y $V_{-}$. La resistencia de salida es la fuente de energ\'ia dependiente. Como es ideal, entonces es cero.
\subsection{Amplificador no inversor}
Ecuaciones del Operacional con realimentaci\'on negativa:
\\ \\ \\ \\ \\ \\
\begin{math}
\\
I_{+} = I_{-} = 0\\
e_{+} = e_{-}\\
e_{+} = V_{in}\\
e_{-} = -R_{1}I_{1} \Rightarrow I_{1} = -\frac{e_{-}}{R_{1}} = -\frac{V_{N}}{R_{1}}\\
V_{out} = -I_{1}(R_{1} + R_{2}) = (1+\frac{R_{2}}{R_{1}})V_{in} \Rightarrow \textup{ Ganancia } A_{v} = 1 + \frac{R_{2}}{R_{1}}\\
I_{out} = \frac{V_{out}}{R_{L}} = \frac{1}{R_{L}}(1+\frac{R_{2}}{R_{1}})V_{in}\\
V_{0} = I_{out} - I_{1} = \frac{1}{R_{L}}(1+\frac{R_{2}}{R_{1}})V_{in} + \frac{V_{in}}{R_{1}}
\end{math}\\
\\
Las caracter\'isticas m\'as notorias:
\begin{itemize}
\item La Ganancia del circuito es independiente de la ganancia del operacional.
\item La Ganancia fijada por el cociente de dos resistencias.
\item La Resistencia de entrada es infinita.
\item La entrada y la salida son del mismo signo.
\end{itemize}
\subsection{Amplificador inversor}
Ecuaciones del Operacional con realimentaci\'on negativa:
\\ \\ \\ \\ \\ \\
\begin{math}
\\
I_{+} = I_{-} = 0\\
e_{+} = e_{-}\\
e_{+} = 0\\
e_{-} = V_{in}-R_{1}I_{1}=0 \Rightarrow I_{1} = -\frac{V_{in}}{R_{1}}\\
v_{out}=-I_{1}R_{2}=-\frac{R_{2}}{R_{1}}V_{in}\Rightarrow\textup{ Ganancia } A_{v} = -\frac{R_{2}}{R_{1}}\\
I_{out} = \frac{V_{out}}{R_{L}}=-\frac{1}{R_{L}}\cdot\frac{R_{2}}{R_{L}R_{1}}\cdot V_{in}\\
I_{0} = I_{out}-I_{1} = \frac{1}{R_{1}}\cdot\frac{R_{2}}{R_{1}}V_{in}+\frac{V_{in}}{R_{1}}
\end{math}\\
Sus principales caracter\'isticas son:\\
\begin{itemize}
\item La Ganancia del circuito es independiente de la operacional.
\item La Ganancia es fijada por el cociente de dos resistencias.
\item La resistencia de entrada es $R_{1}$
\item La entrada y la salida son de signo contrario (Inversor).
\end{itemize}
\subsection{Seguidor de tensi\'on, Separador o Buffer}
Si en un amplificador no tensi\'on, hacemos 0 la resistencia $R_{2}$, queda en:\\
\begin{math}
\\
Avf = (1+\frac{0}{R_{1}})=1\Rightarrow V_{0}=V_{s}\\
R_{in}=\infty\\
R_{out}=0\\
\end{math}
$V_{0} = V_{s}$ implica que el valor de salida de tensi\'on sigue a la tensi\'on de entrada.\\
\\ \\ \\ \\ \\
Debido a que la ganancia de lazo cerrado en el circuito $Avf = 1$ no depende de $R_{1}$ al eliminarla, resulta el siguiente circuito\\
\\ \\ \\ \\ \\
Esto permite acoplar una fuente de tensi\'on con resistencia \emph{relativamente elevada} a una carga con resistencia \emph{relativamente baja} sin que haya un efecto de carga.
\subsection{Comparador}

\subsection{Sumador}
El amplificador sumador inversor es una aplicaci\'on del inversor de tensi\'on.\\
\\ \\ \\ \\ \\ \\
Aplicando la ley de corrientes de Kirchhoff al nodo "x", y considerando el cortocircuito virtual del AO ideal.
\begin{equation}
I_{F} = \sum\limits_{i=1}^{n}I_{i}
\end{equation}
Sustituyendo...
\begin{equation}
\frac{-V_{0}}{R_{F}}=\sum\limits_{i=1}^{n}\frac{V_{i}}{R_{i}}
\end{equation}
\begin{equation}
\frac{V_{0}}{R_{F}}=-(\sum\limits_{i=1}^{n}\frac{V_{i}}{R_{i}})
\end{equation}
El voltaje de salida es igual a la suma, con signo opuesto, de los voltajes de entrada. Si todas las resistencias son iguales, la resistencia total es igual al valor de cualquera de las resistencias ($R_{F}=R$) y $V_{0}=-(\sum\limits_{i=1}^{n}V_{i})$
\subsection{Integrador inversor}
Un circuito integrador inversor proporciona una salida de tensi\'on $V_{0}$ es proporcional a la integral de la se\~nal de entrada $V_{s}$.\\
Seg\'un la Teor\'ia de Circuitos, en el condensador se cumple que\\
\\ \\ \\
\begin{equation}
I_{c} = C\frac{d(V_{1}-V_{2}}{dt}
\end{equation}
\\ \\ \\ \\ \\ \\
considerando el \emph{cortocircuito virtual}, y aplicando la LCK al nodo que corresponde a la entrada inversora.
\begin{math}
I_{s}=I_{c}\\
\frac{V_{1}}{R}=C\frac{d(-V_{0}}{dt}\\
\end{math}\\
Despejando...\\
\begin{equation}
d(-V_{0}) = \frac{1}{CR}V_{1}dt
\end{equation}
Integrando entre los instantes $t_{0}$ y $t$, y considerando la tensi\'on de salida en el instante inicial es $V_{0}(t_{0})$
\begin{equation}
V_{0}=-\frac{1}{RC}\int\limits_{0}^{-\infty}V_{t}dt+V_{0}(t_{0})
\end{equation}
Si hay una tensi\'on "$V_{i}=V$" cntinua en la entrada dle integrador, suponiendo que en el instante $t-{0}=0$, $V_{0}(t_{0})=0$
\begin{equation}
V_{0}=-\frac{1}{RC}\int\limits_{0}^{t}Vdt = -\frac{1}{RC}Vt
\end{equation}
Para evitar cualquier tensi\'on cntinua en la entrada del integrador que lo lleve a su nivel de saturaci\'on, se coloca una resistencia $R_{c}$ en paralelo con el condensador para limitar la ganancia en CC a $-R_{c}/R$. Este valor se suele tomar como 
\begin{equation}
R_{c}\geq\frac{10}{2\pi f_{s}C}.
\end{equation}
\subsection{Derivador}

\section{Filtros}

\subsection{Pasa-bajas}
Tiene la propiedad de transmitir las componentes de se\~nales de excitaci\'on de baja frecuencia, incluyendo las se\~nales de corriente directa, mientras que las componentes de  altas frecuencias, incluyendo  las  infinitas, son bloqueadas. La magnitud de una funci\'on pasa-bajo tiene la siguiente apariencia ideal\\
\\ \\ \\ \\ \\
Y tienen la siguiente forma general de ecuaci\'on
\begin{equation}
T(s)=\frac{H}{D(s)}
\end{equation}
Donde $H$ es una constante y el polinomio $D(s)$ depende de los elementos de la red.
\subsection{Pasa-altas}
Tienen como propiedad bloquear las frecuencias que se encuentren por debajo  de la  frecuencia  de  corte ($\omega_{c}$) y  transmitir  todas  aquellas  componentes  de  frecuencia  que  sean mayores a esta frecuencia. La banda de rechazo se extiende desde $DC$ hasta $\omega_{c}$ y, la banda de paso, en teoría,  se extiende desde $\omega_{c}$ hasta una frecuencia infinita.\\
Las  funciones  pasa-altas  con  caracter\'isticas  en  magnitud  com\'unmente  tienen  sus  ceros 
localizados  en  el  origen  del  plano  de  la  frecuencia  compleja.  Por  lo  tanto,  las  funciones 
racionales tienen la forma
\begin{equation}
T(s)=\frac{H\cdot s^{n}}{D(s)}
\end{equation}
Donde $H$ es una constante y $n$ es el grado del denominador polinomial $D(s)$
\subsection{Pasa-bandas}
El filtro pasa-bandas se caracterisa por la transmisi\'on de un rango de frecuencias y el rechazo de otras dos, las cuales son un rango menor y uno mayor ($\omega_{2}$) que las de paso ($\omega_{1}$). A este rango transmitido se le llama \emph{ancho de banda} o $BW$ y es la diferencia entre las frecuencias que definen los l\'imites de la banda de paso
\begin{equation}
BW= \omega_{1}- \omega_{2}
\end{equation}
La frecuencia central se define como la media geom\'etrica de las frecuencias l\'imite
\begin{equation}
\omega_{0} = \sqrt{\omega_{1}\omega_{2}}
\end{equation}
La magnitud del filtro pasa-bandas a las frecuencias cero e infinita debe ser cero. Todos los filtros pasa-bandas cn caracter\'isticas en magnitud tienen la mitad de sus ceros en el origen y la otra mitad en el infinito.
\begin{equation}
T(s)=\frac{H\cdot s^{\frac{1}{2}}}{D(s)}
\end{equation}
Donde H es constante, $n$ el grado del polinomio denominador $D(s)$ y siempre n\'umero par.
\section{Decibelio}
Es una unidad relativa de medida que describe la ganancia o atenuaci\'on de potencia. Se utiliza para c\'alculos en sistemas de audio, de microondas, de enlaces a sat\'elites y en general en sistemas de comunicaci\'on.\\
La relaci\'on se calcula con el logaritmo de la relaci\'on de la potencia medida con respecto a la potencia de referencia y multiplicado por 10
\begin{equation}\label{Db1}
dB = 10\cdot log(\frac{P_{2}}{P_{1}})
\end{equation}
A partir de dos tensiones, la ecuaci\'on (\ref{Db1}) se puede escribir como
\begin{equation}
dB=10\cdot log(\frac{\frac{V_{2}^{2}}{R_{2}}}{\frac{V_{2}^{2}}{R_{1}}})
\end{equation}
Si $R_{1}=R_{2}$, tenemos...
\begin{equation}
dB=10\cdot log(\frac{V_{2}^{2}}{V_{2}^{2}})
\end{equation}
\end{document}