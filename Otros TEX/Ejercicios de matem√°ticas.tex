\documentclass[12pt,spanish,lettersize]{article}
\usepackage[latin1]{inputenc}
\usepackage[spanish]{babel}
\usepackage[dvips]{graphicx}
\usepackage[usenames,dvipsnames]{xcolor}

\title{\color{Maroon}Ejercicios de matem\'aticas}
\author{L.C.C. Efra\'in Serna Gracia}
\date{\color{gray}\today}

\begin{document}
\maketitle
\section{Aritm\'etica}
\subsection{N\'umeros enteros}
Resuelve estas sumas y restas
\begin{math}
\\
43+325+43-12-65+23-6=\\
21+34-5-76+23-12+654=\\
54+756-32+5-87+98-34+21=\\
\end{math}

Resuelve las siguientes multiplicaciones
\begin{math}
\\
4*5*6*7=\\
2*5*3*7*4=\\
12*45*2=
\end{math}\\

Existe algo que se llama \emph{Jerarqu\'ia Operacional} y que quiere decir que hay un orden para resolver las operaciones y \'esta es la siguiente:\\
\begin{enumerate}
\item Resuelve lo que est\'e dentro de par\'entesis
\item Resuelve lo que est\'e elevado a una potencia
\item Resuelve las multiplicaciones y divisiones
\item Resuelve las sumas y restas
\end{enumerate}


\subsection{N\'umeros fraccionarios}
\section{Gemoetr\'ia}
\subsection{Figuras planas}
\subsection{Cuerpos geom\'etricos}
\begin{math}
i_{1} = \frac{E}{R}\{(2\omegaRC)^{-1}\}
\end{math}
\end{document}