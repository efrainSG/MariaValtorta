\documentclass[12pt,spanish,lettersize]{article}
\usepackage[latin1]{inputenc}
\usepackage[spanish]{babel}
\usepackage[dvips]{graphicx}
\usepackage[usenames,dvipsnames]{xcolor}
\usepackage{mathrsfs}
\title{\color{Maroon}Sistemas Din\'amicos}
\author{Efra\'in Serna Gracia}
\date{\color{gray}\today}
\begin{document}
\maketitle
\begin{eqnarray}
\left.
\begin{array}{l}\label{sistema_original}
x''-4y' = 0 \\
x''+x'+y''=0 \\
\end{array}
\right\rbrace \Rightarrow \\
\left.
\begin{array}{l}\label{Sist_Ordenado_Original}
D^2x-4Dy = 0 \\
D^2x+Dx+D^2y=0 \\
\end{array}
\right\rbrace \Rightarrow \\
\left.
\begin{array}{rcccr}
D^2x     &       & -4Dy & = & 0 \\
(D^2+D)x & +D^2y &      & = & 0 \\
\end{array}
\right\rbrace
\left.
\begin{array}{l}
(D) \\
(4) \\
\end{array}
\right. \Rightarrow \\
\left.
\begin{array}{rccr}
D^3x           & -4D^2y & = & 0 \\
4(D^2+D)x      & +4D^2y & = & 0 \\
\hline
D^3x+4(D^2+D)x &        & = & 0 \\ 
\end{array}
\right\rbrace \Rightarrow
\end{eqnarray}
Multiplicando, factorizando y resolviendo para obtener los valores para $D_i$ y la ecuaci\'on general para $x(t)$
\begin{eqnarray}
\nonumber D^3x+4D^2x+4Dx=0 \Rightarrow \\
\nonumber Dx(D^2+4D+4) = 0 \Rightarrow \\
\nonumber Dx(D+2)(D+2)=0\Rightarrow \\
\nonumber D_{1}=0;D_{2}=-2;D_{3}=-2\Rightarrow \\
\nonumber x(t)=C_{1}e^{D_{1}t}+C_{2}e^{D_{2}t}+C_{3}e^{D_{3}t}\Rightarrow \\
\nonumber x(t)=C_{1}e^{(0)t}+C_{2}e^{-2t}+C_{3}e^{-2t} \Rightarrow \\
x(t)=C_{1}+C_{2}e^{-2t}+C_{3}e^{-2t}
\end{eqnarray}
Tomando nuevamente (\ref{Sist_Ordenado_Original}) como punto de partida para quitar todo lo que esté en funci\'on de $x$, y factorizando de la misma forma tenemos:\\

\begin{eqnarray}
\left.
\begin{array}{rcccr}
D^2x     &       & -4Dy & = & 0 \\
(D^2+D)x & +D^2y &      & = & 0 \\
\end{array}
\right\rbrace
\left.
\begin{array}{l}
(D^2+D) \\
(-D^2) \\
\end{array}
\right. \Rightarrow \\
\left.
\begin{array}{rcccr}
D^2(D^2+D)x  &            & +4(D^2+D)Dy & = & 0 \\
-D^2(D^2+D)x & -D^2(D^2y) &             & = & 0 \\
\hline \\
             & -D^2(D^2y) & +4(D^2+D)Dy & = & 0
\end{array}
\right\rbrace \Rightarrow\\
\end{eqnarray}
Multiplicando, factorizando y resolviendo para obtener los valores para $D_i$ y la ecuaci\'on general para $y(t)$
\begin{eqnarray}
\nonumber D^4y+4D^3y+4D^2y = 0\\
\nonumber D^2y(D^2+4D+4)=0\\
\nonumber D^2y(D+2)(D+2)=0\Rightarrow\\
\nonumber D_{4}=0;D_{5}=-2;D_{6}=-2\Rightarrow \\
\nonumber y(t)=C_{4}e^{D_{4}t}+C_{5}e^{D_{5}t}+C_{6}e^{D_{6}t}\Rightarrow \\
\nonumber y(t)=C_{4}e^{(0)t}+C_{5}e^{-2t}+C_{6}e^{-2t} \Rightarrow \\
y(t)=C_{4}+C_{5}e^{-2t}+C_{6}e^{-2t}
\end{eqnarray}
Calculando las primera y segunda derivadas de $x(t)$ e $y(t)$ tenemos:
\begin{eqnarray}
\nonumber x(t)=C_1+C_2e^{-2t}+C_3e^{-2t}\\
x'(t)  = -2C_2e^{-2t}-2C_3e^{-2t}\\
x''(t) =  4C_2e^{-2t}+4C_3e^{-2t}\\
\nonumber \\
\nonumber y(t)=C_4+C_5e^{-2t}+C_6e^{-2t} \\
y'(t)  = -2C_5e^{-2t}-2C_6e^{-2t}\\
y''(t) =  4C_5e^{-2t}+4C_6e^{-2t}
\end{eqnarray}
Para determinar los valores para $C_1$, $C_2$, $C_3$, $C_4$, $C_5$ y $C_6$, tomamos la primera ecuaci\'on de (\ref{sistema_original}) y sustituimos por sus respectivas funciones:
\begin{eqnarray}
\nonumber x''-4y'=0 \\
\nonumber 4C_2e^{-2t}+4C_3e^{-2t}-4(-2C_5e^{-2t}-2C_6e^{-2t})=0 \\
\nonumber 4e^{-2t}(C_2+C_3)+4e^{-2t}(2C_5+2C_6)=0 \\
\nonumber 4e^{-2t}[(C_2+C_3)+(2C_5+2C_6)]=0 \\
\nonumber 4e^{-2t}[C_2+C_3+2C_5+2C_6]=0 \\
\nonumber C_2+C_3+2C_5+2C_6=0 \\
C_2+2C_5=-C_3-2C_6
\end{eqnarray}
Igualando cada lado de la ecuac\'ion a cero, tenemos
\begin{eqnarray}
\left.
\begin{array}{l}
C_2+2C_5=0\\
C_3+2C_6=0
\end{array}
\right\rbrace\Rightarrow\\
\nonumber \left.
\begin{array}{l}
2C_5=-C_2\\
2C_6=-C_3
\end{array}
\right\rbrace\Rightarrow\\
\left.
\begin{array}{l}
C_5=-\frac{C_2}{2}\\
C_6=-\frac{C_3}{2}
\end{array}
\right\rbrace
\end{eqnarray}
Sustituimos los valores en $x(t)$ e $y(t)$, quedando como sigue:
\begin{eqnarray}
x(t)=C_1+C_2e^{-2t}+C_e^{-2t}\\
\label{Sol_Gral}y(t)=C_4-\frac{C_2}{2}e^{-2t}-\frac{C_3}{2}e^{-2t}
\end{eqnarray}
Siendo (\ref{Sol_Gral}) La soluci\'on general. Para encontrar la soluci\'on particular, tomamos en cuenta las condiciones iniciales planteadas.
\end{document}