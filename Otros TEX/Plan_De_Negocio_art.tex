\documentclass[12pt,spanish,lettersize]{article}
\usepackage[dvips]{graphicx}
\usepackage[latin1]{inputenc}
\usepackage[makeroom]{cancel}
\usepackage[spanish]{babel}
\usepackage[usenames,dvipsnames]{xcolor}
\usepackage{amssymb}
\usepackage{chngcntr}
\usepackage{epstopdf}
\usepackage{fancyhdr}
\usepackage{graphics}
\usepackage[hidelinks]{hyperref}
\usepackage{mathrsfs}
\usepackage{multicol}
\usepackage{setspace}
\usepackage{subcaption}
\usepackage{vmargin}
\pagestyle{fancy}
\fancyhf{}
\rhead{Efra\'in Serna Gracia}
\lhead{Plan de negocio MSG}
\rfoot{\thepage}
\setmargins
{1.25in} %left
{0.5in}  %top
{6in}    %width / right
{8in}  %height / bottom
{1in}    %head height
{0.5in}  %head sep
{1in}    %foot height
{0.5in}  %foot skip
\author{Efrain Serna Gracia}
\begin{document}
\begin{center}
MSG Fast Service\\
\vspace*{0.15in}
Plan de negocio \\
\vspace*{0.6in}
\vspace*{0.2in}
\begin{Large}
\textbf{Plan de negocio para puesta en marcha de MSG Fast Service como empresa poblana} \\
\end{Large}
\vspace*{0.3in}
\begin{large}
Propuesta de plan para puesta en marcha de empresa de tecnolog\'ias inform\'aticas y desarrollo de sistemas de alto impacto de apoyo cient\'ifico, acad\'emico y empresarial\\
\end{large}
\vspace*{0.3in}
\rule{80mm}{0.1mm}\\
\vspace*{0.1in}
\begin{large}
Dirigido por: \\
L.C.C. Efra\'in Serna Gracia \\
\vspace*{0.3in}
67 oriente 631-1. Infonavit La Margarita\\
C.P.: 72560. Puebla, Puebla.\\
Tel\'efono: (222)233-6305\\
Correo electr\'onico: efrain.serna@msg-fs.com\\
Sitio web: http://msg-fs.com
\end{large}
\end{center}

\tableofcontents
\section{Resumen ejecutivo}
\subsection{Necesidades}
Existe una necesidad de tener software econ\'omico y que cubra necesidades de un mercado espec\'ifico como lo es M\'exico. Ya hay sistemas para muchas actividades, pero es dif\'icil de costear para escuelas o empresarios que inician, por ello es importante contar con sistemas de entrenamiento y que aporten resultados al nivel que los sistemas l\'ideres mundiales, por ejemplo: ArchiCAD y AutoCAD. Contar con sistemas acad\'emicos an\'alogos en funciones, caracter\'isticas y resultados.\\

Tambi\'en se necesitan desarrollar sistemas que faciliten actividades diarias, en especial por parte de las \'areas de soporte t\'ecnico y que ofrezcan soluciones.\\

Realizar investigaci\'on en software para desarrollar nuevas opciones en vez de crear m\'as grupos de desarrollo de herramientas administrativas.\\

En la actualidad existe una gran capacidad de procesamiento a precios econ\'omicos, por lo que aprovecharlo para cambiar el enfoque de los sistemas ser\'ia s\'umamente provechoso.

\subsection{Objetivos}
\subsubsection{General}
Crear un grupo de profesionales multidisciplinarios que puedan desarrollar sistemas, aplicando los conocimientos en Ciencias de la Computaci\'on, como soluciones innovadoras, \'agiles y productivas en Tecnolog\'ias de la Informaci\'on, que busque resolver necesidades mediante la propuesta de soluciones y planes de adquisici\'on de estas en vez de proponer la venta indiscriminada de equipos lo cual deja a los clientes (empresas, organizaciones, personas e instituciones) la tarea de buscar, comparar, reunir y tratar de unificar, adem\'as, que pueda educar, instruir y capacitar a las personas en el uso de sistemas y equipos para que \'estas puedan sacar el mejor provecho a los recursos de que disponen y cuenten con el conocimiento que les permita realizar \'as f\'acilmente sus tareas, y puedan cuidar sus equipos para extender lo m\'as posible la vida \'util de estos.
\subsubsection{Espec\'ificos}
\begin{itemize}
\item Crear un grupo que pueda desarrollar sistemas innovadores con un enfoque en Ciencias computacionales que pueda obtener el mejor aprovechamiento con el uso de Ciencias exactas y no solo en la programaci\'on mayormente utilizada.
\item Crear un grupo capaz de identificar necesidades en equipamiento y proponer soluciones completas y planes de implementaci\'on, que cuente con las destrezas para llevar de la mano a los clientes desde su concepci\'on como soluci\'on hasta su plena implementaci\'on.
\item Conseguir que el grupo sea unido, interesado en la investigaci\'on, capacitaci\'on y desarrollo, que cuente con capacidad interdisciplinaria y colaborativa.
\end{itemize}
\section{Planteamiendo de problema y propuesta de soluci\'on}
\subsection{Problem\'atica}
Existe mucho software dise\~nado en el extrangero para extrangeros a precios de extrangeros. Las empresas se tienen que adaptar a dichos sistemas y la adaptaci\'on de los mismos es costosa o a veces imposible
\subsection{Propuesta de soluci\'on}
\subsubsection{Productos}
\begin{description}
\item[Equipos de hardware] Orientados a cubrir las necesidades actuales y futuras en un rango de 3 a\~nos de los clientes.
\item[Sistemas desarrollados] como prop\'osito general basados en sistemas solicitados por clientes, sin comprometer los secretos comerciales de los mismos.
\item[Soluciones en equipamiento de redes] Existe gran cantidad de proveedores de equipos para redes. La propuesta consiste en plantear un esquema de instalaci\'on, equipos requeridos y opciones de configuraci\'on
\end{description}
\subsubsection{Servicios}
\begin{description}
\item[Desarrollo de sistemas] Consistente en sistemas con una forma de trabajo diferente que expote los recursos de que dispone el cliente, desde un enfoque distinto. Sistemas con cierto grado de inteligencia y autonom\'ia
\item[Soporte t\'ecnico] Con un seguimiento muy cercano de
\item[Audotor\'ias inform\'aticas] consistentes en determinar con detalle la cantidad de equipo, caracter\'isticas, distribuci\'on, usuarios, software y formas de uso para un control m\'as preciso por parte de la empresa solicitante
\item[Capacitaci\'on]
\end{description}
\subsubsection{Competidores}
\begin{description}
\item[Sysne de M\'exico] Empresa desarrolladora de sistemas, consultor\'ia y capacitaci\'on. Se enfoca principalmente en el uso de tecnolog\'ias Microsoft  su integraci\'on con otras tecnolog\'ias. Partner Microsoft
\item[Grupo CASI]
\item[GISSA] Desarrollo de sistemas: Escolares, Universitarios, Controles de acceso, personal, gastos, n\'ominas, contables, compras, planificadores de recursos empresariales e industriales
\item[ETIX] Sistemas ERP y CRM
\item[OCTANA] Desarrollo de software web y movil, ERPs, CRMs, Inventarios, ventas, recolecci\'on de datos
\item[Idwasoft] Desarrollo web y paquetes de servicios web
\item[Total Tech]
\item[web Villa Net]
\item[EVICITI] Desarrollo de software, facturaci\'on electr\'onica, punto de venta, contabilidad, Soluciones empresariales de manufactura, sluciones de maquinaria y activos. Partner SAP
\item[GABO Software]
\item[Go Systems]
\end{description}
\subsubsection{Diferenciadores}
La mayor\'ia de las empresas de desarrollo de software revisadas, realizan proyectos de administraci\'on empresarial, administraci\'on de recursos, manejo de bases de datos relacionales, sistemas estad\'isticas y de reporteo, algunos proyectos manejan elementos predictivos basados en proyecciones como los cubos de decisi\'on o cubos OLAP.\\
No se han encontrado empresas de desarrollo de software que utilice herramientas de an\'alisis como C\'alculo, procesamiento matem\'atico de an\'alisis num\'erico, procesamiento mediante m\'aquinas de estados o elementos similares.
\subsection{Misi\'on y Visi\'on}
\begin{description}
\item[Misi\'on]
Entregar a nuestros clientes soluciones innovadoras en TICs que satisfagan sus necesidades, ya sean laborales o personales.
\item[Visi\'on]
Ser una empresa competitiva, identificada por el tipo de soluciones entregadas a nuestros clientes.
\item[Valores]
Honestidad, Compromiso, Precisi\'on, Claridad.
\end{description}
\section{Caracter\'isticas de MSG-FS y perfiles de integrantes}
\subsection{Caracter\'isticas de MSG-FS}
\subsubsection{Fortalezas}
\begin{description}
\item[Aplicaci\'on de la matem\'atica]. Buscamos en cada proyecto, la mejor forma de utilizar las matem\'aticas, ya que estas permiten una operaci\'on m\'as fluida y \'agil en cuanto a c\'alculos se refiere.
\item[B\'usqueda de alternativas] hacia software de bajo costo y alto rendimiento. Software libre y desarrollos a la medida.
\item[Razonamiento l\'ogico-abstracto]
\item[B\'usqueda de soluciones de aplicaci\'on general]. Ya que cada proyecto no solo lo enfocamos a la tarea en particular que se requiere, sino tambi\'en hacia tareas similares con cierto grado de variabilidad.
\end{description}
\subsubsection{Oportunidades}
\subsubsection{Debilidades}
\begin{description}
\item[Equipo de trabajo reducido]. Nuestro equipo de trabajo se compone actualmente de una sola persona que funge como DBA, Arquitecto, Desarrollador, Director, Ejecutivo de ventas y Promotor.
\item[Herramientas de hardware de bajo rendimiento]. El equipo de trabajo con que se cuenta actualmente, ya es obsoleto para proyectos de alto impacto en velocidad y capacidad de procesador. Un equipo es portatil y tiene mediana capacidad de almacenamiento pero con procesador de 32 bits y pantalla de 15". Otro equipo es de escritorio y cuenta con procesador de 64 bits pero baja capacidad de almacenamiento y tiene pantalla de 18". Ambos tienen 4 GB de RAM.
\item[Manejo de pocos lenguajes]. Actualmente se manejan C\#, Object Pascal, HTML, JavaScript, PHP, ANSI SQL, T-SQL, VisualBasic. Se est\'a en proceso de aprendizaje de jQuery, Python, y pr\'oximamente CSS3 y HTML5.
\item[Desarrollo de pocos sistemas]. Contamos con un sistema de gesti\'on acad\'emica de \'amplia capacidad y flexibilidad, un sistema de gesti\'on de bibliotecas, una micro-plataforma en l\'inea para registro de historias cl\'inicas. Falta aprender a programar aplicaciones Android y Windows Phone.
\end{description}
\subsubsection{Amenazas}
\begin{description}
\item[Equipos de desarrollo consolidados]. Muchos cuentan con m\'as de un desarrollador, al menos un arquitecto, un DBA, un Director de desarrollo y/o general. Quienes hacen las promociones suelen ser lso directivos y l\'ideres de proyectos.
\item[Empresas de software de gran envergadura]
\item[usuarios acostumbrados a adaptarse a sistemas]
\item[Situaci\'on econ\'omica]. Obliga a que las empresas se lo piensen m\'as de una vez antes de invertir en el desarrollo de un sistema.
\item[Desconocimiento de MSG] por parte de clientes, quienes, por tanto, consideran que MSG tiene poco tiempo de funcionamiento y carece de experiencia.
\end{description}
\subsection{Perfiles}
\begin{description}
\item[Director] B\'usqueda de nuevos proyectos, nuevas tecnolog\'ias. Inter\'es por la constante capacitaci\'on e investigaci\'on, Proactivo, Investigador
\item[Administrador] de recursos econ\'omicos, encargado de la adquisici\'on de material de trabajo, como papeler\'ia, consumibles, pagos de servicios.
\item[Desarrollador] Inter\'es por la constante preparaci\'on, Capacidad de abstracci\'on, Gusto por desarrollar sistemas desde un enfoque no convencional, Proactivo, Investigador
\item[Mercad\'ologo] O afin al \'area de mercadotecnia para la promoci\'on de los servicios, productos y eventos de MSG
\item[Soporte] Proactivo, Inter\'es en preparaci\'on constante, B\'usqueda de realizar sus actividades de la forma m\'as eficiente posible (equilibrio de menos esfuerzo-mayor calidad), Organizado
\item[Instructor]
\item[Abogado]
\item[Contador]
\end{description}
\section{Mercado destino}
\begin{description}
\item[Escuelas]
\item[Universidades]
\item[Empresas]
\end{description}
\section{Plan de mercadeo}
\section{Fuentes de ingresos}
\section{Ganancias y p\'erdidas}
\subsection{Plan de ganancias}
\subsection{Plan de p\'erdidas}

\section{Estad\'isticas}
\end{document}