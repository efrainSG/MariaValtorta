% !TEX TS-program = pdflatex
% !TEX encoding = UTF-8 Unicode

% This is a simple template for a LaTeX document using the "article" class.
% See "book", "report", "letter" for other types of document.

\documentclass[12pt]{book} % use larger type; default would be 10pt

\usepackage[utf8]{inputenc} % set input encoding (not needed with XeLaTeX)
\usepackage[spanish]{babel}

%%% Examples of Article customizations
% These packages are optional, depending whether you want the features they provide.
% See the LaTeX Companion or other references for full information.

%%% PAGE DIMENSIONS
\usepackage{geometry} % to change the page dimensions
\geometry{a4paper} % or letterpaper (US) or a5paper or....
% \geometry{margin=2in} % for example, change the margins to 2 inches all round
% \geometry{landscape} % set up the page for landscape
%   read geometry.pdf for detailed page layout information

\usepackage{graphicx} % support the \includegraphics command and options

% \usepackage[parfill]{parskip} % Activate to begin paragraphs with an empty line rather than an indent

%%% PACKAGES
\usepackage{booktabs} % for much better looking tables
\usepackage{array} % for better arrays (eg matrices) in maths
\usepackage{paralist} % very flexible & customisable lists (eg. enumerate/itemize, etc.)
\usepackage{verbatim} % adds environment for commenting out blocks of text & for better verbatim
\usepackage{subfig} % make it possible to include more than one captioned figure/table in a single float
% These packages are all incorporated in the memoir class to one degree or another...

%%% HEADERS & FOOTERS
\usepackage{fancyhdr} % This should be set AFTER setting up the page geometry
\pagestyle{fancy} % options: empty , plain , fancy
\renewcommand{\headrulewidth}{0pt} % customise the layout...
\lhead{}\chead{}\rhead{}
\lfoot{}\cfoot{\thepage}\rfoot{}

%%% SECTION TITLE APPEARANCE
\usepackage{sectsty}
\allsectionsfont{\sffamily\mdseries\upshape} % (See the fntguide.pdf for font help)
% (This matches ConTeXt defaults)

%%% ToC (table of contents) APPEARANCE
\usepackage[nottoc,notlof,notlot]{tocbibind} % Put the bibliography in the ToC
\usepackage[titles,subfigure]{tocloft} % Alter the style of the Table of Contents
\renewcommand{\cftsecfont}{\rmfamily\mdseries\upshape}
\renewcommand{\cftsecpagefont}{\rmfamily\mdseries\upshape} % No bold!

%%% END Article customizations

%%% The "real" document content comes below...

\title{Ensayos}
\author{Efraín Serna Gracia}
%\date{} % Activate to display a given date or no date (if empty),
         % otherwise the current date is printed 

\begin{document}
\maketitle
\tableofcontents

\chapter{Ética ¿Pasada de moda o moda que no debe pasar?}

\section{Plagio: ¿astucia, comodidad o deshonestidad?}
¿Qué es el plagio? ¿Qué efectos puede tener? ¿Cómo me afecta si lo realizo? ¿Puedo hacer algo para promoverlo? ¿Puedo hacer algo en contra? ¿Me afecta si lo sufro? Estas son algunas de las preguntas que podemos plantear, ver y analizar el plagio desde el punto de vista de quien lo realiza como de quien lo sufre, analizar brevemente sus consecuencias y su impacto en la vida diaria.\\

¿Qué es el plagio? ¿Qué efectos puede tener? ¿Cómo me afecta si lo realizo? ¿Puedo hacer algo para promoverlo? ¿Puedo hacer algo en contra? ¿Me afecta si lo sufro? Estas son algunas de las preguntas que podemos plantear, ver y analizar el plagio desde el punto de vista de quien lo realiza como de quien lo sufre, analizar brevemente sus consecuencias y su impacto en la vida diaria.\\

Si es bueno o es malo, lo iremos dilucidando a lo largo del trabajo mediante situaciones vividas por Raúl Rojas Soriano y expresadas en su trabajo formación de Investigadores educativos de 1992, junto a situaciones de las que he sabido.
La RAE en su diccionario lo define como Acción y efecto de plagiar (copiar obras ajenas) y como americanismo que refiere a secuestrar a alguien. Ya de entrada no suena muy bien eso de "secuestrar". Por el lado de "copiar obras ajenas", más que copiarlas, refiere a transcribir o parafrasear la obra de otra persona o parte de ella sin decir que esa obra o esa parte pertenecen al trabajo de otro.\\

Raúl Rojas Soriano, en su trabajo Formación de investigadores educativos de 1992, platica la situación donde él fungía como parte de un jurado y que una concursante presentó un trabajo como propio, más tarde, Rojas se encontró con una conocida que le obsequió un ejemplar de uno de sus trabajos y al leerlo se dio cuenta que el trabajo de la concursante era un plagio, por lo que antes de deliberar y para evitar una situación vergonzosa para la concursante, le pidieron en privado que se retirara del concurso.\\
 Otra situación de este tipo la supe por unos compañeros de universidad que supieron que una maestra solicitó una serie de proyectos de fin de semestre, los cuales terminó presentando como propios para una evaluación que tuvo a su vez ella, resulta que uno de sus estudiantes descubrió tal situación e interpuso una demanda, porque él registró su proyecto a fin de tener derechos de autoría intelectual.\\
 
Como nos damos cuenta, el plagio no es bueno, y de afectarnos en caso de incurrir en él, seguro que sí nos afectará, tal vez no se descubra de inmediato, pero en cuanto suceda habrá consecuencias.\\

Esto nos habla sobre la importancia de la honestidad, de la honradez, sobre la empatía que debe existir entre nosotros, porque si yo hago algo que me cuesta, como decimos, tiempo, dinero y esfuerzo, con toda sinceridad y justicia no voy a querer que otro me lo robe (en el folklore mexicano: "agandalle", "madrugue", "fusile", "haga reverencia con sombrero ajeno"). Y si no lo queremos nos lo hagan ¿por qué hacerlo a otros? La situación cambia cuando voluntariamente lo ofrecemos a la comunidad para su uso.

\section{Ética, Comprensión y Humanidad}
Hablando de empatía, honradez, honestidad, es interesante lo que plantea Edgar Morín en su trabajo Los siete saberes necesarios para la educación del futuro en su capítulo VI, pues siendo la empatía la clave para comprender al otro, Edgar expone que vivimos una paradoja, por un lado tomamos conciencia de solidaridad, sabemos de personas en otro lado del planeta, los estudiamos y somos estudiados, comprendemos sus culturas, mientras que por el otro lado, entre nosotros y los que nos rodean existe cada vez más incomprensión, existe un "ruido" cada vez mayor que provoca malos entendidos, ambigüedad en lo que decimos, de forma tal que lo que decimos en un sentido los demás suelen darle otra interpretación. ¿Cuántas veces no se ha escuchado decir a alguien (o uno mismo) "yo no quise decir eso" o "lo que quise decir" o "no me mal interpretes" o "no me entiendes"? Recuerdo que un escritor dijo "cada quien llama barbarie a lo que les extraño", esto nos muestra que solemos no comprender a personas de otras culturas, ideas o filosofías, es difícil comprender a otros cuando uno mismo no se comprende (Morín, E, los siente saberes necesarios para la educación del futuro, Cap. VI) \emph{"Temet nosce"} ("conócete a ti mismo") reza un cartel colocado en el dintel de una puerta de cierto personaje y es lo más difícil de realizar pero es la clave para ir comprendiendo a los demás, precisamente es lo opuesto a conocerse uno mismo lo que lleva a, como dice Morín nuevamente, la indiferencia, el ego, etno y sociocentrismo (creerse a uno mismo, a su raza y su sociedad como el centro de todo), ampliar el abandono de la disciplina y obligación, la autoglorificación, autojustificación y a adjudicar a los demás los males que nos aquejan ("tú tienes la culpa", "no me entiendes", "no me das nada"); el trabajo de conocerse a uno mismo es largo y profundo, es difícil, intenso, agotador, es ampliar la visión y lo contrario, es sencillo: reducir todo lo complejo a su característica más notoria según el momento (dice el refrán "el árbol no deja ver el bosque"), sin embargo, conocerse un poquito cada vez más es una tarea gratificante en muchos aspectos, los réditos son mayores. Bien, si nos conocemos un poquito más veremos que tenemos cualidades y virtudes, defectos, manías y problemas y que no somos "moneditas de oro" entonces si a nosotros mismos nos cuesta dominar esos puntos álgidos que tenemos ¿cómo es que exigimos que otros hagan rápidamente lo que a nosotros nos cuesta muchísimo trabajo? Hay que dar hasta que duela, decía la madre Teresa de Calcuta, el Señor Jesús enseñaba "Da al que te pide, y al que te quita lo tuyo, no se lo reclames. Traten a los demás como quieren que ellos les traten a ustedes. No juzguen y no serán juzgados; no condenen y no serán condenados; perdonen y serán perdonados. Den, y se les dará; se les echará en su delantal una medida colmada, apretada y rebosante. Porque con la medida que ustedes midan, serán medidos ustedes. ¿Y por qué te fijas en la pelusa que tiene tu hermano en un ojo, si no eres consciente de la viga que tienes en el tuyo? ¿Cómo puedes decir a tu hermano: 'Hermano, deja que te saque la pelusa que tienes en el ojo', si tú no ves la viga en el tuyo? Hipócrita, saca primero la viga de tu propio ojo para que veas con claridad, y entonces sacarás la pelusa del ojo de tu hermano." (Lc. 6: 30-31, 37-38, 41-42).\\

La comprensión, su ética es exactamente de esta forma, pide comprender de forma desinteresada, sin esperar ser comprendido, comprender la incomprensión, argumentar y refutar en vez de excomulgar y anatematizar, la comprensión ni excusa ni acusa.\\
 Sabiendo comprender estaremos en vías de humanizar las relaciones humanas. Además, la comprensión hacia los demás necesita la conciencia de la complejidad humana (Morín, 1999) porque como humanos no somos entes simples, sino llenos de facetas, pareceres, humores y cambiamos con cada situación y decisión que tomamos.\\
 
Morín también explica que “La verdadera tolerancia no es indiferente a las ideas o escepticismos generalizados”, involucra convicción, fe, elección ética, aceptación y nos recuerda cuatro grados de tolerancia: respetar el derecho a proferir un propósito que nos parece innoble, nutrirse de opiniones diversas y antagónicas, lo opuesto a una idea profunda es otra idea profunda y la conciencia de las enajenaciones humanas. La tolerancia vale para las ideas, no para los insultos, agresiones o actos homicidas.\\

Inevitablemente, todo esto me suena conocido y estoy seguro que no soy el único con esa sensación, de hecho se puede encontrar algo análogo escrito hace siglos ya, por san Pablo "El amor es paciente, es servicial; el amor no es envidioso, no hace alarde, no se envanece, no procede con bajeza, no busca su propio interés, no se irrita, no tienen en cuenta el mal recibido, no se alegra de la injusticia, sino que se regocija con la verdad. El amor todo lo disculpa, todo lo cree, todo lo espera, todo lo soporta." ¿Por qué me suena familiar? Si pensamos en lo que es la ética, lo que es el respeto, la tolerancia de lo que se ha hablado, la honestidad, empatía que Morín y Rojas mencionan en sus trabajos, veremos que en síntesis es equivalente a decir "trata a los demás como quieras ser tratado", "ama a tu prójimo como a ti mismo", sabemos que cuando verdaderamente se ama, nunca intentaremos lastimar, engañar, robar a quien amamos.

\section{Democracia ¿Qué tienes que ver con nosotros?}
Tolerar a los demás habla de que todos tenemos defectos, de que cada uno es diferente a los demás\ldots diversidad de personas, de razas, credos, pensares, de ahí que la real forma de ponernos de acuerdo, de soportarnos y no sentirnos agraviados radica en el conceso, en encontrar lo que la mayoría puede requerir sin excluir a lo que la minoría pide, siempre en el marco del crecimiento y respeto de las comunidades ¿no es esto la democracia?\\

Morín nos explica que como tal, y nacida de sociedades complejas formadas por individuos complejos, resulta en un sistema complejo que tiene su fuerza y debilidad en los antagónicos, es la unión entre lo unido y lo desunido, tolera y se alimenta de conflictos, es plural y debe conservarse así para seguir existiendo.

\section{Conclusiones}
Pues analizando los textos anteriores, sus conceptos, similitudes, valores que plantean podemos idear formas para desarrollar conciencias de respeto, conciencias que lleven al crecimiento de una ética que se base en la riqueza de la diversidad, como por ejemplo:
Al inicio de los ciclos escolares, entre profesores y alumnos es adecuado establecer un acuerdo en las reglas que operarán a lo largo del curso, de esta forma se fomenta el respeto a las decisiones tomadas en grupo.\\

Dejar de lado la costumbre de mentir para “salir del apuro”, incluso dejar de recurrir a las “mentiras piadosas”, ya que esto es negativo para fomentar la honestidad y honradez.\\

Proporcionar los materiales de fuentes legales, si son películas, aunque sea alquiladas, si son libros, buscar ejemplares, en bibliotecas.\\

Fomentar el trabajo en equipo y entre los equipos que existan desacuerdos hay que intervenir, no para decir “él tiene la razón”, sino para generar empatía entre los involucrados, de forma que se abran al diálogo y lleguen a acuerdos que los beneficien.
Los beneficios que se pueden alcanzar serán mayores, ya que tanto profesores como alumnos iremos educándonos y re-educándonos en la verdad, la honestidad, el aprecio por las diferentes opiniones, el diálogo, la empatía y esto se irá poniendo en práctica en los ambientes que nos rodean.

\section{Referencias}
\begin{itemize}
\item Rojas, R. (1992) Formación de investigadores educativos. México: Editorial Plaza y Valdés.
\item Morín, E. (1999) Los siete saberes necesarios para la educación del futuro. Correo de la UNESCO
\item Sagrada Biblia, Santo Evangelio según san Lucas
\item Sagrada Biblia, Primera carta de san Pablo a los Corintios
\end{itemize}
\chapter{Valores y Democracia: ¿una dupla adecuada?}

\section{Introducción}
Piensa en algo o alguien que sea muy importante para ti. Piensa cómo sería tu vida sin ello. ¿De qué serías capaz por conseguir eso que te es muy importante? ¿Puedes ponerle un precio?, entonces, ¿Es preciado o es valioso? Si a lo que pensaste le puedes poner un precio, sería algo preciado, si no pudiste ponerle precio, es algo valioso. ¿Qué será mejor, algo preciado o algo valioso? Pues esas “sutiles diferencias serán abarcadas en la primera parte de este trabajo, junto con otros conceptos interesantes.\\

Ahora algo un poco delicado, ¿Te consideras alguien que tiene creencias espirituales o no?, si la respuesta no es ni “si” ni “no”, define qué si y qué no entran en tus creencias. Así como puedes decir “si”, “no”, “más o menos” o “a veces”, también hay conceptos que nos pueden ayudar a saber dónde encajamos. En la segunda parte aclararemos brevemente esos conceptos, pues el objetivo de este trabajo no es excluir a nadie, sino incluir y ayudar a clarificar algunas cosas que causan confusión.\\

Algo más, ¿Cómo crees que debería ser la educación? ¿Se cumple lo que constitucionalmente se dicta para la educación? ¿Conoces lo que marca la constitución? Pues respecto de eso trataré en la tercera parte, para finalmente descubrir si hay alguna forma de mejorar, si involucrar los valores con la educación puede ayudar o no, si es constitucional o se violan las leyes.

\section{Diferencia entre valor, moral, principio, regla y costumbre.}
Retomando de las primeras preguntas de la introducción algunos términos y para mantener cierta imparcialidad, diré que algo valioso será aquello que tenga o represente un valor para nosotros y que de acuerdo a la RAE (2010), “valor (Del lat. valor, -ōris). Grado de utilidad o aptitud de las cosas, para satisfacer las necesidades o proporcionar bienestar o deleite. Cualidad que poseen algunas realidades, consideradas bienes, por lo cual son estimables. Los valores tienen polaridad en cuanto son positivos o negativos, y jerarquía en cuanto son superiores o inferiores.”, mientras que algo preciado es algo que tiene un precio, que está definido como “1. adj. Precioso, excelente y de mucha estimación”. Otro término que se le parece o que tiene relación sería principio y si hemos oído el término de principio físico o principio matemático o principio social, salen otros más, el de ley, regla y costumbre, mismos que la RAE (2010) define como “principio. (Del lat. principĭum). Base, origen, razón fundamental sobre la cual se procede discurriendo en cualquier materia. Cada una de las primeras proposiciones o verdades fundamentales por donde se empiezan a estudiar las ciencias o las artes. Norma o idea fundamental que rige el pensamiento o la conducta.”, “ley. (Del lat. lex, legis). Regla y norma constante e invariable de las cosas, nacida de la causa primera o de las cualidades y condiciones de las mismas. Precepto dictado por la autoridad competente, en que se manda o prohíbe algo en consonancia con la justicia y para el bien de los gobernados. En el régimen constitucional, disposición votada por las Cortes y sancionada por el jefe del Estado. Cada una de las disposiciones comprendidas, como última división, en los títulos y libros de los códigos antiguos, equivalentes a los artículos de los actuales.”, “regla. (Del lat. regŭla). Aquello que ha de cumplirse por estar así convenido por una colectividad. Conjunto de preceptos fundamentales que debe observar una orden religiosa. Estatuto, constitución o modo de ejecutar algo. En las ciencias o artes, precepto, principio o máxima. Razón que debe servir de medida y a que se han de ajustar las acciones para que resulten rectas. Orden y concierto invariable que guardan las cosas naturales.”, “costumbre. (Del lat. *cosuetumen, por consuetūdo, -ĭnis). Hábito, modo habitual de obrar o proceder establecido por tradición o por la repetición de los mismos actos y que puede llegar a adquirir fuerza de precepto. Aquello que por carácter o propensión se hace más comúnmente. Conjunto de cualidades o inclinaciones y usos que forman el carácter distintivo de una nación o persona.”. Algo que es frecuente, es relacionar “valor” con “moral” y ésta con “religión”, pero veamos qué dice la RAE (2010) sobre “moral” y “religión”: “moral. (Del lat. morālis). Perteneciente o relativo a las acciones o caracteres de las personas, desde el punto de vista de la bondad o malicia. Que no pertenece al campo de los sentidos, por ser de la apreciación del entendimiento o de la conciencia. Prueba, certidumbre moral. Que no concierne al orden jurídico, sino al fuero interno o al respeto humano. Aunque el pago no era exigible, tenía obligación moral de hacerlo. Ciencia que trata del bien en general, y de las acciones humanas en orden a su bondad o malicia. Conjunto de facultades del espíritu, por contraposición a físico. Ánimos, arrestos. Estado de ánimo, individual o colectivo. En relación a las tropas, o en el deporte, espíritu, o confianza en la victoria.”, “religión. (Del lat. religĭo, -ōnis). Conjunto de creencias o dogmas acerca de la divinidad, de sentimientos de veneración y temor hacia ella, de normas morales para la conducta individual y social y de prácticas rituales, principalmente la oración y el sacrificio para darle culto. Virtud que mueve a dar a Dios el culto debido. Profesión y observancia de la doctrina religiosa. Obligación de conciencia, cumplimiento de un deber. La religión del juramento. Orden (instituto religioso)”.\\

En resumen, podemos decir que un valor es algo que es deseable, que es bueno o que proporciona bienestar, un valor puede tener un grado de estimación, desde completamente no estimado hasta completamente estimado, además son distintos de los principios, leyes y reglas en cuanto que un principio forma una base de la cual parten diversas cosas y un valor no, un valor es poseído por algo y es buscado, es un fin, y en cuanto a su aplicación es un medio también, las leyes son reglas y normas entre grupos a las cuales es deseable cumplir para el bienestar del grupo y que suele ser establecida por representantes de dicha comunidad o por la comunidad completa, mientras que las reglas, definen lo que se debe cumplir o el modo en que se realizan diversas actividades al interior de una colectividad. Y aunque una costumbre, que es un hábito, puede llegar a convertirse en un precepto y forman parte de la identidad de una persona o grupo de personas, no es obligado que se transforme en ley o regla. Dijimos que “valor” suele relacionarse con “religión” y con “moral”, así que, de acuerdo a las definiciones, la moral se entiende como la valoración que se hace de las acciones y que es fundamentalmente de manera interna, y aunque pueden ser coincidentes por el general de las personas, no es en la misma medida para todos y son coadyuvantes para la adquisición de los valores; por parte de la religión, son conjunto de creencias, principios y reglas que debe cumplir un grupo en relación a una divinidad con la finalidad de darle culto, y en ese sentido de reglas y principios, encontramos que existen cosas deseables a fin de trascender de acuerdo a esas creencias. Por tanto, podemos decir que aunque la religión define reglas y principios con metas alcanzables, y la moral dicta a cada persona la forma de comportarse correctamente a su particular forma de ver y ambos definen elementos deseables, no es de su exclusividad el establecer lo que son los valores, sino cuáles de estos son más o menos importantes para el bienestar particular y colectivo.\\

Bien, eso suena bien, pero… ¿Qué es lo que consideran los entendidos del pensamiento como valor y dónde lo podemos encontrar? Como es de esperarse en un mundo de diversidad, la concepción de valor no es la excepción, es por ello que los pensadores han hecho la labor de crear dos corrientes que los definen, una objetivista, donde los valores son objetos, y la otra es subjetivista, donde los valores dependen de las personas. Según Max Scheler, Brentano, Husserl y Hartmann, los valores son cualidades independientes a los objetos (que son sus portadores) y a los fines (a donde apuntan) e inmutables, existen, sean captados o no, son absolutos en sí, inaccesibles a la razón, relacionados con una intencionalidad de un sujeto hacia un objeto, misma que no es intelectual, sino emocional y moral, lo que nos da una gradanción en la preferencia, existiendo además “axiomas” axiológicos (que no pueden ser demostrados), aún así, puede existir una racionalidad en el acto de valorar, como en el caso de la alegría racional al tener la certeza sobre la existencia de algo que se valora como positivo. Sirven a la vez como medida, su objetivismo puede ser captado al ser afectado o atrapado por el valor. Ahora, atendiendo al enfoque subjetivista, representado por Federico Nietsche (1884-1900), Alexius Meinong (1853-1921), Christian Von Ehrenfels (1850-1932) y Ralph Barton Perry (1875-1957) dicen que los valores son una creación de los hombres estabilizados temporalmente, cuyo cambio es necesario para el progreso humano, un objeto posee valor en tanto pueda suministrar una base afectiva, el valor produce agrado tanto por la existencia como inexistencia del objeto. La base para los valores radica en el apetito o deseo, que son quienes confieren valor a las cosas y cuyo interés consiste en la actitud afectivo-motora a favor o en contra del objeto, lo que le da el valor y no al revés. Es ese interés lo que refiere como agrado-desagrado, deseo-aversión, búsqueda-rechazo.
Definición de laicidad.\\

Aunque el ser laico no es un valor, sino una característica, y pareciera salir de contexto, es necesario saber qué es laico para fines del presente trabajo, así pues, siguiendo la misma mecánica, de acuerdo a la REA (2010): “laico, ca adj. No eclesiástico ni religioso, civil: misionero laico. También s.: los laicos colaboran con la Iglesia. [Escuela o enseñanza] que prescinde de la instrucción religiosa: colegio laico”.\\

Miguel Ángel Muñoz (2008), profesor de filosofía va definiendo al ser laico como una persona que es aconfesa o que no profesa una religión en particular. Más adelante retoma definiciones etimológicas y semánticas para definir laico como laicus, “que no tiene órdenes religiosas o que no pertenece al clero” y del término griego laos “pueblo, multitud indiferenciada”. Es precisamente de aquí de donde lo toma el Cristianismo para oponerlo a kleros “jerarquía eclesiástica, autoridad de la ekklesía”.\\

Es importante esta explicación, puesto que por mucha gente, el término laico se asocia directamente como “no ser católico”, mientras que para los cristianos (más globalmente) refiere a los que formamos parte de la Iglesia y que no pertenecemos a la jerarquía eclesiástica. Entonces, ser laico sería no exactamente “no ser católico”, sino simplemente “no ser religioso” o no pertenecer a una orden religiosa, sea la confesión o credo que sea. Y dentro del cristianismo, los laicos somos “el pueblo o multitud indiferenciado, carente de órdenes religiosas y que no pertenece al clero” .

\section{Diferencia entre Laico y Ateísmo.}
Durante un debate, celebrado el 22 de mayo de 2008, los profesores Esteban Cortijo, profesor de filosofía en Bachillerato y Presidente del Ateneo, autor de diversos estudios sobre la obra filosófica de Mario Roso de Luna e Isidoro Reguera, profesor de filosofía en la Universidad de Extremadura, dijeron que en este país no hemos sido laicos, aunque sí, en buena parte, anticlericales, que se explica por la presencia de un clericalismo agresivo y totalitarista. Se puede ser laico y no ser ateo, el laicismo, que es de sentido común, se puede decir que defiende la neutralidad del Estado ante las confesiones religiosas. Que dicha neutralidad permitirá la mejor convivencia entre las personas. Y que por lo tanto es un principio democrático incuestionable. El laicismo no es una cuestión religiosa, sino que atañe a lo político y lo social y que no supone una postura beligerante contra las iglesias, y recordaron como Ortega y Gasset en 1910 en una conferencia pronunciada en Bilbao defendía el laicismo y especialmente la escuela laica cuando dijo que hay que educar a la ciudad para educar al individuo y que la escuela laica debería ser prioritaria, una escuela laica instituida por el Estado, porque la sociedad es la única educadora y el fin de la educación es la sociedad.\\

La RAE (2010) define al Ateísmo como: “Sistema de ideas que niega la fe en lo sobrenatural (espíritus, dioses, vida de ultratumba, \&c.). El objeto del ateísmo es explicar las fuentes y causas del origen y existencia de la religión, criticar las creencias religiosas desde el punto de vista de la visión científica del mundo, aclarar el papel social de la religión, señalar de qué manera pueden superarse los prejuicios religiosos”.\\

En el Diccionario Enciclopédico Hispano-Americano (1887-1910) se menciona que el ateísmo es, primordialmente, una negación referente a la concepción de un dios y a su vez, el mismo diccionario cita a Proudhon y a madame Stael al decir que: ``es menos lógico el ateísmo que la fe'', ``¿El ateísmo espiritualiza la materia o materializa el espíritu?'', además que cita la exigencia de D'Alambert para distinguir ``la ignorancia o desconocimiento de Dios'' de ``la posesión de su idea, que es después rechazada o negada'' y que es a lo que refiere el ateísmo. Por último, J. Reynaud dice (y que resulta interesante): ``se puede negar determinada concepción de la Divinidad, sin por ello negar la existencia de Dios. No lo entienden así los hombres intolerantes, para quienes no existe más Dios que su Dios (el que ellos conciben o dogmáticamente creen y confiesan), y para ellos oponerse a su creencia equivale a profesar el ateísmo. De esto resulta que no hay nombre más frecuentemente atribuido por los apóstoles de todas las religiones a sus adversarios que el de ateo''.\\

Entonces, mientras que laico es la persona que es indiferenciable en el ámbito religioso, pero que no es carente de credo, sino que lo profesa, no importando cuál fuese y que además, el ser laico es una cuestión política y social, y que además no está en una postura contraria a las religiones, el ateo es aquel que niega la existencia de una divinidad, que busca el fundamento de la existencia de las religiones, sin embargo, no agrede a una religión particular, sino que toma una postura científica, inquisitiva, investigadora que lo lleva a buscar una verdad.

\section{Educación laica.}
La Constitución Política de los Estados Unidos Mexicanos, en su artículo tercero (1993) expresa que todo individuo tiene derecho a la educación, y que la misma, impartida por el Estado tenderá a “desarrollar armónicamente todas las facultades del ser humano y fomentar en el, a la vez, el amor a la patria y la conciencia de la solidaridad internacional, en la independencia y en la justicia… Garantizada por el Artículo veinticuatro, dicha educación será laica y por tanto, se mantendrá por completo ajena a cualquier doctrina religiosa… esa educación se basará en los resultados del progreso científico, luchará contra la ignorancia y sus efectos, las servidumbres, los fanatismos y los prejuicios\ldots Será democrática, considerando la democracia\ldots como un sistema de vida fundado en el constante mejoramiento económico, social y cultural del pueblo…” Esto define a la educación como independiente de credos religiosos, lo cual coincide con la definición de laicidad que la RAE expresa y que el profesor Miguel Ángel Muñoz menciona, además de que expresa claramente “se mantendrá ajena”, es decir, ni estará a favor de religión alguna ni en su contra y sí reconocerá la existencia de las mismas, lo cual los confiere en un Estado y una Educación “no ateos”, un Estado y una Educación basados en el respeto de las personas y el reconocimiento de sus creencias. Más aún, el mismo artículo hace mención de otra característica: la democracia, a la cual define como “sistema de vida fundado en el constante mejoramiento económico, social y cultural del pueblo”. Es por tanto, y de acuerdo a nuestra Constitución, algo deseable el alcanzar la democracia. Pero, aparte de definirse como un sistema de vida… ¿Qué es la democracia?\\

\section{Democracia, ¿qué es y cómo se relaciona con los valores y la laicidad?}
Nuevamente, la RAE (2010) define a la democracia como “Doctrina política favorable a la intervención del pueblo en el gobierno” y “Predominio del pueblo en el gobierno político de un Estado”. Etimológicamente hablando, sabemos que el término democracia proviene del antiguo griego ($\delta\eta\mu o\kappa\rho\tau\iota\alpha$) a partir de los vocablos $\delta\eta\mu o\varsigma$ (``demos'', que puede traducirse como ``pueblo'') y $K\rho\alpha\tau o\varsigma$ (krátos, que puede traducirse como ``poder'' o ``gobierno''). Aún así, esta significación no es suficiente actualmente, ya que implica la imposición de lo que la mayoría dicta. Hoy en día, en los gobiernos demócratas se busca un equilibro de respeto y cordialidad donde las decisiones que tome la mayoría incluya de la mejor manera lo que la minoría busca, esto mediante mecanismos que garanticen su consecución. Se han planteado diferentes formas de democracia: directa, en la que la totalidad de la comunidad toma decisiones referentes a asuntos que afectan a todos, indirecta, donde el pueblo elige representantes que tomarán decisiones en beneficio de sus representados y mixta, donde los representantes realizan el trabajo del consenso y la decisión final en ciertos aspectos la toma el pueblo.\\

Es interesante cómo, pese a que la democracia es algo deseable, alcanzado en diferentes grados en diversos países, de acuerdo al informe “El desarrollo de la democracia en América Latina”, publicado en fechas posteriores al 2004, el 54.7\% de los latinoamericanos estaban dispuestos a aceptar un gobierno autoritario, siempre que éste pudiera resolver la situación económica, y esto se refleja en la pérdida de confianza en los partidos políticos, en el bajo crecimiento del PIB per cápita (3856 dólares en AL frente a 36,100 en EUA para el año 2003), aunque los niveles de pobreza, en forma relativa reflejaron una disminución, en términos absolutos, aumentó la cantidad de personas que se ubicaban por debajo de la línea de pobreza, pasando de 190 millones de personas en 1998 a 209 millones en el año 2001. Esto nos hace pensar en que debe haber algo más en la democracia que la simple elección o toma de decisiones que beneficien a la mayoría sin agraviar a las minorías, algo debe estar faltando o en algo debemos estar errando el rumbo.\\

Interesante también lo que el profesor Isidoro Reguera, citando a Wittgenstein recordó que el laicismo, que es de sentido común, se puede decir: Que defiende la neutralidad del Estado ante las confesiones religiosas. Que dicha neutralidad permitirá la mejor convivencia entre las personas. Y que por lo tanto es un principio democrático incuestionable. El laicismo no es una cuestión religiosa, sino que atañe a lo político y lo social y que no supone una postura beligerante contra las iglesias. Interesante porque muestra una relación entre valores, laicidad, religión y democracia, su relación muestra a la democracia como algo deseable profundamente, para beneficio de los grupos sociales, lo que la constituye en un valor, sin embargo, la democracia por sí sola, nos ha mostrado ser insuficiente para la mejora de los pueblos, y América Latina es un claro ejemplo, tal como los datos que expuse líneas antes nos lo muestra.\\

Entonces ¿Qué falta en la democracia? ¿Qué falla en nuestra educación? ¿Qué es lo que necesitamos para que la democracia sea algo efectivo como ha sido en otras naciones? Si hablamos de democracia como el sistema de vida incluyente que busca hacer cumplir la voluntad de las mayorías sin excluir o agraviar a las minorías, entonces hablamos de respeto, de tolerancia, si hablamos de democracia como forma de gobierno representativo donde un grupo elegido por la comunidad, delibera a favor de sus representados, entonces hablamos de comunicación (representantes-representados), de honestidad (lo que se diga sea la verdad), de empatía (comprender la situación de los representados), nuevamente de respeto y tolerancia, en cualquiera de estos dos enfoques, implica comprender que el beneficio propio debe contribuir al beneficio colectivo y el aumento de beneficio colectivo conduce al incremento del beneficio propio.\\

Tolerancia, Honestidad, Empatía, Respeto, Democracia, en sí mismos son cosas deseables, que benefician a la comunidad, coadyuvados por y coadyuvantes de la comunicación. ¡Momento!, si estos son elementos deseables para beneficio colectivo y personal, existentes en sí mismos y que pueden ser percibidos en mayor o menor medida, aplicados a otros elementos (comunidad, persona, organización, etc.), entonces nos encontramos ante la existencia y búsqueda de valores mediante la aplicación y cumplimiento de leyes en un estado que es independiente de cualquier credo religioso y que a su vez los respeta sin coartar su proceder ni imponer alguno en particular y por tanto es LAICO.

\emph{Conclusión}
Si nos encontramos ante un estado laico, que refleja en sus leyes el deseo de mejorar y crecer, de regular el comportamiento ciudadano sin importar creencia, sin coartarlas ni imponerlas, de regirse por medio de un sistema democrático (con todo lo que ello implica), entonces nos encontramos ante la situación de que es necesario plantear la forma de alcanzar aquello que es deseable para beneficio de todos, para ello es necesario enseñar a buscar y practicar aquello que nos ayude a crecer en forma positiva, aquello que llamamos VALORES, como lo son la tolerancia, el respeto, la empatía, la honestidad. Es curioso como en realidad, aquello que se busca como Estado laico, también se busca en cuanto a credos y religiones, ambos buscan en realidad el crecimiento de sus comunidades que son la misma: la comunidad de seres humanos. Es interesante ver esta extraña y a la vez obvia simbiosis que nos negamos muchas veces a ver, que es necesaria para el buen crecimiento de las personas, puesto que las personas somos más que los individuos que vemos, es interesante comprender que si bien se busca un Estado democrático laico, éste no tiene que ser ateo ni inhibidor de las prácticas piadosas y muestras de fe que cada persona profese. Si tuviéramos la disposición a escuchar lo que la otra parte tiene que decir, descubriríamos que son más los puntos que tenemos en común que los que nos diferencian y veríamos que lo que buscamos por estos medios es exactamente lo mismo: el bien común y la trascendencia.\\

Entonces ¿qué concluimos? Concluimos que si bien el Estado tiene la obligación de educar en valores a fin de lograr una democracia plena que lleve al crecimiento de las naciones, no debería impedir, coartar ni obstaculizar el accionar de ninguna de las organizaciones religiosas, en particular el de aquellas cuyos fines son análogos a los que son deseables para el crecimiento del colectivo (“No se lo impidáis, pues el que no está contra vosotros, está por vosotros” Lc. 9:50).

\section{Referencias}
\begin{itemize}
\item Biblia de Jerusalén (1975). Evangelio de san Lucas. Bilbao. Editorial Española Desclée de Brouwer
\item Cáceres Laica(2008). Laicismo y Religión, una visión desde la Filosofía. Recuperado el 27 de octubre de 2010 de http://cacereslaica.wordpress.com/2008/05/23/laicismo-y-religion-una-vision-desde-la-filosofia/
\item Diccionario de la lengua española © 2005 Espasa-Calpe.
\item Diccionario enciclopédico hispano-americano (1887-1910). Recuperado el 26 de octubre de 2010 de http://www.e-torredebabel.com/Enciclopedia-Hispano-Americana/V2/ateismo-filosofia-D-E-H-A.htm
\item Diccionario Soviético de Filosofía (1965). Recuperado el 30 de octubre de 2010 de http://www.filosofia.org/enc/ros/ateismo.htm
\item Gerardo Remolina Vargas (2005). La formación en valores.
\item La democracia en América Latina. El desarrollo de la democracia en América Latina (2005).
\item Miguel Ángel Muñoz, Laicismo día tras día Fundamentación filosófico-política del laicismo, recuperado el 2 de noviembre de 2010 de\\http://www.europalaica.com/colaboraciones/\\c070524\_Laicismo\_dia\_tras\_dia.pdf
\item Pablo Latapi. (2001). “Valores y educación” Ingenierías Vol. IV, No. 11, 59-69
\item Wikipedia (2010). Laico (Religioso). Recuperado el 30 de octubre de 2010 de http://es.wikipedia.org/wiki/Laico\_\%28religioso\%29
\end{itemize}
\end{document}