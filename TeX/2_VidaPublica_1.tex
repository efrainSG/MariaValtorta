\documentclass[12pt, twoside, openright]{book} % use larger type; default would be 10pt

\usepackage[utf8]{inputenc} % set input encoding (not needed with XeLaTeX)
\usepackage{substr}
\usepackage[hidelinks]{hyperref}
%%% PAGE DIMENSIONS
\usepackage[lmargin=1.5in, rmargin=1in, tmargin=1in, bmargin=1in]{geometry} % to change the page dimensions
\geometry{letterpaper} % or letterpaper (US) or a5paper or....
% \geometry{margin=2in} % for example, change the margins to 2 inches all round
% \geometry{landscape} % set up the page for landscape
%   read geometry.pdf for detailed page layout information
\usepackage[spanish]{babel}

\usepackage{graphicx} % support the \includegraphics command and options

% \usepackage[parfill]{parskip} % Activate to begin paragraphs with an empty line rather than an indent

%%% PACKAGES
\usepackage{booktabs} % for much better looking tables
\usepackage{array} % for better arrays (eg matrices) in maths
\usepackage{paralist} % very flexible & customisable lists (eg. enumerate/itemize, etc.)
\usepackage{verbatim} % adds environment for commenting out blocks of text & for better verbatim
\usepackage{subfig} % make it possible to include more than one captioned figure/table in a single float
\usepackage{titlesec}
% These packages are all incorporated in the memoir class to one degree or another...

%%% HEADERS & FOOTERS
\usepackage{fancyhdr} % This should be set AFTER setting up the page geometry
\pagestyle{fancy} % options: empty , plain , fancy
\renewcommand{\headrulewidth}{0pt} % customise the layout...
\lhead{}\chead{}\rhead{}
\lfoot{}\cfoot{\thepage}\rfoot{}

%%% SECTION TITLE APPEARANCE
\usepackage{sectsty}
\allsectionsfont{\sffamily\mdseries\upshape} % (See the fntguide.pdf for font help)
% (This matches ConTeXt defaults)

%%% ToC (table of contents) APPEARANCE
\usepackage[nottoc,notlof,notlot]{tocbibind} % Put the bibliography in the ToC
\usepackage[titles,subfigure]{tocloft} % Alter the style of the Table of Contents
\renewcommand{\cftsecfont}{\rmfamily\mdseries\upshape}
\renewcommand{\cftsecpagefont}{\rmfamily\mdseries\upshape} % No bold!

\titleformat{\chapter}{\normalfont\Large\scshape}{\thechapter}{1em}{}
%%% END Article customizations
\setlength{\marginparwidth}{0pt}
%%% The "real" document content comes below...

\begin{document}

\begin{titlepage}
\centering
\vspace{3.5cm}
{\scshape\LARGE El Evangelio\\como me fue revelado\par}
{\scshape - Primer año de vida pública de Jesús - \par}
\vfill
{\scshape\Large María Valtorta \par}
\vfill
{\itshape \url{http://www.reinadelcielo.org/el-evangelio-como-me-fue-revelado-poema-de-el-hombre-dios-mar%EE%A1%ADvaltorta/} \par}
\vfill
Reina del Cielo
\end{titlepage}

\frenchspacing
\chapter*{Primer año de la vida pública de Jesús. \\ \normalfont\normalsize\textit{Adiós a la Madre y salida de Nazaret. Llanto y oración de la Corredentora.}}
\addcontentsline{toc}{chapter}{\normalfont\scshape{Primer año de la vida pública de Jesús.}}
 
El interior de la casa de Nazaret. Veo una habitación. Parece un comedor, donde la Familia come o está en las horas de descanso. Es una estancia muy reducida. Tiene una sencilla mesa rectangular frente a una especie de arquibanco que está pegando a una de las paredes: éste es el asiento de uno de los lados. En las otras paredes hay: un telar y un taburete; otros dos taburetes y un bazar, que tiene encima algunas lamparitas de aceite y otros objetos. Una puerta da a un pequeño huerto. Debe estar atardeciendo, pues no hay son un recuerdo de sol sobre la copa de un alto árbol que apenas verdece con las primeras hojas.
Jesús está sentado a la mesa. Está comiendo. María le sirve, yendo y viniendo por una puertecita que supongo conduce al lugar donde está el fuego, cuyo resplandor se ve desde la puerta entreabierta.
Jesús le dice a María dos o tres veces que se siente... y que también coma Ella. Pero Ella no quiere; menea la cabeza sonriendo tristemente, y trae, primero, unas verduras hervidas — me parece una sopa —; después, unos peces asados; luego, un queso más bien blando (como de oveja, fresco) de forma redondeada (semeja a esas piedras que se ven en los torrentes), y unas aceitunas pequeñas y oscuras. El pan, en pequeños moldes circulares (de la anchura de un plato común) y poco alto, está ya en la mesa. Es más bien oscuro, como si no se le hubiera separado el salvado. Jesús tiene delante un ánfora con agua y una copa; come en silencio, mirando a la Madre con doloroso amor.
María — se ve claramente — está apenada. Va, viene... para que no se le note. Enciende — aunque haya todavía luz suficiente — una lamparita y la pone junto a Jesús (al alargar el brazo acaricia disimuladamente la cabeza de su Hijo), abre una bolsa de color castaño — que a mí me parece hecha de esos paños de lana virgen tejidos a mano y, por tanto, impermeable —, comprueba si está vacía, sale al huertecito, va hasta el otro lado de éste, a una especie de despensa, de donde sale con unas manzanas ya más bien rugosas – conservadas desde el verano — y las mete en la bolsa; después coge un pan y mete también un pequeño queso, aunque Jesús no quiera y diga que ya tiene suficiente.
María se acerca a la mesa de nuevo, por la parte más estrecha, a la izquierda de Jesús. Le mira mientras come. Le mira con verdadera congoja, con adoración, con el rostro aún más pálido de lo normal y como más envejecido por la pena, con los ojos agrandados por una sombra que los marca, indicio de lágrimas vertidas; parecen, incluso, más claros que de costumbre, como lavados por el llanto que ya está casi apareciendo en ellos: ojos de dolor, cansados.
Jesús, que come despacio, claramente sin ganas, por complacer a su Madre, y que está más pensativo de lo habitual, levanta la cabeza y la mira. Se encuentra con una mirada llena de lágrimas, y baja la cabeza para que no se sienta cohibida, limitándose a cogerle la delicada mano que tiene apoyada en el borde de la mesa. La toma con la mano izquierda y se la lleva a la cara; Jesús apoya en ella su mejilla como rozándola un momento para sentir la caricia de esa pobre mano temblorosa, y la besa en el dorso con gran amor y respeto.
Veo a María llevándose la mano libre, la izquierda, hacia la boca, como para ahogar un sollozo; luego se seca con los dedos una lágrima grande que ha rebasado el borde del párpado y estaba regando la mejilla.
Jesús continúa comiendo. María sale rápidamente al huertecillo, donde ya hay poca luz... y desaparece. Jesús apoya el codo izquierdo sobre la mesa, y sobre la mano la frente, deja de comer y se sumerge en sus pensamientos.
Luego un momento de atención... Se levanta de la mesa. Sale Él también al huerto, mira a uno y otro lado y se dirige hacia la derecha respecto al lado de la casa, entra por una abertura de una pared rocosa, dentro de lo que reconozco como el taller de carpintero; esta vez todo ordenado, sin tablas, sin virutas, sin fuego encendido; el banco de carpintero y las herramientas, todas en su sitio, nada más.
Replegada sobre sí, en el banco, María llora. Parece una niña. Tiene la cabeza apoyada en el brazo izquierdo doblado, y llora, en voz baja pero con mucho dolor. Jesús entra despacio y se le acerca con tanta delicadeza, que Ella comprende que está allí sólo cuando su Hijo le deposita la mano sobre la cabeza inclinada, llamándola "Mamá" con voz de amorosa reprensión.
María levanta la cabeza y mira a Jesús entre un velo de llanto, y se apoya, con las dos manos unidas, en su brazo derecho. Jesús con un extremo de su ancha manga le seca la cara y la abraza, la estrecha contra su pecho, la besa en la frente.
Jesús tiene aspecto majestuoso, parece más viril de lo habitual, y María más niña, salvo en la cara marcada por el dolor.
- Ven, Mamá - le dice Jesús, y, apretándola estrechamente con el brazo derecho, se encamina de nuevo hacia el huerto; allí se sienta en un banco que está apoyado en la pared de la casa. El huerto está silencioso y ya oscuro. Hay sólo un hermoso claro de luna y la luz que sale de la estancia. La noche está serena.
Jesús le habla a María. No percibo al principio las palabras, apenas susurradas, a las que María asiente con la cabeza.
Después oigo:
- Y di a la familia..., a las mujeres de la familia, que vengan. No te quedes sola. Estaré más tranquilo, Madre, y tú sabes la necesidad que tengo de estar tranquilo para cumplir mi misión. Mi amor no te faltará. Vendré frecuentemente y, cuando esté en Galilea y no pueda acercarme a casa, te avisaré; entonces vendrás tú adonde este Yo. Mamá, esta hora debía llegar. Empezó aquí, cuando el Ángel se te apareció; ahora se cumple y debemos vivirla, ¿no es verdad, Mamá? Después vendrá la paz de la prueba superada, y la alegría. Antes es necesario atravesar este desierto, como los antiguos Padres para entrar en la Tierra Prometida. Pero el Señor Dios nos ayudará como hizo con ellos, y su ayuda será como maná espiritual para nutrir nuestro espíritu en el esfuerzo de la prueba. Digamos juntos al Padre nuestro...».
Jesús se levanta y María con Él, y levantan la cara al cielo. Dos hostias vivas que resplandecen en la oscuridad. Jesús dice lentamente, pero con voz clara y remarcando las palabras, la oración del Señor. Hace mucho hincapié en las frases: «venga a nosotros tu Reino, hágase tu voluntad», distanciando mucho estas dos frases de las otras. Ora con los brazos abiertos (no exactamente en cruz, sino como los sacerdotes cuando dicen: «El Señor esté con vosotros»), María tiene las manos juntas.
Entran de nuevo en casa, y Jesús — a quien no he visto nunca beber vino — echa en una copa un poco de vino blanco de un ánfora de la despensa y la lleva a la mesa; coge de la mano a María y la obliga a sentarse junto a Él y a beber de ese vino (en que moja una rebanada de pan que le ofrece). Tanto insiste, que María cede. El resto lo bebe Jesús. Luego estrecha a su Madre contra su costado, y así la sujeta, contra su persona, en el lado del corazón. Ni Jesús ni María están reclinados, sino sentados como nosotros. No hablan más. Esperan. María acaricia la mano derecha de Jesús y sus rodillas. Jesús acaricia el brazo y la cabeza de María.
Jesús se levanta y con Él María, se abrazan y se besan amorosamente una y otra vez; y una y otra vez parece que quieren despedirse, pero María vuelve a estrechar contra su pecho a su Hijo. Es la Virgen, pero es una madre a fin de cuentas, una madre que debe separarse de su hijo y que sabe a dónde conduce esa separación. Que ya no se me venga a decir que María no ha sufrido. Antes lo creía poco, ahora no lo creo en absoluto.
Jesús coge el manto (azul oscuro), se lo echa a los hombros y con él se cubre la cabeza a manera de capucha. Luego se pone en bandolera la bolsa, de forma que no le obstaculice el camino. María le ayuda, nunca termina de ajustarle la túnica y el manto y la capucha, y, mientras, lo vuelve a acariciar.
Jesús va hacia la puerta después de trazar un gesto de bendición en la estancia. María lo sigue y, en la puerta, ya abierta, se besan una vez más.
La calle está silenciosa y solitaria, blanca de luna. Jesús se pone en camino. Dos veces se vuelve aún a mirar a su Madre, que está apoyada en la jamba, más blanca que la Luna, toda reluciente de llanto silencioso. Jesús se va alejando por la callejuela blanca. María continúa llorando apoyada en la puerta. Y Jesús desaparece en una equina de la calle.
Ha empezado su camino de Evangelizador, que terminará en el Gólgota. María entra llorando y cierra la puerta.
También para Ella ha comenzado el camino que la llevará al Gólgota. Y por nosotros...
Dice Jesús:
- Éste es el cuarto dolor de María, Madre de Dios: el primero fue la presentación en el Templo; el segundo, la huida a Egipto; el tercero, la muerte de José; el cuarto, mi separación de Ella. Conociendo el deseo del Padre, te dije ayer por la noche que voy a acelerar la descripción de "nuestros" dolores para que se den a conocer. Pero, como ves, ya algunos de mi Madre habían sido ilustrados. He explicado antes que la Presentación la permanencia en Egipto, porque había necesidad de hacerlo ese día. Yo sé las cosas. Y tú comprendes. Mi proyecto es alternar tus contemplaciones, y mis consiguientes explicaciones, con los dictados propiamente dichos, para aliviarte a ti y a tu espíritu dándote la beatitud de ver, y también porque así queda clara la diferencia estilística entre tu forma de redactar y la mía. Además, ante tantos libros que hablan de mí y que, tocando y retocando, cambiando y acicalando, se han transformado en irreales, tengo el deseo de dar a quien en mí cree una visión devuelta a la verdad de mi tiempo mortal. No salgo disminuido; antes al contrario, magnificado en mi humildad, que se hace pan para vosotros para enseñaros a ser humildes y semejantes a mí, que fui hombre como vosotros y que llevé en mi aspecto humano la perfección de un Dios. Debía ser Modelo vuestro, y los modelos deben ser siempre perfectos. No mantendré en las contemplaciones una línea cronológica correspondiente a la de los Evangelios. Tomaré los puntos que considere más útiles en ese día para ti o para otros, siguiendo una línea mía de enseñanza y bondad. La enseñanza que proviene de la contemplación de mi separación se dirige especialmente a los padres e hijos a quienes la voluntad de Dios llama a la recíproca renuncia por un amor más alto; en segundo lugar está dirigida a todos aquellos que se encuentran frente a una renuncia penosa (¡y cuántas encontráis en la vida!). Son espinas en la Tierra que traspasan el corazón; lo sé. Pero para quien las acoge con resignación — mirad, no digo: "para quien las desea y las acoge con alegría" (esto ya es perfección), digo "con resignación — se transforman en eternas rosas. Pero pocos las acogen con resignación. Como burritos tozudos, os resistís obstinadamente a la voluntad del Padre, aunque no tratéis de herir con patadas y mordiscos espirituales, o sea, con rebelión y blasfemias contra el buen Dios. Y no digáis: "Pero si yo sólo tenía este bien y Dios me lo ha quitado; sólo este afecto, y Dios me lo ha arrancado". También María, mujer noble, amorosa hasta la perfección (porque en la Toda Gracia también las formas afectivas y sensitivas eran perfectas), sólo tenía un bien y un amor en la tierra: su Hijo. No le quedaba más que Él: los padres, muertos desde hacía tiempo; José, muerto desde hacía algunos años. Sólo quedaba Yo para amarla y hacerle sentir que no estaba sola. Los parientes, por causa mía, desconociendo mi origen divino, le eran un poco hostiles, como hacia una madre que no sabe imponerse a su hijo que se aparta del común buen juicio o que rechaza un matrimonio propuesto que podría honrar a la familia e incluso ayudarla. Los parientes, voz del sentido común, del sentido humano —vosotros lo llamáis sensatez, pero no es más que sentido humano, o sea, egoísmo — habrían querido que yo hubiera vivido estas cosas. En el fondo era siempre el miedo de tener un día que soportar molestias por mi causa; que ya osaba expresar ideas — según ellos demasiado idealistas — que podían poner en contra a la sinagoga. La historia hebrea estaba llena de enseñanzas sobre la suerte de los profetas. No era una misión fácil la del profeta, y frecuentemente le ocasionaba la muerte a él mismo y disgustos a la parentela. En el fondo, siempre el pensamiento de tener que hacerse cargo un día de mi Madre. Por ello, el ver que Ella no me ponía ningún obstáculo y parecía en continua adoración ante su Hijo, los ofendía. Este contraste habría de crecer durante los tres años de ministerio, hasta culminar en abiertos reproches cuando, estando yo entre las multitudes, se llegaban hasta mí, y se avergonzaban de mi manía — según ellos — de herir a las castas poderosas. Reprensión a mí y a Ella; ¡pobre Mamá! Y, no obstante, María, que conocía el estado de ánimo de sus parientes — no todos fueron como Santiago, Judas o Simón, ni como la madre de estos, María de Cleofás — y que preveía el estado de ánimo futuro; María, que conocía su suerte durante esos tres años, y la que le esperaba al final de los mismos y la suerte mía, no opuso resistencia como hacéis vosotros. Lloró. Y ¿quién no habría llorado ante una separación de un hijo que la amaba como Yo la amaba; ante la perspectiva de los largos días, vacíos de mi presencia, en la casa solitaria; ante el futuro del Hijo destinado a chocar contra la malevolencia de quien era culpable y se vengaba de serlo agrediendo al Inculpable hasta matarlo? Lloró porque era la Corredentora y la Madre del género humano renacido a Dios, y debía llorar por todas las madres que no saben hacer de su dolor de madres una corona de gloria eterna. ¡Cuántas madres en el mundo a quienes la muerte arranca de los brazos una criatura! ¡Cuántas madres a quienes un querer sobrenatural arrebata de su lado a un hijo! Por todas sus hijas, como Madre de los cristianos, por todas sus hermanas, en el dolor de madre despojada, ha llorado María. Y por todos los hijos que, nacidos de mujer, están destinados a ser apóstoles de Dios o mártires por amor a Dios, por fidelidad a Dios, o por crueldad humana. Mi Sangre y el llanto de mi Madre son la mixtura que fortalece a estos signados para heroica suerte; la que anula en ellos las imperfecciones, o también las culpas cometidas por su debilidad, dando, además del martirio — en cualquier caso, enseguida — la paz de Dios y, si sufrido por Dios, la gloria del Cielo. Las lágrimas de María las encuentran los misioneros como llama que calienta en las regiones donde la nieve impera, las encuentran como rocío allí donde el sol arde. La caridad de María las exprime. Estas han brotado de un corazón de lirio. Tienen, por ello: de la caridad virginal desposada con el Amor, el fuego; de la virginal pureza, la perfumada frescura, semejante a la del agua recogida en el cáliz de un lirio después de una noche de rocío. Las encuentran los consagrados en ese desierto que es la vida monástica bien entendida: desierto, porque no vive más que la unión con Dios, y cualquier otro afecto cae, transformándose únicamente en caridad sobrenatural hacia los parientes, los amigos, los superiores, los inferiores. Las encuentran los consagrados a Dios en el mundo, en el mundo que no los entiende y no los ama, desierto también para ellos, en el que viven como si estuvieran solos: ¡muy grande es, en efecto, la incomprensión que sufren, y las burlas, por mi amor! Las encuentran mis queridas "víctimas", porque María es la primera de las víctimas por amor a Jesús. A sus discípulas Ella les da con mano de Madre y de Médico, sus lágrimas, que confortan y embriagan para más alto sacrificio. ¡Santo llanto de mi Madre! María ora. Porque Dios le dé un dolor, no se niega a orar. Recordadlo. Ora junto con Jesús. Ora al Padre nuestro y vuestro. El primer "Pater noster" fue pronunciado en el huerto de Nazaret para consolar la pena de María, para ofrecer "nuestras" voluntades al Eterno en el momento en que comenzaba para estas voluntades el período de una renuncia cada vez mayor, que habría de culminar en la renuncia de la vida para Mí y de la muerte de un Hijo para María. Y, aunque nosotros no tuviéramos nada que necesitara el perdón del Padre, por humildad incluso, nosotros, los Sin Culpa, pedimos el perdón del Padre para afrontar, perdonados (absueltos incluso de un suspiro), dignamente nuestra misión. Para enseñaros que cuanto más se está en gracia de Dios más bendecida y fructuosa resulta la misión; para enseñaros el respeto a Dios y la humildad. Ante Dios Padre aun nuestras dos perfecciones de Hombre y de Mujer se sintieron nada y pidieron perdón, como también pidieron el "pan de cada día". ¿Cuál era nuestro pan? ¡Oh!, no el que amasaron las manos puras de María, cocido en el pequeño horno, para el cual yo muchas veces había recogido haces y manojos de leña — que es también necesario mientras se está en esta Tierra —, no ese pan, sino que "nuestro" pan cotidiano era el de llevar a cabo, día a día, nuestra parte de misión. Que Dios nos la diera cada día, porque llevar a cabo la misión que Dios da es la alegría de "nuestro" día, ¿no es verdad, pequeño Juan? ¿No lo dices también tú, que te parece vacío el día, como si no hubiera existido, si la bondad del Señor te deja, un día, sin tu misión de dolor? María ora con Jesús. Es Jesús quien os justifica, hijos. Soy Yo quien hace aceptables y fructuosas vuestras oraciones ante el Padre. Yo he dicho: "Todo lo que pidáis al Padre en mi nombre, Él os lo concederá", y la Iglesia acredita sus oraciones diciendo: "Por Jesucristo Nuestro Señor". Cuando oréis, uníos siempre, siempre, siempre a mí. Yo rogaré en voz alta por vosotros, cubriendo vuestra voz de hombres con la mía de Hombre – Dios. Yo pondré sobre mis manos traspasadas vuestra oración y la elevaré al Padre. Será hostia de valor infinito. Mi voz, fundida con la vuestra, subirá como beso filial al Padre, y la púrpura de mis heridas hará preciosa vuestra oración. Estad en mí si queréis tener al Padre en vosotros, con vosotros, para vosotros. Has terminado la narración diciendo: "Y por nosotros...", y querías decir: "Por nosotros que somos tan ingratos hacia estos Dos que han subido el Calvario por nosotros". Has hecho bien en poner esas palabras. Ponlas cada vez que te muestre un dolor nuestro. Que sea como la campana que suena y que llama a meditar y a arrepentirse. Nada más. Descansa. La paz esté contigo".

\chapter*{Predicación de Juan el Bautista \\ \normalfont\normalsize\textit{y Bautismo de Jesús. La manifestación divina.}}
\addcontentsline{toc}{chapter}{\normalfont\scshape{Predicación de Juan el Bautista.}}



\tableofcontents

\end{document}