\documentclass[12pt]{book} % use larger type; default would be 10pt

\usepackage[utf8]{inputenc} % set input encoding (not needed with XeLaTeX)

%%% PAGE DIMENSIONS
\usepackage{geometry} % to change the page dimensions
\geometry{a4paper} % or letterpaper (US) or a5paper or....
% \geometry{margin=2in} % for example, change the margins to 2 inches all round
% \geometry{landscape} % set up the page for landscape
%   read geometry.pdf for detailed page layout information
\usepackage[spanish]{babel}

\usepackage{graphicx} % support the \includegraphics command and options

% \usepackage[parfill]{parskip} % Activate to begin paragraphs with an empty line rather than an indent

%%% PACKAGES
\usepackage{booktabs} % for much better looking tables
\usepackage{array} % for better arrays (eg matrices) in maths
\usepackage{paralist} % very flexible & customisable lists (eg. enumerate/itemize, etc.)
\usepackage{verbatim} % adds environment for commenting out blocks of text & for better verbatim
\usepackage{subfig} % make it possible to include more than one captioned figure/table in a single float
% These packages are all incorporated in the memoir class to one degree or another...

%%% HEADERS & FOOTERS
\usepackage{fancyhdr} % This should be set AFTER setting up the page geometry
\pagestyle{fancy} % options: empty , plain , fancy
\renewcommand{\headrulewidth}{0pt} % customise the layout...
\lhead{}\chead{}\rhead{}
\lfoot{}\cfoot{\thepage}\rfoot{}

%%% SECTION TITLE APPEARANCE
\usepackage{sectsty}
\allsectionsfont{\sffamily\mdseries\upshape} % (See the fntguide.pdf for font help)
% (This matches ConTeXt defaults)

%%% ToC (table of contents) APPEARANCE
\usepackage[nottoc,notlof,notlot]{tocbibind} % Put the bibliography in the ToC
\usepackage[titles,subfigure]{tocloft} % Alter the style of the Table of Contents
\renewcommand{\cftsecfont}{\rmfamily\mdseries\upshape}
\renewcommand{\cftsecpagefont}{\rmfamily\mdseries\upshape} % No bold!

%%% END Article customizations

%%% The "real" document content comes below...

\title{El Evangelio como me fue revelado}
\author{María Valtorta}
%\date{} % Activate to display a given date or no date (if empty),
         % otherwise the current date is printed 

\begin{document}
\maketitle
\chapter{Dios quiso un seno sin mancha}

Dice Jesús: 
\emph{"Hoy escribe esto sólo. La pureza tiene un valor tal, que un seno de criatura pudo contener al Incontenible, porque poseía la máxima pureza posible en una criatura de Dios. La Santísima Trinidad descendió con sus perfecciones, habitó con sus Tres Personas, cerró su Infinito en pequeño espacio- no por ello se hizo menor, porque el amor de la Virgen y la voluntad de Dios dilataron este espacio hasta hacer de él un Cielo – y se manifestó con sus características: El Padre, siendo Creador nuevamente de la Criatura como en el sexto día y teniendo una "hija" verdadera, digna, a su perfecta semejanza. La impronta de Dios estaba estampada en María tan nítidamente, que sólo en el Primogénito del Padre era superior. María puede ser llamada la "segundogénita" del Padre, porque, por perfección dada y sabida conservar, y por dignidad de Esposa y Madre de Dios y de Reina del Cielo, viene segunda después del Hijo del Padre y segunda en su eterno Pensamiento, que ab aeterno en Ella se complació. El Hijo, siendo también para Ella "el Hijo" enseñándole, por misterio de gracia, su verdad y sabiduría cuando aún era sólo un Embrión que crecía en su seno. El Espíritu Santo, apareciendo entre los hombres por un anticipado Pentecostés, por un prolongado Pentecostés, Amor en "Aquella que amó", Consuelo para los hombres por el Fruto de su seno, Santificación por la maternidad del Santo. Dios, para manifestarse a los hombres en la forma nueva y completa que abre la era de la Redención, no eligió como trono suyo un astro del cielo, ni el palacio de un grande. No quiso tampoco las alas de los ángeles como base para su pie. Quiso un seno sin mancha. Eva también había sido creada sin mancha., Mas, espontáneamente, quiso corromperse. María, que vivió en un mundo corrompido – Eva estaba, por el contrario, en un mundo puro – no quiso lesionar su candor ni siquiera con un pensamiento vuelto hacia el pecado. Conoció la existencia del pecado y vio de él sus distintas y horribles manifestaciones, las vio todas, incluso la más horrenda: el deicidio. Pero las conoció para expiarlas y para ser, eternamente, Aquella que tiene piedad de los pecadores y ruega por su redención. Este pensamiento será introducción a otras santas cosas que daré para consuelo tuyo y de muchos".}
 
\chapter{Joaquín y Ana hacen voto al Señor}

Veo un interior de una casa. Sentada a un telar hay una mujer ya de cierta edad. Viéndola con su pelo ahora entrecano, antes ciertamente negro, y su rostro sin arrugas pero lleno de esa seriedad que viene con los años, yo diría que puede tener de cincuenta a cincuenta y cinco años, no más. 

 Al indicar estas edades femeninas tomo como base el rostro de mi madre, cuya efigie tengo, más que nunca, presente estos días que me recuerdan los últimos suyos cerca de mi cama... Pasado mañana hará un año que ya no la veo... Mi madre era de rostro muy fresco bajo unos cabellos precozmente encanecidos. A los cincuenta años era blanca y negra como al final de la vida. Pero, aparte de la madurez de la mirada, nada denunciaba sus años. Por eso, pudiera ser que me equivocase al dar un cierto número de años a las mujeres ya mayores. 

Ésta, a la que veo tejer, está en una habitación llena de claridad. La luz penetra por la puerta, abierta de par en par, que da a un espacioso huerto – jardín. Yo diría que es una pequeña finca rústica, porque se prolonga onduladamente sobre un suave columpiarse de verdes pendientes. Ella es hermosa, de rasgos sin duda hebreos. Ojos negros y profundos que, no sé por qué, me recuerdan al del Bautista. Sin embargo, estos ojos, además de tener gallardía de reina, son dulces, como si su centelleo de águila estuviera velado de azul Ojos dulces, con un trazo de tristeza, como de quien pensara nostálgicamente en cosas perdidas. El color del rostro es moreno, aunque no excesivamente. La boca, ligeramente ancha, está bien proporcionada, detenida en un gesto austero pero no duro. La nariz es larga y delgada, ligeramente combada hacia abajo: una nariz aguileña que va bien con esos ojos. Es fuerte, mas no obesa. Bien proporcionada. A juzgar por su estatura estando sentada, creo que es alta. 
Me parece que está tejiendo una cortina o una alfombra. Las canillas multicolores recorren, rápidas, la trama marrón oscura. Lo ya hecho muestra una vaga entretejedura de grecas y flores en que el verde, el amarillo, el rojo y el azul oscuro se intersecan y funden como en un mosaico. La mujer lleva un vestido sencillísimo y muy oscuro: un morado - rojo que parece copiado de ciertas trinitarias. 

Oye llamar a la puerta y se levanta. Es alta realmente. Abre. 

Una mujer le dice: Ana, ¿me dejas tu ánfora? Te la lleno. 

La mujer trae consigo a un rapacillo de cinco años, que se agarra inmediatamente al vestido de Ana. Ésta le acaricia mientras se dirige hacia otra habitación, de donde vuelve con una bonita ánfora de cobre. Se la da a la mujer diciendo: Tú siempre eres buena con la vieja Ana. Dios te lo pague, en éste y en los otros hijos que tienes y que tendrás. ¡Dichosa tú!

Ana suspira. 

La mujer la mira y no sabe qué decir ante ese suspiro. Para apartar la pena, que se ve que existe, dice: - Te dejo a Alfeo, si no te causa molestias; así podré ir más deprisa y llenarte muchos cántaros. 

Alfeo está muy contento de quedarse, y se ve el porqué una vez que se ha ido la madre: Ana le coge en brazos y lo lleva al huerto, lo aupa hasta una pérgola de uva de color oro como el topacio y dice: Come, come, que es buena" - y le besa en la carita embadurnada del zumo de las uvas que está desgranando ávidamente. 

Luego, cuando el niño, mirándola con dos ojazos de un gris azul oscuro todo abiertos, dice: 
- ¿Y ahora qué me das? - se echa a reír con ganas, y, al punto, parece más joven, borrados los años por la bonita dentadura y el gozo que viste su rostro. Y ríe y juega, metiendo su cabeza entre las rodillas y diciendo: 
- ¿Qué me das si te doy... si te doy?.. ¡Adivina! - Y el niño, dando palmadas con sus manecitas, todo sonriente, dice: 
- ¡Besos, te doy besos, Ana guapa, Ana buena, Ana mamá!... 

Ana, al sentirse llamar "Ana mamá", emite un grito de afecto jubiloso y abraza estrechamente al pequeñuelo, diciendo: - ¡0h, tesoro! ¡Amor! ¡Amor! ¡Amor! - Y por cada "amor" un beso va a posarse sobre las mejillitas rosadas. 
Luego van a un bazar y de un plato bajan tortitas de miel. 

Las he hecho para ti, hermosura de la pobre Ana, para ti que me quieres. Dime, ¿cuánto me quieres? Y el niño, pensando en la cosa que más le ha impresionado, dice: 

Como al Templo del Señor. 

Ana le da más besos: en los ojitos avispados, en la boquita roja. Y el niño se restriega contra ella como un gatito. 

La madre va y viene con un jarro colmado y ríe sin decir nada. Les deja con sus efusiones de afecto. 

Entra en el huerto un hombre anciano, un poco más bajo que Ana, de tupida cabellera completamente cana, rostro claro, barba cortada en cuadrado, dos ojos azules como turquesas, entre pestañas de un castaño claro casi rubio. Está vestido de un marrón oscuro. 

Ana no lo ve porque da la espalda a la puerta. El hombre se acerca a ella por detrás diciendo: ¿Y a mí nada? 

Ana se vuelve y dice:
- ¡Oh, Joaquín! ¿Has terminado tu trabajo? 

 Mientras tanto el pequeño Alfeo ha corrido a sus rodillas diciendo: 

También a ti, también a ti- y cuando el anciano se agacha y le besa, el niño se le ciñe estrechamente al cuello despeinándole la barba con las manecitas y los besos. 

También Joaquín trae su regalo: saca de detrás la mano izquierda y presenta una manzana tan hermosa que parece de cerámica, y, sonriendo, al niño que tiende ávidamente sus manecitas le dice: 

Espera, que te la parto en trozos. Así no puedes. Es más grande que tú - y con un pequeño cuchillo que tiene en el cinturón (un cuchillo de podador) parte la manzana en rodajas, que divide a su vez en otras más delgadas; y parece como si estuviera dando de comer en la boca a un pajarillo que no ha dejado todavía el nido, por el gran cuidado con que mete los trozos de manzana en esa boquita que muele incesantemente. 

- ¡Te has fijado qué ojos, Joaquín! ¿No parecen dos porcioncitas del Mar de Galilea cuando el viento de la tarde empuja un velo de nubes bajo el cielo? 

Ana ha hablado teniendo apoyada una mano en el hombro de su marido y apoyándose a su vez ligeramente en ella: gesto éste que revela un profundo amor de esposa, un amor intacto tras muchos años de vínculo conyugal. 

Joaquín la mira con amor, y asiente diciendo: 

- ¡Bellísimos! ¿Y esos ricitos? ¿No tienen el color de la mies secada por el sol? Mira, en su interior hay mezcla de oro y cobre. 

- ¡Ah, si hubiéramos tenido un hijo, lo habría querido así, con estos ojos y este pelo!... - Ana se ha curvado, es más, se ha arrodillado, y, con un fuerte suspiro, besa esos dos ojazos azul - grises. 

También suspira Joaquín, y, queriéndola consolar, le pone la mano sobre el pelo rizado y canoso, y le dice: 

Todavía hay que esperar. Dios todo lo puede. Mientras se vive, el milagro puede producirse, especialmente cuando se le ama y cuando nos amamos". Joaquín recalca mucho estas últimas palabras. 

Mas Ana guarda silencio, descorazonada, con la cabeza agachada, para que no se vean dos lágrimas que están deslizándose y que advierte sólo el pequeño Alfeo, el cual, asombrado y apenado de que su gran amiga llore como hace él alguna vez, levanta la manita y enjuga su llanto. 

- ¡No llores, Ana! Somos felices de todas formas. Yo por lo menos lo soy, porque te tengo a ti. 

Yo también por ti. Pero no te he dado un hijo... Pienso que he entristecido al Señor porque ha hecho infecundas mis entrañas... 

- ¡Oh, esposa mía! ¿En qué crees tú, santa, que has podido entristecerlo? Mira, vamos una vez más al Templo y por esto, no sólo por los Tabernáculos, hacemos una larga oración... Quizás te suceda como a Sara... o como a Ana de Elcana: esperaron mucho y se creían reprobadas por ser estériles, y, sin embargo, en el Cielo de Dios, estaba madurando para ellas un hijo santo. Sonríe, esposa mía. Tu llanto significa para mí más dolor que el no tener prole... Llevaremos a Alfeo con nosotros. Le diremos que rece. Él es inocente... Dios tomará juntas nuestra oración y la suya y se mostrará propicio. 

Sí. Hagamos un voto al Señor. Suyo será el hijo; si es que nos lo concede... ¡Oh, sentirme llamar "mamá"! Y Alfeo, espectador asombrado e inocente, dice: 
- ¡Yo te llamo "mamá"! 

Sí, tesoro amado... pero tú ya tienes mamá, y yo... yo no tengo niño.... La visión cesa aquí. 

\chapter{En la fiesta de los Tabernáculos.}
\emph{Joaquín y Ana poseían la Sabiduría.}
 
Antes de proseguir hago una observación. 

La casa no me ha parecido la de Nazaret, bien conocida. Al menos la habitación es muy distinta. Con respecto al huerto - jardín, debo decir que es también más amplio; además, se ven los campos, no muchos, pero... los hay. Después, ya casada María, sólo está el huerto (amplio, eso sí, pero sólo huerto). Y esta habitación que he visto no la he observado nunca en las otras visiones. No sé si pensar que por motivos pecuniarios los padres de María se hubieran deshecho de parte de su patrimonio, o si María, dejado el Templo, pasó a otra casa, que quizás le había dado José. No recuerdo si en las pasadas visiones y lecciones recibí alguna vez alusión segura a que la casa de Nazaret fuera la casa natal. 

Mi cabeza está muy cansada. Además, sobre todo por lo que respecta a los dictados, olvido enseguida las palabras, aunque, eso sí, me quedan grabadas las prescripciones que contienen, y, en el alma, la luz. Pero los detalles se borran inmediatamente. Si al cabo de una hora tuviera que repetir lo que he oído, aparte de una o dos frases de especial importancia, no sabría nada más. Las visiones, por el contrario, me quedan vivas en la mente, porque las he tenido que observar por mi misma. Los dictados los recibo. Aquéllas, por el contrario, tengo que percibirlas; permanecen, por tanto, vivas en el pensamiento, que ha tenido que trabajar para advertir sus distintas fases. 

Esperaba un dictado sobre la visión de ayer, pero no lo ha habido. 

Empiezo a ver y escribo. 

Fuera de los muros de Jerusalén, en las colinas, entre los olivos, hay gran multitud de gente. Parece un enorme mercado, pero no hay ni casetas ni puestos de venta ni voces de charlatanes y vendedores ni juegos. Hay muchas tiendas hechas de lana basta, sin duda impermeables, extendidas sobre estacas hincadas en el suelo. Atados a las estacas hay ramos verdes, como decoración y como medio para dar frescor. Otras, sin embargo, están hechas sólo de ramos hincados en el suelo y atados así; éstas crean como pequeñas galerías verdes. Bajo todas ellas, gente de las más distintas edades y condiciones y un rumor de conversación tranquilo e íntimo en que sólo desentona algún chillido de niño. 

Cae la tarde y ya las luces de las lamparitas de aceite resplandecen acá y allá por el extraño campamento. En tomo a estas luces, algunas familias, sentadas en el suelo, están cenando; las madres tienen en su regazo a los más pequeños, muchos de los cuales, cansados, se han quedado dormidos teniendo todavía el trozo de pan en sus deditos rosados, cayendo su cabecita sobre el pecho materno, como los polluelos bajo las alas de la gallina. Las madres terminan de comer como pueden, con una sola mano libre, sujetando con la otra a su hijito contra su corazón. Otras familias, por el contrario, no están todavía cenando. Conversan en la semioscuridad del crepúsculo esperando a que la comida esté hecha. Se ven lumbres encendidas, desperdigadas; en torno a ellas trajinan las mujeres. Alguna nana muy lenta, yo diría casi quejumbrosa, mece a algún niño que halla dificultad para dormirse. 

Encima, un hermoso cielo sereno, azul cada vez más oscuro hasta semejar a un enorme toldo de terciopelo suave de un color negro - azul; un cielo en el que, muy lentamente, invisibles artífices y decoradores estuvieran fijando gemas y lamparitas, ya aisladas, ya formando caprichosas líneas geométricas, entre las que destacan la Osa Mayor y Menor, que tienen forma de carro con la lanza apoyada en el suelo una vez liberados del yugo los bueyes. La estrella Polar ríe con todos sus resplandores. 

Me doy cuenta de que es el mes de Octubre. 

Aparece en la escena Ana. Viene de una de las hogueras con algunas cosas en las manos y colocadas sobre el pan, que es ancho y plano, como una torta de las nuestras, y que hace de bandeja. Trae pegado a las faldas a Alfeo, que va parla que te parla con su vocecita aguda. Joaquín está a la entrada de su pequeña tienda (toda de ramajes). Habla con un hombre de unos treinta años, al que saluda Alfeo desde lejos con un gritito diciendo: "Papá". Cuando Joaquín ve venir a Ana se da prisa en encender la lámpara. 

Ana pasa con su majestuoso caminar regio entre las filas de tiendas; regio y humilde. No es altiva con ninguno. Levanta a un niñito, hijo de una pobre, muy pobre, mujer, el cual ha tropezado en su traviesa carrera y ha ido a caer justo a sus pies. Dado que el niñito se ha ensuciado de tierra la carita y está llorando, ella le limpia y le consuela y, habiendo acudido la madre disculpándose, se lo restituye diciendo: 

- ¡Oh, no es nada! Me alegro de que no se haya hecho daño. Es un niño muy majo. ¿Qué edad tiene?". 

Tres años. Es el penúltimo. Dentro de poco voy a tener otro. Tengo seis niños. Ahora querría una niña... Para una mamá es mucho una niña.... 

- ¡Grande ha sido el consuelo que has recibido del Altísimo, mujer! - Ana suspira. 

La otra mujer dice: Sí. Soy pobre, pero los hijos son nuestra alegría, y ya los más grandecitos ayudan a trabajar. Y tú, señora - todos los signos son de que Ana es de condición más elevada, y la mujer lo ha visto - ¿Cuántos niños tienes? - Ninguno. 

- ¿Ninguno! ¿No es tuyo éste? 

No. De una vecina muy buena. Es mi consuelo... 

 - ¡Oh! - La mujer pobre la mira con piedad. 

Ana la saluda con un gran suspiro y se dirige a su tienda. 

Te he hecho esperar, Joaquín. Me ha entretenido una mujer pobre, madre de seis hijos varones, ¡fíjate! Y dentro de poco va a tener otro hijo. 

Joaquín suspira. 

El padre de Alfeo llama a su hijo, pero éste responde: "Yo me quedo con Ana. Así la ayudo. 

Todos se echan a reír. 

Déjalo. No molesta. Todavía no le obliga la Ley. Aquí o allí... no es más que un pajarito que come- dice Ana, y se sienta con el niño en el regazo; le da un pedazo de torta y, creo, pescado asado. Veo que hace algo antes de dárselo. Quizás le ha quitado la espina. Antes ha servido a su marido. La última que come es ella. 

La noche está cada vez más poblada de estrellas y las luces son cada vez más numerosas en el campamento. Luego muchas luces se van poco a poco apagando: son los primeros que han cenado, que ahora se echan a dormir. Va disminuyendo también lentamente el rumor de la gente. No se oyen ya voces de niños. Sólo resuena la vocecita de algún lactante buscando la leche de su mamá. La noche exhala su brisa sobre las cosas y las personas, y borra penas y recuerdos, esperanzas y rencores. 

Bueno, quizás estos dos sobrevivan, aun cuando hayan quedado atenuados, durante el sueño, en los sueños. 

Ana está meciendo a Alfeo, que empieza a dormirse en sus brazos. Entonces cuenta a su marido el sueño que ha tenido: 

Esta noche he soñado que el próximo año voy a venir a la Ciudad Santa para dos fiestas en vez de para una sola. Una será el ofrecimiento de mi hijo al Templo... ¡Oh! ¡Joaquín!...

Espéralo, espéralo. Ana. ¿No has oído alguna palabra? ¿El Señor no te ha susurrado al corazón nada? 

Nada. Un sueño sólo... 

Mañana es el último día de oración. Ya se han efectuado todas las ofrendas. No obstante, las renovaremos solemnemente mañana. Persuadiremos a Dios con nuestro fiel amor. Yo sigo pensando que te sucederá como a Ana de Elcana. 

Dios lo quiera... ¡Si hubiera, ahora mismo, alguien que me dijera: "Vete en paz. El Dios de Israel te ha concedido la gracia que pides"!...

Si ha de venir la gracia, tu niño te lo dirá moviéndose por primera vez en tu seno. Será voz de inocente y, por tanto, voz de Dios. 

Ahora el campamento calla en la oscuridad de la noche. Ana lleva a Alfeo a la tienda contigua y lo pone sobre la yacija de heno junto a sus hermanitos, que ya están dormidos. Luego se echa al lado de Joaquín. Su lamparita también se apaga, una de las últimas estrellitas de la tierra. Quedan, más hermosas, las estrellas del firmamento, velando a todos los durmientes. 

Dice Jesús: 

Los justos son siempre sabios, porque, siendo como son amigos de Dios, viven en su compañía y reciben instrucción de Él, de Él que es Infinita Sabiduría. Mis abuelos eran justos; poseían, por tanto, la sabiduría. Podían decir con verdad cuanto dice la Escritura cantando las alabanzas de la Sabiduría en el libro que lleva su nombre: "Yo la he amado y buscado desde mi juventud y procuré tomarla por esposa". Ana de Aarón era la mujer fuerte de que habla el Antepasado nuestro. Y Joaquín, de la estirpe del rey David, no había buscado tanto belleza y riqueza cuanto virtud. Ana poseía una gran virtud. Toda las virtudes unidas como ramo fragante de flores para ser una única, bellísima cosa, que era la Virtud, una virtud real, digna de estar delante del trono de Dios. Joaquín, por tanto, había tomado por esposa dos veces a la sabiduría "amándola más que a cualquier otra mujer": la sabiduría de Dios contenida dentro del corazón de la mujer justa. Ana de Aarón no había tratado sino de unir su vida a la de un hombre recto, con la seguridad de que en la rectitud se halla la alegría de las familias. Y, para ser el emblema de la "mujer fuerte", no le faltaba sino la corona de los hijos, gloria de la mujer casada, justificación del vínculo matrimonial, de que habla Salomón; como también a su felicidad sólo le faltaban estos hijos, flores del árbol que se ha hecho uno con el árbol cercano obteniendo copiosidad de nuevos frutos en los que las dos bondades se funden en una, pues de su esposo nunca había recibido ningún motivo de infelicidad. Ella, ya tendente a la vejez, mujer de Joaquín desde hacía varios lustros, seguía siendo para éste "la esposa de su juventud, su alegría, la cierva amadísima, la gacela donosa", cuyas caricias tenían siempre el fresco encanto de la primera noche nupcial y cautivaban dulcemente su amor, manteniéndolo fresco como flor que el rocío refresca y ardiente como fuego que siempre una mano alimenta. Por tanto, dentro de su aflicción, propia de quien no tiene hijos, recíprocamente se decían "palabras de consuelo en las preocupaciones y fatigas". Y la Sabiduría eterna, llegada la hora, después de haberlos instruido en la vida, los iluminó con los sueños de la noche, lucero de la mañana del poema de gloria que había de llegar a ellos, María Santísima., la Madre mía. Si su humildad no pensó en esto, su corazón sí se estremeció esperanzado ante el primer tañido de la promesa de Dios. Ya de hecho hay certeza en las palabras de Joaquín: "Espéralo, espéralo... Persuadiremos a Dios con nuestro fiel amor". Soñaban un hijo, tuvieron a la Madre de Dios. Las palabras del libro de la Sabiduría parecen escritas para ellos: "Por ella adquiriré gloria ante el pueblo... por ella obtendré la inmortalidad y dejaré eterna memoria de mí a aquellos que vendrán después de mí". Pero, para obtener todo esto, tuvieron que hacerse reyes de una virtud veraz y duradera no lesionada por suceso alguno. Virtud de fe. Virtud de caridad. Virtud de esperanza. Virtud de castidad. ¡Oh, la castidad de los esposos! Ellos la vivieron, pues no hace falta ser vírgenes para ser castos. Los tálamos castos tienen por custodios a los ángeles, y de tales tálamos provienen hijos buenos que de la virtud de sus padres hacen norma para su vida. Mas ahora ¿dónde están? Ahora no se desean hijos, pero no se desea tampoco la castidad. Por lo cual Yo digo que se profana el amor y se profana el tálamo. 

\chapter{Ana, con una canción, anuncia que es madre.}
\emph{En su seno está el alma inmaculada de María.}
 
Veo de nuevo la casa de Joaquín y Ana. Nada ha cambiado en su interior, si se exceptúan las muchas ramas florecidas, colocadas aquí y allá en jarrones (sin duda provienen de la podadura de los árboles del huerto, que están todos en flor: una nube que varía del blanco nieve al rojo típico de ciertos corales). 

También es distinto el trabajo que está realizando Ana. En un telar más pequeño, teje lindas telas de lino, y canta ritmando el movimiento del pie con la voz. Canta y sonríe... ¿A quién? A sí misma, a algo que ve en su interior. 

El canto, lento pero alegre, que he escrito aparte para seguirla, porque le repite una y otra vez, como gozándose en él, y cada vez con más fuerza y seguridad, como la persona que ha descubierto un ritmo en su corazón y primero lo susurra calladamente, y luego, segura, va más expedita y alta de tono, dice (y lo transcribo porque, dentro de su sencillez, es muy dulce):- ¡Gloria al Señor omnipotente que ha amado a los hijos de David! 

\begin{verse}
¡Gloria al Señor! \\
Su suprema gracia desde el Cielo me ha visitado. \\
El árbol viejo ha echado nueva rama y yo soy bienaventurada. \\
Por la Fiesta de las Luces echó semilla la esperanza; \\
ahora de Nisán la fragancia la ve germinar. \\
Como el almendro, se cubre de flores mi carne en primavera. \\
Su fruto, cercano ya el ocaso, ella siente llevar. \\
En la rama hay una rosa, hay uno de los más dulces pomos. \\
Una estrella reluciente, un párvulo inocente. \\
La alegría de la casa, del esposo y de la esposa. \\
Loor a Dios, a mi Señor, que piedad tuvo de mí. \\
Me lo dijo su luz: "Una estrella te llegará". \\
¡Gloria, gloria! Tuyo será este fruto del árbol, \\
primero y extremo, santo y puro como don del Señor. \\
Tuyo será. ¡Que por él venga alegría y paz a la tierra! \\
¡Vuela, lanzadera! Aprieta el hilo para la tela del recién nacido\\. 
¡Él nace! Laudatorio a Dios vaya el canto de mi corazón". 
\end{verse}
Entra Joaquín en el momento en que ella iba a repetir por cuarta vez su canto. 

- ¿Estás contenta, Ana? Pareces un ave en primavera. ¿Qué canción es ésta? A nadie se la he oído nunca. ¿De dónde nos viene? 

- De mi corazón, Joaquín. 

 Ana se ha levantado y ahora se dirige hacia su esposo, toda sonriente. Parece más joven y más guapa. 

- No sabía que fueras poetisa - dice su marido mirándola con visible admiración. No parecen dos esposos ya mayores. En su mirada hay una ternura de jóvenes cónyuges. 

- He venido desde la otra parte del huerto oyéndote cantar. Hacía años que no oía tu voz de tórtola enamorada. ¿Quieres repetirme esa canción? 

- Te la repetiría aunque no lo pidieras. Los hijos de Israel han encomendado siempre al canto los gritos más auténticos de sus esperanzas, alegrías y dolores. Yo he encomendado al canto la solicitud de anunciarme y de anunciarte una gran alegría. Sí, también a mí, porque es cosa tan grande que, a pesar de que yo ya esté segura de ella, me parece aún no verdadera... 

Y empieza a entonar de nuevo la canción. Pero cuando llega al punto: "En la rama hay una rosa, hay uno de los más dulces pomos, una estrella"..., su bien entonada voz de contralto primero se oye trémula y luego se rompe; se echa a llorar de alegría, mira a Joaquín y, levantando los brazos, grita: 

- ¡Soy madre, amado mío! - y se refugia en su corazón, entre los brazos que él ha tendido para volver a cerrarlos en torno a ella, su esposa dichosa. Es el más casto y feliz abrazo que he visto desde que estoy en este mundo. Casto y ardiente, dentro de su castidad. 

Y la delicada reprensión entre los cabellos blanco - negros de Ana: 

- ¿Y no me lo decías? 

- Porque quería estar segura. Siendo vieja como soy... verme madre... No podía creer que fuera verdad... y no quería darte la más amarga de las desilusiones. Desde finales de diciembre siento renovarse mis entrañas profundas y echar, como digo, una nueva rama. Mas ahora en esa rama el fruto es seguro... ¿Ves? Esa tela ya es para el que ha de venir. 

- ¿No es el lino que compraste en Jerusalén? 

Sí. Lo he hilado durante la espera... y con esperanza. Tenía esperanza por lo que sucedió el último día mientras oraba en el Templo, lo más que puede una mujer en la Casa de Dios, ya de noche. ¿Te acuerdas que decía: "Un poco más, todavía un poco más?" ¡No sabía separarme de allí sin haber recibido gracia! Pues bien, descendiendo ya las sombras, desde el interior del lugar sagrado al que yo miraba con arrobo para arrancarle al Dios presente su asentimiento, vi surgir una luz. Era una chispa de luz bellísima. Cándida como la luna pero que tenía en sí todas las luces de todas las perlas y gemas que hay en la tierra. Parecía como si una de las estrellas preciosas del Velo, las que están colocadas bajo los pies de los querubines, se separase y adquiriese esplendor de luz sobrenatural...Parecía como si desde el otro lado del Velo sagrado, desde la Gloria misma, hubiera salido un fuego y viniera veloz hacia mí, y que al cortar el aire cantara con voz celeste diciendo: "Recibe lo que has pedido". Por eso canto: "Una estrella te llegará". ¿Y qué hijo será éste, nuestro, que se manifiesta como luz de estrella en el Templo y que dice "existo" en la Fiesta de las Luces? ¿Será que has acertado al pensar en mí como una nueva Ana de Elcana? ¿Cómo la llamaremos a esta criatura nuestra que, dulce como canción de aguas, siento queme habla en el seno con su corazoncito, latiendo, latiendo, como el de una tortolita entre los huecos de las manos?". 

Si es varón, le llamaremos Samuel; si es niña, Estrella, la palabra que ha detenido tu canto para darme esta alegría de saber que soy padre, la forma que ha tomado para manifestarse entre las sagradas sombras del Templo. Estrella. Nuestra Estrella, porque... no lo sé, pero creo que es una niña. Pienso que unas caricias tan delicadas no pueden provenir sino de una dulcísima hija. Porque no la llevo yo, no me produce dolor; es ella la que me lleva por un sendero azul y florido, como si ángeles santos me sostuvieran y la tierra estuviera ya lejana... Siempre he oído decir a las mujeres que el concebir y el llevar al hijo en el seno supone dolor, pero yo no lo siento. Me siento fuerte, joven, fresca; más que cuando te entregué mi virginidad en la lejana juventud. Hija de Dios, porque es más de Dios que nuestra, siendo así que nacerá de un tronco aridecido, que no da dolor a su madre; sólo le trae paz y bendición: los frutos de Dios, su verdadero Padre. 

Entonces la llamaremos María. Estrella de nuestro mar, perla, felicidad, el nombre de la primera gran mujer de Israel. Pero no pecará nunca contra el Señor, que será el único al que dará su canto, porque ha sido ofrecida a Él como hostia antes de nacer. 

Está ofrecida a Él, sí. Sea niño o niña nuestra criatura, se la daremos al Señor, después de tres años de júbilo con ella. Nosotros seremos también hostias, con ella, para la gloria de Dios. 

No veo ni oigo nada más. 

Dice Jesús: 
\emph{La Sabiduría, tras haberlos iluminado con los sueños de la noche, descendió; Ella, que es "emanación de la potencia de Dios, genuino efluvio de la gloria del Omnipotente", y se hizo Palabra para la estéril. Quien ya veía cercano su tiempo de redimir, Yo, el Cristo, nieto de Ana, casi cincuenta años después, mediante la Palabra, obraría milagros en las estériles y en las enfermas, en las obsesas, en las desoladas; los obraría en todas las miserias de la tierra. Pero, entretanto, por la alegría de tener una Madre, he aquí que susurro una arcana palabra en las sombras del Templo que contenía las esperanzas de Israel, del Templo que ya estaba en la frontera de su vida. En efecto, un nuevo y verdadero Templo, no ya portador de esperanzas para un pueblo, sino certeza de Paraíso para el pueblo de toda la tierra, y por los siglos de los siglos hasta el fin del mundo, estaba para descender sobre la tierra. Esta Palabra obra el milagro de hacer fecundo lo que era infecundo, y de darme una Madre, la cual no tuvo sólo óptimo natural, como era de esperarse naciendo de dos santos, y no tuvo sólo un alma buena, como muchos también la tienen, y continuo crecimiento de esta bondad por su buena voluntad, ni sólo un cuerpo inmaculado... Tuvo, caso único entre las criaturas, inmaculado el espíritu. Tú has visto la generación continua de las almas por Dios. Piensa ahora cuál debió ser la belleza de esta alma que el Padre había soñado antes de que el tiempo fuera, de esta alma que constituía las delicias de la Trinidad, Trinidad que ardientemente deseaba adornarla con sus dones para donársela a sí misma. ¡Oh, Todo Santa que Dios creó para sí, y luego para salud de los hombres! Portadora del Salvador, tú fuiste la primera salvación; vivo Paraíso, con tu sonrisa comenzaste a santificar la tierra. ¡Oh, el alma creada para ser alma de la Madre de Dios!.. Cuando, de un más vivo latido del trino Amor, surgió esta chispa vital, se regocijaron los ángeles, pues luz más viva nunca había visto el Paraíso. Como pétalo de empírea rosa, pétalo inmaterial y preciado, gema y llama, aliento de Dios que descendía a animar a una carne de forma muy distinta que a las otras, con un fuego tan vivo que la Culpa no pudo contaminarla, traspasó los espacios y se cerró en un seno santo. La tierra tenía su Flor y aún no lo sabía. La verdadera, única Flor que florece eterna: azucena y rosa, violeta y jazmín, helianto y ciclamino sintetizados, y con ellas todas las flores de la tierra fusionadas en una Flor sola, María, en la cual toda virtud y gracia se unen. En Abril, la tierra de Palestina parecía un enorme jardín. Fragancias y colores deleitaban el corazón de los hombres. Sin embargo, aún ignorábase la más bella Rosa. Ya florecía para Dios en el secreto del claustro materno, porque mi Madre amó desde que fue concebida, mas sólo cuando la vid da su sangre para hacer vino, y el olor de los mostos, dulce y penetrante, llena las eras y el olfato, Ella sonreiría, primero a Dios y luego al mundo, diciendo con su superinocente sonrisa: "Mirad: la Vid que os va a dar el Racimo para ser prensado y ser Medicina eterna para vuestro mal está entre vosotros". He dicho que María amó desde que fue concebida. ¿Qué es lo que da al espíritu luz y conocimiento? La Gracia. ¿Qué es lo que quita la Gracia? El pecado original y el pecado mortal. María, la Sin Mancha, nunca se vio privada del recuerdo de Dios, de su cercanía, de su amor, de su luz, de su sabiduría. Ella pudo por ello comprender y amar cuando no era más que una carne que se condensaba en torno a un alma inmaculada que continuaba amando. Más adelante te daré a contemplar mentalmente la profundidad de las virginidades en María. Te producirá un vértigo celeste semejante a cuando te di a considerar nuestra eternidad. Entre tanto; piensa cómo el hecho de llevar en las entrañas a una criatura exenta de la Mancha que priva de Dios le da a la madre, que, no obstante, la concibió en modo natural, humano, una inteligencia superior, y la hace profeta, la profetisa de su hija, a la que llama "Hija de Dios". Y piensa lo que habría sido si de los Primeros Padres inocentes hubieran nacido hijos inocentes, como Dios quería. Éste, ¡oh, hombres que decís que vais hacia el "superhombre", y que de hecho con vuestros vicios estáis yendo únicamente hacia el super- demonio!, éste habría sido el medio que conduciría al "superhombre": saber estar libres de toda contaminación de Satanás, para dejarle a Dios la administración de la vida, del conocimiento, del bien; no deseando más de cuanto Dios os hubiera dado, que era poco menos que infinito, para poder engendrar, en una continua evolución hacia lo perfecto, hijos que fueran hombres en el cuerpo y, en el espíritu, hijos de la Inteligencia, es decir, triunfadores, es decir, fuertes, es decir, gigantes contra Satanás, que habría mordido el polvo muchos miles de siglos antes de la hora en que lo haga, y con él todo su mal. }
 
\chapter{Nacimiento de la Virgen María.}
\emph{Su virginidad en el eterno pensamiento del Padre.}

Veo a Ana saliendo al huerto - jardín. Va apoyándose en el brazo de una pariente (se ve porque se parecen). Está muy gruesa y parece cansada, quizás también porque hace bochorno, un bochorno muy parecido al que a mí me hace sentirme abatida. 

A pesar de que el huerto sea umbroso, el ambiente es abrasador y agobiante. Bajo un despiadado cielo, de un azul ligeramente enturbiado por el polvo suspendido en el espacio, el aire es tan denso, que podría cortarse como una masa blanda y caliente. Debe persistir ya mucho la sequía, pues la tierra, en los lugares en que no está regada, ha quedado literalmente reducida a un polvo finísimo y casi blanco. Un blanco ligeramente tendente a un rosa sucio. Sin embargo, por estar humedecida, es marrón oscura al pie de los árboles, como también a lo largo de los cortos cuadros donde crecen hileras de hortalizas, .y en torno a los rosales, a los jazmines o a otras flores de mayor o menor tamaño (que están especialmente a lo largo de todo el frente de una hermosa pérgola que divide en dos al huerto hasta donde empiezan las tierras, ya despojadas de sus mieses). La hierba del prado, que señala el final de la propiedad, está requemada; se ve rala. Sólo permanece la hierba más verde y tupida en los márgenes del prado, donde hay un seto de espino blanco silvestre, ya todo adornado de los rubíes de los pequeños frutos; en ese lugar, en busca de pastos y de sombra, hay unas ovejas con su zagalillo. 

Joaquín, con otros dos hombres como ayuda, está dedicado a las hortalizas y a los olivos. A pesar de ser anciano, es rápido y trabaja con gusto. Están abriendo unas pequeñas protecciones de las lindes de una parcela para proporcionar agua a las sedientas plantas. Y el agua se abre camino borboteando entre la hierba y la tierra quemada, y se extiende en anillos que, en un primer momento, parecen como de cristal amarillento para luego ser anillos oscuros de tierra húmeda en torno a los sarmientos y a los olivos colmados de frutos. 

Lentamente, Ana, por la umbría pérgola, bajo la cual abejas de oro zumban ávidas del azúcar de los dorados granos de las uvas, se dirige hacia Joaquín, el cual, cuando la ve, se apresura a ir a su encuentro. 

- ¿Has llegado hasta aquí? 

La casa está caliente como un horno". 

Y te hace sufrir". 

Es mi único sufrimiento en este último período mío de embarazo. Es el sufrimiento de todos, de hombres y de animales. No te sofoques demasiado, Joaquín. 

El agua que hace tanto que esperamos, y que hace tres días que parece realmente cercana, no ha llegado todavía. Las tierras arden. Menos mal que nosotros tenemos el manantial cercano, y muy rico en agua. He abierto los canales. Poco alivio para estas plantas cuyas hojas ya languidecen cubiertas de polvo. No obstante, supone ese mínimo que las mantiene en vida. ¡Si lloviera!... 

Joaquín, con el ansia de todos los agricultores, escudriña el cielo, mientras Ana, cansada, se da aire con un abanico (parece hecho con una hoja seca de palma traspasada por hilos multicolores que la mantienen rígida). 

La pariente dice: Allí, al otro lado del Gran Hermón, están formándose nubes que avanzan velozmente. Viento del norte. Bajará la temperatura y dará agua. 

Hace tres días que se levanta y luego cesa cuando sale la Luna. Sucederá lo mismo esta vez - Joaquín está desalentado. 

Vamos a casa. Aquí tampoco se respira; además, creo que conviene volver - dice Ana, que ahora se le ha puesto de improviso pálida la cara. 

- ¿Sientes dolor? 

No. Siento la misma gran paz que experimenté en el Templo cuando se me otorgó la gracia, y que luego volví a sentir otra vez al saber que era madre. Es como un éxtasis. Es un dulce dormir del cuerpo, mientras el espíritu exulta y se aplaca con una paz sin parangón humano. Yo te he amado, Joaquín, y, cuando entré en tu casa y me dije: "Soy esposa de un justo", sentí paz, como todas las otras veces que tu próvido amor se prodigaba en mí. Pero esta paz es distinta. Creo que es una paz como la que debió invadir, como una deleitosa unción de aceite, el espíritu de Jacob, nuestro padre, después de su sueño de ángeles. O semejante, más bien, a la gozosa paz de los Tobías tras habérseles manifestado Rafael. Si me sumerjo en ella, al saborearla, crece cada vez más. Es como si yo ascendiera por los espacios azules del cielo... y, no sé por qué, pero, desde que tengo en mí esta alegría pacífica, hay un cántico en mi corazón: el del anciano Tobit. Me parece como si hubiera sido compuesto para esta hora... para esta alegría... para la tierra de Israel que es su destinataria... para Jerusalén, pecadora, mas ahora perdonada... bueno... no os riáis de los delirios de una madre... pero, cuando digo: "Da gracias al Señor por tus bienes y bendice al Dios de los siglos para que vuelva a edificar en ti su Tabernáculo", yo pienso que aquel que reedificará en Jerusalén el Tabernáculo del Dios verdadero, será este que está para nacer... y pienso también que, cuando el cántico dice: "Brillarás con una luz espléndida, todos los pueblos de la tierra se postrarán ante ti, las naciones irán a ti llevando dones, adorarán en ti al Señor y considerarán santa tu tierra, porque dentro de ti invocarán el Gran Nombre. Serás feliz en tus hijos porque todos serán bendecidos y se reunirán ante el Señor. ¡Bienaventurados aquellos que te aman y se alegran de tu paz!...", cuando dice esto, pienso que es profecía no ya de la Ciudad Santa, sino del destino de mi criatura, y la primera que se alegra de su paz soy yo, su madre feliz... 

El rostro de Ana, al decir estas palabras, palidece y se enciende, como una cosa que pasase de luz lunar a vivo fuego, y viceversa. Dulces lágrimas le descienden por las mejillas, y no se da cuenta, y sonríe a causa de su alegría. Y va yendo hacia casa entre su esposo y su pariente, que escuchan conmovidos en silencio. 

Se apresuran, porque las nubes, impulsadas por un viento alto, galopan y aumentan en el cielo mientras la llanura se oscurece y tirita por efectos de la tormenta que se está acercando. Llegando al fibra! de la puerta, un primer relámpago lívido surca el cielo. El ruido del primer trueno se asemeja al redoble de un enorme bombo ritmado con el arpegio de las primeras gotas sobre las abrasadas hojas. 

Entran todos. Ana se retira. Joaquín se queda en la puerta con unos peones que le han alcanzado, hablando de esta agua tan esperada, bendición para la sedienta tierra. Pero la alegría se transforma en temor, porque viene una tormenta violentísima con rayos y nubes cargadas de granizo. 

Si rompe la nube, la uva y las aceitunas quedarán trituradas como por rueda de molino. ¡Pobres de nosotros!". 

Joaquín tiene además otro motivo de angustia: su esposa, a la que le ha llegado la hora de dar a luz al hijo. La pariente le dice que Ana no sufre en absoluto. Él está, de todas formas, muy inquieto, y, cada vez que la pariente u otras mujeres (entre las cuales está la madre de Alfeo) salen de la habitación de Ana para luego volver con agua caliente, barreños y paños secados a la lumbre, que, jovial, brilla en el hogar central en una espaciosa cocina, él va y pregunta, y no le calman las explicaciones tranquilizadoras de las mujeres. También le preocupa la ausencia de gritos por parte de Ana. Dice: 

Yo soy hombre. Nunca he visto dar a luz. Pero recuerdo haber oído decir que la ausencia de dolores es fatal...

Declina el día antes de tiempo por la furia de la tormenta, que es violentísima. Agua torrencial, viento, rayos... de todo, menos el granizo, que ha ido a caer a otro lugar. 

Uno de los peones, sintiendo esta violencia, dice: 

Parece como si Satanás hubiera salido de la Gehena con sus demonios. ¡Mira qué nubes tan negras! ¡Mira qué exhalación de azufre hay en el ambiente, y silbidos y voces de lamento y maldición! Si es él, ¡está enfurecido esta noche! 

El otro peón se echa a reír y dice: 

Se le habrá escapado una importante presa, o quizás Miguel de nuevo le habrá lanzado el rayo de Dios, y tendrá cuernos y cola cortados y quemados. 

Pasa corriendo una mujer y grita: 

- ¡Joaquín! ¡Va a nacer de un momento a otro! ¡Todo ha ido rápido y bien! 

Y desaparece con una pequeña ánfora en las manos. 

Se produce un último rayo; tan violento, que lanza contra las paredes a los tres hombres. En la parte delantera de la casa, en el suelo del huerto, queda como recuerdo un agujero negro y humeante. Luego, de repente, cesa la tormenta. De detrás de la puerta de Ana viene un vagido (parece el lamento de una tortolita en su primer arrullo). Mientras, un enorme arco iris extiende su faja semicircular por toda la amplitud del cielo. Surge, o por lo menos lo parece, de la cima del Hermón (la cual, besada por un filo de sol, parece de alabastro de un blanco - rosa delicadísimo), se eleva hasta el más terso cielo septembrino y, salvando espacios limpios de toda impureza, deja debajo las colinas de Galilea y un terreno llano que aparece entre dos higueras, que está al Sur, y luego otro monte, y parece posar su punta extrema en el extremo horizonte, donde una abrupta cadena de montañas detiene la vista. 

- ¡Qué cosa más insólita! 

- ¡Mirad, mirad! 

- Parece como si reuniera en un círculo a toda la tierra de Israel, y... ya... ¡fijaos!, ya hay una estrella y el Sol no se ha puesto todavía. ¡Qué estrella! ¡Reluce como un enorme diamante!.. 

- ¡Y la Luna, allí, ya llena y aún faltaban tres días para que lo fuera! ¡Fijaos cómo resplandece! 

Las mujeres irrumpen, alborozadas, con un "ovillejo" rosado entre cándidos paños. 

¡Es María, la Mamá! Una María pequeñita, que podría dormir en el círculo de los brazos de un niño; una María que al máximo tiene la longitud de un brazo, una cabecita de marfil teñido de rosa tenue, y unos labiecillos de carmín que ya no lloran sino que instintivamente quieren mamar (tan pequeñitos, que no se ve cómo van a poder coger un pezón), y una naricita diminuta entre dos carrillitos redondetes. Si la estimulan abre los ojitos: dos pedacitos de cielo, dos puntitos inocentes y azules que miran, y no ven, entre sutiles pestañas de un rubio tan tenue que es casi rosa. También el vello de su cabeza redondita tiene una veladura entre rosada y rubia como ciertas mieles casi blancas. 

Tiene por orejas dos conchitas rosadas y transparentes, perfectas; y por manitas... ¿qué son esas dos cositas que gesticulan y buscan la boca? Cerradas, como están, son dos capullos de rosa de musgo que hubieran hendido el verde de los sépalos y asomaran su seda rosa tenue; abiertas, como están ahora, dos joyeles de marfil apenas rosa, de alabastro apenas rosa, con cinco pálidos granates por uñitas. ¿Cómo podrán ser capaces de secar tanto llanto esas manitas? ¿Y los piececitos? ¿Dónde están? Por ahora son sólo pataditas escondidas entre los lienzos. Pero, he aquí que la pariente se sienta y la destapa... ¡Oh, los piececitos! De la largura aproximada de cuatro centímetros, tienen por planta una concha coralina; por dorso, una concha de nieve veteada de azul; sus deditos son obras maestras de escultura liliputiense, coronados también por pequeñas esquirlas de granate pálido. Me pregunto cómo podrán encontrarse sandalias tan pequeñas que valgan para esos piececitos de muñeca cuando den sus primeros pasos, y cómo podrán esos piececitos recorrer tan áspero camino y soportar tanto dolor bajo una cruz. Pero esto ahora no se sabe. Se ríe o se sonríe de cómo menea los brazos y las piernas, de sus lindas piernecitas bien perfiladas, de los diminutos muslos, que, de tan gorditos como son, forman hoyuelos y aritos, de su barriguita (un cuenco invertido), de su pequeño tórax, perfecto, bajo cuya seda cándida se ve el movimiento de la respiración y se oye ciertamente, si, como hace el padre feliz ahora, en él se apoya la boca para dar un beso, latir un corazoncito... Un corazoncito que es el más bello que ha tenido, tiene y tendrá la tierra, el único corazón inmaculado de hombre. ¿Y la espalda? Ahora la giran y se ve el surco lumbar y luego los hombros, llenitos, y la nuca rosada, tan fuerte, que la cabecita se yergue sobre el arco de las vértebras diminutas, como la de un ave escrutadora en torno a sí del nuevo mundo que ve, y emite un gritito de protesta por ser mostrada en ese modo; Ella, la Pura y Casta, ante los ojos de tantos, Ella, que jamás volverá a ser vista desnuda por hombre alguno, la Toda Virgen, la Santa e Inmaculada. Tapad, tapad a este Capullo de azucena que nunca se abrirá en la tierra, y que dará, más hermosa aún que Ella, su Flor, sin dejar de ser capullo. Sólo en el Cielo la Azucena del Trino Señor abrirá todos sus pétalos. Porque allí arriba no existe vestigio de culpa que pudiera involuntariamente profanar ese candor. Porque allí arriba se trata de acoger, a la vista de todo el Empíreo, al Trino Dios - Padre, Hijo, Esposo - que ahora, dentro de pocos años, celado en un corazón sin mancha, vendrá a Ella. 

De nuevo está envuelta en los lienzos y en los brazos de su padre terreno, al que asemeja. No ahora, que es un bosquejo de ser humano. Digo que le asemeja una vez hecha mujer. De la madre no refleja nada; del padre, el color de la piel y de los ojos, y, sin duda, también del pelo, que, si ahora son blancos, de joven eran ciertamente rubios a juzgar por las cejas. Del padre son las facciones — más perfectas y delicadas en Ella por ser mujer, ¡y qué Mujer!; también del padre es la sonrisa y la mirada y el modo de moverse y la estatura. Pensando en Jesús como lo veo, considero que ha sido Ana la que ha dado su estatura a su Nieto, así como el color marfil más cargado de la piel; mientras que María no tiene esa presencia de Ana (que es como una palma alta y flexible), sino la finura del padre. También las mujeres, mientras entran con Joaquín donde se encuentra la madre feliz para devolverle a su hijita, hablan de la tormenta y del prodigio de la Luna, de la estrella, del enorme arco iris. 

Ana sonríe ante un pensamiento propio: 

Es la estrella – dice Su signo está en el cielo. ¡María, arco de paz! ¡María, estrella mía! ¡María, Luna pura! ¡María, perla nuestra! 

¿María la llamas? 

Sí. María, estrella y perla y luz y paz... 

Pero también quiere decir amargura... ¿No temes acarrearle alguna desventura? 

Dios está con Ella. Es suya desde antes de que existiera. El la conducirá por sus vías y toda amargura se transformará en paradisíaca miel. Ahora sé de tu mamá... todavía un poco, antes de ser toda de Dios.... 

Y la visión termina en el primer sueño de Ana madre y de María recién nacida. 

Dice Jesús: 
\emph{Levántate y apresúrate, pequeña amiga. Siento ardiente deseo de llevarte conmigo al azul paradisíaco de la contemplación de la Virginidad de María. Saldrás de él con el alma fresca como si tú también hubieras sido recientemente creada por el Padre, una pequeña Eva antes de conocer carne; saldrás con el espíritu lleno de luz, pues te habrás abismado en la contemplación de la obra maestra de Dios; con todo tu ser repleto de amor, pues habrás comprendido cómo sabe amar Dios. Hablar de la concepción de María, la Sin Mancha, significa sumergirse en lo azul, en la luz, en el amor. Ven y lee sus glorias en el Libro del Antepasado: "Dios me poseyó al inicio de sus obras, desde el principio, antes de la creación. Ab aeterno fui erigida, al principio, antes de que la tierra fuera hecha; aún no existían los abismos, y yo ya había sido concebida. Aún no manaba agua de los manantiales, aún no se elevaban con su pesada mole los montes, aún las colinas no eran para el Sol collares... y yo ya había nacido. Dios no había hecho todavía la tierra ni los ríos ni las columnas del mundo, y yo ya existía. Cuando preparaba los cielos, yo estaba presente, cuando con ley inmutable clausuró el abismo bajo la bóveda, cuando fijó arriba la bóveda celeste y colgó de ella las fuentes de las aguas, cuando al mar le establecía sus confines y daba leyes a las aguas, cuando daba leyes a las aguas de no sobrepasar su límite, cuando echaba los fundamentos de la tierra, yo estaba con Él ordenando todas las cosas. Siempre alegre jugueteaba ante Él continuamente, jugueteaba en el universo...". Las habéis aplicado a la Sabiduría, pero hablan de Ella: la hermosa Madre, la santa Madre, la Virgen Madre de la Sabiduría, que soy Yo, el que te habla. He querido que escribieras, como encabezamiento del libro que habla de Ella, el primer verso de este himno, para que fuera confesado y conocido el consuelo y la alegría de Dios; la razón de la constante, perfecta, íntima alegría de este Dios Uno y Trino que os sostiene y ama y que del hombre recibió tantos motivos de tristeza; la razón de que perpetuara la raza aun cuando ésta, con la primera prueba, había merecido la destrucción; la razón del perdón que habéis recibido. Que María le amara... ¡Oh, bien merecía la pena crear al hombre y dejarlo vivir y decretar perdonarlo, para tener a la Virgen bella, a la Virgen santa, a la Virgen inmaculada, a la Virgen enamorada, a la Hija dilecta, a la Madre purísima, a la Esposa amorosa! Mucho os ha dado, y más aún os habría dado, Dios, con tal de poseer a la Criatura de sus delicias, al Sol de su sol y Flor de su jardín. Y mucho os sigue dando por Ella, a petición de Ella, para alegría de Ella, porque su alegría se vierte en la alegría de Dios y la aumenta con destellos que llenan de resplandores la luz, la gran luz del Paraíso, y cada resplandor es una gracia para el universo, para la raza del hombre, para los mismos bienaventurados, que responden con un esplendoroso grito de aleluya a cada milagro que sale de Dios, creado por el deseo del Dios Trino de ver la esplendorosa sonrisa de alegría de la Virgen. Dios quiso poner un rey en ese universo que había creado de la nada. Un rey que, por naturaleza material, fuera el primero entre todas las criaturas creadas con materia y dotadas de materia. Un rey que, por naturaleza espiritual, fuera poco menos que divino, fundido con la Gracia, como en su inocente primer día. Pero la Mente suprema, que conoce la totalidad de los hechos más lejanos en el tiempo, la Mente cuya vista ve incesantemente todo cuanto era, es y será, y que, mientras contempla el pasado y observa el presente, hunde su mirada en el extremo futuro, no ignorando cómo será el morir del último hombre, sin confusión ni discontinuidad, esa Mente no ignoró nunca que ese rey, creado para ser semidivino a su lado en el Cielo, heredero del Padre, cuando llegara como adulto a su Reino después de haber vivido en la casa de su madre — la tierra con la que fue hecho —, durante su niñez de párvulo del Eterno en su jornada sobre la tierra, cometería hacia sí mismo el delito de matarse en la Gracia y el latrocinio de despojarse del cielo. ¿Por qué lo creó entonces? Sin duda muchos se hacen esta pregunta. ¿Habríais preferido no existir? ¿No merece ser vivida esta jornada incluso por sí misma, a pesar de ser tan pobre y desnuda, y tan severa a causa de vuestra maldad, para conocer y admirar la Belleza infinita que la mano de Dios ha sembrado en el universo? ¿Para quién, si no, habría hecho estos astros y planetas que pasan como saetas, como flechas, rayando la bóveda del firmamento, o van — y parecen lentos —, van majestuosos con su paso veloz de bólidos, regalándoos luces y estaciones, y dándoos, eternos, inmutables aunque siempre mutables, a leer en el cielo una nueva página, cada noche, cada mes, cada año, como queriendo deciros: "Olvidaos de la cárcel, abandonad esa imagen vuestra llena de cosas oscuras, podridas, sucias, venenosas, mentirosas, blasfemas, corruptoras, y elevaos, al menos con la mirada, a la ilimitada libertad de los firmamentos; haceos un alma azul mirando tanta limpidez de cielo, haceos con una reserva de luz que podáis llevar a vuestra oscura cárcel; leed la palabra que escribimos cantando en coro nuestra melodía sideral, más armoniosa que si proviniera de un órgano de catedral, la palabra que escribimos resplandeciendo, la palabra que escribimos amando, porque siempre tenemos presente a Aquel que nos dio la alegría de existir, y le amamos por habernos dado este existir, este resplandecer, este movemos, este ser libres y bellos en medio de este cielo delicado allende el cual vemos un cielo aún más sublime, el Paraíso; a Aquel cuyo precepto de amor en su segunda parte cumplimos al amaros a vosotros, prójimo universal nuestro, al amaros proporcionándoos guía y luz, calor y belleza. Leed la palabra que decimos, la palabra a la que ajustamos nuestro canto, nuestro resplandecer, nuestro reír: Dios"? ¿Para quién habría hecho ese líquido azul: para el cielo, espejo; para la tierra, camino; sonrisa de aguas; voz de olas; palabra, también, que, con frufrú de roce de seda, con risitas de muchachas serenas, con suspiros de ancianos que recuerdan y lloran, con bofetadas de violentos, y con envites y bramidos y estruendos, siempre habla y dice: "Dios"? El mar es para vosotros, como lo son el cielo y los astros. Y con el mar los lagos y los ríos, los estanques y los arroyos, y los manantiales puros, que sirven, todos, para transportaros, para nutriros, para apagar vuestra sed y limpiaros, y que os sirven, sirviendo al Creador, sin salir a sumergiros, como merecéis. ¿Para quién habría hecho las innumerables familias de los animales, que son flores que vuelan cantando, que son siervos que trabajan, que corren, que os alimentan, que os recrean a vosotros, los reyes? ¿Para quién habría hecho las innumerables familias de las plantas y de las flores, que parecen mariposas, que parecen gemas e inmóviles avecillas; de los frutos, que parecen collares de oro y piedras preciosas o cofres de gemas? Son alfombra para vuestros pies, protección para vuestras cabezas, recreo, beneficio, alegría para la mente, para los miembros del cuerpo, para la vista y el olfato. ¿Para quién, si no, habría hecho los minerales en las entrañas de la Tierra y las sales disueltas en manantiales de álgidas aguas o de agua hirviendo: los azufres, los yodos, los bromos?.. Ciertamente, para que los gozara uno que no fuera Dios, sino hijo de Dios. Uno: el hombre. Nada le faltaba a la alegría de Dios, nada necesitaba Dios. El se basta a sí mismo. No tiene sino que contemplarse para deleitarse, nutrirse, vivir y descansar. Toda la creación no ha aumentado ni en un átomo su infinidad de alegría, de belleza, de vida, de potencia. He aquí que todo lo ha hecho para la criatura a la que ha querido poner como rey de la obra de sus manos: para el hombre. Aunque sólo fuera por ver una obra divina de tal magnitud y por manifestarle reconocimiento a Dios, que os la otorga, merecería la pena vivir. Y debéis sentir gratitud por el hecho de vivir. Gratitud que deberíais haber tenido aunque no hubierais sido redimidos sino al final de los siglos, porque, a pesar de que hayáis sido, en los Primeros, y ahora aun individualmente, prevaricadores, soberbios, lujuriosos, homicidas, Dios os concede todavía gozar de lo bello del universo, de lo bueno del universo, y os trata como si fuerais personas buenas, hijos buenos a los cuales todo se enseña y todo se concede para hacerles más suave y sana la vida. Cuanto sabéis, lo sabéis por luz de Dios. Cuanto descubrís, lo descubrís porque Dios os lo señala. Esto, en el Bien. Los otros conocimientos y descubrimientos que llevan el signo del mal vienen del Mal supremo: Satanás. La Mente suprema, que nada ignora, antes de que el hombre fuese, sabía que sería ladrón y homicida de sí mismo. Y, dado que la Bondad eterna no conoce límites en su ser buena, antes de que la Culpa fuera, pensó el medio para anular la Culpa. El medio, Yo; el instrumento para hacer del medio un instrumento operante, María. Y la Virgen fue creada en el pensamiento sublime de Dios. Todas las cosas han sido creadas para mí, Hijo dilecto del Padre. Yo- Rey habría debido tener bajo mi pie de Rey divino alfombras y joyas como palacio alguno jamás tuviera, y cantos y voces, y tantos siervos y ministros en torno a Mí como soberano alguno jamás tuviera, y flores y gemas, y todo lo sublime, lo grandioso, lo fino, lo delicado que es posible extraer del pensamiento de todo un Dios. Mas Yo debía ser Carne además de Espíritu. Carne para salvar a la carne. Carne para sublimar la carne, llevándola al Cielo muchos siglos antes de la hora. Porque la carne habitada por el espíritu es la obra maestra de Dios, y para ella había sido hecho el Cielo. Para ser Carne tenía necesidad de una Madre. Para ser Dios tenía necesidad de que el Padre fuese Dios. He aquí que entonces Dios se crea a su Esposa y le dice: "Ven conmigo. Junto a mí ve cuanto Yo hago para el Hijo nuestro. Mira y regocíjate, eterna Virgen, Doncella eterna, y tu risa llene este empíreo y dé a los ángeles la nota inicial y al Paraíso le enseñe la armonía celeste. Yo te miro, y te veo como serás, ¡oh, Mujer inmaculada que ahora eres sólo espíritu: el espíritu en que Yo me deleito! Yo te miro y doy al mar y al firmamento el azul de tu mirada; el color de tus cabellos, al trigo santo; el candor, a la azucena; el color rosa como tu epidermis de seda, a la rosa; de tus dientes delicados copio las perlas; hago las dulces fresas mirando tu boca; a los ruiseñores les pongo en la garganta tus notas y a las tórtolas tu llanto. Leyendo tus futuros pensamientos, oyendo los latidos de tu corazón, tengo el motivo guía para crear. Ven, Alegría mía, séante los mundos juguete hasta que me seas luz danzarina en el pensamiento, sean los mundos para reír tuyo. Tente las guirnaldas de estrellas y los collares de astros, ponte la luna bajo tus nobles pies, adórnate con el chal estelar de Galatea. Son para ti las estrellas y los planetas. Ven y goza viendo las flores que le servirán a tu Niño como juego y de almohada al Hijo de tu vientre. Ven y ve crear las ovejas y los corderos, las águilas y las palomas. Estate a mi lado mientras hago las cuencas de los mares y de los ríos, y alzo las montañas y las pinto de nieve y de bosques; mientras siembro los cereales y los árboles y las vides, y hago el olivo para ti, Pacífica mía, y la vid para ti, Sarmiento mío que llevarás el Racimo eucarístico. Camina, vuela, regocíjate, ¡oh, Hermosa mía!, y que el mundo universo, que en diversas fases voy creando, aprenda de ti a amarme, Amorosa, y que tu risa le haga más bello, Madre de mi Hijo, Reina de mi Paraíso, Amor de tu Dios". Y, viendo a quien es el Error y mirando a la Sin Error, dice: "Ven a mí, tú que cancelas la amargura de la desobediencia humana, de la fornicación humana con Satanás y de la humana ingratitud. Contigo me tomaré la revancha contra Satanás". Dios, Padre Creador, había creado al hombre y a la mujer con una ley de amor tan perfecta, que vosotros no podéis ni siquiera comprender sus perfecciones; vuestra mente se pierde pensando en cómo habría venido la especie si el hombre no la hubiera obtenido con la enseñanza de Satanás. Observad las plantas de fruto y de grano. ¿Obtienen la semilla o el fruto mediante fornicación, mediante una fecundación por cada cien uniones? No. De la flor masculina sale el polen y, guiado por un complejo de leyes meteóricas y magnéticas, va hacia el ovario de la flor femenina. Éste se abre y lo recibe y produce; no como hacéis vosotros, para experimentar al día siguiente la misma sensación, se mancha y luego lo rechaza. Produce, y hasta la nueva estación no florece, y cuando florece es para reproducirse. Observad a los animales. Todos. ¿Habéis visto alguna vez a un macho y a una hembra ir el uno hacia el otro para estéril abrazo y lascivo comercio? No. Desde cerca o desde lejos, volando, arrastrándose, saltando o corriendo, van, llegada la hora, al rito fecundativo, y no se substraen a él deteniéndose en el goce, sino que van más allá de éste, van a las consecuencias serias y santas de la prole, única finalidad que en el hombre, semidiós por el origen de gracia, de esa Gracia que Yo he devuelto completa, debería hacer aceptar la animalidad del acto, necesario desde que descendisteis un grado hacia los brutos. Vosotros no hacéis como las plantas y los animales. Vosotros habéis tenido como maestro a Satanás, lo habéis querido y lo queréis como maestro. Y las obras que realizáis son dignas del maestro que habéis querido. Mas si hubieseis sido fieles a Dios, habríais recibido la alegría de los hijos santamente, sin dolor, sin extenuaros en cópulas obscenas, indignas, ignoradas incluso por las bestias, las bestias sin alma racional y espiritual. Dios quiso oponer, frente al hombre y a la mujer pervertidos por Satanás, al Hombre nacido de una Mujer suprasublimada por Dios hasta el punto de generar sin haber conocido varón: Flor que genera Flor sin necesidad de semilla; sólo por el beso del Sol en el cáliz inviolado de la Azucena- María. ¡La revancha de Dios!Echa resoplidos de odio, Satanás, mientras Ella nace. ¡Esta Párvula te ha vencido! Antes de que fueras el Rebelde, el Tortuoso, el Corruptor, eras ya el Vencido, y Ella es tu Vencedora. Mil ejércitos en formación nada pueden contra tu potencia, ceden las armas de los hombres contra tus escamas, ¡oh, Perenne!, y no hay viento capaz de llevarse el hedor de tu hálito. Y sin embargo este calcañar de recién nacida, tan rosa que parece el interior de una camelia rosada, tan liso y suave que comparada con él la seda es áspera, tan pequeño que podría caber en el cáliz de un tulipán y hacerse un zapatito de ese raso vegetal, he aquí que te comprime sin miedo, te confina en tu caverna. Y su vagido te pone en fuga, a ti que no tienes miedo de los ejércitos; y su aliento libera al mundo de tu hedor. Estás derrotado. Su nombre, su mirada, su pureza son lanza, rayo, losa que te traspasan, que te abaten, que te encierran en tu madriguera de Infierno, ¡oh, Maldito, que le has arrebatado a Dios la alegría de ser Padre de todos los hombres creados! Se demuestra inútil ahora el haber corrompido a quienes habían sido creados inocentes, conduciéndolos a conocer y a concebir por caminos sinuosos de lujuria, privándole a Dios, en su criatura dilecta, de ser Él quien distribuyera magnánimamente los hijos según reglas que, si hubieran sido respetadas, habrían mantenido en la tierra un equilibrio entre los sexos y las razas que hubiera podido evitar guerras entre los hombres y desgracias en las familias. Obedeciendo, habrían conocido también el amor. Es más, sólo obedeciendo lo habrían conocido y lo habrían poseído. Una posesión llena y tranquila de esta emanación de Dios, que de lo sobrenatural desciende hacia lo inferior, para que la carne también se goce santamente en ella, la carne que está unida al espíritu y que ha sido creada por el Mismo que le creó el espíritu. ¿Ahora, ¡oh, hombres!, vuestro amor, vuestros amores, qué son? O libídine vestida de amor o miedo incurable de perder el amor del cónyuge por libídine suya y de otros. Desde que la libídine está en el mundo, ya nunca os sentís seguros de la posesión del corazón del esposo o de la esposa; y tembláis y lloráis y enloquecéis de celos, asesináis a veces para vengar una traición, os desesperáis otras veces u os volvéis abúlicos o dementes. Eso es lo que has hecho, Satanás, a los hijos de Dios. Estos que tú has corrompido habrían conocido la dicha de tener hijos sin padecer dolor, la dicha de nacer y no tener miedo a morir. Mas ahora has sido derrotado en una Mujer y por la Mujer. De ahora en adelante quien la ame volverá a ser de Dios, venciendo a tus tentaciones para poder mirar a su inmaculada pureza. De ahora en adelante, no pudiendo concebir sin dolor, las madres la tendrán a Ella como consuelo. De ahora en adelante será guía para las esposas y madre para los moribundos, por lo que dulce será el morir sobre ese seno que es escudo contra ti, Maldito, y contra el juicio de Dios. María, (se dirige aquí a María Valtorta) pequeña voz, has visto el nacimiento del Hijo de la Virgen y el nacimiento de la Virgen al Cielo. Has visto, por tanto, que los sin culpa desconocen la pena del dar a luz y la pena del morir. Y, si a la superinocente Madre de Dios le fue reservada la perfección de los dones celestes, igualmente, si todos hubieran conservado la inocencia y hubieran permanecido como hijos de Dios en los Primeros, habrían recibido el generar sin dolores (como era justo por haber sabido unirse y concebir sin lujuria) y el morir sin aflicción. La sublime revancha de Dios contra la venganza de Satanás ha consistido en llevar la perfección de la dilecta criatura a una superperfección que anulara, al menos en una, cualquier vestigio de humanidad susceptible de recibir el veneno de Satanás, por lo cual el Hijo vendría no de casto abrazo de hombre sino de un abrazo divino que, en el éxtasis del Fuego, arrebola el espíritu. ¡La Virginidad de la Virgen!... Ven. Medita en esta virginidad profunda que produce al contemplarla vértigos de abismo! ¿Qué es, comparada con ella, la pobre virginidad forzada de la mujer con la que ningún hombre se ha desposado? Menos que nada. ¿Y la virginidad de la mujer que quiso ser virgen para ser de Dios, pero sabe serlo sólo en el cuerpo y no en el espíritu, en el cual deja entrar muchos pensamientos de otro tipo, y acaricia y acepta caricias de pensamientos humanos? Empieza a ser una sombra de virginidad. Pero bien poco aún. ¿Qué es la virginidad de una religiosa de clausura que vive sólo de Dios? Mucho. Pero nunca es perfecta virginidad comparada con la de mi Madre. Hasta en el más santo ha habido al menos un contubernio: el de origen, entre el espíritu y la Culpa, esa unión que sólo el Bautismo disuelve. La disuelve, sí, pero, como en el caso de una mujer separada de su marido por la muerte, no devuelve la virginidad total como era la de los Primeros antes del pecado. Una cicatriz queda, y duele, recordando así su presencia, cicatriz que puede siempre en cualquier momento traducirse de nuevo en una llaga, como ciertas enfermedades agudizadas periódicamente por sus virus. En la Virgen no existe esta señal de un disuelto ligamen con la Culpa. Su alma aparece bella e intacta como cuando el Padre la pensó reuniendo en Ella todas las gracias. Es la Virgen. Es la Única. Es la Perfecta. Es la Completa. Pensada así. Engendrada así. Que ha permanecido así. Coronada así. Eternamente así. Es la Virgen. Es el abismo de la intangibilidad, de la pureza, de la gracia que se pierde en el Abismo de que procede, es decir, en Dios, Intangibilidad, Pureza, Gracia perfectísimas. Así se ha desquitado el Dios Trino y Uno: Él ha alzado contra la profanación de las criaturas esta Estrella de perfección; contra la curiosidad malsana, esta Mujer Reservada que sólo se siente satisfecha amando a Dios; contra la ciencia del mal, esta Sublime Ignorante. Ignorante no sólo en lo que toca al amor degradado, o al amor que Dios había dado a los cónyuges, sino más todavía: en Ella se trata de ignorancia del fomes, herencia del Pecado. En Ella sólo se da la gélida e incandescente sabiduría del Amor divino. Fuego que encoraza de hielo la carne para que sea espejo transparente en el altar en que un Dios se desposa con una Virgen, y no por ello se rebaja, porque su perfección envuelve a Aquella que, como conviene a una esposa, es sólo inferior en un grado al Esposo, sujeta a Él por ser Mujer, pero, como Él, sin mancha". }

\chapter{Purificación de Ana y ofrecimiento de María,}
\emph{que es la Niña perfecta para el reino de los Cielos.}
 
Veo a Joaquín y a Ana, junto a Zacarías y a Isabel, saliendo de una casa de Jerusalén de amigos o familiares. Se dirigen hacia el Templo para la ceremonia de la Purificación. 

Ana lleva en brazos a la Niña, envuelta toda en fajos, toda envuelta en un amplio tejido de lana ligera, pero que debe ser suave y caliente. ¡Con cuánto cuidado y amor lleva a su criaturita! De vez en cuando levanta el borde del fino y caliente tejido para ver si María respira a gusto, y luego vuelve a taparla para protegerla del aire helador de un día sereno pero frío, de pleno invierno. 

Isabel lleva unos paquetes en las manos. Joaquín lleva de una cuerda a dos corderos blanquísimos bien cebados, ya más carneros que corderos. Zacarías no lleva nada. ¡Qué apuesto con ese vestido de lino que un grueso manto de lana, también blanca, deja entrever! Es un Zacarías mucho más joven que el que se veía en el nacimiento del Bautista, entonces ya en plena edad adulta. Isabel es una mujer madura, pero todavía de apariencia fresca; cada vez que Ana mira a la Niña, se curva extasiada hacia esa carita dormida. También Isabel está guapísima con su vestido de un azul tendente al morado oscuro y con el velo que le cubre la cabeza y cae sobre los hombros y sobre el manto, que es más oscuro que el vestido. 

¿Y Joaquín y Ana? ¡Ah..., solemnes con sus vestidos de fiesta! Contrariamente a lo normal, él no lleva la túnica marrón oscura, sino un largo vestido de un rojo oscurísimo (hoy diríamos: rojo S. José). Las orlas de su manto son bonitas y muy nuevas. En la cabeza lleva también una especie de velo rectangular, ceñido con una cinta de cuero. Todo nuevo y fino. 

Ana... ¡oh!, hoy no viste de oscuro. Lleva un vestido de un amarillo muy tenue, casi color marfil viejo, ceñido en la cintura, cuello y muñecas, con una gruesa cinta que parece de plata y oro. Su cabeza está cubierta por un velo ligerísimo y como adamascado, sujeto a la frente con un aro sutil, valioso. En el cuello lleva un collar de filigrana; en las muñecas, pulseras. Parece una reina, incluso por la dignidad con que lleva el vestido, y especialmente el manto, amarillo tenue, orlado con una greca en bordadura muy bonita, también amarilla. 

Me pareces como en el día de tu boda. Entonces yo era poco más que una niña. Todavía me acuerdo de lo guapa y dichosa que se te veía - dice Isabel. 

Pues más feliz me siento ahora... Y he querido ponerme el mismo vestido para este rito. Lo había conservado siempre para esto... aunque ya, para esto, no tenía esperanzas de ponérmelo. 

El Señor te ha amado mucho... - dice suspirando Isabel. 

Por eso precisamente le doy lo que más quiero. Esta flor mía. 

- ¿Y vas a tener fuerzas para arrancártela de tu seno cuando llegue el momento? 

Sí, porque recordaré que no la tenía y que Dios me la dio. En todo caso me sentiré más feliz que entonces. Y, sabiendo que está en el Templo, me diré: "Está orando ante el Tabernáculo, está rezando al Dios de Israel, y también por su madre". Ello me dará paz. Y más paz todavía al decir: "Ella es toda suya. Cuando estos dos felices ancianos, que la recibieron del Cielo, ya no estén en este mundo, Él, el Eterno, seguirá siendo su Padre". Créeme, tengo la firme convicción de que esta pequeñuela no es nuestra. Yo ya no podía hacer nada... Él la puso en mi seno como don divino para enjugar mi llanto y confortar nuestras esperanzas y oraciones. Por tanto, es suya. Nosotros somos los encargados, felices encargados, de cuidarla... ¡y que por ello sea bendito! 

Llegan a los muros del Templo. 
 
Mientras vais a la Puerta de Nicanor, yo voy a advertir al sacerdote. Luego os alcanzo - dice Zacarías; y desaparece tras un arco que introduce a un amplio patio circundado de pórticos. 

La comitiva continúa adentrándose por las sucesivas terrazas (porque — no sé si lo he dicho alguna vez — el recinto del Templo no es una superficie plana, sino que sube escalonadamente en niveles cada vez más altos; a cada uno de ellos se accede mediante escalinatas, y en todos hay patios y pórticos y portones labradísimos, de mármol, bronce y oro). 

Antes de llegar al lugar establecido, se paran para desenvolver las cosas que traen, o sea, tortas — me parece — muy untadas, anchas y finas, harina blanca, dos palomas en una jaulita de mimbre y unas monedas grandes de plata, tan pesadas que era una suerte que en aquella época no hubiera bolsillos, porque los habrían roto. 

Ahí está la bonita Puerta de Nicanor; es por entero un bordado en pesado bronce laminado de plata. Ya está allí Zacarías, al lado de un sacerdote que está todo pomposo con su vestido de lino. 

Asperjan a Ana con agua lustral — supongo — y luego le indican que se dirija hacia el ara del sacrificio. Ya no lleva a la Niña en brazos. La ha tomado en brazos Isabel, que se ha quedado a este lado de la Puerta. 

Joaquín, sin embargo, entra siguiendo a su mujer, y llevando tras sí un desgraciado cordero que va balando. Y yo... hago como para la purificación de María: cierro los ojos para no ver ningún tipo de degüello. 

Ana ya está purificada. 

Zacarías dice en voz baja unas palabras a su compañero de ministerio, el cual, sonriendo, da señales de asentimiento y luego se acerca al grupo, rehecho de nuevo, y, congratulándose con la madre y el padre por su gozo y por su fidelidad a las promesas, recibe el segundo cordero, la harina y las tortas. 

Entonces ¿esta hija está consagrada al Señor? Que su bendición os acompañe a Ella y a vosotros. Mirad, ahí viene Ana. Va a ser una de sus maestras. Ana de Fanuel, de la tribu de Aser. Ven, mujer. Esta pequeñuela ha sido ofrecida al Templo como hostia de alabanza. Tú serás para ella maestra. A tu amparo crecerá santa. 

Ana de Fanuel, ya completamente encanecida, hace mimos a la Niña, que ya se ha despertado y que observa toda esa blancura con esos inocentes y atónitos ojos suyos, y todo ese oro que el sol enciende. 

La ceremonia debe haber terminado. No he visto ningún rito especial para el ofrecimiento de María. Quizás era suficiente con decírselo al sacerdote, y sobre todo a Dios, en el lugar santo. 

Querría dar mi ofrenda al Templo e ir al lugar en que el año pasado vi la luz - dice Ana. 

Ana de Fanuel va con ellos. No entran en el Templo propiamente dicho. Es natural que, siendo mujeres y tratándose de una niña, no vayan ni siquiera a donde fue María para ofrecer a su Hijo. Pero, eso sí, desde muy cerquita de la puerta, que está abierta de par en par, miran hacia el semioscuro interior del que vienen dulces cantos de niñas y en el que brillan ricas lámparas, que expanden luz de oro sobre dos cuadros de flores de cabecitas veladas de blanco, dos verdaderos cuadros de azucenas. 

Dentro de tres años estarás ahí, Azucena mía - le promete Ana a María, que mira como embelesada hacia el interior y sonríe al oír el lento canto. 

Parece como si entendiera - dice Ana de Fanuel - ¡Es una niña muy bonita! La querré como si fuera fruto de mis entrañas. Te lo prometo, madre. Si la edad me lo concede. 

Te lo concederá, mujer - dice Zacarías - La recibirás entre las niñas consagradas. Yo también estaré presente. Quiero estar ese día para decirle que pida por nosotros desde el primer momento... - y mira a su mujer, la cual, habiendo comprendido, suspira. 

La ceremonia ha concluido. Ana de Fanuel se retira, mientras los otros, hablando entre sí, salen del Templo. 

Oigo a Joaquín que dice: 

- ¡No sólo dos, y los mejores, sino que habría dado todos mis corderos por este gozo y para alabar a Dios! 

No veo nada más. 

Dice Jesús: 
\emph{Salomón pone en boca de la Sabiduría estas palabras: "Quien sea niño venga a mí". Y verdaderamente, desde la roca, desde los muros de su ciudad, la eterna Sabiduría le decía a la eterna Niña: "Ven a mí". Se consumía por tenerla. Pasado un tiempo, el Hijo de la Doncella purísima dirá: "Dejad que los niños vengan a mí, porque el Reino de los Cielos es de ellos, y quien no se haga como ellos no tendrá parte en mi Reino". Las voces se buscan recíprocamente y, mientras la voz proveniente del Cielo grita a la pequeñuela María: "Ven a mí", la voz del Hombre dice: "Venid a mí si sabéis ser niños", y al decirlo piensa en su Madre. Os doy el modelo en mi Madre. Ella es la perfecta Niña con corazón de paloma sencillo y puro, Aquélla a quien ni los años ni el contacto con el mundo enrudecen bárbaramente, corrompiendo su espíritu o haciéndole tortuoso o mentiroso. Porque Ella no lo quiere. Venid a mí mirando a María. Tú, que la ves, dime: ¿su mirada de infante es muy distinta de la que viste al pie de la Cruz; o en el júbilo de Pentecostés; o en la hora en que los párpados cubrieron su ojo de gacela para el último sueño? No. Aquí se trata de la mirada incierta y atónita del infante; luego se tratará de esa mirada atónita y ruborosa de la Virgen de la Anunciación, o beata como la de la Madre de Belén, o adoradora, como la de mi primera, sublime Discípula; luego será la mirada lastimera de la Torturada del Gólgota, o radiante, como en la Resurrección y en Pentecostés; luego será esa mirada velada: la del extático sueño de la última visión. Pero, ya se abra para ver por primera vez, ya se cierre, cansado, con la última luz, habiendo visto tanto gozo y tanto horror, este ojo es ese apacible, puro, sosegado trocito de cielo que resplandece siempre igual bajo la frente de María. Ira, mentira, soberbia, lujuria, odio, curiosidad, no lo ensucian jamás con sus fumosas nubes. Es la mirada que mira a Dios con amor, ya llore, ya ría, y que por amor a Dios acaricia y perdona, y todo lo soporta; el amor a su Dios le ha hecho inmune a los asaltos del Mal, que muchas veces se sirve del ojo para penetrar en el corazón; es el ojo puro, tranquilizante, bendecidor que tienen los puros, los santos, los enamorados de Dios. Ya lo dije: "El ojo es luz de tu cuerpo. Si el ojo es puro, todo tu cuerpo estará iluminado; mas si el ojo es túrbido, toda tu persona estará en las tinieblas". Los santos han tenido estos ojos, que son luz para el espíritu y salvación para la carne, porque, como María, durante toda su vida sólo han mirado a Dios; o, más aún, han tenido recuerdo de Dios. Ya te explicaré, pequeña voz, el sentido de estas palabras mías. }
 
\chapter{María niña con Ana y Joaquín.}
\emph{En sus labios ya está la Sabiduría del Hijo.}
 
Sigo viendo todavía a Ana. Desde ayer por la tarde la veo así: sentada donde empieza la pérgola umbrosa; dedicada a un trabajo de costura. Está vestida de un solo color gris arena; es un vestido muy sencillo y suelto, quizás por el mucho calor que parece que hace. 

En el otro extremo de la pérgola se ve a los dalladores segando el heno; heno que no debe ser de mayo. Efectivamente, la uva ya está detrás coloreándose de oro, y un grueso manzano muestra entre sus oscuras hojas sus frutos, que están tomando un color de lúcida cera amarilla y roja; y además el campo de trigo es ya sólo un rastrojal en que ondean ligeras las llamitas de las amapolas y los lirios se elevan, rígidos y serenos, radiados como una estrella, azules como el cielo de oriente. 

De la pérgola umbrosa sale caminando una María pequeñita, que, no obstante, es ya ágil e independiente. Su breve paso es seguro y sus sandalitas blancas no tropiezan en los cantos. Tiene ya esbozado su dulce paso ligeramente ondulante de paloma, y está toda blanca, como una palomita, con su vestidito de lino que le llega a los tobillos, amplio, fruncido en torno al cuello con un cordoncito de color celeste, y con unas manguitas cortas que dejan ver los antebrazos regordetes. Con su pelito sérico y rubio- miel, no muy rizado pero sí todo él formando suaves ondas que en el extremo terminan en un leve ensortijado, con sus ojos de cielo y su dulce carita tenuemente sonrosada y sonriente, parece un pequeño ángel. El vientecillo que le entra por las anchas mangas y le hincha por detrás el vestidito de lino contribuye también a darle aspecto de un pequeño ángel cuando despliega las alas para el vuelo. 

Lleva en sus manitas amapolas y lirios y otras florecillas que crecen entre los trigos y cuyo nombre desconozco. Se dirige hacia su madre. Cuando está ya cerca, inicia una breve carrera, emitiendo una vocecita festiva, y va, como una tortolita, a detener su vuelo contra las rodillas maternas, abiertas un poco para recibirla. Ana ha depositado al lado el trabajo que estaba haciendo para que Ella no se pinche, y ha extendido los brazos para ceñirla. 

Hasta este punto, ayer por la tarde; hoy por la mañana se ha vuelto a presentar y continúa así: 

- ¡Mamá! ¡Mamá! 

La tortolita blanca está toda en el nido de las rodillas maternas, apoyando sus piececitos sobre la hierba corta, y la carita en el regazo materno. Sólo se ve el oro pálido de su pelito sobre la sutil nuca que Ana se inclina a besar con amor. 

Luego la tortolita levanta su pequeña cabeza y entrega sus florecillas: todas para su mamá. Y de cada flor cuenta una historia creada por Ella. 

Ésta, tan azul y tan grande, es una estrella que ha caído del cielo para traerle a su mamá el beso del Señor... ¡Que bese en el corazón, en el corazón, a esta florecilla celeste, y percibirá que tiene sabor a Dios!.. 

Y esta otra, de color azul más pálido, como los ojos de su papá, lleva escrito en las hojas que el Señor quiere mucho a su papá porque es bueno. 

Y esta tan pequeñita, la única encontrada de ese tipo (una miosota), es la que el Señor ha hecho para decirle a María que la quiere. 

Y estas rojas, ¿sabe su mamá qué son? Son trozos de la vestidura del rey David, empapados de sangre de los enemigos de Israel, y esparcidos por los campos de batalla y de victoria. Proceden de esos limbos de regia vestidura hecha jirones en la lucha por el Señor. 

En cambio ésta, blanca y delicada, que parece hecha con siete copas de seda que miran al cielo, llenas de perfumes, y que ha nacido allí, junto al fontanar, se la ha cogido su papá de entre las espinas, está hecha con la vestidura que llevaba el rey Salomón cuando, el mismo mes en que nació esta Niña descendiente suya, muchos años, ¡oh, cuántos, cuántos antes; muchos años antes, él, con la pompa cándida de sus vestiduras, caminó entre la multitud de Israel ante el Arca y ante el Tabernáculo, y se regocijó por la nube que volvía a circundar su gloria, y cantó el cántico y la oración de su gozo. 

Yo quiero ser siempre como esta flor, y, como el rey sabio, quiero cantar toda la vida cánticos y oraciones ante el Tabernáculo" termina así la boquita de María. 

- ¡Tesoro mío! ¿Cómo sabes estas cosas santas? ¿Quién te las dice? ¿Tu padre? 

No. No sé quién es. Es como si las hubiera sabido siempre. Pero quizás me las dice alguien, alguien a quien no veo. Quizás uno de los ángeles que Dios envía a hablarles a los hombres buenos. Mamá, ¿me sigues contando alguna otra historia?... 

- ¡Oh, hija mía! ¿Cuál quieres saber? 

María se queda pensando; seria y recogida como está, habría que pintarla para eternizar su expresión. En su carita infantil se reflejan las sombras de sus pensamientos. Sonrisas y suspiros, rayos de sol y sombras de nubes pensando en la historia de Israel. Luego elige: 

Otra vez la de Gabriel y Daniel, en que está la promesa del Cristo. 

Y escucha con los ojos cerrados, repitiendo en voz baja las palabras que su madre le dice, como para recordarlas mejor. Cuando Ana termina, pregunta: 

- ¿Cuánto falta todavía para tener con nosotros al Emmanuel? 

Treinta años aproximadamente, querida mía. 

- ¡Cuánto todavía! Y yo estaré en el Templo... Dime, si rezase mucho, mucho, mucho, día y noche, noche y día, y deseara 

ser sólo de Dios, toda la vida, con esta finalidad, ¿el Eterno me concedería la gracia de dar antes el Mesías a su pueblo? 

No lo sé, querida mía. El Profeta dice: "Setenta semanas". Yo creo que la profecía no se equivoca. Pero el Señor es tan bueno — se apresura a añadir Ana, al ver que las pestañas de oro de su niña se perlan de llanto — que creo que si rezas mucho, mucho, mucho, se te mostrará propicio. 

La sonrisa aparece de nuevo en esa carita ligeramente alzada hacia la madre, y un ojalito de sol que pasa entre dos pámpanas hace brillar las lágrimas del ya cesado llanto, cual gotitas de rocío colgando de los tallitos sutilísimos del musgo alpino. - Entonces rezaré y me consagraré virgen para esto. - Pero, ¿sabes lo que quiere decir eso? 

Quiere decir no conocer amor de hombre, sino sólo de Dios. Quiere decir no tener ningún pensamiento que no sea para el Señor. Quiere decir ser siempre niña en la carne y ángel en el corazón. Quiere decir no tener ojos sino para mirar a Dios, oídos para oírle, boca para alabarle, manos para ofrecerse como hostias, pies para seguirle velozmente, corazón y vida para dárselos a El. 

- ¡Bendita tú! Pero entonces no tendrás nunca niños, ¿sabes? ; y a ti te gustan mucho los niños y los corderitos y las tortolitas. Un niño para una mujer es como un corderito blanco y crespo, como una palomita de plumas de seda y boca de coral: se le puede amar, besar; se puede oír que nos llama "mamá". 

No importa. Seré de Dios. En el Templo rezaré. Y quizás un día vea al Emmanuel. La Virgen que debe ser Madre suya, como dice el gran Profeta, ya debe haber nacido y estar en el Templo... Yo seré compañera suya... y sierva suya. ¡Oh, sí! Si pudiera conocer, por luz de Dios, a esa mujer bienaventurada, querría servirla. Luego Ella me traería a su Hijo, me conduciría hacia su Hijo y así le serviría también a Él. ¡Fíjate, mamá!.. ¡¡Servir al Mesías!!.. - María se siente sobrepujada por este pensamiento que la sublima y la deja anonadada al mismo tiempo. Con las manitas cruzadas sobre su pecho y la cabecita un poco inclinada hacia adelante, y encendida de emoción, parece una infantil reproducción de la Virgen de la Anunciación que yo vi. Y sigue diciendo: 

- ¿Pero, el Rey de Israel, el Ungido de Dios, me permitirá servirle? 

No lo dudes. ¿No dice el rey Salomón: "Sesenta son las reinas y ochenta las otras esposas y sin número las doncellas" En ello puedes ver que en el palacio del Rey serán sin número las doncellas vírgenes que servirán a su Señor. 

- ¡Oh! ¿Lo ves como debo ser virgen? Debo serlo. Si Él por madre quiere una virgen, es señal de que estima la virginidad por encima de todas las cosas. Yo quiero que me ame a mí, su sierva, por esa virginidad que me hará un poco similar a su dilecta Madre... Esto es lo que quiero... Querría también ser pecadora, muy pecadora, si no temiera ofender al Señor... Dime, mamá, ¿puede una ser pecadora por amor a Dios? 

Pero, ¿qué dices, tesoro? No entiendo. 

Quiero decir: pecar para poder ser amada por Dios hecho Salvador. Se salva a quien está perdido, ¿no es verdad? Yo querría ser salvada por el Salvador para recibir su mirada de amor. Para esto querría pecar, pero no cometer un pecado que le disgustase. ¿Cómo puede salvarme si no me pierdo? 

Ana está atónita. No sabe ya qué decir. 

Viene en su ayuda Joaquín, el cual, caminando sobre la hierba, se ha ido acercando, sin hacer ruido, por detrás del seto de sarmientos bajos. 

Te ha salvado antes porque sabe que le amas y quieres amarle sólo a Él. Por ello tú ya estás redimida y puedes ser virgen como quieres - dice Joaquín. 

- ¿Sí, padre mío?- María se abraza a sus rodillas y le mira con las claras estrellas de sus ojos, muy semejantes a los paternos, y muy dichosos por esta esperanza que su padre le da. 

Verdaderamente, pequeño amor. Mira, yo te traía este pequeño gorrión que en su primer vuelo había ido a posarse junto a la fuente. Habría podido dejarlo, pero sus débiles alas no tenían fuerza para elevarlo en nuevo vuelo, ni sus patitas de seda para fijarlo a las musgosas piedras, que resbalaban. Se habría caído en la fuente. No he esperado a que esto sucediera. Lo he cogido y ahora te lo regalo. Haz lo que quieras con él. El hecho es que ha sido salvado antes de caer en el peligro. Lo mismo ha hecho Dios contigo. Ahora, dime, María: ¿he amado más al gorrión salvándolo antes, o lo habría amado más salvándolo después? 

Ahora lo has amado, porque no has permitido que se hiciera daño con el agua helada. 

Y Dios te ha amado más, porque te ha salvado antes de que tú pecaras. 

Pues entonces yo le amaré completamente, completamente. Gorrioncito bonito, yo soy como tú. El Señor nos ha amado de la misma manera, salvándonos... Ahora voy a criarte y luego te dejaré suelto. Tú cantarás en el bosque y yo en el Templo las alabanzas del Señor, y diremos: "Envía a tu Prometido, envíaselo a quien espera". ¡Oh, papá mío! ¿Cuándo me vas a llevar al Templo? 

Pronto, perla mía. Pero, ¿no te duele dejar a tu padre? 

- ¡Mucho! Pero tú vendrás... y, además, si no doliese, ¿qué sacrificio sería? 

 - ¿Y te vas a acordar de nosotros? - Siempre. Después de la oración por el Emmanuel rezaré por vosotros. Para que Dios os haga dichosos y os dé una larga vida... hasta el día en que Él sea Salvador. Luego diré que os tome para llevaros a la Jerusalén del Cielo. 

La visión me cesa con María estrechada en el lazo de los brazos de su padre... 

Dice Jesús: 
\emph{Llegan ya a mis oídos los comentarios de los doctores de los tiquismiquis: "¿Cómo puede hablar así una niña que no ha cumplido aún tres años? Es una exageración". Pero no piensan que ellos, alterando mi infancia con actos propios de adultos, dan de mí una imagen monstruosa. La inteligencia no llega a todos de la misma manera y al mismo tiempo. La Iglesia ha establecido los seis años como la edad de responsabilidad de las acciones, porque esa es la edad en que incluso un niño retrasado puede distinguir, al menos rudimentariamente, el bien y el mal. Pero hay niños que mucho antes son capaces de discernir, entender y querer, con una razón ya suficientemente desarrollada. Que las pequeñas Imelda Lambertini, Rosa de Viterbo, Nellie Organ, Nennolina os proporcionen una base para creer, ¡oh, doctores difíciles!, que mi Madre podía pensar y hablar así. Sólo he considerado cuatro nombres al azar entre los millares de niños santos que, después de haber razonado como adultos en la tierra durante más o menos años, han venido a poblar mí Paraíso. ¿Qué es la razón? Un don de Dios. Él, por tanto, puede darla con la medida que quiera, a quien quiera y cuando quiera. Es, además, una de las cosas que más os asemejan a Dios, Espíritu inteligente y que razona. La razón y la inteligencia fueron gracias otorgadas por Dios al Hombre en el Paraíso Terrenal. ¡Y qué vivas estaban cuando la Gracia moraba, aún intacta y operante, en el espíritu de los dos Primeros! En el libro de Jesús Bar Sirac está escrito: "Toda sabiduría viene del Señor Dios y con Él ha estado siempre, incluso antes de los siglos". ¿Qué sabiduría, pues, habrían tenido los hombres si hubieran conservado su filiación para con Dios? Vuestras lagunas de inteligencia son el fruto natural de haber venido a menos en la Gracia y en la honestidad. Perdiendo la Gracia, habéis alejado de vosotros, durante siglos, la Sabiduría. Cual estrella fugaz que se oculta tras nebulosidades de kilómetros, la Sabiduría no ha seguido llegándoos con sus netos destellos, sino sólo a través de neblinas cada vez más oprimentes a causa de vuestras prevaricaciones. Luego ha venido el Cristo y os ha vuelto a dar la Gracia, don supremo del amor de Dios. Pero ¿sabéis custodiar limpia y pura esta gema? No. Cuando no la rompéis con la voluntad individual de pecar, la ensuciáis con continuas culpas menores, con debilidades, o gravitando hacia el vicio (y ello, a pesar de no significar una verdadera unión con el septiforme vicio, debilita la luz de la Gracia y su actividad). Luego, además, siglos y siglos de corrupciones, que, deletéreas, repercuten en lo físico y en la mente, han ido debilitando la magnífica luz de la inteligencia que Dios había dado a los Primeros. Pero María era no sólo la Pura, la nueva Eva recreada para alegría de Dios, era la super- Eva, era la Obra Maestra del Altísimo, era la Llena de Gracia, era la Madre del Verbo en la mente de Dios. "Fuente de la Sabiduría" dice Jesús Bar Sirac "es el Verbo". ¿Y el Hijo no va a haber puesto su sabiduría en los labios de su Madre? Si a un Profeta que debía decir las palabras que el Verbo, la Sabiduría, le confiaba para transmitírselas a los hombres, le fue purificada la boca con carbones encendidos, ¿no va a haber depurado y elevado el Amor el habla de esa su Esposa niña que debía llevar en sí la Palabra, a fin de que no hablase primero como niña y luego como mujer, sino sólo y siempre como criatura celeste fundida con la gran luz y sabiduría de Dios? El milagro no está en el hecho de que María, como luego Yo, mostrara en edad infantil una inteligencia superior. El milagro está en el hecho de contener a la Inteligencia infinita, que en Ella moraba, en los diques convenientes para no pasmar a las multitudes y para no despertar la atención satánica. En otra ocasión seguiré hablando de esto, que está en relación con ese "recordarse" que los santos tienen de Dios. }
 
\chapter{María recibida en el Templo.}
\emph{En su humildad, no sabía que era la Llena de Sabiduría.}
 
Veo a María caminando entre su padre y su madre por las calles de Jerusalén. 

Los que pasan se paran a mirar a la bonita Niña vestida toda de blanco nieve y arrollada en un ligerísimo tejido que, por sus dibujos, de ramas y flores, más opacos que el tenue fondo, creo que es el mismo que tenía Ana el día de su Purificación. Lo único es que, mientras que a Ana no le sobrepasaba la cintura, a María, siendo pequeñita, le baja casi hasta el suelo, envolviéndola en una nubecita ligera y lúcida de singular gracia. 

El oro de la melena suelta sobre los hombros, mejor: sobre la delicada nuca, se transparenta a través del sutilísimo fondo, en las partes del velo no adamascadas. Éste está sujeto a la frente con una cinta de un azul palidísimo que tiene, obviamente hecho por su mamá, unas pequeñas azucenas bordadas en plata. 

El vestido, como he dicho, blanquísimo, le llega hasta abajo, y los piececitos, con sus pequeñas sandalias blancas, apenas se muestran al caminar. Las manitas parecen dos pétalos de magnolia saliendo de la larga manga. Aparte del círculo azul de la cinta, no hay ningún otro punto de color. Todo es blanco. María parece vestida de nieve. 

Joaquín lleva el mismo vestido de la Purificación. Ana, en cambio, un oscurísimo morado; el manto, que le tapa incluso la cabeza, es también morado oscuro; lo lleva muy bajo, a la altura de los ojos, dos pobres ojos de madre rojos de llanto, que no quisieran llorar, y que no quisieran, sobre todo, ser vistos llorar, pero que no pueden no llorar al amparo del manto. Éste protege, por una parte, de los que pasan; también, de Joaquín, cuyos ojos, siempre serenos, hoy están también enrojecidos y opacos por las lágrimas (las que ya han caído y las que aún siguen cayendo). Camina muy curvado, bajo su velo a guisa casi de turbante que le cubre los lados del rostro. 

Joaquín está muy envejecido. Los que le ven deben pensar que es abuelo o quizás bisabuelo de la pequeñuela que lleva de la mano. El pobre padre, a causa de la pena de perderla, va arrastrando los pies al caminar; todo su porte es cansino y le hace unos veinte años más viejo de lo que en realidad es; su rostro parece el de una persona enferma además de vieja, por el mucho cansancio y la mucha tristeza; la boca le tiembla ligeramente entre las dos arrugas — tan marcadas hoy — de los lados de la nariz. 

Los dos tratan de celar el llanto. Pero, si pueden hacerlo para muchos, no pueden para María, la cual, por su corta estatura, los ve de abajo arriba y, levantando su cabecita, mira alternativamente a su padre y a su madre. Ellos se esfuerzan en sonreírle con su temblorosa boca, y aprietan más con su mano la diminuta manita cada vez que su hijita los mira y les sonríe. Deben pensar: "Sí. Otra vez menos que veremos esta sonrisa". 

Van despacio, muy despacio. Da la impresión de que quieren prolongar lo más posible su camino. Todo es ocasión para detenerse... Pero, ¡siempre debe tener un fin un camino!.. Y éste está ya para acabarse. En efecto, allí, en la parte alta de este último tramo en subida, están los muros que circundan el Templo. Ana gime, y estrecha más fuertemente la manita de María. - ¡Ana, querida mía, aquí estoy contigo! - dice una voz desde la sombra de un bajo arco echado sobre un cruce de calles. Isabel estaba esperando. Ahora se acerca a Ana y la estrecha contra su corazón, y, al ver que Ana llora, le dice: - Ven, ven un poco a esta casa amiga; también está Zacarías. 

Entran todos en una habitación baja y oscura cuya luz es un vasto fuego. La dueña, que sin duda es amiga de Isabel, si bien no conoce a Ana, amablemente se retira, dejando a los llegados libertad de hablar. 

No creas que estoy arrepentida, o que entregue con mala voluntad mi tesoro al Señor — explica Ana entre lágrimas — ... Lo que pasa es que el corazón... ¡oh, cómo me duele el corazón, este anciano corazón mío que vuelve a su soledad, a esa soledad de quien no tiene hijos!.. Si lo sintieras... 

Lo comprendo, Ana mía... Pero tú eres buena y Dios te confortará en tu soledad. María va a rezar por la paz de su mamá, ¿verdad? 

María acaricia las manos maternas y las besa, se las pone en la cara para ser acariciada a su vez, y Ana cierra entre sus manos esa carita y la besa, la besa... no se sacia de besarla. 

Entra Zacarías y saluda diciendo: 

A los justos la paz del Señor. 

Sí — dice Joaquín —, pide paz para nosotros porque nuestras entrañas tiemblan, ante la ofrenda, como las de nuestro padre Abraham mientras subía el monte; y nosotros no encontraremos otra ofrenda que pueda recobrar ésta; ni querríamos hacerlo, porque somos fieles a Dios. Pero sufrimos, Zacarías. Compréndenos, sacerdote de Dios, y no te seamos motivo de escándalo. 

Jamás. Es más, vuestro dolor, que sabe no traspasar lo lícito, que os llevaría a la infidelidad, es para mí escuela de amor al Altísimo. ¡Ánimo! La profetisa Ana cuidará con esmero esta flor de David y Aarón. En este momento es la única azucena que David tiene de su estirpe santa en el Templo, y cual perla regia será cuidada. A pesar de que los tiempos hayan entrado ya en la recta final y de que deberían preocuparse las madres de esta estirpe de consagrar sus hijas al Templo — puesto que de una virgen de David vendrá el Mesías — no obstante, a causa de la relajación de la fe, los lugares de las vírgenes están vacíos. Demasiado pocas en el Templo; y de esta estirpe regia ninguna, después de que, hace ya tres años, Sara de Elíseo salió desposada. Es cierto que aún faltan seis lustros para el final, pero bueno, pues esperemos que María sea la primera de muchas vírgenes de David ante el Sagrado Velo. Y... ¿quién sabe?... — Zacarías se detiene en estas palabras y... mira pensativo a María. Luego prosigue diciendo: - También yo velaré por Ella. Soy sacerdote y ahí dentro tengo mi influencia. Haré uso de ella para este ángel. Además, Isabel vendrá a menudo a verla... 

- ¡Oh, claro! Tengo mucha necesidad de Dios y vendré a decírselo a esta Niña para que a su vez se lo diga al Eterno. 

Ana ya está más animada. Isabel, buscando confortarla aún más, pregunta: 

- ¿No es éste tu velo de cuando te casaste?, ¿o has hilado más muselina? 

Es aquél. Lo consagro con Ella al Señor. Ya no tengo ojos para hilar... Además, por impuestos y adversidades, las posibilidades económicas son mucho menores... No me era lícito hacer gastos onerosos. Sólo me he preocupado de que tuviera un ajuar considerable para el tiempo que transcurra en la Casa de Dios y para después... porque creo que no seré yo quien la vista para la boda... Pero quiero que sea la mano de su madre, aunque esté ya fría e inmóvil, la que la haya ornado para la boda y le haya hilado la ropa y el vestido de novia. 

- ¡Oh, por qué tienes que pensar así? 

Soy vieja, prima. Jamás me he sentido tan vieja como ahora bajo el peso de este dolor. Las últimas fuerzas de mi vida se las he dado a esta flor, para llevarla y nutrirla, y ahora... y ahora... el dolor de perderla sopla sobre las postreras y las dispersa. - No digas eso. Queda Joaquín. 

Tienes razón. Trataré de vivir para mi marido. 

Joaquín ha hecho como que no ha oído, atento como está a lo que le dice Zacarías; pero sí que ha oído, y suspira fuertemente, y sus ojos brillan de llanto. 

Estamos entre tercia y sexta. Creo que sería conveniente ponernos en marcha" dice Zacarías. 

Todos se levantan para ponerse los mantos y comenzar a salir. 

Pero María se adelanta y se arrodilla en el umbral de la puerta con los brazos extendidos, un pequeño querubín suplicante: 

- ¡Padre, Madre, vuestra bendición! 

No llora la fuerte pequeña; pero los labiecitos sí tiemblan, y la voz, rota por un interno sollozo, presenta más que nunca el tembloroso gemido de una tortolita. La carita está más pálida y el ojo tiene esa mirada de resignada angustia que — más fuerte, hasta el punto de llegar a no poderse mirar sin que produzca un profundo sufrimiento — veré en el Calvario y ante el Sepulcro. 

Sus padres la bendicen y la besan. Una, dos, diez veces. No se sacian de besarla... Isabel llora en silencio. Zacarías, aunque quiera no dar muestras de ello, está también conmovido. 

Salen. María entre su padre y su madre, como antes; delante, Zacarías y su mujer... 

Ahora están dentro del recinto del Templo. 

- Voy a ver al Sumo Sacerdote. Vosotros subid hasta la Gran Terraza. 

Atraviesan tres atrios y tres patios superpuestos... Ya están al pie del vasto cubo de mármol coronado de oro. Cada una de las cúpulas, convexas como una media naranja enorme, resplandece bajo el sol, que cae a plomo, ahora que es aproximadamente mediodía, en el amplio patio que rodea a la solemne edificación, y llena el vasto espacio abierto y la amplia escalinata que conduce al Templo. Sólo el pórtico que hay frente a la escalinata, a lo largo de la fachada, está en sombra, y la puerta, altísima, de bronce y oro, con tanta luz, aparece aún más oscura y solemne. 

Por el intenso sol, María parece aún más de nieve. Ahí está, al pie de la escalinata, entre sus padres. ¡Cómo debe latirles el corazón a los tres! Isabel está al lado de Ana, pero un poco retrasada, como medio paso. 

Un sonido de trompetas argentinas y la puerta gira sobre los goznes, los cuales, al moverse sobre las esferas de bronce, parecen producir sonido de cítara. Se ve el interior, con sus lámparas en el fondo. Un cortejo viene desde allí hacia el exterior. Es un pomposo cortejo acompañado de sonidos de trompetas argénteas, nubes de incienso y luces. 

Ya ha llegado al umbral; delante, el que debe ser el Sumo Sacerdote: un anciano solemne, vestido de lino finísimo, cubierto con una túnica más corta, también de lino, y sobre ésta una especie de casulla, recuerda en parte a la casulla y en parte al paramento de los Diáconos, multicolor: púrpura y oro, violáceo y blanco se alternan en ella y brillan como gemas al sol; y dos piedras preciosas resplandecen encima de los hombros más vivamente aún (quizás son hebillas con un engaste precioso); al pecho lleva una ancha placa resplandeciente de gemas sujeta con una cadena de oro; y colgantes y adornos lucen en la parte de abajo de la túnica corta, y oro en la frente sobre la prenda que cubre su cabeza (una prenda que me recuerda a la de los sacerdotes ortodoxos, con su mitra en forma de cúpula en vez de en punta como la mitra católica). 

El solemne personaje avanza, solo, hasta el comienzo de la escalinata, bajo el oro del sol, que le hace todavía más espléndido. Los otros esperan, abiertos en forma de corona, fuera de la puerta, bajo el pórtico umbroso. A la izquierda hay un cándido grupo de niñas, con Ana, la profetisa, y otras maestras ancianas. 

El Sumo Sacerdote mira a la Pequeña y sonríe. ¡Debe parecerle bien pequeñita al pie de esa escalinata digna de un templo egipcio! Levanta los brazos al cielo para pronunciar una oración. Todos bajan la cabeza como anonadados ante la majestad sacerdotal en comunión con la Majestad eterna. 

Luego... una señal a María, y Ella se separa de su madre y de su padre y sube, sube como hechizada. Y sonríe, sonríe a la zona del Templo que está en penumbra, al lugar en que pende el preciado Velo... Ha llegado a lo alto de la escalinata, a los pies del Sumo Sacerdote, que le impone las manos sobre la cabeza. La víctima ha sido aceptada. ¿Alguna vez había tenido el Templo una hostia más pura? 

Luego se vuelve y, pasando la mano por el hombro de la Corderita sin mancha, como para conducirla al altar, la lleva a la puerta del Templo y, antes de hacerla pasar pregunta: 

María de David, ¿conoces tu voto? 

Ante el "sí" argentino que le responde, él grita: 

Entra, entonces. Camina en mi presencia y sé perfecta. 

Y María entra y desaparece en la sombra, y el cortejo de las vírgenes y de las maestras, y luego de los levitas, la ocultan cada vez más, la separan... Ya no se la ve... 

La puerta se vuelve, girando sobre sus armoniosos goznes. Una abertura, cada vez más estrecha, permite todavía ver al cortejo, que se va adentrando hacia el Santo. Ahora es sólo una rendija. Ahora ya nada. Cerrada. 

Al último acorde de los sonoros goznes responde un sollozo de los dos ancianos y un grito único: " ¡María! ¡Hija!". Luego dos gemidos invocándose mutuamente: " ¡Ana, Joaquín!". Luego, como final: "Glorifiquemos al Señor, que la recibe en su Casa y la conduce por sus caminos". 

Y todo termina así. 

Dice Jesús: 
\emph{El Sumo Sacerdote había dicho: "Camina en mi presencia y sé perfecta". El Sumo Sacerdote no sabía que estaba hablándole a la Mujer que, en perfección, es sólo inferior a Dios. Mas hablaba en nombre de Dios y, por tanto, su imperativo era sagrado. Siempre sagrado, pero especialmente a la Repleta de Sabiduría. María había merecido que la "Sabiduría viniera a su encuentro tomando la iniciativa de manifestarse a Ella", porque "desde el principio de su día Ella había velado a su puerta y, deseando instruirse, por amor, quiso ser pura para conseguir el amor perfecto y merecer tenerla como maestra". En su humildad, no sabía que la poseía antes de nacer y que la unión con la Sabiduría no era sino un continuar los divinos latidos del Paraíso. No podía imaginar esto. Y cuando, en el silencio del corazón, Dios le decía palabras sublimes, Ella, humildemente, pensaba que fueran pensamientos de orgullo, y elevando a Dios un corazón inocente suplicaba: "¡Piedad de tu sierva, Señor!". En verdad, la verdadera Sabia, la eterna Virgen, tuvo un solo pensamiento desde el alba de su día: "Dirigir a Dios su corazón desde los albores de la vida y velar para el Señor, orando ante el Altísimo", pidiendo perdón por la debilidad de su corazón, como su humildad le sugería creer, sin saber que estaba anticipando la solicitud de perdón para los pecadores que haría al pie de la Cruz junto con su Hijo moribundo. "Luego, cuando el gran Señor lo quiera, Ella será colmada del Espíritu de inteligencia", y entonces comprenderá su sublime misión. Por ahora no es más que una párvula que, en la paz sagrada del Templo, anuda, "reanuda", cada vez de forma más estrecha, sus coloquios, sus afectos, sus recuerdos, con Dios. Esto es para todos. Pero, para ti, pequeña María (se dirige aquí a María Valtorta), ¿no tiene ninguna cosa particular que decir tu Maestro? "Camina en mi presencia, sé por tanto perfecta". Modifico ligeramente la sagrada frase y te la doy por orden. Perfecta en el amor, perfecta en la generosidad, perfecta en el sufrir. Mira una vez más a la Madre. Y medita en eso que tantos ignoran, o quieren ignorar, porque el dolor es materia demasiado ingrata para su paladar y para su espíritu. El dolor. María lo tuvo desde las primeras horas de la vida. Ser perfecta como Ella era poseer también una perfecta sensibilidad. Por eso, el sacrificio debía serle más agudo; mas, por eso mismo, más meritorio. Quien posee pureza posee amor, quien posee amor posee sabiduría, quien posee sabiduría posee generosidad y heroísmo, porque sabe el porqué de por qué se sacrifica. ¡Arriba tu espíritu, aunque la cruz te doble, te rompa, te mate! Dios está contigo". }
 
\chapter{La muerte de Joaquín y Ana fue dulce,}
\emph{después de una vida de sabia fidelidad a Dios en las pruebas.}
 
Dice Jesús: 
\emph{Como un rápido crepúsculo de invierno en que un viento de nieve acumule nubes en el cielo, la vida de mis abuelos conoció rápida la noche, una vez que su Sol se había quedado fijo resplandeciendo ante la sagrada Cortina del Templo. Pero, ¿acaso no fue dicho: "La Sabiduría inspira vida a sus hijos, toma bajo su protección a los que la buscan... Quien la ama ama la vida, y quien está en vela por ella gozará de su paz. Quien la posee heredará la vida... Quien la sirve rendirá obediencia al Santo, y a quien la ama Dios lo ama mucho... Si cree en ella la tendrá como herencia y le será como tal confirmada a su posteridad porque lo acompaña en la prueba. En primer lugar le elige, luego enviará sobre él temores, miedos y pruebas, le atormentará con el flagelo de su disciplina, hasta haberle probado en sus pensamientos y poder fiarse de él. Mas luego le dará estabilidad, volverá a él por recto camino y le alegrará. Le descubrirá sus arcanos, pondrá en él tesoros de ciencia y de inteligencia en la justicia"? Sí, todo esto fue dicho. Los libros sapienciales son aplicables a todos los hombres, que en ellos tienen un espejo de sus comportamientos y una guía. Mas dichosos aquellos que puedan ser reconocidos como amantes espirituales de la Sabiduría. Yo me circundé de una parentela mortal de sabios. Ana, Joaquín, José, Zacarías y, más aún, Isabel y luego el Bautista, ¿no son, acaso, verdaderos sabios? Y eso sin hablar de mi Madre, en la cual la Sabiduría había hecho morada. Desde la juventud hasta la tumba, la Sabiduría había inspirado a mis abuelos la manera de vivir de forma grata a Dios; y, como un toldo que protege de la violencia de los elementos, los había protegido del peligro de pecar. El santo temor de Dios es base del árbol de la sabiduría, que, a partir de aquél, se desarrolla impetuoso con todas sus ramas para alcanzar con su copa el amor tranquilo en su paz, el amor pacífico en su seguridad, el amor seguro en su fidelidad, el amor fiel en su intensidad, el amor total, generoso, activo de los santos. "Quien la ama ama la vida y recibirá en herencia la Vida" dice el Eclesiástico. Pues bien, esto se funde con mi: 'Aquel que pierda la vida por amor mío, la salvará". Porque no se habla de la pobre vida de esta tierra, sino de la eterna; no de las alegrías de una hora, sino de las inmortales. Joaquín y Ana la amaron en ese sentido. Y ella estuvo con ellos en las pruebas.¡Cuántas, vosotros, que, pensando que no sois completamente malvados, querríais no tener que llorar ni sufrir nunca! ¡Cuántas pruebas sufrieron estos dos justos que merecieron tener por hija a María! La persecución política que los arrojó de la tierra de David, empobreciéndolos excesivamente. La tristeza de ver caer en la nada los años sin que una flor les dijese: "Yo os continuaré". Y luego la congoja por haberla tenido a una edad en que ciertamente no la iban a ver hacerse mujer. Y, más tarde, el tener que arrancarse de su corazón esta flor para depositarla sobre el altar de Dios. Y el vivir en un silencio más oprimente aún que el primero, ahora que se habían acostumbrado al gorjeo de su tortolita, al rumor de sus pasitos, a las sonrisas, a los besos de su criatura; y esperar en el recuerdo la hora de Dios. Y más, y más todavía: enfermedades, calamidades por la intemperie, abusos de los poderosos... muchos golpes de ariete contra el débil castillo de su modesta prosperidad. Y no acaba aquí todo: el dolor de esa criatura lejana, que se quedaba sola y pobre, y que, a pesar de todas las atenciones y todos los sacrificios, no tendría sino un resto del bien paterno. ¿Y cómo podía encontrarlo, si durante años todavía quedaría yermo, cerrado, esperándola? Temores, miedos, pruebas y tentaciones. Y fidelidad, fidelidad, fidelidad, siempre, a Dios. La tentación más fuerte: no negarse el consuelo de su hija en torno a su vida ya declinante. Pero, los hijos son de Dios antes que de los padres. Todos los hijos pueden decir lo que Yo le dije a mi Madre: "¿No sabes que debo ocuparme de los intereses del Padre de los Cielos?". Y todas las madres y todos los padres deben aprender la actitud a guardar en estos casos, mirando a María y a José en el Templo, a Ana y a Joaquín en la casa de Nazaret, cada vez más vacía y triste, aunque, no obstante, en ella una cosa no disminuyese nunca, sino que, al contrario, crecía cada vez más: la santidad de dos corazones, la santidad de una unión matrimonial. ¿Qué luz le queda a Joaquín, enfermo; qué luz le queda a su dolorida esposa en las largas y silenciosas tardes propias de ancianos que se sienten morir? Los vestiditos, las primeras sandalitas, los pobres juguetitos de su criatura lejana, y los recuerdos, los recuerdos, los recuerdos. Y, con éstos, una paz que proviene del poder decir: "Sufro, pero he cumplido mi deber de amor hacia Dios". Pues bien, he aquí que se produce una alegría sobrehumana de celestial brillo, no conocida por los hijos de este mundo, y que no se opaca por el hecho de que un grave párpado descienda sobre dos ojos que mueren, sino que en la postrera hora resplandece más, e ilumina verdades que habían estado dentro durante toda la vida, cerradas como mariposas en su capullo, que daban señales de estar dentro de ellos sólo por unos suaves movimientos de ligeros destellos, mientras que ahora abren sus alas de sol mostrando las palabras que las decoran. Y la vida se apaga en el conocimiento de un futuro beato para ellos y para su estirpe, bendiciendo a su Dios. Así fue la muerte de mis abuelos, como era justo que fuera por su vida santa. Por la santidad merecieron ser los primeros depositarios de la Amada de Dios, y, sólo cuando un Sol mayor se mostró en su vital ocaso, ellos intuyeron la gracia que Dios les había concedido. Por la santidad que tuvieron, Ana no padeció la tortura propia de la puérpera, sino que experimentó el éxtasis de quien llevó a la Sin Culpa. No sufrieron la angustia de la agonía, sino que fueron languidez que se apaga, como dulcemente se apaga una estrella cuando el Sol sale con la aurora. Y, si bien no experimentaron el consuelo de tenerme como Encamada Sabiduría, como me tuvo José, Yo, no obstante, estaba allí, invisible Presencia que decía sublimes palabras, inclinado hacia su almohada para adormecerlos en la paz en espera del triunfo. Hay quien dice: "¿Por qué no debieron sufrir al generar y al morir, puesto que eran hijos de Adán?". A éste le respondo: "Si el Bautista, hijo de Adán y concebido con la culpa de origen, fue presantificado en el seno de su madre porque Yo le visité, ¿ninguna gracia va a haber recibido la madre santa de la Santa sin Mancha, de la Preservada por Dios que llevó consigo a Dios en su espíritu casi divino y en el corazón embrional, y que no se separó nunca de Él desde que fue pensada por el Padre, desde que fue concebida en un seno, hasta que retornó a poseer a Dios plenamente en el Cielo para una eternidad gloriosa?". A éste le respondo: "La recta conciencia proporciona una muerte serena y las oraciones de los santos os obtienen tal muerte". Joaquín y Ana tenían toda una vida de recta conciencia a sus espaldas, y ésta se alzaba como sosegado panorama y los guió hasta el Cielo; y tenían a la Santa en oración por ellos, sus padres lejanos, ante el Tabernáculo de Dios. Dios, Bien supremo, era antes que ellos, pero Ella amaba a sus padres, como querían la ley y el sentimiento, con un amor sobrenaturalmente perfecto. }

\chapter{Cántico de María.}
\emph{Ella recordaba cuanto su espíritu había visto en Dios.}
 
Hasta ayer por la tarde, viernes, no se me ha iluminado la mente para ver. Y he visto solamente esto. He visto a una María muy joven, una María de como mucho doce años, cuyo rostro no presenta ya esas redondeces propias de la infancia, sino que devela los futuros contornos de la mujer en el perfil oval que ya se va alargando. Por lo que respecta al pelo, ya no es aquel que caía suelto sobre el cuello con sus ligeros rizos, sino que está recogido en dos gruesas trenzas de un oro palidísimo, de lo claro que es el pelo, parece como si estuviera mezclado con plata, que siguiendo los hombros bajan hasta las caderas. El rostro aparece más pensativo, más maduro, aunque siga siendo el rostro de una niña, de una hermosa y pura niña que, toda vestida de blanco, cose en una habitacioncita muy pequeña y también toda blanca, por cuya ventana abierta de par en par se ve el edificio imponente y central del Templo, y toda la bajada de las escalinatas de los patios, de los pórticos, y, al otro lado de la muralla, la ciudad con sus calles y casas y jardines, y, al fondo, la cima protuberante y verde del Monte de los Olivos. 

Cose y canta en voz baja. No sé si se trata de un canto sacro. Dice:

\begin{verse}
Como una estrella dentro de un agua clara\\
me resplandece una luz en el fondo del corazón.\\
Desde la infancia, de mí no se separa\\
y dulcemente me guía con amor.

En lo más hondo del corazón hay un canto. \\
¿De dónde venir podrá? \\
¡Oh, hombre, tú lo ignoras! \\
De donde descansa el Santo.

Yo miro mi estrella clara\\
y no quiero cosa que no sea,\\
aunque fuera la más dulce y estimada,\\
esta dulce luz que es toda mía. 

Me trajiste de los altos Cielos, \\
Estrella, al interior de un seno de madre.\\
Ahora vives en mí; mas allende los velos\\
te veo, rostro glorioso del Padre.

¿Cuándo a tu sierva darás el honor\\
de ser humilde esclava del Salvador? \\
Manda, del Cielo mándanos al Mesías.\\ 
Acepta, Padre Santo, la ofrenda de María.
\end{verse} 

María calla, sonríe y suspira, y luego se pone de rodillas en oración. Su carita es toda una luz. Alta, elevada hacia el azul terso de un bonito cielo estival, parece como si aspirase toda su luminosidad y la irradiara. O, más exactamente, parece como si de su interior un escondido Sol irradiase sus luces y encendiera la nieve apenas rosada de la carne de María y se vertiera, llegando a las cosas y al Sol que resplandece sobre la tierra, bendiciendo y prometiendo abundancia de bienes. 

Estando María a punto de ponerse en pie después de su amorosa oración, permaneciendo en su rostro una luminosidad de éxtasis, entra la anciana Ana de Fanuel y se detiene atónita, o, por lo menos, admirada del acto y del aspecto de María. 

La llama: "María", y la Niña se vuelve con una sonrisa, distinta pero como siempre muy bonita, y saluda diciendo: "Ana, 

paz a ti". 

- ¿Estabas orando? ¿No te es suficiente nunca la oración? 

La oración me sería suficiente. Pero yo hablo con Dios. Ana, tú no puedes saber qué cercano a mí lo siento; más que cercano, en el corazón. Dios me perdone tal soberbia. Es que yo no me siento sola. ¿Ves? Allí, en aquella casa de oro y de nieve, detrás de la doble Cortina, está el Santo de los Santos, y jamás ojo alguno, aparte del del Sumo Sacerdote, puede detenerse en el Propiciatorio, sobre el que descansa la gloria del Señor. Mas yo no tengo necesidad de mirar con toda el alma veneradora a ese doble Velo bordado, que palpita con las ondas de los cantos virginales y de los levitas y que huele a preciosos inciensos, como para perforar su cohesión y ver así la luz irradiada por el Testimonio. ¡Pero sí que miro! No temas que no mire con ojo venerador como todo hijo de Israel. No temas que el orgullo me ciegue haciéndome pensar esto que ahora te digo. Yo miro, y no hay ningún humilde siervo en el pueblo de Dios que mire más humildemente la Casa de su Señor que como yo la miro, convencida como estoy de ser la más pequeña de todos. Pero, ¿qué es lo que veo? Un velo. ¿Qué pienso al otro lado del Velo? Un Tabernáculo. ¿Y en él? Mas si miro a mi corazón, he aquí que veo a Dios resplandecer en su gloria de amor y decirme: "Te amo", y yo le digo: "Te amo", y me deshago y me rehago con cada uno de los latidos del corazón en este beso recíproco... Estoy entre vosotras, mis queridas maestras y compañeras, pero un círculo de fuego me aísla de vosotras. Dentro de ese círculo, Dios y yo. Y os veo a través del Fuego de Dios y así os amo... mas no puedo amaros según la carne, como jamás podré amar a nadie según la carne, sino sólo a Este que me ama, y según el espíritu. Conozco mi destino. La Ley secular de Israel quiere de toda niña una esposa y de toda esposa una madre. Pero yo, no sin obedecer a la Ley, obedezco a la Voz que me dice: "Yo te quiero para mí", y permaneceré siempre virgen. ¿Cómo podré hacerlo? Esta dulce, invisible Presencia que está conmigo me ayudará, porque ella desea eso. Yo no temo. Ya no tengo ni padre ni madre... y sólo el Eterno sabe cómo en ese dolor se quemó cuanto yo tenía de humano. Ardió con dolor atroz. Ahora sólo tengo a Dios. A Él, por tanto, le presto obediencia ciegamente... Lo habría hecho incluso contra el padre y la madre, porque la Voz me enseña que quien quiere seguirla debe pasar por encima del padre y de la madre, amorosas patrullas de ronda en torno a los muros del corazón filial, al que quieren conducir a la alegría según sus caminos... y no saben que hay otros caminos de infinita alegría. Yo les habría dejado los vestidos y el manto, con tal de seguir la Voz que me dice: "¡Ven, dilecta mía, esposa mía!". Les habría dejado todo; y las perlas de las lágrimas — porque habría llorado por tener que desobedecer —, y los rubíes de mi sangre — que hasta a la muerte habría desafiado por seguir la Voz que llama — les habrían dicho que hay algo más grande que el amor de un padre y una madre, y más dulce: la Voz de Dios. Pero ahora su voluntad me ha dejado libre incluso de este lazo de piedad filial. Ya de por sí no habría habido lazo. Eran dos justos, y Dios, ciertamente, hablaba en ellos como me habla a mí. Habrían seguido la justicia y la verdad. Cuando pienso en ellos, pienso que están en la quietud de la espera entre los Patriarcas, y acelero con mi sacrificio la venida del Mesías para abrirles las puertas del Cielo. En la tierra yo me rijo, o sea, es Dios quien rige a su pobre sierva diciéndole sus preceptos, y yo los cumplo, porque cumplirlos es mi alegría. Cuando llegue la hora, le diré a mi esposo mi secreto... y él lo acogerá en su interior. 

Pero, María... ¿con qué palabras lo vas a persuadir? Tendrás en contra el amor de un hombre, la Ley y la vida. 

Tendré conmigo a Dios... Dios abrirá a la luz el corazón de mi esposo... la vida perderá sus aguijones de sentido para ser pura flor con perfume de caridad. La Ley... Ana, no me llames blasfema. Yo creo que la Ley pronto va a sufrir un cambio. Pensarás: "¿quién puede cambiarla, si es divina?". Sólo quien la puede mutar: Dios. El tiempo está más próximo de lo que pensáis, yo os lo digo. Leyendo a Daniel, una gran luz que venía del centro del corazón se me ha iluminado, y la mente ha comprendido el sentido de las arcanas palabras. Serán abreviadas las setenta semanas por las oraciones de los justos. ¿Será cambiado el número de los años? No. La profecía no miente; mas, la medida del tiempo profético no es el curso del Sol, sino el de la Luna, y por ello os digo: "Cercana está la hora que oirá el vagido del Nacido de una Virgen". ¡Oh, si esta Luz que me ama quisiera decirme — pues muchas cosas me dice — dónde está la mujer feliz que dará a luz el Hijo a Dios y el Mesías a su pueblo! Caminando descalza recorrería la tierra; ni frío y hielo, ni polvo y canícula, ni fieras y hambre me serían obstáculo para llegar a Ella y decirle: "Concédele a tu sierva y a la sierva de los siervos del Cristo vivir bajo tu techo. Haré girar la rueda del molino y la prensa; como esclava ponme en el molino; como pastora, a tu rebaño; o para lavar los pañalitos a tu Nacido; ponme en tus cocinas, en tus hornos... donde tú quieras, pero recíbeme. ¡Que yo lo pueda ver, que pueda oír su voz, recibir su mirada!". Y, si no me admitiese, yo viviría, mendiga, a su puerta, de limosnas y escarnios, al raso o bajo el sol intenso, con tal de oír la voz del Mesías niño y el eco de su risa, y luego verle pasar... y, quizás, un día recibiría de Él el óbolo de un pan... ¡Oh, aunque el hambre me desgarrara las entrañas y desfalleciera después de tanto ayuno, yo no me comería ese pan! Lo tendría como un saquito de perlas contra mi corazón y lo besaría para sentir el perfume de la mano del Cristo, y ya no tendría ni hambre ni frío, porque su contacto me proporcionaría éxtasis y calor, éxtasis y alimento... 

- ¡Tú deberías ser la Madre del Cristo, tú que le amas de esa forma! ¿Por eso es por lo que quieres permanecer virgen? 

- ¡Oh, no! Yo soy miseria y polvo. No oso levantar la mirada hacia la Gloria. Por eso es por lo que prefiero mirar dentro de mi corazón más que mirar al doble Velo, tras el cual sé que está la invisible Presencia de Yeohveh. Allí está el Dios terrible del Sinaí. Aquí, en mí, veo al Padre nuestro, veo un amoroso Rostro que me sonríe y bendice, porque soy pequeña como un pajarillo que el viento sujeta sin sentir su peso, y débil como tallito de muguete silvestre que sólo sabe florecer y perfumar, y no opone más resistencia al viento que la de su perfumada y pura dulzura. ¡Dios, mi viento de amor! No, no es por eso, sino porque al Nacido de Dios y de una Virgen, al Santo del Santísimo no le puede gustar sino lo que en el Cielo ha elegido como Madre y lo que en la tierra le habla del Padre celestial: la Pureza. Si la Ley meditara en esto, si los rabíes, que la han multiplicado con todas las sutilezas de su enseñanza, volviendo la mente a horizontes más altos, se sumergieran en lo sobrenatural, dejando de lado lo humano y la ganancia que pretenden olvidando el Fin supremo, deberían, sobre todo, volver su enseñanza a la Pureza, para que el Rey de Israel, cuando venga, la encuentre. Con el olivo del Pacífico, con las palmas del Triunfador, esparcid azucenas y azucenas y azucenas... ¡Cuánta Sangre tendrá que derramar para redimirnos el Salvador! ¡Cuánta! De los miles de heridas que Isaías vio en el Hombre de dolores, cae, cual rocío de un recipiente poroso, una lluvia de Sangre. ¡Que no caiga en el lugar de la profanación y la blasfemia esta Sangre divina, sino en copas de fragante pureza que la acojan y recojan, para luego esparcirla sobre los enfermos del espíritu, sobre los leprosos del alma, sobre los muertos a Dios! ¡Dad azucenas, azucenas dad para enjugar, con la cándida vestidura de los pétalos puros, los sudores y las lágrimas del Cristo! ¡Dad azucenas, azucenas dad para el ardor de su fiebre de Mártir! ¡Oh, ¿dónde estará esa Azucena que te lleva dentro; dónde, la que aplacará la quemazón que padeces; dónde, la que se pondrá roja con tu Sangre y morirá por el dolor de verte morir; dónde, la que llorará ante tu Cuerpo desangrado?! ¡Oh, Cristo, Cristo, suspiro mío!... 

María queda en silencio, llorando y abatida. 

Ana está un rato en silencio. Luego, con su voz blanca de anciana conmovida, dice: 

- ¿Tienes algo más que enseñarme, María? 

María se estremece. Debe haber creído, en su humildad, que su maestra la haya reprendido y dice: - ¡Perdón! Tú eres maestra, yo soy una pobre nada. Es que esta Voz me sube del corazón. Yo la tengo bien vigilada, para no hablar; pero, cual río que por el ímpetu de la ola rompe las presas, ahora me ha prendido y se ha desbordado. No tengas en cuenta mis palabras y mortifica mi presunción. Las arcanas palabras deberían estar en el arca secreta del corazón al que Dios, en su bondad, favorece. Lo sé. Pero, tan dulce es esta invisible Presencia, que me embriaga... ¡Ana, perdona a tu pequeña sierva! 

Ana la estrecha contra sí, y todo el viejo rostro rugoso tiembla y brilla de llanto. Las lágrimas se insinúan entre las arrugas como agua por terreno accidentado que se transforma en un trémulo regatillo. No obstante, la anciana maestra no suscita risa, sino que, al contrario, su llanto promueve la más alta veneración. 

María está entre sus brazos, su carita contra el pecho de la anciana maestra, y todo termina así. 

Dice Jesús: 
\emph{María tenía el recuerdo de Dios. Soñaba con Dios. Creía soñar. No hacía sino ver de nuevo cuanto su espíritu había visto en el fulgor del Cielo de Dios, en el instante en que había sido creada para ser unida a la carne concebida en la tierra. Condividía con Dios, si bien de forma mucho menor, por exigencia de justicia, una de las propiedades de Dios: la de recordar, ver y prever, por el atributo de una inteligencia no lesionada por la Culpa, y, por tanto, poderosa y perfecta. El hombre ha sido creado a imagen y semejanza de Dios. Una de las semejanzas está en la posibilidad, para el espíritu, de recordar, ver y prever. Esto explica la facultad de leer el futuro, facultad que viene, muchas veces y directamente, por voluntad divina, otras por el recuerdo, que se alza, como Sol en una mañana, iluminando un cierto punto del horizonte de los siglos precedentemente visto desde el seno de Dios. Son misterios demasiado altos como para que podáis comprenderlos plenamente. Eso sí, reflexionad. ¿Esa Inteligencia suprema, ese Pensamiento que lo sabe todo, esa Vista que lo ve todo, que os crea con un movimiento de su voluntad y con el hálito de su amor infinito, haciéndoos hijos suyos por origen e hijos suyos por destino, podrá daros algo que sea distinto de Él? Os lo da en proporción infinitesimal, porque la criatura no podría contener al Creador, mas esa parte es, en su infinitesimal, perfecta y completa. ¡Cuán grande el tesoro de inteligencia que dio Dios al hombre, a Adán! La culpa lo ha menoscabado, mas mi Sacrificio lo reintegra y os abre los fulgores de la Inteligencia, sus ríos, su ciencia. ¡Oh, sublimidad de la mente humana unida por la Gracia a Dios, copartícipe de la capacidad de Dios de conocer!.. De la mente humana unida por la Gracia a Dios. No hay otro modo; que lo tengan presente los que anhelan conocer secretos ultrahumanos. Toda cognición que no venga de alma en gracia — y no está en gracia aquel que se manifiesta contrario a la Ley divina, cuyos preceptos son muy claros — sólo puede venir de Satanás, y difícilmente corresponde a verdad por lo que se refiere a cuestiones humanas, y nunca responde a verdad por lo que respecta a lo sobrehumano, porque el Demonio es padre de la mentira y a quien arrastra consigo lo lleva por el sendero de la mentira. No existe ningún otro método para conocer la verdad, sino el que viene de Dios. Y Dios habla y dice o hace recordar, del mismo modo como un padre a un hijo le hace recordar la casa paterna y dice: "¿Te acuerdas cuando conmigo hacías esto, veías aquello, oías aquello otro? ¿Te acuerdas cuando yo te despedía con un beso? ¿Te acuerdas cuando me viste por primera vez, cuando viste el fulgurante sol de mi rostro en tu alma virgen, instantes antes creada y aún exenta — puesto que acababa de salir de mí — de la debilidad que después te consumiera? ¿Te acuerdas de cuando comprendiste en un latido de amor lo que es el Amor y cuál es el misterio de nuestro Ser y Proceder?". Y cuando la capacidad limitada del hombre en gracia no llega a comprender, entonces el Espíritu de ciencia habla y enseña. Pero para poseer al Espíritu es necesaria la Gracia. Y para poseer la Verdad y la Ciencia es necesaria la Gracia. Y para tener consigo al Padre es necesaria la Gracia, Tienda en que las tres Personas hacen morada, Propiciatorio en que reside el Eterno y habla, no desde dentro de la nube, sino mostrando su Rostro al hijo fiel. Los santos tienen el recuerdo de Dios, de las palabras oídas en la Mente creadora y resucitadas por la Bondad en su corazón para elevarlos como águilas en la contemplación de la Verdad, en el conocimiento del Tiempo. María era la Llena de Gracia. Toda la Gracia Una y Trina estaba en Ella. Toda la Gracia Una y Trina la preparaba como esposa para la boda, como tálamo para la prole, como divina para su maternidad y para su misión. Ella es la que cierra el ciclo de las profetisas del Antiguo Testamento y abre el de los "portavoces de Dios" en el Nuevo Testamento. Verdadera Arca de la Palabra de Dios, mirando en su interior eternamente inviolado, descubría, trazadas por el dedo de Dios sobre su corazón inmaculado, las palabras de ciencia eterna, y recordaba, como todos los santos, haberlas oído ya al ser generada con su espíritu inmortal por Dios Padre, creador de todo lo que tiene vida. Y, si no recordaba todo de su futura misión, era porque en toda perfección humana Dios deja algunas lagunas, por ley de una divina prudencia que es bondad y mérito para y hacia la criatura. María, segunda Eva, tuvo que conquistarse su parte de mérito de ser la Madre del Cristo; con una fiel, buena voluntad. Esto quiso también Dios en su Cristo para hacerle Redentor. El espíritu de María estaba en el Cielo. Su parte moral y su carne estaban en la tierra, y tenían que pisotear tierra y carne para llegar hasta el espíritu y unirlo al Espíritu en un abrazo fecundo.}

Nota mía. Todo el día de ayer había estado pensando que vería la noticia de la muerte de los padres, y, además — por qué, no lo sé —, dado por Zacarías. Igualmente pensaba, a mi manera, cómo trataría Jesús el punto del "recuerdo de Dios por parte de los santos". Esta mañana, cuando empezó la visión, he dicho: "Eso es, ahora le dirán que es huérfana". Y ya sentía encogido mi corazón porque... se trataba de oír y ver la misma tristeza mía de estos días. Sin embargo, no hay nada de cuanto había pensado ver y oír; pero es que ni una palabra por equivocación. Esto me consuela porque me dice que verdaderamente no hay nada mío, ni siquiera una honesta sugestión respecto a un determinado punto. Todo viene realmente de otra fuente. Mi continuo miedo cesa... hasta la próxima vez, porque este miedo de ser engañada y de engañar me acompañará siempre. 
 
\chapter{María confía su voto al Sumo Sacerdote.}
\emph{3 de septiembre de 1944.}
 
¡Qué noche de infierno! Verdaderamente parecía como si los demonios hubieran salido a la Tierra a pasear. Cañonazos, truenos, relámpagos, peligro, miedo, sufrimiento por estar en una cama que no es mía... (estaban en la Segunda Guerra Mundial y la guerra se desarrollaba cerca de su pueblo) Y, en medio, como una flor toda blanca y suave entre fogonazos y angustias, la presencia de María, un poco más adulta que en la visión de ayer, pero todavía jovencita, con sus trenzas rubias sobre los hombros, su vestido blanco y su mansa, recogida sonrisa, una sonrisa interior, vuelta al misterio glorioso que lleva dentro de su corazón. Paso la noche comparando su aspecto dulce con la crueldad que hay en el mundo, y evocando sus palabras de ayer por la mañana, canto de caridad viva, en contraste con el odio que hace que los hombres se despedacen... 

Pues bien, esta mañana, de nuevo en el silencio de mi habitación, presencio esta escena. 

María sigue estando en el Templo, y ahora sale del Templo propiamente dicho entre otras vírgenes. 

Debe haberse llevado a cabo alguna ceremonia, pues un olor a inciensos se esparce por la atmósfera toda roja de un hermoso ocaso, que yo diría que es de otoño avanzado, porque un cielo ya dulcemente cansado, como lo está en un octubre sereno, se arquea sobre los jardines de Jerusalén, en los que el amarillo ocre de las hojas que pronto caerán dispone manchas dorado- rojizas entre el verde- plata de los olivos. 

La comitiva — mejor sería llamarla enjambre — cándida de las vírgenes cruza el patio posterior, sube la escalinata, atraviesa un pórtico, entra en otro patio menos suntuoso, cuadrado, que como aperturas no tiene sino la que sirve para acceder a él. Debe ser el patio dedicado a acoger las pequeñas moradas de las vírgenes reservadas para el Templo, porque cada una de las jovencitas se dirige a su celda como una palomita a su nido, y asemejan verdaderamente a una bandada de palomas separándose tras haberlas tenido agrupadas. Muchas — podría decir todas — hablan entre sí antes de dejarse, en voz baja pero al mismo tiempo festiva. María guarda silencio. Sólo las saluda con afecto antes de separarse; luego se dirige a su habitacioncita, que está en una de las esquinas a la derecha. 

Se llega hasta Ella una maestra anciana, aunque no tanto como Ana de Fanuel. 

María, el Sumo Sacerdote te espera. 

María la mira con cierto asombro, pero no hace preguntas. Se limita a responder: 

Voy inmediatamente. 

No sé si la espaciosa sala en que entra es de la casa del Sacerdote o forma parte de los aposentos de las mujeres que están dedicadas al Templo. Sé que es vasta y luminosa, puesta con gusto, y que en ella, además del Sumo Sacerdote (que con las vestiduras que lleva aparece muy elegante), están Zacarías y Ana de Fanuel. 

María se inclina profundamente en el umbral de la puerta y no entra hasta que el Sumo Sacerdote no le dice: "Pasa, María. No temas". Ella se yergue y alza la cara, y entra lentamente, no por desgana, sino por un algo de involuntaria solemnidad que la hace parecer más mujer. 

Ana le sonríe para animarla y Zacarías la saluda con un: "Paz a ti, prima". 

El Pontífice la observa atentamente. Luego le dice a Zacarías: 

Es patente en Ella la estirpe de David y Aarón. 

Hija, conozco tu gracia y tu bondad. Sé que cada día has ido creciendo en ciencia y gracia ante los ojos de Dios y de los hombres. Sé que la voz de Dios susurra a tu corazón las más dulces palabras. Sé que eres la Flor del Templo de Dios y que un tercer querubín está ante el Testimonio desde que tú llegaste; y quisiera que tu perfume siguiera subiendo con los inciensos cada nuevo día. Pero, la Ley se expresa en modo distinto. Tú ya no eres una niña, sino una mujer. Y en Israel todas las mujeres deben casarse para ofrecer a su hijo varón al Señor. Tú seguirás el precepto de la Ley. No temas, no te ruborices. No me olvido de tu ofrecimiento. De hecho ya te la tutela la Ley al ordenar que todo hombre reciba de su estirpe la mujer; pero, aunque no fuera así, yo lo haría, para no corromper tu magnífica sangre. ¿No conoces, María, a alguno de tu estirpe que pudiera ser tu marido? 

María levanta su cara, todo roja de pudor, y, con un primer titileo de llanto, que resplandece orlando los párpados, y con voz temblorosa, responde: 

Ninguno. 

No puede conocer a ninguno, puesto que entró aquí siendo niña, y la estirpe de David está demasiado castigada y demasiado dispersa como para que las distintas ramas puedan reunirse y formar con sus frondas la copa de la palma regia - dice Zacarías. 

Entonces le dejaremos a Dios que elija. 

Las lágrimas, contenidas hasta ese momento, brotan y descienden hasta la trémula boca. María dirige una mirada suplicante a su maestra. 

Ana la socorre diciendo: 

María se ha prometido al Señor para gloria de Dios y para la salvación de Israel. Era sólo una niña que apenas sabía pronunciar y ya se había ligado con un voto. 

Se debe a esto entonces tu llanto. No es por resistencia a la Ley. 

Es por esto... no por otro motivo. Yo te obedezco, Sacerdote de Dios. 

Esto confirma cuanto de ti me ha sido referido siempre. ¿Desde hace cuántos años eres virgen consagrada? 

Yo creo que desde siempre. Antes de venir a este Templo ya me había ofrecido al Señor. 

Pero, ¿no eres tú la Niña que vino hace doce inviernos a pedirme entrar? 

Sí. 

Y ¿cómo, entonces, puedes decir que ya eras de Dios? 

Si miro hacia atrás yo me veo ya consagrada... No tengo memoria de la hora en que nací, ni de cómo empecé a amar a mi madre y a decirle a mi padre: "¡Oh, padre, yo soy tu hija!"... Pero sí recuerdo, aunque no a partir de cuándo, haber dado mi corazón a Dios. Quizás fue con el primer beso que supe dar, con la primera palabra que supe pronunciar, con el primer paso que supe dar... Sí, eso es, creo que mi primer recuerdo de amor lo encuentro junto a mi primer paso seguro... Mi casa... Mi casa tenía un jardín lleno de flores... un huerto de árboles frutales y campos cultivados... y había un manantial allí, en el fondo, al pie del monte, que manaba de una roca ahuecada en forma de gruta... estaba llena de hierbas largas y finas que pendían de todas partes asemejando cascaditas verdes, y parecía como si llorasen porque las livianas hojitas, que en su espesura parecían un bordado, tenían, todas, una gotita de agua que al caer sonaba como un cascabelito diminuto. Y también cantaba el manantial. Y había aves en los olivos y en los manzanos de la pendiente que estaba hacia arriba del manantial, y palomas blancas venían a lavarse en la balsa límpida de la fuente... Ya no me acordaba de todo esto porque había puesto todo mi corazón en Dios y, aparte de mi padre y de mi madre, a quienes amé en vida y después de muertos, todas las demás cosas de la tierra habían desaparecido de mi corazón... Pero tú me haces pensar en ello, Sacerdote... Debo buscar el momento en que me di a Dios... y vuelven a la mente las cosas de los primeros años... Me gustaba esa gruta porque en ella oía una Voz, más dulce que el canto del agua y de los pájaros, que me decía: "Ven, dilecta mía". Me gustaban esas hierbas diamantinas con sus gotas sonoras porque en ellas veía el signo de mi Señor y me perdía diciéndome: "¿Ves qué grande es tu Dios, alma mía! El mismo que ha hecho los cedros del Líbano para el aquilón ha hecho estas hojitas que ceden bajo el peso de un mosquito para alegría de tu ojo y para que protejan tu piececito". Me gustaba aquel silencio de cosas puras: el viento leve, el agua de plata, la pulcritud de las palomas... me gustaba esa paz que amparaba la gruta, descendiendo de los manzanos y de los olivos, ya enteramente en flor, ya repletos de frutos... Y, no sé... me parecía que la Voz me dijese a mí, justamente a mí: "Ven, tú, aceituna especiosa; ven, tú, dulce pomo; ven, tú, fuente sigilada; ven, tú, paloma mía"... Dulce era el amor de mi padre y de mi madre... dulce su voz cuando me llamaba... ¡Ah, pero ésta, ésta...! ¡Oh!, yo creo que así la oiría en el Paraíso Terrenal aquella que fue culpable, y no sé cómo pudo preferir un silbido a esta Voz de amor, cómo pudo apetecer otro conocimiento que no fuera Dios... Aún con el sabor a leche materna en los labios, pero con el corazón ebrio de miel celestial, yo dije entonces: "Sí, voy. Tuya. Y mi carne no tendrá otro señor aparte de Ti, Señor, de la misma forma que mi espíritu no tiene otro amor"... Y al decir esto me parecía estar repitiendo cosas ya dichas precedentemente y estar cumpliendo un rito que ya había sido cumplido, y no me resultaba extraño el Esposo elegido, puesto que yo ya conocía su ardor y mi vista se había formado bajo su luz y mi capacidad de amar había hallado cumplimiento entre sus brazos. ¿Cuándo?.. No lo sé. Yo diría que más allá de la vida, porque tengo la impresión de que siempre ha sido mío, y de que yo siempre he sido suya, y de que yo existo porque Él me ha querido para sí, para alegría de su Espíritu y del mío... 'Ahora obedezco, Sacerdote; pero, dime tú cómo debo actuar... No tengo ni padre ni madre. Sé tú mi guía. - Dios te dará el esposo, y será santo, dado que en Dios te abandonas. Lo que harás será manifestarle tu voto. 

- ¿Y aceptará? 

- Espero que sí. Ora, hija, para que él pueda comprender tu corazón. Ahora puedes marcharte. Que Dios te acompañe siempre. 

María se retira con Ana y Zacarías se queda con el Pontífice. La visión cesa aquí. 
 
\chapter{José designado para esposo de la Virgen.}
 
Veo una rica sala, con un suelo bonito, cortinas, alfombras y muebles taraceados. Debe formar parte del Templo todavía. Se deduce que hay sacerdotes (entre los cuales Zacarías) y muchos hombres de las más diversas edades, o sea, de los veinte a los cincuenta años aproximadamente. 

Están hablando unos con otros, bajo pero animadamente. Se los ve inquietos por algo que desconozco. Todos están vestidos de fiesta, con vestidos nuevos o, al menos, recién lavados, como si estuvieran ataviados para una celebración. Muchos se han quitado el paño con que se cubren la cabeza, otros todavía lo tienen puesto, especialmente los ancianos, mientras que los jóvenes muestran sus cabezas descubiertas: unas rubio- oscuras, otras moreno- oscuras, algunas negrísimas, una — sólo ella — rojo- cobre. Las cabelleras son generalmente cortas, pero algunas de ellas llegan hasta los hombros. No deben conocerse todos entre sí porque se están observando con curiosidad. Pero parecen relacionados pues se ve que los apremia un pensamiento común. 

En una de las esquinas veo a José. Está hablando con un anciano de aspecto robusto y vigoroso. José tendrá unos treinta años. Es un hombre apuesto; pelo corto, más bien rizado, de un castaño oscuro como el de la barba y el bigote, que velan un mentón bien conformado y suben hacia las mejillas moreno- rojizas, no aceitunadas como en el caso de otras personas morenas; tiene ojos oscuros, buenos y profundos, muy serios, incluso yo diría que un poco tristes. Sin embargo, cuando sonríe — como está haciendo en este momento —aparecen alegres y juveniles. Está vestido todo de marrón claro, de forma muy simple pero muy ordenada. 

Entra un grupo de jóvenes levitas. Se disponen entre la puerta y una mesa larga y estrecha que está cerca de la pared en cuyo centro se encuentra la puerta, la cual queda abierta de par en par; sólo una cortina tensa, que pende hasta unos veinte centímetros del suelo, sigue cubriendo el vano. 

La curiosidad se acentúa. Y más aún cuando una mano separa la cortina para dejar paso a un levita que lleva en los brazos un haz de ramas secas sobre el cual ha sido depositada delicadamente una ramilla florecida, una ligera espuma de pétalos blancos que apenas muestran un rosáceo esfumado que desde el centro se irradia, atenuándose cada vez más, hasta el extremo de los livianos pétalos. El levita deposita el haz de ramas encima de la mesa con exquisito cuidado para no lesionar el milagro de esa rama en flor en medio de tanta hojarasca. 

Un murmullo recorre la sala. Los cuellos se alargan, las miradas se hacen más penetrantes, como para poder ver. Zacarías, con los sacerdotes, también trata de ver, estando como está más cerca de la mesa, pero no ve nada. 

José, desde su esquina, apenas dirige los ojos hacia el haz de ramas, y, cuando su interlocutor le dice algo, él hace un gesto denegatorio como de quien dice: "¡Imposible!", y sonríe. 

Un toque de trompeta desde el otro lado de la cortina. Todos guardan silencio y se disponen en perfecto orden mirando hacia la puerta, ahora enteramente abierta, dado que a la cortina la hacen deslizarse sobre sus anillos. Rodeado de otros ancianos, entra el Sumo Pontífice. Todos se postran. El Pontífice se acerca a la mesa y, en pie, comienza a hablar: 

Hombres de la estirpe de David, que habéis convenido en este lugar por convocatoria mía, escuchad. El Señor ha hablado, ¡gloria a Él! De su Gloria un rayo ha descendido y, como sol de primavera, ha dado vida a una rama seca, y ésta ha florecido milagrosamente cuando ninguna rama de la tierra hoy está en flor, hoy, último día de las Luminarias, cuando aún no se ha derretido la nieve caída sobre las alturas de Judá y es lo único cándido que hay entre Sión y Betania. Dios ha hablado haciéndose padre y tutor de la Virgen de David, que no tiene tutor alguno aparte de Dios. Santa doncella, gloria del Templo y de la estirpe, ha merecido la palabra de Dios para conocer el nombre del esposo grato al Eterno. ¡Muy justo debe ser para haber sido elegido por el Señor para tutelar a su amada Virgen! Por ello nuestro dolor de perderla se aplaca, y cesa toda preocupación acerca de su destino como esposa. Y a aquel que ha sido señalado por Dios le confiamos, plenamente seguros, la Virgen que posee la bendición de Dios y la nuestra. El nombre del prometido es José de Jacob, betlemita, de la tribu de David, carpintero en Nazaret de Galilea. José, acércate; el Sumo Sacerdote te lo ordena. 

Gran murmullo. Cabezas que se vuelven, ojos y manos que señalan, expresiones de desilusión y expresiones de alivio. Alguno, especialmente entre los viejos, debe haberse sentido contento de no haber sido destinado para ello. 

José, muy colorado y visiblemente turbado, se abre paso. Ya está ante la mesa, frente al Pontífice, al cual ha saludado con reverencia. 

Venid todos y mirad el nombre grabado en la rama. Coja cada uno su ramilla, para asegurarse de que no hay trampa. 

Los hombres obedecen. Miran la ramilla que delicadamente tiene el Sumo Sacerdote; cada uno coge la suya: unos la rompen, otros la guardan. Todos miran a José: hay quien mira y calla, otros lo felicitan. El anciano con el que antes estaba hablando dice: 

- ¿No te lo había dicho, José? ¡Quien menos se siente seguro es el que vence la partida! Ya han pasado todos. 

El Sumo Sacerdote da a José la ramilla florecida, y, poniéndole la mano en el hombro, le dice: 

No es rica, y tú lo sabes, la esposa que Dios te dona, pero posee todas las virtudes. Hazte cada día más digno de Ella. En Israel no hay flor alguna tan linda y pura como Ella. Salid todos ahora. Que se quede José; y tú, Zacarías, pariente, trae a la prometida. 

Salen todos, excepto el Sumo Sacerdote y José. Vuelven a correr la cortina, cubriendo así la puerta. 

José está todo humilde junto al majestuoso Sacerdote. Una pausa silenciosa y éste le dice: María debe manifestarte un voto que ha hecho. Ayúdala en su timidez. Sé bueno con la mujer buena. 

Pondré mi virilidad a su servicio y ningún sacrificio por Ella me pesará. Estáte seguro de ello. 

 Entra María con Zacarías y Ana de Fanuel. 

Ven, María - dice el Pontífice - Éste es el esposo que Dios te ha destinado. Es José de Nazaret. Regresarás, por tanto, a tu ciudad. Ahora os voy a dejar. Que Dios os dé su bendición. Que el Señor os mire y os bendiga, os muestre su rostro y tenga siempre piedad de vosotros. Que vuelva a vosotros su rostro y os dé la paz. 

Zacarías sale escoltando al Pontífice. Ana felicita al prometido y luego también sale. 

Los dos prometidos están el uno enfrente del otro. María, toda colorada, tiene la cabeza agachada. José, también ruborizado, la observa buscando las primeras palabras que decir. 

Al fin las encuentra y una sonrisa ilumina su rostro. Dice: 

Te saludo, María. Te vi cuando eras una niña de pocos días... Yo era amigo de tu padre y tengo un sobrino de mi hermano Alfeo que era muy amigo de tu madre, su pequeño amigo, pues ahora no tiene más que dieciocho años, y, cuando tú todavía no habías nacido, siendo sólo un niñito, ya alegraba las tristezas de tu madre, que lo quería mucho. No nos conoces porque viniste aquí siendo muy pequeñita. Pero en Nazaret todos te quieren y piensan en ti, y hablan de la pequeña María de Joaquín, cuyo nacimiento fue un milagro del Señor, que hizo verdecer a la estéril... Yo me acuerdo de la tarde en que naciste... Todos la recordamos por el prodigio de una gran lluvia que salvó los campos, y de una violenta tormenta durante la cual los rayos no quebraron ni siquiera un tallito de brezo silvestre, tormenta que terminó con un arco iris de dimensiones y belleza no vistas nunca más. Y... ¿quién no recuerda la alegría de Joaquín? Te mecía enseñándote a los vecinos... Considerándote una flor venida del Cielo, te admiraba, y quería que todos te admirasen. ¡Oh, dichoso y anciano padre que murió hablando de su María, tan bonita y buena y que decía palabras llenas de gracia y de saber!.. ¡Tenía razón al admirarte y al decir que no existe ninguna más hermosa que tú! ¿Y tu madre? Llenaba con su canto el ángulo en que estaba tu casa. Parecía una alondra en primavera durante la gestación, y luego, cuando te amamantaba. Yo hice tu cuna, una cunita toda de entalladuras de rosas, porque así la quiso tu madre. Quizás esté todavía en la casa, ahora cerrada... Yo soy viejo, María. Cuando naciste, yo ya hacía mis primeros trabajos. Ya trabajaba... ¡Quién me iba a decir que te hubiera tenido por esposa! Quizás hubieran muerto más felices los tuyos, porque éramos amigos. Yo enterré a tu padre, llorándole con corazón sincero porque fue para mí maestro bueno durante la vida. 

María levanta muy despacio el rostro, sintiéndose cada vez más segura al oír cómo le habla José, y cuando alude a la cuna sonríe levemente, y cuando José habla de su padre le tiende una mano y dice: Gracias, José - Un "gracias" tímido y delicado. 

José toma entre sus cortas y fuertes manos de carpintero esa manita de jazmín, y la acaricia con un afecto que pretende inspirar cada vez más tranquilidad. Quizás espera otras palabras, pero María vuelve a guardar silencio. Entonces continúa hablando él: 

La casa, como sabes, está intacta, menos la parte que fue derribada por orden consular para transformar en calle el sendero para los convoyes de Roma. Pero las parcelas de cultivo, las que te han quedado — porque ya sabes... la enfermedad de tu padre consumió mucho tus haberes — están un poco abandonadas. Hace ya más de tres primaveras que los árboles y las cepas no conocen podadera de hortelano, y la tierra está sin cultivar y, por tanto, dura. Pero los árboles que te vieron cuando eras pequeñita están todavía allí, y, si me lo permites, yo me ocuparé inmediatamente de ellos. 

Gracias, José. Pero, ya trabajas... 

Trabajaré en tu huerto durante las primeras y las últimas horas del día. Ahora el tiempo de luz se va alargando cada vez más. Para la primavera quiero que todo esté en orden, para alegría tuya. Mira, ésta es una ramilla del almendro que está frente a la casa. Quise coger ésta... — se puede entrar por cualquier parte por el seto destruido, pero ahora le haré de nuevo sólido y fuerte —, quise coger ésta pensando que si yo hubiera sido el elegido — no lo esperaba porque soy consagrado nazareno, y he obedecido porque se trataba de una orden del Sacerdote, no por deseos de casamiento —, pensando, te decía, que el tener una flor de tu jardín te habría alegrado. Aquí la tienes, María. Con ella te doy mi corazón, que, como ella, hasta ahora, ha florecido sólo para el Señor, y que ahora florece para ti, esposa mía. 

María coge la ramita. Se la ve emocionada, y mira a José con una cara cada vez más segura y radiante. Se siente segura de él. Cuando él dice: "Soy consagrado nazareno", su rostro se muestra todo luminoso y encuentra fuerzas para decir: Yo también soy toda de Dios, José. No sé si el Sumo Sacerdote te lo ha dicho... 

Me ha dicho sólo que tú eres buena y pura y que debes manifestarme un voto tuyo, y que fuera bueno contigo. Habla, María. Tu José desea hacerte feliz en todos tus deseos. No te amo con la carne. ¡Te amo con mi espíritu, santa doncella que Dios me otorga! Debes ver en mí un padre y un hermano, además de un esposo. Ábrete a mí como con un padre, abandónate en mí como con un hermano. 

Ya desde la infancia me consagré al Señor. Sé que esto no se hace en Israel, pero yo sentía una Voz que me pedía mi virginidad en sacrificio de amor por la venida del Mesías. ¡Hace mucho tiempo que Israel lo espera!.. ¡No es demasiado el renunciar por esto a la alegría de ser madre! 

José la mira fijamente, como queriendo leer en su corazón, y luego coge las dos manitas que tienen todavía entre los 

dedos la ramita florecida, y dice: Pues yo también uniré mi sacrificio al tuyo, y amaremos tanto con nuestra castidad al Eterno, que Él dará antes a la Tierra al Salvador, permitiéndonos ver su Luz resplandecer en el mundo. Ven, María. Vamos ante su Casa y juremos amarnos como lo hacen los ángeles entre sí. 'Luego iré a Nazaret a prepararlo todo para ti, en tu casa si quieres ir a ella, en otra parte si así lo deseas. 

En mi casa... En el fondo había una gruta... ¿Todavía está? 

Está, pero ya no es tuya... Yo, de todas formas, te haré otra gruta donde estarás fresca y tranquila en las horas más calurosas. La haré lo más parecida posible. Y... dime, ¿quién quieres que esté contigo? 

Nadie. No tengo miedo. La madre de Alfeo, que siempre viene a verme, me hará compañía un poco durante el día, y por la noche prefiero estar sola. Ningún mal me puede suceder. 

Bueno, y ahora estoy yo... ¿Cuándo debo venir a recogerte? - Cuando tú quieras, José. 

Pues entonces vendré cuando la casa esté en orden. No pienso tocar nada. Quiero que encuentres todo como lo dejó tu madre, pero quiero también que esté llena de luz y bien limpia para acogerte sin tristeza. Ven, María. Vamos a decirle al Altísimo que le bendecimos. 

Y no veo nada más. Me queda, eso sí, en el corazón el sentido de seguridad que experimenta María... 
 
\chapter{Esponsales de la Virgen y José,}
\emph{que fue instruido por la Sabiduría para ser custodio del Misterio.}
 
¡Qué guapa está María, rodeada de sus amigas y sus maestras jubilosas, vestida para los esponsales! Entre aquéllas está también Isabel. 

Va toda vestida de blanquísimo lino, tan seríceo y fino que parece de preciosa seda. Ciñe su grácil cintura un cinturón burilado de oro y plata, hecho todo de medallones unidos por delgadas cadenas — cada uno de los medallones es una filigrana engastada en la pesada plata bruñida por el tiempo — y, quizás porque es demasiado largo para Ella, que todavía es una delicada jovencita, le pende por delante con los tres últimos medallones, cayendo entre los pliegues del vestido amplísimo, que a su vez termina en una pequeña cola debido a su largura. Calzan sus piececitos unas sandalias de piel blanquísima con hebillas de plata. 

El vestido está sujeto al cuello por una cadenita de rosetas de oro y de filigrana de plata, que presentan en pequeño el mismo motivo del cinturón. La cadenita pasa a través de los anchos ojales del amplio cuello del vestido, acortándolo, por tanto, en frunces que forman como una pequeña puntilla. El cuello de María sobresale entre ese candor fruncido, con la gracia de un tierno tallo fajado con una gasa preciada, y así parece aún más grácil y blanco: un tallito de azucena culminado por su rostro de lirio, el cual, por la emoción, se ve aún más pálido y más puro: un rostro de hostia purísima. 

El pelo ya no le pende sobre los hombros. Está graciosamente dispuesto en nudo de trenzas. Unas valiosas horquillas de plata bruñida, con un trabajo de filigrana que cubre enteramente la parte superior del arco, sujetan las trenzas. El velo materno se apoya sobre ellas y desciende, formando lindos pliegues, por debajo del estrecho aro que lleva ajustado a la frente blanquísima; desciende hasta las caderas, porque María no tiene la altura de su madre y el velo le llega más abajo de ellas, mientras que a Ana le llegaba sólo a la cintura. 

No lleva anillos en las manos; en las muñecas, unas pulseras. Pero estas muñecas son tan delgadas, que las pesadas pulseras maternas se apoyan sobre el dorso de las manos y quizás, si sacudiera las manos, se caerían al suelo. 

Las compañeras la miran absortas desde todos los puntos, y con maravilla. Con sus preguntas y con sus frases de admiración crean un festivo trinar de gorrioncillos. 

- ¿Son de tu madre? 

Antiguas, ¿verdad? 

- ¡Qué bonito, Sara, ese cinturón! 

- ¿Y este velo, Susana? ¡Mira que finura! ¡Fíjate estas azucenas tejidas en el velo! 

¡Déjame ver las pulseras, María! ¿Eran de tu madre? 

Las llevó ella, pero son de la madre de Joaquín, mi padre. 

- ¡Oh, mira! Tienen el sigilo de Salomón entrelazado con sutiles ramitas de palma y olivo, y entre ellas hay azucenas y rosas. ¡Oh! ¿Quién habrá realizado un trabajo tan perfecto y minucioso? 

Son de la casa de David - explica María - Hace ya siglos que las llevan las mujeres de esta estirpe cuando se van a casar, y van pasando a las herederas. 

- ¡Ah, ya! Tú eres hija heredera... 

- ¿Te han traído todo de Nazaret? 

No. Cuando murió mi madre, mi prima se llevó a su casa el ajuar para conservarlo sin que se dañase. Ahora me lo ha traído. 

- ¿Dónde está? ¿Dónde está? Enséñanoslo a las amigas. 

María no sabe qué hacer... Quisiera ser amable, pero no querría remover todas las cosas, que están ordenadas en tres pesados baúles. 

Vienen en su ayuda las maestras: 

El novio está para llegar. No es el momento de crear confusión. Dejadla. Que la cansáis. Id a prepararos". 

El gárrulo enjambre se aleja un poco enfadado. María puede así gozar en paz de la compañía de sus maestras, las cuales le dirigen palabras de alabanza y bendición. 

Isabel también se ha acercado, y, dado que María, emocionada, llora porque Ana de Fanuel la llama hija y la besa con un afecto verdaderamente maternal, le dice: 

María, tu madre no está presente, pero sí está presente. Su espíritu se regocija junto al tuyo, y, mira, las cosas que llevas te traen de nuevo su caricia. En ellas sientes aún el sabor de sus besos. Un día ya lejano, el día en que viniste al Templo, me dijo: "Le he preparado los vestidos y el ajuar para cuando se case, porque quiero ser yo la que le haya hilado las telas y le haya hecho los vestidos, para no estar ausente en el día de su alegría". Mira, al final, cuando yo la asistía, ella quería todas las noches acariciar tus primeros vestidos y este que llevas ahora, y decía: "Aquí siento el olor de jazmín de mi pequeñuela, aquí quiero que Ella sienta el beso de su mamá". ¡Cuántos besos dio a este velo que cubre tu frente! ¡Más besos que hilos tiene!.. Y, cuando uses estas telas hiladas por ella, piensa que más que la estambre los ha hecho el amor de tu madre. Y estas joyas... Tu padre las salvó para ti incluso en los momentos difíciles, para que te embellecieran, como corresponde a una princesa de David, en este momento. Alégrate, María. No estás huérfana; los tuyos están contigo, y quien va a ser tu marido es tan perfecto, que es para ti padre y madre... 

- ¡Oh, sí! ¡Eso es verdad! No puedo quejarme de él, ciertamente. En menos de dos meses ha venido dos veces, y hoy viene por tercera vez, desafiando a las lluvias y al tiempo ventoso, declarándose sujeto a mí... Fíjate: ¡sujeto a mí! ¡Yo, que soy una pobre mujer, y mucho más joven que él! Y no me ha negado nada. Es más, ni siquiera espera a que yo pida. Parece como si un ángel le dijera lo que deseo, y me lo dice él antes de que yo hable. La última vez me dijo: "María, creo que preferirás estar en tu casa paterna. Dado que eres hija heredera, lo puedes hacer, si lo ves oportuno. Yo iré a tu casa. Solamente para observar el rito, tú vas durante una semana a casa de Alfeo, mi hermano. María te quiere ya mucho. De allí partirá la tarde de la boda el cortejo que te llevará a casa". ¿No es amable por su parte? No le ha importado ni siquiera el dar pie a la gente para decir que él no tiene una casa que me guste... A mí me hubiera gustado en todo caso, por estar él, que es tan bueno, en ella. Pero sin duda prefiero la mía... por los recuerdos... ¡Oh, José es bueno! 

- ¿Qué dijo del voto? Todavía no me has comentado nada. 

- No puso ninguna objeción. Es más, conocidas las razones del mismo, dijo: "Uniré mi sacrificio al tuyo". 

- ¡Es un joven santo!- dice Ana de Fanuel. 

El "joven santo" entra en este momento, acompañado de Zacarías. 

Su figura es, literalmente hablando, espléndida. Todo de amarillo oro, parece un soberano oriental. Bolsa y puñal penden de un espléndido cinturón: aquélla, de tafilete bordado en oro; el puñal, en una vaina con guarniciones bordadas en oro, también de tafilete. Cubre su cabeza un turbante, la típica faja de tela como la llevan todavía ciertos pueblos de África, los beduinos por ejemplo; lo sujeta en torno un valioso arito de oro, delgado, que ciñe unos ramitos de mirto. Viste majestuosamente un manto completamente nuevo con muchas franjas. Está radiante de alegría. En las manos lleva unos ramitos de mirto en flor. 

Saluda diciendo: 

- ¡A ti la paz, mi prometida! Paz a todos. 

Recibido el saludo de respuesta, dice: 

Vi tu alegría el día en que te di la ramita de tu huerto. He pensado traerte este mirto que procede de la gruta que tanto estimas. Quería haberte traído las rosas que están enfrente de tu casa, las primeras que están floreciendo ahora; pero las rosas no duran varios días de viaje... Habría llegado trayendo sólo espinas, y yo a ti, dilecta mía, te quiero ofrecer sólo rosas, y quiero sembrar tu camino de flores blandas y perfumadas, para que apoyes tu pie sobre ellas y no encuentres ni inmundicias ni asperezas. 

- ¡Oh, gracias, hombre de corazón bueno! ¿Cómo has logrado que llegara fresco? 

He atado a la silla un recipiente y he metido dentro estas ramitas con las flores todavía en capullo. Durante el viaje han florecido. Tómalas, María. Que tu frente se enguirnalde de pureza, símbolo de la mujer prometida; aunque siempre será mucho menor que la pureza que hay en tu corazón. 

Isabel y las maestras engalanan a María con la florida guirnaldita que se forma al fijar en el precioso aro los ramitos cándidos del mirto, e intercalan unas pequeñas, cándidas rosas, que había en un jarrón encima de un arca. 

María hace ademán de coger su amplio manto cándido para colocárselo prendido a los hombros. Pero su prometido le precede en el gesto y le ayuda a fijar con dos hebillas de plata, en los hombros, este amplio manto suyo. Las maestras disponen los pliegues con amor y gracia. 

Todo está preparado. Mientras esperan a no sé qué, José dice (lo dice apartándose un poco con María): 

He pensado este tiempo en tu voto. Ya te dije que lo comparto. Pero, cuanto más pienso en ello, más me doy cuenta de que no es suficiente el nazireato temporal, aunque se vaya renovando. Yo te he comprendido, María. No merezco todavía la palabra de la Luz, pero sí me llega un murmullo de su voz, y ello me pone en condiciones de leer tu secreto, al menos en sus líneas maestras. Soy un pobre ignorante, María. Soy un pobre obrero. Ni sé de letras ni tengo tesoros, mas a tus pies pongo mi tesoro, para siempre. Mi castidad absoluta, para ser digno de estar a tu lado, Virgen de Dios, "hermana mía, novia, cerrado huerto, fuente sellada", como dice el Antepasado nuestro, que quizás escribió el Cantar viéndote a ti... Yo seré el guardián de este huerto de perfumes en que se dan las más preciadas frutas, donde mana una vena de agua viva con ímpetu suave: ¡tu dulzura, prometida mía, que con tu candor — ¡oh, llena de hermosura! — me has conquistado el espíritu! ¡Oh, tú, más hermosa que una aurora; Sol, que resplandeces porque te resplandece el corazón; oh, toda amor para con tu Dios y para con el mundo al que quieres dar el Salvador con tu sacrificio de mujer! ¡Ven, mi amada! 

Y coge delicadamente su mano para guiarla hacia la puerta. 

Los siguen todos los demás. Afuera se añaden las joviales compañeras, enteramente de blanco todas ellas y con velos. 

Van por patios y pórticos, entre la muchedumbre observadora, hasta llegar a un punto que ya no pertenece al Templo; parece, más bien, una sala dada para el culto, como se deduce de la existencia en ella de lámparas y rollos de pergaminos como en las sinagogas. Los novios caminan hasta llegar frente a un alto atril (casi una cátedra), y esperan. Los demás, perfectamente en orden, se ponen detrás de ellos. Otros sacerdotes y gente simplemente curiosa se agolpan en el fondo de la sala. Entra, solemne, el Sumo Sacerdote. Rumor de los curiosos: - ¿Es él el que los casa? 

Sí, porque es de casta real y sacerdotal. La novia es flor de David y Aarón, y virgen del Templo; el novio, de la tribu de David. 

El Pontífice pone la mano derecha de la novia en la del novio y los bendice solemnemente: 

El Dios de Abraham, Isaac y Jacob esté con vosotros. Que El os una y se cumpla en vosotros su bendición, dándoos su paz y una numerosa descendencia con larga vida y muerte beata en el seno de Abraham. 

Luego se retira, solemne como había entrado. 

Se lleva a cabo la promesa recíproca. María es la prometida- esposa de José. 

Todos salen y, en perfecto orden, van a una sala, en la cual se redacta el contrato de matrimonio, donde se dice que María, hija heredera de Joaquín de David y Ana de Aarón, da como dote a su prometido- esposo su casa y bienes anejos y su ajuar personal así como cualquier otro bien heredado de su padre. Todo queda cumplido. 

Los esposos salen al patio, lo atraviesan, van hacia la salida, que está cerca de la sección de las mujeres dedicadas al Templo. Los está esperando un carro cómodo y voluminoso. Va provisto de una cortina protectora. En él ya están colocados los pesados baúles de María. 

Despedidas, besos y lágrimas, bendiciones, consejos, recomendaciones... María sube con Isabel y se pone en el interior del carro; en la parte de delante se ponen José y Zacarías. Se han quitado los mantos de fiesta y se han arrollado en unas capas oscuras. 

El carro se pone en marcha, al trote pesado de un caballazo oscuro. Los muros del Templo se alejan, y luego los de la ciudad. Ya se ve el campo, nuevo, fresco, florido bajo los primeros soles de la primavera, con los trigos ya alzados un buen palmo del suelo, que parecen esmeraldas transformadas en hojitas ondulantes bajo una brisa ligera con sabor a flores de melocotonero y manzano, con sabor a tréboles en flor y a hierbabuenas silvestres. 

María llora en voz baja, al amparo de su velo, y, de vez en cuando, corre un poco la cortina y mira una vez más al Templo lejano, a la ciudad dejada... 

La visión cesa así. 

Dice Jesús: 
\emph{¿Qué dice el libro de la Sabiduría al cantar sus alabanzas?: "En la sabiduría está presente, efectivamente, el espíritu de inteligencia, santo, único, múltiple, sutil". Y continúa enumerando sus dotes, para terminar el período con estas palabras: "... que todo lo puede, todo lo prevé; que comprende a todos los espíritus, inteligente, puro, sutil. La sabiduría penetra con su pureza, es vapor de la virtud de Dios... por ello en ella no hay nada impuro... imagen de la bondad de Dios. Es única y, no obstante, lo puede todo; es inmutable y da vida nueva a todas las cosas; se comunica a las almas santas; forma a los amigos de Dios y a los profetas". Ya has visto cómo José, no por cultura humana, sino por instrucción sobrenatural, sabe leer en el libro sellado de la Virgen sin mancha; y cómo se acerca extremamente a las verdades proféticas con ese su "ver" un misterio sobrehumano donde los demás veían únicamente una gran virtud. Impregnado de esta sabiduría, que es vapor de la virtud de Dios y emanación cierta del Omnipotente, se conduce con espíritu seguro por el mar de este misterio de gracia que es María, se armoniza con Ella con espirituales contactos — en que se hablan, más que los labios, los dos espíritus en el sagrado silencio de las almas — donde sólo Dios oye voces que perciben también los que le son gratos por servirle con fidelidad y por estar llenos de Él. La sabiduría del Justo, que aumenta por la unión con la Toda Gracia y por la cercanía a Ella, le prepara a penetrar en los secretos más altos de Dios y a poderlos tutelar y defender de insidias humanas y demoníacas. Y contemporáneamente lo va renovando. Del justo hace un santo; del santo, el custodio de la Esposa y del Hijo de Dios. Sin quitar el sello de Dios, él, el casto, que ahora lleva su castidad a heroísmo angélico, puede leer la palabra de fuego escrita sobre el diamante virginal por el dedo de Dios, y en él lee aquello que su prudencia no dice, y que es mucho más grande que lo que leyó Moisés en las tablas de piedra. Y a fin de que ningún ojo profano alcance este Misterio, él se pone, como sello sobre el sello, como arcángel de fuego, a la entrada del Paraíso, dentro del cual el Eterno encuentra sus delicias "paseando al fresco del atardecer" y hablando con Aquella que es su amor, bosque de azucena en flor, aura perfumada de aromas, viento suave de frescura matutina, hermosa estrella, delicia de Dios. La nueva Eva está allí, en su presencia. No es hueso de sus huesos ni carne de su carne; sí, compañera de su vida, Arca viva de Dios. Él la recibe para tutelarla, y a Dios debe restituírsela, pura como la ha recibido. "Desposada con Dios" estaba escrito en ese libro místico de inmaculadas páginas... Y cuando la duda, sibilante, en la hora de la prueba, le sugirió su tormento, él, como hombre y como siervo de Dios, sufrió, como ninguno, por causa del temido sacrilegio. Pero ésta fue la prueba futura. Ahora, en este tiempo de gracia, él ve y se pone a sí mismo al servicio más auténtico de Dios. Luego vendrá la tempestad de la prueba, como para todos los santos, para ser probados y venir así a ser ayudantes de Dios. ¿Qué se lee en el Levítico? "Di a Aarón, tu hermano, que no entre en cualquier tiempo en el santuario que está detrás del Velo, ante el Propiciatorio que cubre al Arca, para no morir — pues Yo apareceré en la nube sobre el oráculo —, si no hace antes estas cosas: ofrecerá un novillo por el pecado y un carnero como holocausto; llevará la túnica de lino y con calzones de lino cubrirá su desnudez". Y verdaderamente José entra, cuando Dios quiere y cuanto Dios quiere, en el santuario de Dios; y traspasa el velo que cela el Arca sobre la cual está suspendido el Espíritu de Dios; y se ofrece a sí mismo y ofrecerá al Cordero, holocausto por el pecado del mundo, expiación de tal pecado? Y esto lo hace, vestido de lino, mortificados los miembros viriles para abolir su sensualidad, la cual, una vez, al inicio de los tiempos, triunfó, lesionando el derecho de Dios sobre el hombre; mas ahora será conculcada en el Hijo, en la Madre y en el padre adoptivo, para restituir a los hombres a la Gracia y devolverle a Dios su derecho sobre el hombre. Esto lo hace con su castidad perpetua. ¿No estaba José en el Gólgota? ¿Os parece que no está en el número de los corredentores? En verdad os digo que fue el primero de ellos, y que grande es, por tanto, ante los ojos de Dios. Grande por el sacrificio, la paciencia, la constancia y la fe. ¿Qué fe será mayor que ésta, que creyó sin haber visto los milagros del Mesías? Sea alabado mi padre adoptivo, ejemplo para vosotros de aquello que en vosotros más falta: pureza, fidelidad y perfecto amor. Gloria al magnífico lector del Libro sellado, que fue instruido por la Sabiduría para saber comprender los misterios de la Gracia y que fue elegido para tutelar la Salvación del mundo contra las insidias de todos los enemigos. }
 
\chapter{Los Esposos llegan a Nazaret.}
 
El más azul de los cielos de un apacible febrero se extiende sobre las colinas de Galilea. Las suaves colinas que no he visto nunca en este ciclo de la Virgen niña, y que me son ya tan familiares al ojo como si hubiera nacido entre ellas. 

La calzada principal, refrescada por lluvia reciente, caída quizás la noche anterior, no tiene polvo, mas tampoco barro. Presenta aspecto compacto y limpio, como si fuera una calle de ciudad, y avanza, sinuosa, entre dos hileras de espino albar en flor: una nevada con sabor amargoso y a bosque, interrumpida una y otra vez por las monstruosas aglomeraciones de los cactus, con sus hojas carnosas en forma de paleta, erizadas de pinchos y decoradas con los enormes granates de sus originales frutos, crecidos sin tallo sobre las hojas, las cuales, por su color y forma, evocan siempre en mí profundidades marinas y bosques de corales y medusas, u otros animales de los mares profundos. 

Las hileras de espino sirven como cercas de las propiedades privadas, por lo cual se extienden en todas las direcciones formando un caprichoso trazado geométrico de curvas y de ángulos, de rombos, cuadrados, semicírculos, triángulos con las más inverosímiles formas agudas u obtusas; es un trazado enteramente asperjado de blanco: como una cinta llena de fantasía que hubieran extendido así, por diversión, a lo largo de los campos; sobre ella vuelan, pían, cantan, a centenares, pajaritos de toda especie, sintiendo la alegría del amor y dedicados a rehacer sus nidos. Al otro lado de las hileras de espino están los campos, con los trigos todavía verdes, pero aquí ya más altos que en los campos de Judea, y prados llenos de flores, y en ellos — como contrapunto de las ligeras nubecillas del cielo, que el ocaso tiñe de rosa o de un lila tenue o violeta o de un opalino colorado de azul o de un naranja- coral —, a centenares, las nubes vegetales de los árboles frutales, blancas, rosadas, rojas, en todas las tonalidades del blanco, rosa y rojo. 

Con el suave viento de la tarde, caen revoloteando de los árboles florecidos los primeros pétalos: parecen bandadas de mariposas buscando polen en las flores del campo. Entre árbol y árbol, festones de vid aún desnuda: sólo en la parte alta de los festones, en la parte donde más da el sol, las primeras hojitas se abren, inocentes, extrañadas, palpitantes. 

El Sol se pone, sereno, en el cielo — ¡qué apacible con ese azul suyo que la luz hace aún más claro! — y a lo lejos titilan, reflejándolo, las nieves del Hermón y de otras cumbres lejanas. 

Un carro avanza por la calzada, el carro que lleva a José y a María y a los primos de Ella; el viaje está tocando a su fin. 

María mira con el ojo ansioso de quien quiere conocer, o mejor, reconocer, aquello que ya un día vio, pero no lo recuerda, y sonríe cuando una sombra de recuerdo vuelve y se posa, como una luz, en esta o aquella cosa, en este o aquel punto. Isabel le ayuda a recordar, y también Zacarías y José, señalando esta o aquella cumbre, esta o aquella casa. 

Casas, sí. Porque Nazaret ya aparece extendida sobre la ondulación de su colina. Recibiendo por la izquierda el Sol ya ocultándose, muestra, con pinceladas de rosa, el color blanco de sus casitas, anchas y bajas, culminadas por una terraza. Algunas de ellas, al darles el sol de lleno, parecen, de lo rojas que se han puesto las fachadas, estar al lado de un fuego. Y el sol enciende también el agua de los bajos pozos, que no tienen casi brocal, de donde suben, chirriando, los cubos para la casa o los odres para la huerta. 

Niños y mujeres se acercan al borde de la calzada, queriendo ver el interior del carro, y saludan a José, que es muy conocido en el lugar. Pero luego se muestran titubeantes y tímidos ante las otras tres personas. 

Sin embargo, dentro ya de la pequeña ciudad, no hay titubeos ni temor. Mucha, mucha gente de todas las edades está a la entrada del pueblo bajo un rústico arco hecho con flores y ramas, y nada más que el carro aparece por detrás del recodo de la última casa de campo, que está colocada oblicuamente, se produce un verdadero gorjeo de voces agudas y un agitarse de ramas y flores. Son las mujeres, las chiquillas y los niños de Nazaret que saludan a la novia. Los hombres, más contenidos, están detrás de este seto agitado y gorjeante, y saludan con gravedad. 

María, ahora que la cortina ha sido quitada, dejando al descubierto el carro — lo habían hecho ya antes de llegar al pueblo, porque el sol ya no molestaba, y para permitirle a María el ver bien su tierra natal — aparece en su belleza de flor. Blanca y rubia como un ángel, sonríe con bondad a los niños, que le echan flores y besos, a las jóvenes de su edad, que la llaman por el nombre, a las mujeres casadas, a las madres, a las ancianas, que la bendicen con sus voces cantadoras. Inclina su cabeza ante los hombres, y especialmente ante uno de ellos, que quizás es el rabino o la personalidad principal del pueblo. 

El carro prosigue por la calle principal a paso lento, seguido de la muchedumbre por un buen trecho, muchedumbre para la que esta llegada es un acontecimiento. 

Esa es tu casa, María- dice José señalando con el látigo una casita que está justo en la base de una ondulación de la colina, y que tiene en la parte de atrás un hermoso y amplio huerto, exuberante, que termina en un pequeño olivar. Más allá, la consabida cerca de espino albar y cácteas señala el límite de la propiedad. Las tierras, que fueron de Joaquín, están al otro lado... 

Te ha quedado poco, ¿ves?- dice Zacarías - La enfermedad de tu padre fue larga y económicamente cara. Y caros fueron también los gastos para reparar el daño que hizo Roma. ¿Lo ves? La calle le ha cortado a la casa sus tres principales habitaciones. Se ha quedado más pequeña. Para ampliarla sin gastos excesivos, se cogió una parte del monte que forma una gruta; Joaquín tenía en ese lugar las provisiones y Ana sus telares. Haz con esto lo que creas más oportuno. 

- ¡Que sea poco no importa! Siempre me será suficiente. Me pondré a trabajar... 

No, María — es José quien habla — Yo seré quien trabaje. Tú sólo tejerás y coserás las cosas de la casa. Soy joven y fuerte, y soy tu esposo. No me atormentes viéndote trabajar. 

Haré como tú quieras. 

Sí, en esto yo quiero. Para todas las demás cosas tu deseo es ley, pero en esto no. 

Ya han llegado. El carro se detiene. 

Dos mujeres y dos hombres, respectivamente de unos cuarenta y cincuenta años, están a la puerta, y muchos niños y jovencitos están con ellos. 

Dios te dé paz, María - dice el hombre más anciano. Una de las mujeres se acerca a María, la abraza y la besa. 

Es mi hermano Alfeo, y María, su mujer, y éstos son sus hijos. Han venido expresamente para recibirte y felicitarte y decirte que su casa es tuya, si así lo deseas - dice José. 

Sí, ven, María, si te resulta penoso vivir sola. El campo es bonito en primavera y nuestra casa está en medio de campos floridos. Tú serás su más hermosa flor - dice María de Alfeo. 

Gracias, María. Yo iría con mucho gusto, y alguna vez iré; iré, sin duda, para la boda... Pero, deseo vivamente ver, reconocer mi casa. La dejé siendo muy pequeña y se me ha desdibujado su imagen... Ahora esta imagen la encuentro de nuevo... y me parece como si encontrara de nuevo a mi madre perdida, a mi padre amado, el eco de las palabras de ellos... y el aroma de su último respiro. Siento como si ya no fuera huérfana, porque me abrazan de nuevo estas paredes... Compréndeme, María - Aparece un poco el llanto en la voz de María, y también en sus pestañas. María de Alfeo responde: 

Querida mía, como tú quieras. Quiero que me sientas hermana y amiga y un poco madre incluso, porque soy mucho más mayor que tú. 

La otra mujer, que se ha acercado entretanto, dice: 

María, quiero saludarte. Soy Lía, la amiga de tu madre. Te vi nacer. Este es Alfeo, sobrino de Alfeo y muy amigo de tu madre. Lo que hice por tu madre, si quieres, lo haré por ti. Mira, mi casa es la que está más cerca de la tuya y tus parcelas de terreno son ahora nuestras. Pero, si quieres venir hazlo cuando te apetezca, en cualquier momento. Abrimos un paso en el cercado y así estaremos juntas, sin dejar de estar cada una en su casa. Este es mi marido. 

Os doy las gracias a todos y por todo; por todo el amor que habéis tenido a los míos, y por todo el amor que me tenéis a mí. Que Dios todopoderoso os bendiga por ello. 

Descargan los pesados baúles y los meten en la casa. Entran. Reconozco ahora que es la casita de Nazaret, como será luego, durante la vida de Jesús. 

José toma de la mano — un gesto habitual en él — a María, y entra así. Pero en el umbral de la puerta le dice: 

Ahora, aquí, en el umbral de esta puerta, quiero de ti una promesa: que cualquier cosa que te suceda, o cualquier cosa que necesites, tu único amigo, la única persona en quien pienses para solicitar ayuda, sea yo, y que, bajo ningún motivo, debas sufrir sola ninguna pena. Yo estoy a tu entera disposición, y para mí será una satisfacción el hacerte feliz el camino, y, dado que la felicidad no siempre está en nuestra mano, al menos, hacértelo tranquilo y seguro. 

Te lo prometo, José. 

La siguiente cosa es abrir puertas y ventanas... El último sol entra curioso. 

María se ha quitado el manto y el velo. Menos las flores de mirto, todavía va vestida como en los esponsales. Sale al huerto, que presenta un aspecto exuberante. Mira, sonríe, y, todavía de la mano de José, da un paseo. Se la ve como quien volviera a tomar posesión de un lugar perdido. 

José le muestra el resultado de sus trabajos: 

Mira, aquí he cavado para recoger el agua de la lluvia, porque estas cepas están siempre sedientas. A este olivo le he vuelto a cortar las ramas más viejas para darle vigor; y he plantado estos manzanos, porque dos estaban muertos; y luego, allí he plantado unas higueras. Cuando crezcan resguardarán a la casa del sol excesivo y de las miradas curiosas. La pérgola es la misma que había; lo único que he hecho ha sido cambiar los palos que estaban deteriorados, y también una labor de poda. Espero que dé muchas uvas. Y aquí, mira - y la lleva, orgulloso, hacia el terreno en pendiente que resguarda la casa por detrás y que es límite del huerto por el lado de tramontana - y. aquí he excavado una pequeña gruta, y la he reforzado, y, cuando agarren estas plantas, será casi igual que la que tenías. Falta el manantial... pero, espero hacer llegar aquí desde el manantial un regatillo. 

Pienso trabajar durante las largas tardes de verano cuando venga a verte... 

- ¿Cómo es eso? - dice Alfeo. " ¿No vais a celebrar la boda este verano? 

No. María quiere tejer los paños de lana, que es lo único que le falta a su ajuar. Y a mí eso me satisface. María es tan joven, que el esperar un año o más no es nada. Entretanto se ambienta a la casa... 

- ¡Bueno! Tú siempre has sido un poco distinto de los demás, y lo sigues siendo. No sé quién pudiera no tener prisa en tener por esposa a una flor como María, ¡y tú metes meses por medio!.. 

Alegría muy esperada, alegría más intensamente gustada - responde José con una sonrisa sutil. 

El hermano se encoge de hombros y dice: 

- ¿Y entonces? Según tus planes, ¿cuándo vas a pensar en la boda? 

Cuando María cumpla dieciséis años. Después de la fiesta de los Tabernáculos. ¡Dulces serán las tardes de invierno para los recién casados!.. - Y sigue sonriendo mirando a María: una sonrisa que conlleva un pacto secreto y delicado; de una castidad fraterna consoladora. 

Luego continúa caminando y explicando: 

Ésta es la habitación grande que había en el monte. Si te parece bien, cuando venga, instalaré en ella mi taller. Está unida, pero no forma parte de la casa. Así no molestaré con los ruidos, o creando otros trastornos. No obstante, si no quieres que sea así... 

No, José; así está muy bien. 

Vuelven a entrar en la casa. Encienden las lámparas. 

María está cansada - dice José - Dejémosla tranquila con sus primos. 

Saludos de todos los que se marchan... José se queda todavía unos minutos y habla con Zacarías en voz baja. 

Tu primo te deja a Isabel durante un poco. ¿Contenta? Yo sí, porque te ayudará a... ser una perfecta ama de casa; con ella podrás colocar como quieras tus cosas y tu ajuar, y yo vendré todas las tardes a ayudarte; con ella podrás conseguir lana y todo lo que necesites, y yo me encargaré de los gastos. Acuérdate de que has prometido que recurrirías a mí para todo. Adiós, María. Duerme el primer sueño de señora en esta casa tuya, y que el ángel de Dios te lo haga sereno. Que el Señor sea siempre contigo. 

Adiós, José. Queda tú también bajo las alas del ángel de Dios. 

Gracias, José, por todo. En la medida en que pueda, te pagaré por tu amor, con el mío. José saluda a los primos y sale. 

Y con él cesa la visión. 
 
\chapter{Como conclusión del Pre- Evangelio.}
\emph{6 de Septiembre de 1944.}

Dice Jesús: 
\emph{El ciclo ha terminado. Y con él, tan dulce y delicado como ha sido, tu Jesús te ha mantenido (habla a María Valtorta), sin movimientos bruscos, al margen de la agitación de estos días. Como a niño envuelto en blandos paños de lana y depositado sobre mullidos almohadones, a ti te han envuelto estas beatas visiones, para que no sintieras, con el consiguiente terror, la crueldad de los hombres que se odian en vez de amarse. No serías capaz ya de soportar ciertas cosas, y no quiero que mueras por causa de ello: Yo cuido a mi "portavoz". Está para desaparecer del mundo ya la causa de todas las desesperaciones que han torturado a las víctimas. Por tanto, María, también cesa para ti el tiempo de este tremendo sufrimiento por demasiadas causas tan en contraposición con tu modo de sentir. No terminará tu sufrir: eres víctima; pero, parte de él, ésta, cesa. Después llegará el día en que Yo te diga, como a María de Magdala moribunda: "Descansa. Ahora es tiempo de descanso para ti. Dame tus espinas. Ahora es tiempo de rosas. Descansa y espera. Te bendigo, mujer bendita". Esto es lo que te decía — y era una promesa y tú no la entendiste — cuando llegaba, el tiempo en que habías de ser sumergida, revolcada, en espinas, encadenada, colmada de espinas hasta en los más hondos recovecos de tu ser... Esto es lo que ahora te repito, con una alegría como sólo el Amor puede experimentar — y Yo soy el Amor — cuando puede hacer cesar un dolor en su dilecto amado. Esto es lo que te digo ahora, ahora que ese tiempo de sacrificio cesa. Y Yo, que sé, por el mundo que no sabe, por Italia, por Viareggio, por esta pequeña población, a donde tú me has portado — medita el sentido de estas palabras - Yo te expreso mi agradecimiento, como corresponde a las víctimas por su sacrificio. Cuando te mostré a Cecilia (Santa Cecilia), virgen- esposa, te dije que ella se había echado mis perfumes, y con ellos atrajo a su marido, a su cuñado, a sus domésticos, a sus familiares, a sus amigos. Tú has hecho — no lo sabes, pero te lo digo Yo, Yo que conozco las cosas — el papel de Cecilia en medio de este mundo enloquecido. Te has saturado de Mí, de mi palabra, has llevado mis deseos a las personas, y las mejores han comprendido, y siguiéndote a ti, que eres víctima, muchísimas otras víctimas han surgido. Si tu patria, y los lugares que tú más quieres, no han sido completamente destruidos, ha sido porque muchas hostias han sido sacrificadas a raíz de tu ejemplo y de tu ministerio. Gracias, mujer bendita. Continúa así. Tengo gran necesidad de salvar a la Tierra, de volver a comprar la Tierra; las monedas sois vosotras, las víctimas. La Sabiduría que ha instruido a los santos, y que te instruye a ti con un magisterio directo, te eleve cada vez más en la comprensión de la Ciencia de vida y en practicarla. Levanta tú también tu pequeña tienda ante la casa del Señor. Te digo más aún: hinca las estacas que sostienen tu misma morada en la morada de la Sabiduría y mora en ella sin jamás dejarla. Descansarás así, protegida por el Señor, que te ama, como ave entre ramas florecidas, y Él será tu amparo ante cualquier tipo de intemperie espiritual y estarás en la luz de la gloria de Dios de donde descenderán para ti palabras de paz y verdad. Puedes ir en paz. Te bendigo, mujer bendita". }

Inmediatamente después dice María (la Virgen): 
\emph{A María el regalo de Mamá por su fiesta. Una cadena de regalos. Y si hay alguna espina entre ellos, no te quejes al Señor, que te ha amado como a bien pocos ama. Te dije al principio: "Escribe acerca de mí. De todo lo que sufras recibirás consuelo". ¿Ves como ha sido verdad? Te estaba reservado este regalo para este tiempo de agitación, porque no sólo cuidamos el espíritu, sino que sabemos también cuidar la materia, que no es reina, sino sierva útil al espíritu en el cumplimiento de su misión. Sé agradecida al Altísimo, que, incluso en el sentido afectuosamente humano, es verdaderamente Padre tuyo, que te acuna con éxtasis suaves para ocultarte lo que te asustaría. Ámame cada vez más. Te he conducido conmigo al secreto de mis primeros años. Ahora ya sabes todo acerca de Mamá. Ámame como hija y como hermana en el destino victimal. Y ama a Dios Padre, a Dios Hijo, a Dios Espíritu Santo, con perfección de amor. La bendición del Padre, del Hijo y del Espíritu pasa por mis manos; recibe el perfume de mi materno amor hacia ti, a ti desciende y en ti se deposita. Sé sobrenaturalmente devota. }

\chapter{La Anunciación.}
 
Lo que veo. María, muchacha jovencísima (al máximo quince años a juzgar por su aspecto), está en una pequeña habitación rectangular; verdaderamente, una habitación de jovencita. Contra una de las dos paredes más largas, está el lecho: una cama baja, sin armadura, cubierta por gruesas esteras o tapetes — diríase que éstos están extendidos sobre una tabla o sobre un entramado de cañas porque están muy rígidos y sin pliegues como los de nuestras camas —. Contra la otra pared, un estante con una lámpara de aceite, unos rollos de pergamino y una labor de costura — parece un bordado — cuidadosamente doblada. 

A uno de los lados del estante, hacia la puerta, que da al huerto, abierta ahora, aunque tapada por una cortina que se mueve movida por un ligero vientecillo, en un taburete bajo está sentada la Virgen. Está hilando un lino candidísimo y suave como la seda. Sus manitas, sólo un poco más oscuras que el lino, hacen girar rápidamente el huso. Su carita juvenil, preciosa, está ligeramente inclinada y ligeramente sonriente, como si estuviera acariciando o siguiendo algún dulce pensamiento. 

Hay un gran silencio en la casita y en el huerto. Y mucha paz, tanto en la cara de María como en el espacio que la rodea. Paz y orden. Todo está limpio y ordenado. La habitación, de humildísimo aspecto y mobiliario, casi desnuda como una celda, tiene un aire austero y regio, debido a su gran limpieza y a la cuidadosa colocación de la cobertura del lecho, de los rollos, de la lámpara y del jarroncito de cobre que está cerca de ésta con un haz de ramitas floridas dentro, ramitas de melocotonero o de peral, no lo sé; lo que sí está claro es que son de árboles frutales, de un blanco ligeramente rosado. 

María comienza a cantar en voz baja. Luego alza ligeramente la voz. No llega al pleno canto, pero su voz ya vibra en la habitación, sintiéndose en aquélla una vibración del alma. No entiendo la letra, que sin duda es en hebreo, pero, dado que, de vez en cuando repite "Yeohveh", intuyo que se trata de algún canto sagrado, acaso un salmo. Quizás María recuerda los cantos del Templo. Debe tratarse de un dulce recuerdo. Efectivamente, deja sobre su regazo sus manos, y con ellas el hilo y el huso, y levanta la cabeza para apoyarla en la pared, hacia atrás. Su rostro está encendido de un lindo rubor; los ojos, perdidos tras algún dulce pensamiento, brillantes por un golpe de llanto, que no los rebosa pero sí los agranda. Y, a pesar de todo, loa ojos ríen, sonríen ante ese pensamiento que ven y que los abstrae de lo sensible. Resaltando de su vestido blanco sencillísimo, circundado por las trenzas, que lleva recogidas como corona en torno a la cabeza, el rostro rosado de María parece una linda flor. 

El canto pasa a ser oración: 

- Señor Dios Altísimo, no te demores más en mandar a tu Siervo para traer la paz a la tierra. Suscita el tiempo propicio y la virgen pura y fecunda para la venida de tu Cristo. Padre, Padre santo, concédele a tu sierva ofrecer su vida para esto. Concédeme morir tras haber visto tu Luz y tu Justicia en la Tierra, sabiendo que la Redención se ha cumplido. ¡Oh, Padre Santo, manda a la Tierra el Suspiro de los Profetas! Envía el Redentor a tu sierva. Que cuando cese mi día se me abra tu Casa por haber sido abiertas sus puertas por tu Cristo para todos aquellos que en ti hayan esperado. Ven, ven, Espíritu del Señor. Ven a los fieles tuyos que te esperan. ¡Ven, Príncipe de la Paz!.. 

María se queda así ensimismada... 

La cortina late más fuerte, como si alguien la estuviera aventando con algo o quisiera descorrerla. Y una luz blanca de perla fundida con plata pura hace más claras las paredes tenuemente amarillentas, hace más vivos los colores de las telas, más espiritual el rostro alzado de María. En la luz se prosterna el Arcángel. La cortina no ha sido descorrida ante el misterio que se está verificando; es más, ya no late: pende, rígida, pegada a las jambas, separando, como una pared, el interior del exterior. 

El Arcángel necesariamente debe adquirir un aspecto humano; pero es un aspecto ultra- humano. ¿De qué carne está compuesta esta figura bellísima y fulgurante? ¿Con qué sustancia la ha materializado Dios para hacerla sensible a los sentidos de la Virgen? Sólo Dios puede poseer estas substancias y usarlas de esa manera perfecta. Es un rostro, es un cuerpo, son ojos, boca, cabellos y manos como las nuestras; pero no se trata de nuestra opaca materia: es una luz que ha tomado color de carne, de ojos, de cabellera, de labios, una luz que se mueve y sonríe y mira y habla. 

- ¡Salve, María, llena de Gracia, salve! - La voz es un dulce arpegio como de perlas chocadas contra un metal precioso. 

María se estremece y baja la mirada. Su estremecimiento aumenta cuando ve a la fúlgida criatura arrodillada aproximadamente a un metro de distancia de Ella, con las manos cruzadas sobre el pecho, mirándola con una veneración infinita. 

María, repentinamente, se pone en pie y aprieta su cuerpo contra la pared. Palidece y se ruboriza alternativamente. Su rostro expresa estupor y turbación. Inconscientemente aprieta sus manos contra el pecho escondiéndolas en las anchas mangas. Se recoge sobre sí misma como queriendo esconder lo más posible su cuerpo: un acto de delicado pudor. 

- No. No temas. ¡El Señor está contigo! ¡Bendita tú entre todas las mujeres! 

A pesar de estas palabras, María sigue temiendo. ¿De dónde viene ese ser extraordinario? ¿Es un enviado de Dios, o del Engañador? 

- ¡No temas, María! - insiste el Arcángel - Yo soy Gabriel, el Ángel de Dios. Mi Señor me ha enviado a ti. No temas, porque has hallado gracia ante Dios. Tú concebirás en tu seno y darás a luz un Hijo, y le pondrás por nombre "Jesús". Será grande, será llamado Hijo del Altísimo, y verdaderamente lo será. El Señor Dios le dará el trono de David, su padre, y reinará para siempre en la casa de Jacob, y su Reino no tendrá fin nunca. Comprende, santa Virgen amada del Señor, Hija bendita suya, llamada a ser Madre de su Hijo, comprende qué Hijo vas a engendrar. 

- ¿Cómo puede suceder esto si yo no conozco hombre? ¿Acaso el Señor ya no acoge el ofrecimiento de su sierva? ¿Es que ya no quiere que sea virgen por amor a Él? 

No vas a ser madre por obra de varón, María. Tú eres la eterna Virgen, la Santa de Dios. El Espíritu Santo descenderá sobre ti y la potencia del Altísimo te cubrirá con su sombra. Por eso, Santo se llamará el que nacerá de ti, e Hijo de Dios. Todo lo puede el Señor, Dios nuestro. Isabel, la estéril, en su vejez ha concebido un hijo que será el Profeta de tu Hijo, el que preparará sus caminos. El Señor la ha liberado de su oprobio y su memoria quedará en las gentes unida a tu nombre, como el nombre de su hijo al de tu Hijo Santo, y hasta el final de los siglos las gentes os llamarán bienaventuradas por la gracia del Señor que habéis recibido, y a ti especialmente, María, porque habrán recibido la Gracia por medio de ti. Isabel está ya en su sexto mes, y su peso, paradójicamente, la regocija, y más aún la regocijará cuando conozca el motivo de tu gozo. Para Dios nada es imposible, María, llena de Gracia. ¿Qué debo responderle a mi Señor? No te turbe ningún tipo de pensamiento. Él tutelará tus intereses si te pones en sus manos. ¡El mundo, el Cielo, Dios eterno esperan tu respuesta! 

María, cruzando a su vez sus manos sobre el pecho e inclinándose con gesto reverente dice: 

He aquí la esclava de Dios. Hágase de mí según su palabra. 

El Ángel resplandece de alegría y se pone en actitud adorante, puesto que, sin duda, ve al Espíritu de Dios descender sobre la Virgen, inclinada en gesto de adhesión; luego desaparece sin mover la cortina, dejándola cerrada cubriendo el Misterio santo. 

\chapter{La desobediencia de Eva y la obediencia de María.}
 
Dice Jesús: 
\emph{¿No se lee en el Génesis que Dios hizo al hombre dominador de todo lo que había sobre la tierra, es decir, de todo excepto de Dios y de sus ángeles ministros? ¿No se lee que hizo a la mujer como compañera del hombre en la alegría y en el dominio sobre todos los seres vivos? ¿No se lee que de todo podían comer excepto del árbol de la ciencia del Bien y del Mal? ¿Por qué? ¿Cuál es el sentido que subyace en las palabras "para que domine"; cuál, en el árbol de la ciencia del Bien y del Mal? ¿Os habéis preguntado alguna vez esto, vosotros, que os hacéis tantas preguntas inútiles y que no sabéis preguntarle nunca a vuestra alma acerca de las celestes verdades? Vuestra alma, si estuviera viva, os las manifestaría. Esa alma que, cuando está en gracia, es como una flor entre las manos de vuestro ángel; esa alma que, cuando está en gracia, es como una flor besada por el sol y asperjada por el rocío, besada y asperjada por el Espíritu Santo, que le da calor y la ilumina, que la riega y la adorna de celestes luces. ¡Cuántas verdades os manifestaría vuestra alma, si supierais conversar con ella, si la amarais como a quien os proporciona la semejanza con Dios, que es Espíritu, como espíritu es vuestra alma! ¡Qué gran amiga tendríais, si amarais a vuestra alma en vez de odiarla hasta matarla; qué grande, sublime amiga con quien hablar de cosas celestes; vosotros que tenéis tanta avidez de hablar y os destruís los unos a los otros con amistades que, aun no siendo indignas (alguna vez lo son), sí son casi siempre inútiles, y se os transforman en un bullicio vano o nocivo de palabras y sólo palabras, todas terrenas! ¿No dije Yo: "Quien me ama observará mi palabra y el Padre mío le amará e iremos a él y haremos morada en él"? El alma que está en gracia posee el amor y, poseyéndolo, posee a Dios, o sea, al Padre que la conserva, al Hijo que la instruye, al Espíritu que la ilumina. Posee, por tanto, el Conocimiento, la Ciencia, la Sabiduría. Posee la Luz. Imaginaos, pues, qué conversaciones más sublimes podría establecer con vosotros vuestra alma, que son las conversaciones que han llenado los silencios de las cárceles, los silencios de las celdas, los silencios del yermo, los silencios de las habitaciones de los enfermos santos; las que han confortado a los presos que en la cárcel esperaban el martirio, a los cenobitas, que habían elegido el claustro en pos de la Verdad, a los eremitas, que anhelaban conocer anticipadamente a Dios, a los enfermos, para que soportaran o, mejor dicho, amaran su cruz. Si supierais preguntar a vuestra alma, ella os diría que el significado verdadero, exacto, vasto cuanto la creación, de la palabra "domine" es éste: "Para que el hombre domine todo: sus tres estratos (el inferior, animal; el estrato intermedio, moral; el estrato superior, espiritual), y oriente los tres hacia un único fin: poseer a Dios". Poseerlo mereciéndolo con este férreo dominio que tiene sujetas todas las fuerzas del yo haciéndolas esclavas de esta única finalidad: merecer poseer a Dios. Vuestra alma os diría que Dios había prohibido el conocimiento del Bien y del Mal, porque el Bien lo había donado con generosidad y gratuitamente a sus criaturas, y el Mal no quería que lo conocierais, porque es un fruto dulce al paladar, pero que, una vez que baja con su jugo a la sangre, ocasiona una fiebre que mata y produce ardiente sequedad en la garganta, por lo cual, cuanto más se bebe de su jugo traidor, más sed de él se tiene. Vuestra objeción será: "¿Y por qué lo ha puesto?". ¿Por qué! El Mal es una fuerza que ha nacido sola, como ciertos males monstruosos en el más sano de los cuerpos. Lucifer era un ángel, el más hermoso de los ángeles. Espíritu perfecto. Sólo Dios era superior a él. Pues bien, con todo, en su ser luminoso nació un vapor de soberbia, y Lucifer no lo dispersó, sino que, por el contrario, lo condensó dándole vida en su interior. De esta incubación nació el Mal. Este ya existía antes del hombre. Dios había arrojado fuera del Paraíso al Incubador maldito del Mal, al que ensuciaba el Paraíso. Pero ha seguido siendo y es el eterno Incubador del Mal, y, no pudiendo seguir ensuciando el Paraíso, ha ensuciado la Tierra. Ese metafórico árbol pone en evidencia esta verdad. Dios había dicho al hombre y a la mujer: "Conoced todas las leyes y los misterios de la creación. Pero no pretendáis usurparme el derecho de ser el Creador del hombre. Para propagar la estirpe humana bastará el amor mío que circulará por vosotros, y, sin libídine sensual, sólo por latido de caridad, dará vida a los nuevos hombres como Adán de la estirpe. Todo os lo doy; sólo me reservo este misterio de la formación del hombre". Satanás quiso quitarle al hombre esta virginidad intelectual y, con su lengua serpentina, hechizó y halagó miembros y ojos de Eva, suscitando en ellos reflejos y sutilezas que antes no tenían porque no estaban intoxicados de Malicia. Ella "vio", y, viendo, quiso probar. Había sido despertada la carne. ¡Ah, si hubiera llamado a Dios¡ Si hubiera corrido a decirle: "¡Padre, estoy enferma; la serpiente me ha halagado y me siento turbada!". El Padre la habría purificado, la habría curado con su aliento, pues lo mismo que le había infundido la vida podía infundirle de nuevo la inocencia, quitándole el recuerdo del tóxico serpentino, es más, introduciendo en ella una repugnancia hacia la Serpiente (como les sucede a los que han sufrido una enfermedad, que, una vez curados, sienten hacia ella una instintiva repugnancia). Pero no, Eva no va al Padre, Eva vuelve donde la Serpiente. Esa sensación le es dulce. "Viendo que el fruto del árbol se podía comer y que era bonito y de aspecto agradable, lo cogió y comió de él". Y "comprendió". Ya la malicia había penetrado y le mordía las entrañas. Vio con ojos nuevos y oyó con oídos nuevos los usos y la voz de las bestias; y los deseó febrilmente. Inició sola el pecado. Lo consumó con su compañero. Por eso sobre la mujer pesa una condena mayor. Por ella el hombre se hizo rebelde a Dios, y por ella conoció la lujuria y la muerte. Por ella perdió el dominio sobre sus tres reinos: el del espíritu, porque permitió que el espíritu desobedeciera a Dios; el de lo moral, porque permitió que las pasiones le sometieran a su señorío; el de la carne, porque la rebajó a las leyes instintivas de las bestias. "La Serpiente me ha seducido" dice Eva. "La mujer me ha ofrecido el fruto y yo he comido de él" dice Adán. Y el triple, desenfrenado apetito, desde entonces, tiene entre sus garras los tres reinos del hombre. Sólo la Gracia logra aflojar la presa de este monstruo despiadado; y, si vive, si está vivísima, si la voluntad del hijo fiel la mantiene cada vez más viva, llega incluso a estrangular al monstruo. Ya no habrá nada que temer: ni a los tiranos internos (o sea, la carne y las pasiones), ni a los tiranos externos (el mundo y los que en el mundo tienen poder), ni a las persecuciones, ni a la muerte. Es como dice el apóstol Pablo: "Nada de esto yo temo, y no considero ya mía mi vida, con tal de cumplir mi misión y llevar a cabo el ministerio recibido del Señor Jesús para dar testimonio del Evangelio de la Gracia de Dios"". }

Dice María (la Virgen): 
\emph{Gozoso — pues, efectivamente, cuando comprendí la misión a que Dios me llamaba, mi corazón se llenó de gozo — mi corazón se abrió como una azucena en capullo y vertió la sangre que habría de ser terreno para la Semilla del Señor. Alegría de ser madre. Me había consagrado a Dios desde mi más tierna edad, porque la luz del Altísimo me había iluminado acerca de la causa del mal del mundo; yo deseé, por lo que de mí dependía, borrar de mí la huella de Satanás. No sabía que no tenía mancha. No podía pensarlo. El solo hecho de pensarlo habría sido presunción y soberbia porque, habiendo nacido de padre y madre humanos, no me era lícito pensar que justamente yo era la Elegida para ser la Sin Mancha. El Espíritu de Dios me había instruido acerca del dolor del Padre ante la corrupción de Eva, que había aceptado degradarse — siendo una criatura de gracia — a un nivel de criatura inferior. Yo tenía la intención de suavizar ese dolor, poniendo de nuevo mi carne en la situación de pureza angélica, conservándome intacta de pensamientos, deseos y contactos humanos. Sólo para Él sería mi latido de amor; sólo para El, mi ser. No había en mí sed camal, pero sí sentía el sacrificio de no ser madre. La maternidad, exenta de lo que ahora la humilla, le había sido concedida por el Padre creador también a Eva. ¡Dulce y pura maternidad sin el peso del sentido! ¡Yo la experimenté! ¡Cuán grande la pérdida de Eva, renunciando a esta riqueza! Mayor que la pérdida de la inmortalidad. No, no creáis que es una exageración. Mi Jesús, y con Él yo, su Madre, conocimos la languidez de la muerte. Yo, el dulce languidecer de quien, cansado, se duerme; Él, ese languidecer atroz de quien muere por haber sido condenado. A nosotros, pues, también nos vino la muerte. Sin embargo, la maternidad exenta de cualquier tipo de violación me vino solamente a mí, la nueva Eva, para que yo pudiera manifestarle al mundo cuan dulce era el destino de la mujer, llamada a ser madre sin el dolor de la carne. El deseo de esta pura maternidad, siendo, como es, la gloria de la mujer, podía estar, y estaba, en la Virgen toda de Dios. Añadid a vuestra consideración el honor en que era tenida la mujer madre en el pueblo israelita, y comprenderéis mejor la naturaleza del sacrificio cumplido al consagrarme a esta privación. Ahora a su sierva el eterno Bueno le ofrecía este don, sin privarme del candor de que yo me había vestido para ser flor en su trono. Por ello exultaba, con el doble gozo de ser madre de un hombre y de ser Madre de Dios. Alegría porque a través de mí se restablecía la paz entre el Cielo y la Tierra. ¡Oh... haber deseado esta paz por amor a Dios y por amor al prójimo, y saber que por medio de mí, pobre esclava del Poderoso, aquélla venía al mundo! ¡Decir: "Hombres, no lloréis más. Yo traigo conmigo el secreto que os hará felices. No os lo puedo manifestar, porque está sellado en mí, en mi corazón, de la misma forma que el Hijo dentro del intacto seno. Ya os lo traigo, ya cada hora que pasa está más cercano el momento en que le veréis y sabréis su Nombre santo"! Alegría de haber hecho feliz a Dios: alegría del creyente que ve feliz a su Dios. ¡Oh... haber quitado del corazón de Dios la amargura de la desobediencia de Eva, de la soberbia de Eva, de su incredulidad! Mi Jesús ha explicado con qué culpa se manchó la Pareja primera. Yo he anulado esa culpa recorriendo en sentido inverso, para ascender, las etapas de su descenso. El principio de la culpa estuvo en la desobediencia: "No comáis y no toquéis de ese árbol", había dicho Dios. Pero el hombre y la mujer, los reyes de la creación, que podían tocar todo y comer todo excepto aquello — porque Dios quería hacerlos sólo inferiores a los ángeles — no tomaron en consideración ese veto. El árbol: el medio para probar la obediencia de los hijos. ¿Qué es la obediencia al mandato divino? Es un bien porque Dios no ordena sino el bien. ¿Qué es la desobediencia? Es un mal porque pone al corazón en las disposiciones de rebelión sobre las cuales Satanás puede obrar. Eva va al árbol, a ese árbol del que vendría: alejándose, su bien; acercándose, su mal. La arrastra a él la curiosidad ingenua de ver qué es lo que podía tener en sí de especial; la arrastra la imprudencia, que hace que le parezca inútil el mandato divino, dado que ella es fuerte y pura, reina del Edén, donde todo le presta obediencia, donde nada podrá causarle mal alguno. Su presunción la pierde. La presunción es ya levadura de soberbia. En el árbol encuentra al Seductor, el cual, a su inexperiencia, a su tan hermosa y virgen inexperiencia, a esa inexperiencia que no supo tutelar, le canta la canción de la mentira: "¿Tú crees que aquí hay mal? No. Dios te lo ha dicho porque quiere teneros bajo la esclavitud de su poder. ¿Creéis que sois reyes? No tenéis ni siquiera la libertad de las fieras. Ellas tienen concedido el amarse con amor verdadero, vosotros no. A las fieras se les ha concedido el ser creadoras como Dios. Ellas engendrarán hijos y verán a su gusto crecer la familia, vosotros no. A vosotros os ha sido negado este contento. ¿En razón de qué, pues, que seáis hombre y mujer, para tener que vivir de ese modo? Sed dioses. ¿No sabéis qué alegría supone el ser dos en una sola carne creadora de una tercera, de muchas otras terceras! No creáis en las promesas de Dios acerca del gozo de una descendencia viendo a vuestros hijos crearse nuevas familias, dejando por ellas padre y madre. Os ha dado un simulacro de vida. La verdadera vida está en conocer las leyes de la vida. Entonces seréis como dioses y podréis decirle a Dios: 'Somos tus iguales'". Y la seducción continuó, porque no hubo voluntad de interrumpirla, sino, más bien, de continuarla, y de conocer aquello que no le pertenecía al hombre. He aquí pues que el árbol prohibido vino a ser, para la raza, realmente mortal, porque de sus ramos pendía el fruto del amargo saber que venía de Satanás; y la mujer vino a ser hembra, y, con la levadura del conocimiento satánico en el corazón, fue a Adán a corromperlo. Humillada así la carne, corrompida la parte moral, degradado el espíritu, conocieron el dolor y la muerte: del espíritu privado de la Gracia; de la carne privada de la inmortalidad. Y la herida de Eva engendró el sufrimiento, que no se calmará hasta la extinción de la última pareja de la tierra. Yo recorrí en sentido inverso el camino de los dos pecadores. Obedecí. Obedecí en todos los modos. Dios me había pedido ser virgen. Obedecí. Habiendo amado la virginidad, que me hacía pura como la primera de las mujeres antes de conocer a Satanás, Dios me pidió ser esposa. Obedecí, llevando al matrimonio a la pureza que tuvo, a ese grado de pureza que Dios tenía en su pensamiento cuando creó a los dos Primeros. Convencida de mi destino de soledad en el matrimonio y de desprecio del prójimo por mi esterilidad santa, ahora Dios me pedía ser Madre. Obedecí. Creí que ello era posible y que esa palabra venía de Dios, porque la paz iba entrando en mí al oírla. No pensé: "Lo he merecido". No me dije a mí misma: "Ahora el mundo me admirará, porque soy semejante a Dios dando ser a la carne de Dios". No. Me anonadé en la humildad. La alegría brotó de mi corazón como un tallo de rosa florecida. Pero enseguida se adornó de punzantes espinas y quedó abrazada por la maraña del dolor, como esas ramas envueltas en campanillas de enredadera. El dolor del dolor de mi esposo: ésta era la angustia dentro de mi gozo. El dolor del dolor de mi Hijo: éstas eran las espinas de mi gozo. Eva quiso el disfrute, el triunfo, la libertad: yo acepté el dolor, el anonadamiento, la esclavitud. Renuncié a mi vida tranquila, a la estima de mi esposo, a la propia libertad. No me quedé con nada. Me hice la Esclava de Dios en la carne, en la parte moral, en el espíritu, confíándome a Él, no sólo respecto a la concepción virginal, sino también a la defensa de mi honor, a la consolación de mi esposo, al medio con que conducirlo a él también a la sublimación del matrimonio, de manera que los dos fuéramos quienes devolvieran al hombre y a la mujer la dignidad perdida. Abracé la voluntad del Señor por mí, por mi esposo, por mi Hijo. Dije "sí" por los tres, segura como estaba de que Dios no faltaría a su promesa de socorrerme en mi dolor de esposa que se ve juzgada culpable, en mi dolor de madre que ve que engendra para entregar a su Hijo al dolor. "Sí" dije. Sí, y basta. Ese "sí" ha anulado el "no" que Eva opuso al mandato divino. "Sí, Señor, como Tú quieras. Conoceré lo que Tú quieras. Viviré como Tú quieras. Estaré gozosa si Tú lo quieres. Sufriré por lo que Tú quieras. Sí, siempre sí, mi Señor, desde el momento en que tu rayo me hizo Madre hasta el momento en que me llamaste a ti. Sí, siempre sí. Todas las voces de la carne, todas las pasiones de lo moral, bajo el peso de este sí mío perpetuo. Y encima, como encima de un pedestal de diamante, mi espíritu, al cual le faltan las alas para volar a ti, pero es señor de todo el yo, domado y siervo tuyo, siervo en la alegría, siervo en el dolor. ¡Sonríe, oh Dios! ¡Alégrate! La culpa ha sido vencida, cancelada, destruida; yace bajo mi talón, ha sido lavada en mi llanto, destruida por mi obediencia. De mi seno nacerá el Árbol nuevo que dará el Fruto que conocerá todo el Mal por haberlo padecido en sí y dará todo el Bien. A éste sí podrán acercarse los hombres, y yo me sentiré feliz de que cojan de él, aunque no piensen que de mí nace. Con tal de que el hombre se salve y Dios sea amado, hágase de su esclava lo mismo que de la base de terreno en que un árbol crece: escalón para subir". María, hay que saber ser siempre escalón para que los demás suban a Dios. Si nos pisan, no importa, con tal de que logren ir a la Cruz. Es el nuevo árbol que posee el fruto del conocimiento del Bien y del Mal, porque le dice al hombre lo que está mal y lo que está bien, para que sepa elegir y vivir; y sabe, al mismo tiempo, hacer de sí elixir para curar a los que se han intoxicado con el mal que quisieron gustar. Nuestro corazón bajo los pies de los hombres, con tal de que el número de los redimidos crezca y que la Sangre de mi Jesús no sea derramada sin fruto. Este es el destino de las esclavas de Dios. Mas luego mereceremos recibir en nuestro seno la Hostia santa, y, a los pies de la Cruz, embebida en su Sangre y en nuestro llanto, decir: "He aquí, oh Padre, la Hostia inmaculada que te ofrecemos para salud del mundo. Míranos, oh Padre, fundidas con Ella, y por sus méritos infinitos danos tu bendición". Y yo te doy una caricia. Descansa, hija (María Valtorta). El Señor está contigo. }

Dice Jesús: 
\emph{Las palabras de mi Madre deberían disolver cualquier vacilación de pensamiento, incluso en los más atrapados por las fórmulas. Había dicho: "metafórico árbol"; ahora diré: "simbólico árbol". Quizás así entenderéis mejor. Su símbolo es claro: de cómo los dos hijos de Dios actuasen respecto a él, se comprendería la medida de su tendencia al Bien y al Mal. Cual agua regia que prueba el oro, cual balanza del orfebre que pesa los quilates del oro, ese árbol; que vino a ser una "misión" a causa del mandato divino respecto a él, dio la medida de la pureza del metal de Adán y de Eva. Llega ya a mis oídos vuestra objeción: "¿No fue excesiva la condena y pueril el medio que condujo a ella?". No lo fue. Una desobediencia actualmente en vosotros, que sois sus herederos, es menos grave de lo que lo fue en ellos. Vosotros estáis redimidos por Mí, pero el veneno de Satanás, como ciertos morbos que no desaparecen nunca totalmente de la sangre, está siempre pronto para reactivarse. Ellos, los dos progenitores, eran posesores de la Gracia sin haber tenido nunca el más mínimo contacto con la Desgracia. Por tanto, eran más fuertes, estaban más respaldados por esa Gracia que generaba inocencia y amor. Infinito era el don que Dios les había dado; mucho más grave, por tanto, su caída poseyendo ese don. También el fruto ofrecido, y comido, era simbólico. Era el fruto de una experiencia voluntariamente llevada a cabo por instigación satánica contra el imperativo de Dios. Yo no les había prohibido a los hombres el amor. Quería únicamente que se amaran sin malicia; de la misma forma que Yo los amaba con mi santidad, ellos habrían de amarse en santidad de afectos, de afectos limpios de toda libídine. No se debe olvidar que la Gracia es foco de luz, y, que quien la posee conoce aquello que es útil y bueno conocer. La Llena de Gracia conoció todo, porque la Sabiduría la instruía (la Sabiduría, que es Gracia), y supo guiarse a sí misma santamente. Eva conocía, por tanto, aquello que le era bueno conocer; no más de eso. Porque es inútil conocer lo que no es bueno. No tuvo fe en las palabras de Dios y no fue fiel a su promesa de obediencia. Prestó fe a Satanás, infringió la promesa, quiso conocer lo no bueno, lo amó sin remordimiento, transformó en cosa corrompida, envilecida, ese amor que Yo había otorgado tan santo. Ángel caído, se revolcó en barro y paja, mientras que podía haber corrido dichosa entre las flores del Paraíso Terrenal y ver florecer a su alrededor la prole, de la misma forma que un árbol se cubre de flores sin combar su copa y meterla en el pantano. No seáis como esos niños estúpidos de que hablo en el Evangelio, los cuales oían cantar y se tapaban los oídos, oían tocar y no bailaban, oían llorar y querían reír. No seáis mezquinos ni negadores. Aceptad la Luz, aceptadla sin malicia, sin testarudez, sin ironía o incredulidad. Y ya basta sobre esto. Para que entendáis cuánto debéis sentiros agradecidos a Aquel qué murió para levantaros y orientaros de nuevo al Cielo y para vencer la concupiscencia de Satanás, he querido hablaros, en este tiempo de preparación a la Pascua, de este primer eslabón de la cadena con que el Verbo del Padre, el Cordero Divino, fue llevado a la muerte, al matadero. Os he querido hablar de ello porque al presente el noventa por ciento de vosotros está, como Eva, intoxicado por el hálito y por la palabra de Lucifer, y no vivís para amaros sino para saciaros de sensualidad, no vivís para el Cielo sino para el barro; ya no sois criaturas dotadas de alma y razón, sino perros sin alma y sin razón. Habéis matado el alma, habéis depravado la razón. En verdad os digo que las bestias, en sus amores, son más honestas que vosotros. }

\tableofcontents
\end{document}
