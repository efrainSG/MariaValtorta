% !TEX TS-program = pdflatex
% !TEX encoding = UTF-8 Unicode

% This is a simple template for a LaTeX document using the "article" class.
% See "book", "report", "letter" for other types of document.

\documentclass[11pt]{article} % use larger type; default would be 10pt

\usepackage[utf8]{inputenc} % set input encoding (not needed with XeLaTeX)

%%% Examples of Article customizations
% These packages are optional, depending whether you want the features they provide.
% See the LaTeX Companion or other references for full information.

%%% PAGE DIMENSIONS
\usepackage{geometry} % to change the page dimensions
\geometry{a4paper} % or letterpaper (US) or a5paper or....
% \geometry{margin=2in} % for example, change the margins to 2 inches all round
% \geometry{landscape} % set up the page for landscape
%   read geometry.pdf for detailed page layout information

\usepackage{graphicx} % support the \includegraphics command and options

% \usepackage[parfill]{parskip} % Activate to begin paragraphs with an empty line rather than an indent

%%% PACKAGES
\usepackage{booktabs} % for much better looking tables
\usepackage{array} % for better arrays (eg matrices) in maths
\usepackage{paralist} % very flexible & customisable lists (eg. enumerate/itemize, etc.)
\usepackage{verbatim} % adds environment for commenting out blocks of text & for better verbatim
\usepackage{subfig} % make it possible to include more than one captioned figure/table in a single float
% These packages are all incorporated in the memoir class to one degree or another...

%%% HEADERS & FOOTERS
\usepackage{fancyhdr} % This should be set AFTER setting up the page geometry
\pagestyle{fancy} % options: empty , plain , fancy
\renewcommand{\headrulewidth}{0pt} % customise the layout...
\lhead{}\chead{}\rhead{}
\lfoot{}\cfoot{\thepage}\rfoot{}

%%% SECTION TITLE APPEARANCE
\usepackage{sectsty}
\allsectionsfont{\sffamily\mdseries\upshape} % (See the fntguide.pdf for font help)
% (This matches ConTeXt defaults)

%%% ToC (table of contents) APPEARANCE
\usepackage[nottoc,notlof,notlot]{tocbibind} % Put the bibliography in the ToC
\usepackage[titles,subfigure]{tocloft} % Alter the style of the Table of Contents
\renewcommand{\cftsecfont}{\rmfamily\mdseries\upshape}
\renewcommand{\cftsecpagefont}{\rmfamily\mdseries\upshape} % No bold!

%%% END Article customizations

%%% The "real" document content comes below...

\title{Brief Article}
\author{The Author}
%\date{} % Activate to display a given date or no date (if empty),
         % otherwise the current date is printed 

\begin{document}
\maketitle

\section{Energía}
\begin{enumerate}
\item Una muchacha que pesa 80 lb está sentada en un columpio  cuyo peso es insignificante. Si se imparte una velocidad inicial de 20 ft/s. ¿A qué altura se elevará?\\
\emph{Tenemos que igualar la energía cinética del niño con la energía potencial, de esta manera buscaremos la altura del mismo, tenemos que:}\\
\begin{tabular}{lcc}
Datos & Fórmulas & Desarrollo\\
\hline\\
$m = 80 lb$ & $E_c = E_p$ & $\frac{1}{2}(20)^2 = (32.18)h$\\
$v = 20 ft/s$ & $\frac{1}{2}mv^2 = mgh$ & $\frac{400}{2} = (32.18)h$\\
$g = 9.81 m/s^2 = 32.18ft/s^2$ & $\frac{1}{2}v^2 = gh$ & $\frac{400}{2(32.18)} = h$\\
& & $h = 6.21 ft$
\end{tabular}

\emph{Por tanto, la altura que se elevará será de 6.21 ft. Simplemente se realizo un balance de energía para lograr encontrar la ubicación final.}
\end{enumerate}
\section{Potencia}
\begin{enumerate}
\item Una masa de 40kg se eleva hasta 20m en 3s. ¿Qué potencia promedio se ha utilizado?\\
\begin{tabular}{lcc}
Datos & Fórmulas & Desarrollo\\
\hline\\
$m = 40 kg$ & $P_{med} = \frac{mgh}{t} $ & $P_{med} = \frac{(40)(9.81)(20)}{3}$\\
$h = 20 m$ & & $P_{med} = \frac{7848}{3}$\\
$t = 3 s$ & & \\
$g = 9.81 m/s^2$ & & $P_{med} = 2616 $
\end{tabular}

\item Un motor de 90kW eleva una carga de 1200kg. ¿Cuál es la velocidad promedio durante el ascenso?\\
\begin{tabular}{lcc}
Datos & Fórmulas & Desarrollo\\
\hline\\
$P = 90kW$ & $P = mav$ & $90000 = (1200)(9.81)v$ \\
$m = 1200 kg$ & $v = \frac{P}{ma}$ & $v = \frac{90000}{(1200)(9.81)}$ \\
$a = 9.81m/s^2$ & & $v = \frac{90000}{11772}$ \\
& & $v = 7.64 m/s$ \\
\end{tabular}

\item Un estudiante de 800N sube corriendo una escalera y asciende 6m en 8s. ¿Cuál es la potencia promedio que ha desarrollado?
\end{enumerate}
\section{Leyes de Keppler}
\begin{enumerate}
\item
\end{enumerate}
\section{Densidad}
\begin{enumerate}
\item
\end{enumerate}
\section{Presión de fluidos}
\begin{enumerate}
\item
\end{enumerate}
\section{Prensa hidráulica}
\begin{enumerate}
\item
\end{enumerate}
\section{Principio de Arquímides}
\begin{enumerate}
\item
\end{enumerate}
\end{document}
