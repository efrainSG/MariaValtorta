\documentclass[12pt,spanish,lettersize]{article}

\title{Gu\'ia para ex\'amenes de Programaci\'on en .NET}
\author{L.C.C. Efra\'in Serna Gracia}
\date{\today}

\begin{document}
\maketitle
\section{Conceptos}
\begin{description}
\item [P.O.O.] Metodolog\'ia o forma para programar sistemas, trabajando los elementos como objetos concretos o abstractos dentro del mundo real.
\item [Herencia] Permite transferir las caracter\'isticas de un objeto hacia sus descendientes, lo cual permite ahorrar tiempo de programaci\'on.
\item [Polimorfismo] Es la caracter\'istica que permite a un objeto descendiente definir su propia forma de trabajo en una funci\'on dada y que previamente est\'a definida en su clase padre, por lo cual, puede realizar esta funci\'on de dos formas distintas seg\'un se indique.
\item [Encapsulamiento]. Consiste en ocultar la forma de trabajo y definici\'on de una clase hacia el exterior, lo cual permite enfocarse en \emph{Qu\'e hace} y no en \emph{C\'omo lo hace}
\item [Sobrecarga]. Permite definir el mismo m\'etodo con distintos argumentos y/o resultado.
\item [Clase] Define el conjunto de caracter\'isticas y funciones que posee y puede realizar un conjunto de objetos.
\item [Objeto] Instancia derivada de una clase y que se diferenc\'ia de otras instancias mediante los valores que tiene en sus atributos.
\item[Interfaz]. Un contrato que define las propiedades, m\'etodos y eventos que ha de implementar una clase. Contribuye a definir \emph{herencia m\'ultiple}.
\item [Propiedad] Define las caracter\'isticas que pueden tener un objeto.
\item [M\'etodo]. Son las acciones o funciones que puede realizar un objeto.
\item [Evento]. Son mecanismos que permiten al objeto responder a est\'imulos externos.
\item[Alcance p\'ublico] Indica que la propiedad o m\'etodo pertenecientes a una clase pueden ser invocados desde cualquier lugar externo a la propia clase.
\item[Alcance privado] Indica que la propiedad o m\'etodo pertenecientes a una clase no pueden ser invocados desde el exterior.
\item[Alcance protegido]  Indica que la propiedad o m\'etodo pertenecientes a una clase solo pueden ser invocados desde cualquier clase descendiente a \'esta.
\item [NET Framework]. Es un marco de trabajo que sustenta, mediante un conjunto de librer\'ias, el desarrollo estandarizado de software medisnte herramientas de Microsoft
\end{description}
\section{C\'odigos}
\begin{verbatim}
using System; // Espacio de nombres cuyas clases y m\'etodos
              // se van a utilizar

// Espacio de nombres definido para las clases que se han de
// programar.
namespace Guia{

  // Definición de una clase pública 
  public class ProgramacionNET{

    // Definición de variables privadas
    private int version;
    private string titulo;

    // Definición de una propiedad pública
    public int Version{
      get; set;
    }

    // Definición de una propiedad pública
    public string Titulo{
      get{return titulo + "...";}
      set{titulo = value; }
    }

    // Método constructor
    public ProgramacionNET(){
      version = 0;
      titulo = "Programacion .NET";
    }

    // Método constructor sobrecargado
    public ProgramacionNET(int v, string t){
      version = v;
      titulo = t;
    }

    // Método que no devuelve resultado
    public void cambiarTitulo(t){
      titulo = t + "::..";
    }

    // Método que devuelve un resultado
    public int cambiar version(int i){
      version += i;
      return 0;
    }
  }
}
\end{verbatim}
\end{document}