\documentclass[12pt,spanish,lettersize,twocolumn]{article}
\usepackage[latin1]{inputenc}
\usepackage[spanish]{babel}
\usepackage[dvips]{graphicx}
\usepackage[usenames,dvipsnames]{xcolor}
\usepackage{mathrsfs}
\usepackage{amssymb}
\usepackage{amsmath}
\usepackage[makeroom]{cancel}
\usepackage{vmargin}
\setmargins{2.5cm}
{1.5cm}
{16.5cm}
{23.42cm}
{10pt}
{1cm}
{0pt}
{2cm}
\title{\color{Maroon}Sistemas Din\'amicos}
\author{Efra\'in Serna Gracia}
\date{\color{gray}\today}
\begin{document}
\maketitle
Sea
\begin{equation}\label{Axt+gt}
X'(t)=Ax(t)+g(t)
\end{equation}
Entonces $X(t)=X_c(t)+X_p(t)$ es soluci\'on general, con $X_p(t)$ Soluci\'on particular y $X_c(t)$ Soluci\'on complementaria.\\
Y para $X_p(t)$
\begin{equation}
X'(t)=Ax(t)
\end{equation}
\begin{equation}
\left\{v^{(1)}e^{r_1t},v^{(2)}e^{r_2t},\dots,v^{(n)}e^{r_nt}\right\}
\end{equation}
Es conjunto de soluciones.\\
Sea $\Psi(t)$ definida como la siguiente matr\'iz
\begin{equation}
\left[
\begin{array}{cccc}
v_{11}e^{r_1t}&v_{21}e^{r_2t}&\dots&v_{n1}e^{r_nt}\\
v_{12}e^{r_1t}&v_{22}e^{r_2t}&\dots&v_{n2}e^{r_nt}\\
\vdots&\vdots & &\vdots \\
v_{1n}e^{r_1t}&v_{2n}e^{r_2t}&\dots&v_{nn}e^{r_nt}
\end{array}
\right]
\end{equation}
Al $det\{\Psi(t)\}$ se le denomina \emph{Wronskiano}.\\
$\Psi(t)$ debe cumplir las siguientes 2 propiedades:
\begin{itemize}
\item $\Psi$ es no singular\\
\item $\Psi$ satisface el sistema homogeneo
\end{itemize}
Entonces a la siguiente ecuaci\'on se le denomina \emph{Combinaci\'on lineal del conjunto soluci\'on}\\
\begin{equation}
x(t)=C_1v_1e^{r_1t}+C_2v_2e^{r_2t}+\dots+C_nv_ne^{r_nt}
\end{equation}
Sea
\begin{equation}
\mathbb{C}=[C_1, C_2, \dots, C_n]^T \Rightarrow \Psi(t)\mathbb{C}
\end{equation}
Usando \emph{Variaci\'on de par\'ametros}
\begin{equation}
X_p(t)=\Psi(t)\mathbb{U}(t)
\end{equation}
Donde
\begin{eqnarray}
\mathbb{U}(t)=\left[
\begin{array}{c}
u_1(t)\\
u_2(t)\\
\vdots\\
u_n(t)
\end{array}
\right]\Rightarrow\\
X'_p(t)=\Psi'(t)U(t)+\Psi(t)U'(t)\Rightarrow
\end{eqnarray}
Sustituyendo en (\ref{Axt+gt}) tenemos
\begin{equation}
\Psi'(t)U(t)+\Psi(t)U'(t)=A\Psi(t)\mathbb{U}(t)+g(t)
\end{equation}
Aplicando la segunda propiedad\\
\begin{equation}
\nonumber \Psi'(t)=A\Psi(t)
\end{equation}
\begin{eqnarray}
\nonumber \Psi'(t)U(t)=A\Psi(t)U(t)\Rightarrow\\
\nonumber \cancel{\Psi'(t)U(t)}+\Psi(t)U'(t)=\cancel{A\Psi(t)\mathbb{U}(t)}+g(t)\Rightarrow\\
\Psi(t)U(t)=g(t)\Rightarrow\\
\nonumber U'(t)=\Psi^{-1}(t)=g(t)\Rightarrow\\
U(t)=\int\limits{}{t}\Psi'(\tau)g(\tau)d\tau\Rightarrow
\end{eqnarray}
\begin{equation}
X_p(t)=\Psi(t)\int\limits_{}{t}\Psi^{-1}(\tau)g(\tau)d\tau
\end{equation}
\section{Ejercicio}
Hallar la soluci\'on al siguiente sistema de ecuaciones diferenciales:\\
\begin{eqnarray}
\left.
\begin{array}{l}
X'_1(t)=2x_1(t)-x_2(t)+e^t\\
X'_2(t)=3x_1(t)-2x_2(t)+t
\end{array}
\right\rbrace
\end{eqnarray}
Para encontrar la soluci\'on...\\
\begin{equation*}
\mathbb{X}(t)=\left[
\begin{array}{c}
x_1(t)\\
x_2(t)
\end{array}
\right]; A=\left[
\begin{array}{rr}
2 & -1\\
3 & -2
\end{array}
\right]; g=\left[
\begin{array}{c}
e^t\\
t
\end{array}
\right]
\end{equation*}
Ahora\\
\begin{eqnarray}
\left|A-rI\right|=\left|\begin{array}{cc}
2-r & -1\\
3 & -2-r
\end{array}
\right|\Rightarrow\\
\nonumber |A-rI|=(r^2-4)+3\Rightarrow\\
\nonumber |A-rI|=r^2-1=(r-1)(r+1)\Rightarrow\\
r_1=1;r_2=-1
\end{eqnarray}
Para $r_1$\\
\begin{equation}
X^{(1)}=\left[\begin{array}{c}
1\\
1
\end{array}\right]e^t
\end{equation}
Para $r_2$\\
\begin{eqnarray}
(A-r_2I)\left(
\begin{array}{c}
v_1\\
v_2
\end{array}
\right)=\left(
\begin{array}{c}
0\\
0
\end{array}
\right)\Rightarrow\\
\nonumber(A+r_1I)\left(
\begin{array}{c}
v_1\\
v_2
\end{array}
\right)=\left(
\begin{array}{c}
0\\
0
\end{array}
\right)\Rightarrow
\end{eqnarray}
\begin{equation*}
\left(
\begin{array}{rr}
3 & -1\\
3 & -1\\
\end{array}
\right)=
\left(
\begin{array}{c}
0\\
0\\
\end{array}
\right)\Rightarrow\\
\end{equation*}
\begin{equation*}
\left(
\begin{array}{rr}
3 & -1\\
0 & 0\\
\end{array}
\right)
\left(
\begin{array}{c}
v_1\\
v_2\\
\end{array}
\right)=
\left(
\begin{array}{c}
0\\
0\\
\end{array}
\right)\Rightarrow
\end{equation*}
\begin{equation}
3v_1=v_2\Rightarrow
\mathbb{X}^{(2)}=
\left[\begin{array}{c}
1\\
3
\end{array}
\right]e^t
\end{equation}
Entonces, el conjunto soluci\'on es
\begin{equation}
\left\{\left[
\begin{array}{c}
e^t\\
e^t
\end{array}
\right],\left[
\begin{array}{c}
e^{-t}\\
3e^{-t}
\end{array}
\right]
\right\}\Rightarrow\\
\Psi(t)=
\left[
\begin{array}{cc}
e^t & e^{-t}\\
e^t & 3e^{-t}
\end{array}
\right]
\end{equation}
Verificando el Wronskiano
\begin{equation*}
|\Psi(t)|=
\left|
\begin{array}{cc}
e^t & e^{-t}\\
e^t & 3e^{-t}
\end{array}
\right|=
\left|
\begin{array}{cc}
1 & 1\\
1 & 3
\end{array}
\right|=2\neq 0
\end{equation*}
Resolviendo para $v'_2(t)$
\begin{equation*}
\begin{array}{cc}
\left.
\begin{array}{rcrcr}
v'_1(t)e^t&+&e^{-t}v'_2(t) & = & e^t\\
v'_1(t)e^t&+&3e^{-t}v'_2(t) & = & t
\end{array}
\right\} & \begin{array}{c}
(-1)\\
(1)
\end{array}
\end{array}
\end{equation*}
\begin{equation}
2e^{-t}v'_2(t)  =  e^t\Rightarrow
\end{equation}
\begin{eqnarray}
v'_2(t)=\frac{1}{2}(te^t-e^{2t})\Rightarrow\\
v_2(t)=\int\frac{1}{2}(te^t-e^{2t})dt\Rightarrow\\
v_2(t)=\int\frac{1}{2}te^tdt-\int\frac{1}{2}e^{2t}dt\label{intV2}
\end{eqnarray}
Resolviendo para $v'_1(t)$
\begin{equation*}
\begin{array}{cc}
\left.
\begin{array}{rcrcr}
v'_1(t)e^t&+&e^{-t}v'_2(t) & = & e^t\\
v'_1(t)e^t&+&3e^{-t}v'_2(t) & = & t
\end{array}
\right\} & \begin{array}{c}
(3)\\
(-1)
\end{array}
\end{array}
\end{equation*}
\begin{eqnarray}
\nonumber 2v'_1(t)e^t  =  3e^t-t\Rightarrow\\
v'_1(t)= \frac{1}{2}(3e^t-t)e^{-t}\Rightarrow\\
v_1(t)=\int\frac{3}{2}-\frac{1}{2}te^{-t}dt\label{intV1}
\end{eqnarray}
\begin{equation}
X_p=\left[
\begin{array}{cc}
e^t & e^{-t}\\
e^t & 3e^{-t}
\end{array}
\right]\left[
\begin{array}{c}
v_1(t)\\
v_2(t)
\end{array}
\right]
\end{equation}
Por tanto, la soluci\'on general para $\mathbb{X}(t)$ se da por la siguiente ecuaci\'on
\begin{equation}
\mathbb{X}(t)=C_1\left(\begin{array}{c}e^t\\e^t\end{array}\right)+C_2\left(\begin{array}{c}e^{-t}\\3e^{-t}\end{array}\right)+\left(\begin{array}{c}x_p1(t)\\x_p2(t)\end{array}\right)
\end{equation}
Siendo el resultado para (\ref{intV2})
\begin{eqnarray}
\nonumber\int\frac{1}{2}te^tdt-\int\frac{1}{2}e^{2t}dt = \\
\nonumber\frac{1}{2}\int te^tdt-\frac{1}{2}\int e^{2t}dt = \\
\nonumber\frac{1}{2}\left(te^t-\int e^tdt\right)-\frac{1}{2}\left(\frac{e^t}{2}\right)=\\
\nonumber\frac{1}{2}\left(te^t-e^t\right)-\frac{1}{2}\left(\frac{e^t}{2}\right)=\\
\nonumber \frac{1}{2}\left(te^t-e^t-\frac{e^{2t}}{2}\right)=\\
\frac{2te^t-2e^t-e^{2t}}{4}\Rightarrow\\
v_2(t)=-\frac{e^t(e^{t}-2t+2)}{4}
\end{eqnarray}
Y siendo el resultado para (\ref{intV1})
\begin{eqnarray}
\nonumber\int\frac{3}{2}-\frac{1}{2}te^{-t}dt=\\
\nonumber\frac{3}{2}t-\frac{1}{2}\int te^{-t}dt=\\
\nonumber\frac{3}{2}t-\frac{1}{2}\left(-te^{-t}-\int -e^{-t}dt\right)=\\
\nonumber\frac{3}{2}t-\frac{1}{2}\left(-te^{-t}+\int e^{-t}dt\right)=\\
\nonumber\frac{3}{2}t-\frac{1}{2}\left(-te^{-t}-e^{-t}\right)=\\
\nonumber \frac{3t}{2}+\frac{te^-t}{2}+\frac{e^{-t}}{2}\Rightarrow\\
v_1(t)=\frac{e^{-t}(t+1)+3t}{2}
\end{eqnarray}
\end{document}