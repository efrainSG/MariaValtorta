\documentclass[12pt,spanish,lettersize,twocolumn]{article}
\usepackage[latin1]{inputenc}
\usepackage[spanish]{babel}
\usepackage[dvips]{graphicx}
\usepackage[usenames,dvipsnames]{xcolor}

\title{\color{Maroon}Comandos b\'asicos en GNU/Linux: UBUNTU}
\author{L.C.C. Efra\'in Serna Gracia}
\date{\color{gray}\today}

\begin{document}
\maketitle
\begin{description}
\item[ls]: listar. Es el primer comando que todo linuxero debe aprender. Nos muestra el contenido de la carpeta que le indiquemos despu\'es. Por ejemplo. Si queremos que nos muestre lo que contiene /etc: \emph{\$ ls /etc}\\

Si no ponemos nada interpretar\'a que lo que queremos ver es el contenido de la carpeta donde estamos actualmente:

\emph{\$ ls}\\

Adem\'as acepta ciertos argumentos que pueden ser interesantes. Para mostrar todos los archivos y carpetas, incluyendo los ocultos:

\emph{\$ ls -a}\\

Para mostrar los archivos y carpetas junto con los derechos que tiene, lo que ocupa, etc:

\emph{\$ ls -l}\\

Adem\'as se pueden solapar los argumentos. Si quisi\'eramos mostrar los archivos de la misma forma que antes, pero que muestre tambi\'en los ocultos:

\emph{\$ ls -la}\\

\item[cd]: cambiar directorio. Podemos usarlo con rutas absolutas o relativas. En las absolutas le indicamos toda la ruta desde la ra\'iz (/). Por ejemplo, estemos donde estemos, si escribimos en consola\ldots

\emph{\$ cd /etc/apt}\\

\ldots nos llevar\'a a esa carpeta directamente. Del mismo modo si escribimos \ldots

\emph{\$ cd /}\\

\ldots nos mandar\'a a la ra\'iz del sistema de ficheros.

Las rutas relativas son relativas a algo, y ese algo es la carpeta donde estemos actualmente. Imaginad que estamos en /home y queremos ir a una carpeta que se llama temporal dentro de vuestra carpeta personal. Con escribir \ldots

\emph{\$ cd tucarpeta/temporal}\\

\ldots nos situar\'a all\'i. Como v\'eis hemos obviado el /home inicial ya que si no lo introducimos toma como referencia el directorio donde estamos, que es ese.

¿Y qu\'e sucede si escribimos tan s\'olo \ldots

\emph{\$ cd}\\

S\'i, s\'olo ''cd''. Esto lo que hace es que te lleva a tu carpeta personal directamente y estemos donde estemos. Es algo realmente muy pr\'actico, muy simple y que no todos conocen.

\emph{\$ cd -}\\
Nos devuelve al último directorio en el que estuvi\'eramos. Lo descubr\'i por accidente.
Es muy pr\'actico cuando queremos editar ficheros en dos lugares.
\emph{\$ cd /etc/X11/}\\
\emph{\$ cd /backup/}\\
\emph{\$ cd -}\\
Nos lleva a /etc/X11
\emph{\$ cd -}\\
nos devuelve a /backup

\item[mkdir]: hacer directorio. Crea una carpeta con el nombre que le indiquemos. Nuevamente podemos usar rutas absolutas y relativas. Podemos indicarle toda la ruta que le precede al directorio que queremos crear, o si estamos ya en la carpeta que lo va a contener basta con poner tan s\'olo el nombre:

\emph{\$ mkdir /home/tucuenta/pepino}\\

Si ya estamos en /home/tucuenta\ldots

\emph{\$ mkdir pepino}\\

\item[rm]: borrar. Borra el archivo o la carpeta que le indiquemos. Como antes se puede indicar la ruta completa o el nombre del archivo. Esto a partir de ahora lo vamos a obviar, creo que ya ha quedado claro con los dos comandos anteriores.

Para borrar un archivo:

\emph{\$ rm nombrearchivo}\\

Para borrar una carpeta vac\'ia:

\emph{\$ rm nombrecarpeta}\\

Para borrar una carpeta que contiene archivos y/o otras carpetas que pueden incluso contener m\'as:

\emph{\$ rm -r nombrecarpeta}\\

Otras opciones: ''-f'' no te pide una confirmaci\'on para eliminar o ''-v'' va mostrando lo que va borrando.

\item[cp]: copiar. Copia el archivo indicado donde le digamos. Aqu\'i podemos tambi\'en jugar con las rutas, tanto para el fichero origen, como en el del destino. Tambi\'en pod\'eis poner el nombre que le quer\'eis poner a la copia. Por ejemplo, si estuvi\'eramos en /etc/X11 y quisi\'eramos hacer una copia de seguridad de xorg.conf en nuestra carpeta personal:

\emph{\$ cp xorg.conf /home/tucarpeta/xorg.conf.backup}\\

Para copiar un directorio se usa cp -r
\emph{\$ cp -r /etc /backup/}\\
Tendremos /backup/etc, y dentro de ese directorio tendremos lo mismo que hay en /etc

\item[mv]: mover. Es igual que el anterior, s\'olo que en lugar de hacer una copia, mueve directamente el archivo con el nombre que le indiquemos, puede ser otro distinto al original:

\emph{\$ mv /etc/pepino.html /home/tucarpeta/esepepino.html}\\

Otro uso muy pr\'actico que se le puede dar es para renombrar un archivo. Basta con indicar el nuevo nombre en el segundo argumento con la misma ruta del primero. En este ejemplo suponemos que ya estamos en la carpeta que lo contiene:

\emph{\$ mv pepino.html esepepino.html}\\

Tambien permite mover directorios o renombrarlos
\emph{\$ mv /home/yo/misdocumentos /home/yo/misdocumentos}\\

\item[find]: encontrar. Busca el archivo o carpeta que le indiques:

\emph{\$ find / -name pepino}\\

El comando anterior buscar\'ia en todos los sitios las carpetas y archivos que se llamen pepino. Si tuvi\'eramos la seguridad de que se encuentra en /var por ejemplo, se lo indicar\'iamos:

\emph{\$ find /var -name pepino}\\

Si no estamos muy seguros del nombre podemos indic\'arselo con comodines. Supongamos que el nombre de lo que buscamos contiene ''pepi'', en la misma carpeta de antes:

\emph{\$ find /var -name *pepi*}\\

Tiene otras opciones. Por ejemplo podemos decirle que encuentre los archivos/carpetas de m\'as de 1500 KB:

\emph{\$ find / -size +1500}\\

O los archivos/carpetas contienen el nombre ''pepi'' y tienen menos de 1000 KB:

\emph{\$ find / -name *pepi* -size -1000}\\

\item[clear]: despejar. Limpia la pantalla/consola qued\'andola como si acab\'aramos de abrirla.

\emph{\$ clear}\\

\item[ps]: estado de los procesos. Nos muestra lo que queramos saber de los procesos que est\'an corriendo en nuestro sistema. Cada proceso est\'a identificado con un número llamado PID. Si hacemos \ldots

\emph{\$ ps -A}\\

\ldots nos mostrar\'a un listado de todos los procesos, su PID a la izquierda y su nombre a la derecha. Si queremos m\'as informaci\'on:

\emph{\$ ps aux}\\

\item[kill]: matar. Elimina el proceso que le indiquemos con su PID:

\emph{\$ kill}\\

En ocasiones el proceso no ''muere'' del todo, pero se le puede forzar al sistema para que lo mate con seguridad del siguiente modo:

\emph{\$ kill -9}\\

\item[sudo]: hacer como superusuario. La cuenta de usuario en Ubuntu es relativamente normal. Tiene derechos de administrador a medias. Me explico, los tiene, pero cada vez que se haga algo importante y de riesgo para el sistema, hay que hacerlo mediante el prefijo ''sudo'' y escribiendo despu\'es la contrase\~na.

Por ejemplo, algo que hemos hecho muchas veces en los tutoriales es hacer una copia de seguridad del fichero xorg.conf. \'Este est\'a localizado en la carpeta /etc/X11 y ah\'i ningún usuario puede hacer modificaciones o borrar nada si no es el administrador o tiene derechos como tal, gracias a sudo. Por eso hac\'iamos siempre:

apenas lo uso, en su lugar utilizo killall, que permite matar por nombre de proceso, y los procesos llamarse siempre igual (no as\'i sus pid), permite no tener que hacer un ps para saber qu\'e tienes que poner.
\emph{\$ killall firefox-bin}\\
Eso s\'i, no es útil si tienes varios procesos abiertos con el mismo nombre, puesto que los cerrar\'a todos.
Tambi\'en acepta entre killall y el nombre del proceso un -X, donde X es la se\~nal que quieras (9 para matarlo, 15 para que intente cerrar \'el solo)
\emph{\$ killall -15 firefox-bin (vaya, no va)}\\
\emph{\$ killall -9 firefox-bin}\\

\emph{\$ sudo cp /etc/X11/xorg.conf /etc/X11/xorg.conf}\\

Siempre que necesitemos hacer un apt-get/aptitude update o install y acciones de este tipo, tendremos que poner antes el ''sudo''.

\item[passwd]: contrase\~na. Con este comando podremos cambiar la contrase\~na de nuestra cuenta. Primero nos pedir\'a la contrase\~na actual como medida de seguridad. Despu\'es nos pedir\'a que introduzcamos dos veces seguidas la nueva contrase\~na.

\emph{\$ passwd}\\

\item[su]: superusuario. Mediante su podremos loguearnos como superusuario. Tras escribirlo nos pedir\'a la contrase\~na de root y estaremos como administrador. Podremos hacer todo lo que queramos.

\emph{\$ su}\\

Este comando tambi\'en nos permite hacer login con otra cuenta distinta. Por ejemplo, imaginemos que tenemos otra cuenta, adem\'as de root y la nuestra, llamada ''invitado''. Para hacer login como tal bastar\'ia con poner:

\emph{\$ su invitado}\\

y despu\'es escribir la contrase\~na de esa cuenta.

\item[sudo passwd]
No es un comando propiamente dicho, pero es interesante que lo conozc\'ais. Gracias a la uni\'on de estos dos comandos podr\'eis cambiar la contrase\~na de root (la del super-usuario).

\emph{\$ sudo passwd}\\

\item[apt]: herramienta avanzada de paquetes. Es uno de los comandos m\'as útiles que se han desarrollado en los sistemas GNU/Linux debian o basados en esta distro. Nos permite comprobar actualizaciones, actualizar todo el sistema. Tambi\'en nos ofrece funcionalidad para buscar, descargar e instalar paquetes con una sola orden.

Tenemos variantes, las m\'as usadas son las siguientes:

\emph{\$ apt-cache search nombrepaquete}\\

Busca nombrepaquete para ver si existe literal o aproximadamente ofreci\'endonos los paquetes que pudieran ser en caso de que hayamos puesto un nombre aproximado.

\emph{\$ apt-get update}\\

Actualiza los repositorios que son los que contienen los paquetes. Los repositorios son como las direcciones que contienen nuestros paquetes. apt-get update lo que hace es actualizar el listado de todos esos paquetes, con la direcci\'on de d\'onde obtenerlos para que a la hora de hacer la búsqueda y su posterior descarga sea m\'as r\'apida haci\'endolo en nuestro ordenador.

\emph{\$ apt-get upgrade}\\

Actualiza nuestro sistema con todas las posibles actualizaciones que pudiera haber. La actualizaci\'on no se realiza s\'olo sobre el propio sistema operativo, sino tambi\'en sobre las aplicaciones que est\'en contenidas en los repositorios. Una útil forma de estar siempre al d\'ia.

\emph{\$ apt-get install nombrepaquete}\\

Localizado el nombre del paquete que queremos descargar e instalar, este comando se encargar\'a del proceso. Buscar\'a en nuestro \'indice (el que se crea con update) de d\'onde tiene que descargarse el paquete, lo hace y posteriormente lo instala.

\emph{\$ apt-get remove [--purge] nombrepaquete}\\

Elimina el paquete especificado del sistema. Damite el argumento ''-purge'' (corchetes = opcional) para que borre tambi\'en los ficheros de configuraci\'on.

\emph{\$ apt-get autoremove}\\

Elimina paquetes que han quedado inservibles tras realizar algún apt-get remove, los llamados hu\'erfanos. Normalmente tras hacer este último te avisa en un mensaje que lo realices.

Todos estos comandos necesitan tener privilegios de administraci\'on, as\'i que si no los us\'ais como root, deb\'eis agregar primero el conocido ''sudo''.

\item[aptitude]: aptitud, habilidad. En el fondo juega con las siglas de apt para crear aptitude. Es una versi\'on mejorada de apt. Si os hab\'eis fijado en todos los manuales y entradas donde hab\'ia un proceso de instalaci\'on he usado aptitude en lugar de apt. El segundo es quiz\'a el m\'as extendido al ser el que vio la luz primero.

aptitude naci\'o como un front-end de apt, es decir, como una especie de aplicaci\'on gr\'afica y en modo texto para realizar todo lo que hace apt. Pero lo cierto es que sus caracter\'isticas son mejores.

apt cuando instala algo te puede realizar una sugerencia para que instales algo m\'as que te podr\'ia venir bien, pero no lo hace. Hay programas que a la vez usan otros para algunas de sus funciones u opciones. apt no instalar\'ia los segundos, como mucho te avisar\'ia. Sin embargo aptitude s\'i que lo instalar\'a porque sabe que de alguna forma es indispensable para el que has pedido.

De la misma forma, si con apt instalas luego ese programa que es usado por otro, cuando desinstalas el principal, no se desinstalar\'a el secundario, aunque \'este ya no tenga mucho sentido que est\'e instalado, y lo mismo sucede con librer\'ias. aptitude est\'a capacitado para desinstalar lo que \'el mismo ha instalado como recomendaci\'on. Te deja el sistema m\'as limpio tras las desinstalaciones.

Para abrir el interfaz gr\'afico de aptitude, tan s\'olo hay que teclearlo: \emph{\$ aptitude}

Sin embargo, tambi\'en se puede usar exactamente igual que apt, pero con las caracter\'isticas que he comentado de aptitude:

\emph{\$ aptitude search nombrepaquete}\\
\emph{\$ aptitude install nombrepaquete}\\
\emph{\$ aptitude remove nombrepaquete}\\
\emph{\$ aptitude purge nombrepaquete}\\
\emph{\$ aptitude update}\\
\emph{\$ aptitude upgrade}\\

Y al igual que antes, necesitar\'eis usarlo con el sudo delante si no est\'ais como administrador.

\item[dpkg]: despaquetar. Los paquetes cuando se instalan sufren un proceso de despaquetaje. En el fondo un paquete .deb contiene una serie de scripts de pre-instalaci\'on, post-instalaci\'on y los archivos en cuesti\'on del paquete.

Este comando lo usaremos para instalar un paquete .deb que ya tengamos descargados en nuestro sistema. En muchas ocasiones hay una aplicaci\'on que no est\'a en los repositorios y nos hemos bajado el .deb para instalarlo con el interfaz gr\'afico que corresponda (GDebi en el caso de GNOME).

En el fondo estas interfaces gr\'aficas est\'an basadas en dpkg. Si queremos instalar un paquete ya descargado mediante consola usaremos el argumento '-i' (i=install):

\emph{\$ dpkg -i nombrepaquete}\\

Para desinstalarlo '-r' (r=remove):

\emph{\$ dpkg -r nombrepaquete}\\

Para desinstalar el paquete y los ficheros de configuraci\'on ''-purge'' (purgar):

\emph{\$ dpkg -r -purge nombrepaquete}\\

\item[alien]
A petici\'on de lector.

Alien: de otro pa\'is, de otro planeta. Aunque Debian -y por extensi\'on Ubuntu- dispone de una ingente cantidad de paquetes en sus repositorios, puede que alguien tenga algún problema en encontrar una aplicaci\'on espec\'ifica empaquetada como le interesa aunque ha visto el paquete que quiere para otras distros.

alien es bastante pr\'actico para estas situaciones ya que nos permite transformar un paquete de un gestor de paquetes determinado en otro. Por ejemplo podemos pasar de un .deb (Debian) a un .rpm (Red Hat) y viceversa. Las extensiones soportadas son:
\begin{itemize}
\item deb (Debian)
\item rpm (Red Hat)
\item slm (Stampede)
\item tgz (Slackware)
\item pkg (Solaris)
\end{itemize}
Su uso es sencillo. Lo que debemos saber es el argumento que transformar\'a el paquete original en la extensi\'on objetivo:

\begin{itemize}
\item ''-to-deb'' o ''-d'' para transformar a .deb
\item ''-to-rpm'' o ''-r'' para transformar a .rpm
\item ''-to-tgz'' o ''-t'' para transformar a .tgz
\item ''-to-pkg'' o ''-p'' para transformar a .pkg
\item ''-to-slp'' para transformar a .slp
\end{itemize}

Como ejemplo, pasaremos un supuesto paquete de Red Hat llamado ''pepino.rpm'' a ''pepino.deb'':
\emph{\$ alien -d pepino.rpm}\\

\item[man]: manual. Es otro de los comandos de gran potencia en linux. Normalmente queda programa o comando viene con un archivo de ayuda muy completo sobre su uso y sus argumentos. Cuando desconozc\'ais c\'omo se usa y qu\'e argumentos tiene un comando o aplicaci\'on tan s\'olo ten\'eis que escribir en consola:

\emph{\$ man nombre}\\

En ocasiones la informaci\'on que nos ofrece man puede llegar a ser excesiva. Casi todos los comandos y aplicaicones aceptan el argumento ''-help'' para que muestre cierta ayuda m\'as resumida. Por ejemplo con aptitude:

\emph{\$ aptitude -help}
\item[alias]: permite crear atajos para comandos
ejemplo:

\emph{\$ alias ll=''ls -l''}\\
\emph{\$ alias instala=''sudo aptitude install''}\\

Tambi\'en se pueden dejar permanentes, para agregar los alias permanentemente editan el archivo \emph{\$ nano /home/usuario/.bashrc}
y agregan al final del archivo sus alias

\begin{verbatim}
alias actualiza=''sudo aptitude update''
alias upgrade=''sudo aptitude upgrade''
alias instala=''sudo aptitude install''
alias purge=''sudo aptitude purge''
alias show=''sudo aptitude show''
alias busca=''sudo aptitude search''
alias apagate=''sudo shutdown -h now''
\end{verbatim}
\end{description}
Todos los comandos que os he mostrado tienen muchos m\'as argumentos. Os he puesto los m\'as usados o necesarios, as\'i que si hay alguno que os interesa particularmente conocer m\'as de \'el, ten\'eis ''man'' o ''-help'' para obtener m\'as detalles.\\

Fuente: \emph{http://ubunturoot.wordpress.com/2007/11/06/comandos-basicos-para-linux/}
\end{document}